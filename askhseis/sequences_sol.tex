\input{preamble/preamble.tex}
\input{preamble/definitions.tex}

\everymath{\displaystyle}

\begin{document}

\begin{center}
    \minibox[frame,c,pad=5pt]{\large \bfseries Ακολουθίες\\ \large Ασκήσεις}
\end{center}

\vspace{\baselineskip}


\begin{enumerate}
    \item Φραγμένες Ακολουθίες

        \begin{enumerate}[i)]

            \item Να δείξετε ότι η ακολουθία $ a_{n} = \frac{n}{3^{n}} $ είναι 
                φραγμένη. 

                \begin{proof}
                \item {}
                    $ \inlineequation[eq:ask1a]{3^{n}=(1+2)^{n} \geq 1+2n > 2n}, \; 
                    \forall n \in \mathbb{N} $. Άρα 
                    \[
                        \abs{a_{n}} = \abs{\frac{n}{3^{n}}} =  \frac{n}{3^{n}} 
                        \overset{\eqref{eq:ask1a}}{<} \frac{n}{2n} 
                        = \frac{1}{2}, \; \forall n \in \mathbb{N}
                    \] 
                \end{proof}

            \item Να δείξετε ότι η ακολουθία $ a_{n} = \frac{n!}{n^{n}} $ είναι 
                φραγμένη. 

                \begin{proof}
                    \[
                        \abs{a_{n}} = \abs{\frac{n!}{n^{n}}} = \frac{n!}{n^{n}} = 
                        \frac{1 \cdot 2 \cdots n}{n^{n}} \leq 
                        \frac{\overbrace{n \cdot n \cdots n}^ 
                        {n-\text {φορές}}}{n^{n}} = \frac{n^{n}}{n^{n}} 
                        = 1, \; \forall n \in \mathbb{N}
                    \]
                \end{proof}

            \item Να δείξετε ότι η ακολουθία $ a_{n} = \frac{\cos{n} + n 
                \sin{n}}{n^{2}} $ είναι φραγμένη. 

                \begin{proof}
                    \begin{align*}
                        \abs{\frac{\cos{n} + n \sin{n}}{n^{2}}} = 
                        \frac{\abs{\cos{n} + n \sin{n}}}{n^{2}} \leq 
                        \frac{\abs{\cos{n}} + \abs{n \sin{n}}}{n^{2}} 
               &= \frac{\abs{\cos{n}} + n \abs{\sin{n}}}{n^{2}} \leq 
               \frac{1 + n\cdot 1}{n^{2}} = \\
               &= \frac{1}{n^{2}} + \frac{1}{n} \leq 
               1 + 1 = 2, \; \forall n \in \mathbb{N}
                    \end{align*} 
                \end{proof}

            \item Να δείξετε ότι η ακολουθία $ a_{n} = \frac{5 \cos^{3}{n}}{n+2} $ 
                είναι φραγμένη.

                \begin{proof}
                    \[
                        \abs{a_{n}} = \abs{\frac{5 \cos^{3}{n}}{n+2}} = 
                        \frac{5 \cdot \abs{\cos^{3}{n}}}{n+2} = \frac{5\cdot 
                        \abs{\cos{n}}^ {3}}{n+2} \leq  \frac{5 \cdot 1^{3}}{n+2} < 
                        \frac{5}{2}, \; \forall n \in \mathbb{N}
                    \]
                \end{proof}

            \item Να δείξετε ότι η ακολουθία $ a_{n} = \frac{3 \sin{3n}}{n^{2}} $ 
                είναι φραγμένη.

                \begin{proof}
                    \[
                        \abs{a_{n}} = \abs{\frac{3 \sin{3n}}{n^{2}}} = 
                        \frac{3 \cdot \abs{\sin{3n}}}{n^{2}} \leq 
                        \frac{3 \cdot 1}{n^{2}} \leq 3, \; \forall n \in \mathbb{N}  
                    \] 
                \end{proof}

            \item Να δείξετε ότι η ακολουθία $ a_{n} = \frac{(-1)^{n}}{n} $ είναι 
                φραγμένη.

                \begin{proof}
                    \[
                        \abs{\frac{(-1)^{n}}{n}} = \frac{\abs{(-1)^{n}}}{n} = 
                        \frac{\abs{-1}^{n}}{n} = \frac{1}{n} \leq 1, \; \forall n \in 
                        \mathbb{N} 
                    \] 
                \end{proof}

            \item Να δείξετε ότι η ακολουθία $ a_{n} = \frac{n}{2^{n}}  $ είναι 
                φραγμένη.

                \begin{proof}
                \item {}
                    $ 2^{n} = (1+1)^{n} \geq 1+1\cdot n > n, \; \forall n \in 
                    \mathbb{N} $ 
                    \[
                        0 < a_{n} = \frac{n}{2^{n}} < \frac{n}{n} = 1, \; \forall n \in 
                        \mathbb{N} 
                    \]
                \end{proof}

            \item Να δείξετε ότι η ακολουθία $ a_{1} = 3, \; a_{n+1} =
                \frac{a_{n}+4}{2}, \; \forall n \in \mathbb{N} $ είναι άνω φραγμένη.

                \begin{proof}
                \item {}
                    Προφανώς, $ a_{n} > 0, \; \forall n \in \mathbb{N} $ 

                    Παρατηρούμε ότι, $ a_{2}=3.5, \; a_{3}=3.75, \; a_{4}=3.875, \; 
                    \ldots $

                    Θ.δ.ο. $ a_{n} < 4, \; \forall n \in \mathbb{N} $ 
                    \begin{itemize}
                        \item Για $ n=1 $, έχουμε: $ a_{1}=3 < 4 $, ισχύει.
                        \item Έστω ότι ισχύει για $n$, δηλαδή $ a_{n} < 4 $
                        \item Θ.δ.ο. ισχύει για $ n+1 $. Πράγματι
                            \[
                                a_{n}<4 \Leftrightarrow a_{n}+4 < 8 \Leftrightarrow 
                                \frac{a_{n}+4}{2} < 4 \Leftrightarrow a_{n+1} <4, \; 
                                \forall n \in \mathbb{N} 
                            \] 
                    \end{itemize}
                \end{proof}

            \item Να δείξετε ότι η ακολουθία $ a_{1} = \sqrt{2}, \; a_{n+1} =
                \sqrt{2+ a_{n}}, \; \forall n \in \mathbb{N} $ είναι άνω φραγμένη.

                \begin{proof}
                \item {}
                    Προφανώς, $ a_{n} > 0, \; \forall n \in \mathbb{N} $ 

                    Παρατηρούμε ότι, $ a_{2}= \sqrt{2 + \sqrt{2}} < \sqrt{2+2} =2, \; 
                    a_{3}= \sqrt{2+ \sqrt{2 + \sqrt{2}}} < \sqrt{2+ \sqrt{2+2}} = 2, 
                    \ldots $ 

                    Θ.δ.ο. $ a_{n} < 2, \; \forall n \in \mathbb{N} $ 
                    \begin{itemize}
                        \item Για $ n=1 $, έχουμε: $ a_{1}= \sqrt{2} < 2 $, ισχύει.
                        \item Έστω ότι ισχύει για $n$, δηλαδή $ a_{n} < 2 $
                        \item Θ.δ.ο. ισχύει για $ n+1 $. Πράγματι
                            \[
                                a_{n}<2 \Leftrightarrow 2+a_{n} < 4 \Leftrightarrow 
                                \sqrt{2+ a_{n}} < 2\Leftrightarrow a_{n+1} <2, \; 
                                \forall n \in \mathbb{N} 
                            \] 
                    \end{itemize}
                \end{proof}


            \item Να δείξετε ότι η ακολουθία $ a_{n} = 1 + \frac{1}{1!} +
                \frac{1}{2!} + \frac{1}{n!} $ είναι άνω φραγμένη.

                \begin{proof}
                \item {} 
                    Ισχύει ότι $ n! > 2^{n-1}, \; \forall n \in \mathbb{N} $, άρα 
                    \begin{align*}
                        a_{n} &= 1 + \frac{1}{1!} + \frac{1}{2!} + \frac{1}{3!} + 
                        \frac{1}{4!} + \cdots \frac{1}{n!} < 1 + 
                        \underbrace{\frac{1}{2^{0}} 
                            + \frac{1}{2^{1}} + \frac{1}{2^{2}} + \frac{1}{2^{3}} 
                        + \cdots \frac{1}{2^{n-1}}}_ {\text{γεωμ. πρόοδος}} = 
                        1 + \frac{\left(\frac{1}{2}\right)^{n}-1}{\frac{1}{2} -1} = \\ 
                              &= 1 + 2\left(1- \frac{1}{2^{n}}\right) = 1+2- 
                              \frac{1}{2^{n-1}} < 3 
                    \end{align*} 
                \end{proof}

            \item Να δείξετε ότι η ακολουθία $ a_{n} = 2^{n} $ δεν είναι άνω 
                φραγμένη.

                \begin{proof}
                \item {} 
                    Έστω ότι η $ a_{n}=2^{n} $ είναι άνω φραγμένη. Τότε 

                    \begin{minipage}{0.35\textwidth}
                        \begin{itemize}
                            \item $ \exists M \in \mathbb{R} \; : \; 2^{n} \leq M, \; 
                                \forall n \in \mathbb{N} $ \hfill \tikzmark{a}
                            \item $ 2^{n}=(1+1)^{n} \geq 1+n, \; \forall n \in 
                                \mathbb{N} $ \hfill \tikzmark{b}
                        \end{itemize}    
                    \end{minipage}

                    \mybrace{a}{b}[$ M \geq 1+n \Leftrightarrow n \leq M-1, \; 
                    \forall n \in \mathbb{N} $] 
                    άτοπο, γιατί $ \mathbb{N} $ δεν είναι άνω φραγμένο.
                \end{proof}

            \item Να δείξετε ότι η ακολουθία $ a_{n} = \frac{n^{3} + \sin{5n}}{n} $ 
                δεν είναι φραγμένη.

                \begin{proof}
                \item {}
                    Έστω ότι η $ a_{n}= \frac{n^{3}+ \sin{5n}}{n} $ είναι φραγμένη. 
                    Τότε 

                    \begin{minipage}{0.6\textwidth}
                        \begin{itemize}
                            \item $ \exists M>0 \; : \; \abs{\frac{n^{3}+ 
                                \sin{5n}}{n}} \leq M \Leftrightarrow \abs{n^{3}+ 
                            \sin{5n}} \leq nM  \; $ 
                                \hfill \tikzmark{a}
                            \item $ n^{2}-1 < n^{3}-1 \leq \abs{n^{3}}- \abs{\sin{5n}} 
                                \leq \abs{n^{3}+ \sin{5n}}$ 
                                \hfill \tikzmark{b}
                        \end{itemize}
                    \end{minipage}

                    \mybrace{a}{b}[$n^{2}-nM-1<0 $] 
                    Το οποίο ισχύει ανν $ r_{1}<n< r_{2} $,
                    όπου $ r_{1}, r_{2} $ οι ρίζες του τριωνύμου, το οποίο έχει $ 
                    \Delta= M^{2} + 4 > 0$. Όμως αυτό είναι άτοπο, γιατί $ \mathbb{N} $ 
                    δεν είναι άνω φραγμένο.
                \end{proof}

            \item Να δείξετε ότι η ακολουθία $ a_{n} = \frac{n^{2}}{3n+ 
                \sin^{2}{n}} $ δεν είναι άνω φραγμένη.

                \begin{proof}
                \item {}
                    Έστω ότι η $ a_{n}= \frac{n^{2}}{3n+ \sin^{2}{n}} $ είναι φραγμένη.
                    Τότε 
                    \[
                        \exists M>0 \; : \; \abs{\frac{n^{2}}{3n+ \sin^{2}{n}}} 
                        \leq M \Leftrightarrow n^{2} \leq M(3n+ \sin^{2}{n}) \leq 
                        M(3n+1) \Leftrightarrow n^{2}-3Mn-M \leq 0
                    \]
                    Το οποίο ισχύει ανν $ r_{1}<n< r_{2} $,
                    όπου $ r_{1}, r_{2} $ οι ρίζες του τριωνύμου, το οποίο έχει $ 
                    \Delta= 9M^{2}+4M >0 $. Όμως αυτό είναι άτοπο, γιατί $ \mathbb{N} $ 
                    δεν είναι άνω φραγμένο.
                \end{proof}

        \end{enumerate}

    \item Μονότονες Ακολουθίες

        \begin{enumerate}[i)]

            \item Να δείξετε ότι ακολουθία $ a_{n} = \frac{n}{5n -1} $ είναι 
                γνησίως φθίνουσα.

                \begin{proof}
                    \begin{align*} 
                        a_{n+1}- a_{n} = \frac{n+1}{5(n+1)-1} - \frac{n}{5n -1} = 
                        \frac{n+1}{5n+4} - \frac{n}{5n-1} 
                    &= \frac{5n^{2}+5n-n-1-5n^{2}-4n}{(5n+4)(5n-1)} \\ 
                    &= - \frac{1}{(5n+4)(5n-1)} < 0 
                    \end{align*} 
                \end{proof}

            \item Να δείξετε ότι ακολουθία $ a_{n} = \frac{n}{3^{n}} $ είναι 
                γνησίως φθίνουσα.

                \begin{proof}
                    \[
                        a_{n+1}- a_{n} = \frac{n+1}{3^{n+1}} - \frac{n}{3^{n}} = 
                        \frac{n+1-3n}{3^{n+1}} = \frac{1-2n}{3^{n+1}} < 0, \; 
                        \forall n \in \mathbb{N}
                    \] 
                \end{proof}

            \item Να δείξετε ότι η ακολουθία $ a_{1}=0, a_{n+1}= 
                \frac{2 a_{n}+4}{3}, \; \forall n \in \mathbb{N} $ είναι γνησίως 
                αύξουσα.

                \begin{proof}
                \item {}
                    \begin{description}
                        \item 
                            Θα δέιξουμε ότι $ a_{n+1}> a_{n}, \; \forall n \in 
                            \mathbb{N} $. Πράγματι,
                            \begin{itemize}
                                \item Για $ n=1 $, έχουμε $ a_{2}= \frac{4}{3} > 0 
                                    = a_{1}$, ισχύει.
                                \item Έστω ότι ισχύει για $n$, δηλ. $ a_{n+1} > a_{n} $.
                                \item Θ.δ.ο. ισχύει για $ n+1 $. Πράγματι,
                                    \[
                                        a_{n}< a_{n+1} \Leftrightarrow 2 a_{n} 
                                        < 2 a_{n+1} 
                                        \Leftrightarrow 2 a_{n}+4 < 2 a_{n+1}+4 
                                        \Leftrightarrow \underbrace{\frac{2 a_{n}+4}{3}}
                                        _{a_{n+1}} < 
                                        \underbrace{\frac{2 a_{n+1}+4}{3}}_{a_{n+2}}, 
                                        \; \forall n \in \mathbb{N}
                                    \]
                            \end{itemize}

                        \item [Β᾽ Τρόπος:]
                            \[
                                a_{n+1}- a_{n} > 0 \Leftrightarrow \frac{2 a_{n}+4}{3} 
                                - a_{n}>0 
                                \Leftrightarrow \frac{4 - a_{n}}{3} > 0 
                                \Leftrightarrow a_{n}< 4, \; \forall n \in \mathbb{N}
                            \] 
                            Θα δείξουμε ότι $ a_{n}<4, \; \forall n \in \mathbb{N} $.
                            Πράγματι,
                            \begin{itemize}
                                \item Για $ n=1 $, έχουμε $ a_{1}=0 < 4 $, ισχύει.
                                \item Έστω ότι ισχύει για $n$, δηλ. $ a_{n}<4 $
                                \item Θ.δ.ο. ισχύει για $ n+1 $. Πράγματι,
                                    \[
                                        a_{n+1}= \frac{2 a_{n}+4}{3} < \frac{2\cdot 4 
                                        + 4}{3} = \frac{12}{3} = 4
                                    \] 
                            \end{itemize}
                    \end{description}
                \end{proof}

            \item Να δείξετε ότι ακολουθία $ a_{n} = (-1)^{n} \frac{1}{n^{2}+2} $ 
                δεν είναι μονότονη.

                \begin{proof}
                \item {}
                    Θα δείξουμε ότι η ακολουθία δεν διατηρεί πρόσημο. Πράγματι, 
                    $ \forall n \in \mathbb{N} $
                    \begin{align*}
                        a_{n+1}- a_{n} = \frac{(-1)^{n+1}}{(n+1)^{2}+2} - 
                        \frac{(-1)^{n}} {n^{2}+2} 
                        &= \frac{(-1)^{n+1}}{(n+1)^{2}+2} + 
                        \frac{(-1)^{n+1}}{n^{2}+2}= \\
                        &= (-1)^{n+1}\Biggl[\underbrace{\frac{1}{(n+1)^{2}+2} + 
                                \frac{1}{n^{2}+2}}_{b_{n} > 0, \; \forall n \in 
                        \mathbb{N}}\Biggr] 
                        = \begin{cases}
                            -b_{n}, & n \; \text{περιττός} \\
                            b_{n}, & n \; \text{άρτιος} 
                        \end{cases}
                    \end{align*} 
                \end{proof}

            \item Να δείξετε ότι ακολουθία $ a_{n} = \frac{2n^{2}-1}{n} $ είναι γνησίως 
                αύξουσα.

                \begin{proof}
                    \[
                        a_{n+1}- a_{n} = \frac{2(n+1)^{2}-1}{n+1} - \frac{2n^{2}-1}{n} 
                        = \cdots = \frac{2n^{2}+2n+1}{n(n+1)} >0, \; \forall n 
                        \in \mathbb{N}
                    \] 
                \end{proof}

            \item Να δείξετε ότι ακολουθία $ a_{1}=1, a_{n} = \sqrt{a_{n}+1}, \; 
                \forall n \in \mathbb{N}$ είναι γνησίως αύξουσα.

                \begin{proof}
                    \begin{itemize}
                        \item Για $ n=1 $, έχω $ a_{1}=1 < \sqrt{2} = a_{2} $, ισχύει.
                        \item Έστω ότι ισχύει για $n$, δηλ. $ a_{n}< a_{n+1} $
                        \item Θ.δ.ο. ισχύει για $ n+1 $. Πράγματι,
                            \[
                                a_{n} < a_{n+1} \Leftrightarrow a_{n}+1 < a_{n+1}+1 
                                \Leftrightarrow \sqrt{a_{n}+1} < \sqrt{a_{n+1}+1} 
                                \Leftrightarrow a_{n+1}< a_{n+2}
                            \] 
                    \end{itemize}
                \end{proof}

        \end{enumerate}
\end{enumerate}

% Η ακολουθία $ a_{n}=n, \; \forall n \in \mathbb{N} $ δεν συγκλίνει. 
%    \begin{proof}(Με άτοπο)
%        Έστω ότι συγκλίνει στο $ a \in \mathbb{R} $.
%        Τότε για $ \varepsilon =1>0 $ έχουμε ότι 
%        \[ \forall n \geq n_{0}, \; \exists n_{0} \in \mathbb{N} \;
%        : \; \abs{n-a} < 1 \Leftrightarrow a - 1 < n < 
%        a+ 1\] 
%        Δηλαδή, $ \forall n \geq n_{0}, \; n < a+1 $, άτοπο, γιατί $ \mathbb{N} $ 
%        όχι άνω φραγμένο.
%    \end{proof}




\end{document}
