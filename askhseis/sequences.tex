\input{preamble/preamble.tex}
\input{preamble/definitions.tex}

\everymath{\displaystyle}

\begin{document}

\begin{center}
    \minibox[frame,c,pad=5pt]{\large \bfseries Ακολουθίες\\ \large Ασκήσεις}
\end{center}

\vspace{\baselineskip}


\setcounter{chapter}{1}
\section{Φραγμένες Ακολουθίες}

\begin{enumerate}
    \item Να δείξετε ότι η ακολουθία $ a_{n} = \frac{n}{3^{n}} $ είναι 
        φραγμένη. 
        \hfill Απ: $ \abs{a_{n}}< \frac{1}{2} $
    \item Να δείξετε ότι η ακολουθία $ a_{n} = \frac{n!}{n^{n}} $ είναι 
        φραγμένη. 
        \hfill Απ: $ 0 \leq a_{n} \leq 1 $ 
    \item Να δείξετε ότι η ακολουθία $ a_{n} = \frac{\cos{n} + n 
        \sin{n}}{n^{2}} $ είναι φραγμένη. 
        \hfill Απ: $ \abs{an} \leq 2 $ 
    \item Να δείξετε ότι η ακολουθία $ a_{n} = \frac{5 \cos^{3}{n}}{n+2} $ 
        είναι φραγμένη.
        \hfill Απ: $ \abs{a_{n}} < \frac{5}{2}  $ 
    \item Να δείξετε ότι η ακολουθία $ a_{n} = \frac{3 \sin{3n}}{n^{2}} $ 
        είναι φραγμένη.
        \hfill Απ: $ \abs{a_{n}} \leq 3 $ 
    \item Να δείξετε ότι η ακολουθία $ a_{n} = \frac{(-1)^{n}}{n} $ είναι 
        φραγμένη.
        \hfill Απ: $ \abs{a_{n}} \leq 1 $ 
    \item Να δείξετε ότι η ακολουθία $ a_{n} = \frac{n}{2^{n}}  $ είναι 
        φραγμένη.
        \hfill Απ: $ 0 < a_{n} < 1 $ 
    \item Να δείξετε ότι η ακολουθία $ a_{1} = 3, \; a_{n+1} =
        \frac{a_{n}+4}{2}, \; \forall n \in \mathbb{N} $ είναι άνω φραγμένη.
        \hfill Απ: $ a_{n} < 4 $ 
    \item Να δείξετε ότι η ακολουθία $ a_{n} = 1 + \frac{1}{1!} +
        \frac{1}{2!} + \frac{1}{n!} $ είναι άνω φραγμένη.
        \hfill Απ: $ a_{n} < 3 $ 
    \item Να δείξετε ότι η ακολουθία $ a_{n} = 2^{n} $ δεν είναι άνω 
        φραγμένη.
    \item Να δείξετε ότι η ακολουθία $ a_{n} = \frac{n^{2}+1}{3n+ 
        \sin^{3}{n}} $
        δεν είναι άνω 
        φραγμένη.
\end{enumerate}

\section{Μονότονες Ακολουθίες}

\begin{enumerate}
    \item Να δείξετε ότι ακολουθία $ a_{n} = \frac{n}{3^{n}} $ είναι 
        γνησίως φθίνουσα.
    \item Να δείξετε ότι η ακολουθία $ a_{1}=0, a_{n+1}= 
        \frac{2 a_{n}+4}{3}, \; \forall n \in \mathbb{N} $ είναι γνησίως 
        αύξουσα.
    \item Να δείξετε ότι ακολουθία $ a_{n} = (-1)^{n} \frac{1}{n^{2}+2} $ 
        δεν είναι μονότονη.
\end{enumerate}

\section{Ορισμός του Ορίου}

\begin{enumerate}
    \item Να δείξετε με τη βοήθεια του ορισμού τα παρακάτω όρια.
        \begin{enumerate}[i)]
            \item $ \lim_{n \to \infty} \frac{3n -2}{2n+1} = \frac{3}{2} $ 
            \item $ \lim_{n \to \infty} \frac{5n-4}{2-3n} = - \frac{5}{3} $ 
        \end{enumerate}
\end{enumerate}

\section{Άλγεβρα και θεωρήματα Ορίων}

\begin{enumerate}
    \item Να υπολογιστούν τα παρακάτω όρια με τη βοήθεια του Κριτηρίου 
        Παρεμβολής.

        \begin{enumerate}[i)]
            \item $ \lim_{n \to \infty} \sqrt{n+1} - \sqrt{n}  $ \hfill Απ:0
            \item $ \lim_{n \to \infty} \frac{\cos{n} + 3 \sin{4n}}{ 2
                \sqrt{n} -1} $ \hfill Απ: 0  
            \item $ \lim_{n \to \infty} \sqrt[n]{3^{n}+4^{n}+n} $ \hfill Απ:
                4 
        \end{enumerate}

    \item Να υπολογιστούν τα παρακάτω όρια με τη βοήθεια του ορίου 
        $ \lim_{n \to \infty} \left(1+ \frac{1}{n}\right)^{n}=e $

        \begin{enumerate}[i)]
            \item $ \lim_{n \to \infty} \left(1+ \frac{1}{n-2}\right)^{n}, 
                n \geq 2 $ 
                \hfill Απ: $e$  
            \item $ \lim_{n \to \infty} \left(1 + \frac{1}{3n}\right)^{n} $ 
                \hfill Απ: $ \sqrt[3]{e} $ 
            \item $ \lim_{n \to \infty} \left(\frac{2n +3}{2n} 
                \right)^{3n+2}  $
                \hfill Απ: $ e^{4}\cdot \sqrt{e} $ 
            \item $ \lim_{n \to \infty} \left(1+ \frac{2}{n}\right)^{n} $ 
                \hfill Απ: $ e^{2} $ 
            \item $ \lim_{n \to \infty}\left(1-\frac{1}{n^{2}} \right)^{n} $ 
                \hfill Απ: 1 
            \item $ \lim_{n \to \infty} \left(\frac{n^{2}-1}{n^{2}+1} 
                \right)^{n^{2}} $
                \hfill Απ: $ \frac{1}{e^{2}} $ 
        \end{enumerate}

    \item Να υπολογιστούν τα παρακάτω όρια.

        \begin{enumerate}[i)]
            \item $ \lim_{n \to \infty}  \frac{n^{2}}{2^{n}} $ \hfill Απ: 0 
            \item $ \lim_{n \to \infty} \frac{3^{n}}{n!} $ \hfill Απ: 0 
        \end{enumerate}


\end{enumerate}



\section{Θέματα}

Να υπολογιστούν τα παρακάτω όρια.

\begin{enumerate}[i)]
    \item $ \lim_{n \to \infty} \sqrt[n]{\frac{1}{2^{n}+ \frac{1}{3^{n}}} } $
        \hfill Απ: $ \frac{1}{2} $ 
    \item $ \lim_{n \to \infty} \frac{n^{3}}{3^{n}} $ \hfill Απ: $ 0 $
    \item $ \lim_{n \to \infty} \sqrt[n]{\frac{1}{3^{n}} + \frac{1}{4^{n}} 
        + \frac{1}{5^{n}}} $ \hfill Απ: $ \frac{1}{3} $
    \item $ \lim_{n \to \infty} \frac{n^{4}+5n-6}{2^{n}} $ \hfill Απ: 1 
\end{enumerate}

\end{document}
