\documentclass[main.tex]{subfiles}


\begin{document}


\section{Ορισμός Ακολουθίας}

\begin{dfn}
    \textcolor{Col\thechapter}{Ακολουθία} πραγματικών αριθμών ονομάζεται 
    κάθε συνάρτηση με πεδίο ορισμού τους φυσικούς αριθμούς. 
    \begin{align*}
        a \colon &\mathbb{N} \to \mathbb{R} \\
                 &n \to a(n)=a_{n}
    \end{align*} 
    Οι ακολουθίες συμβολίζονται ως $ (a_{n})_{n \in \mathbb{N}} $  
    ή $ \{ a_{n} \} _{n \in \mathbb{N}} $  ή $ \{ a_{n} \} _{n=1}^{+\infty}$
    , κλπ.
\end{dfn}

\begin{dfn}
    \textcolor{Col\thechapter}{Σύνολο Τιμών} (Σ.Τ.) της ακολουθίας 
    $ (a_{n})_{n \in \mathbb{N}} $, 
    ονομάζουμε το σύνολο των όρων της, δηλαδή το σύνολο 
    $ \{ a_{1}, a_{2}, \ldots, a_{n} \} $ το οποίο μπορεί να είναι 
    πεπερασμένο ή άπειρο.
\end{dfn}

\begin{examples}
\item {}
    \begin{enumerate}[i)]
        \item $ a_{n} = n, \; \forall n \in \mathbb{N} $. Έχει Σ.Τ. 
            το σύνολο $  \{ 1,2,3, \ldots \} $.
        \item $\left(\frac{1}{n}\right)_{n \in \mathbb{N}} $. Έχει Σ.Τ. το 
            σύνολο 
            $  \left\{ 1, \frac{1}{2}, \frac{1}{3}, \ldots \right\} $.
        \item $ \{(-1)^{n}\}_{n=1}^{+ \infty}, $. Έχει Σ.Τ. 
            το σύνολο $ \{ -1,1 \} $.
        \item $ a_{n} = c, \; \forall n \in \mathbb{N}, c \in \mathbb{R} $.
            Έχει Σ.Τ. το σύνολο $ \{ c \} $ και ονομάζεται
            \textcolor{Col\thechapter}{σταθερή ακολουθία}.
        \item $ a_{n}=2n, \; \forall n \in \mathbb{N} $. Έχει Σ.Τ. το 
            σύνολο $ \{ 2,4,6, \ldots, 2n, \ldots \} $. Πρόκειται για την 
            \textcolor{Col\thechapter}{ακολουθία των άρτιων} φυσικών 
            αριθμών.
        \item $ a_{n}= 2n-1, \; \forall n \in \mathbb{N} $. Έχει Σ.Τ. το 
            σύνολο $ \{ 1,3,5, \ldots, 2n-1, \ldots \} $. Πρόκεται για την 
            \textcolor{Col\thechapter}{ακολουθία των περιττών} φυσικών 
            αριθμών.
        \item \label{ex:anadr} $ a_{1}= a_{2} = 1 $ και $ a_{n+2}=a_{n+1}
            +a_{n}, \; \forall n \in \mathbb{N}$. Έχει Σ.Τ. το 
            σύνολο $ \{ 1,1,2,3,5,8, 13,21,34, \ldots\} $. 
            Πρόκειται για την \textcolor{Col\thechapter}{ακολουθία 
            Fibonacci}. 
    \end{enumerate}
\end{examples}

\begin{rem}
\item {}
    \begin{enumerate}[i)]
        \item Ουσιαστικά οι ακολουθίες είναι λίστες πραγματικών αριθμών.
        \item Η ακολουθία~\ref{ex:anadr}, όπου κάθε επόμενος όρος,
            ορίζεται με τη βοήθεια του προηγούμενου, λέγεται
            \textcolor{Col\thechapter}{αναδρομική ακολουθία}. 
            Προτάσεις που αφορούν αναδρομικές ακολουθίες, 
            αποδεικνύονται με Μαθηματική Επαγωγή.
    \end{enumerate}
\end{rem}

\begin{dfn}
    Δυο ακολουθίες, $(a_{n})_{n \in \mathbb{N}}$  και $ (b_{n})_{n \in 
    \mathbb{N}} $ ονομάζονται \textcolor{Col\thechapter}{ίσες}, αν $ a_{n} 
    = b_{n}, \; \forall n \in \mathbb{N} $.
\end{dfn}

\begin{dfn}
    Οι πράξεις μεταξύ ακολουθιών, ορίζονται όπως ακριβώς και για τις 
    συναρτήσεις.
\end{dfn}

\section{Φραγμένες Ακολουθίες}

\begin{dfn}
\item {}
    Μια ακολουθία $ (a_{n})_{n \in \mathbb{N}} $ ονομάζεται:
    \begin{enumerate}[i)]
        \item \textcolor{Col\thechapter}{άνω φραγμένη} 
            $ \overset{\text{ορ.}}{\Leftrightarrow} \exists M \in 
            \mathbb{R} \; : \; a_{n} \leq M, \; \forall n \in \mathbb{N}$.
        \item \textcolor{Col\thechapter}{κάτω φραγμένη} 
            $ \overset{\text{ορ.}}{\Leftrightarrow} \exists m \in 
            \mathbb{R} \; : \; m \leq a_{n}, \; \forall n \in \mathbb{N}  $
        \item \textcolor{Col\thechapter}{φραγμένη} 
            $ \overset{\text{ορ.}}{\Leftrightarrow} \exists$ 
            είναι άνω και κάτω φραγμένη.
    \end{enumerate}

    \begin{prop}
        $ (a_{n})_{n \in \mathbb{N}} $ φραγμένη $ \Leftrightarrow \exists 
        M>0, \; M \in \mathbb{R} \; : \; \abs{a_{n}} \leq M, \; \forall 
        n \in \mathbb{N} $
        (Απολύτως φραγμένη).
    \end{prop}
\end{dfn}

\begin{examples}
\item {}  
    \begin{enumerate}[i)]
        \item $ a_{n}= \frac{1}{n}, \; \forall n \in \mathbb{N} $, είναι 
            φραγμένη.

            Πράγματι, $ 0 \leq \frac{1}{n} \leq 1, \; \forall n \in 
            \mathbb{N} $. 

            Επίσης, $ \abs{\frac{1}{n}} = \frac{1}{n} \leq 1 $, άρα και 
            απολύτως φραγμένη.
        \item $ a_{n}=(-1)^{n} \frac{1}{n}, \; \forall n \in \mathbb{N} $ 
            είναι απολύτως φραγμένη. Πράγματι,

            \[
                \abs{a_{n}} = \abs{(-1)^{n} \frac{1}{n}} = \abs{(-1)^{n}} 
                \cdot \abs{\frac{1}{n}} = \abs{-1}^{n} \cdot \frac{1}{n}
                = 1 \cdot \frac{1}{n} = \frac{1}{n} \leq 1, \; \forall n 
                \in \mathbb{N}
            \] 

        \item $ a_{n}= \frac{(n-1)!}{n^{n}}, \; \forall n \in \mathbb{N} $
            είναι φραγμένη. Πράγματι, 
          
            \begin{itemize}
                \item $ a_{n} > 0, \; \forall n \in 
                    \mathbb{N}$, άρα 0 κ.φ. της $( a_{n})_{n \in 
                    \mathbb{N}} $. 
                \item Επίσης 
                    \[
                        a_{n}= \frac{(n-1)!}{n^{n}} = \frac{1 \cdot 2 
                            \cdots (n-1)}{n^{n}} < \frac{\overbrace{n 
                            \cdot n \cdots n} ^{n-1 \; 
                    \text{φορές}}}{n^{n}} = \frac{n^{n-1}}{n^{n}} =
                        \frac{1}{n} \leq 1, \; \forall n \in \mathbb{N},
                    \]
                    άρα το 1 είναι α.φ. της $(a_{n})_{n \in \mathbb{N}}$. 
            \end{itemize}

        \item $ a_{n}= 1 + \left(- \frac{1}{2} \right) + \left(- 
                \frac{1}{2}\right)^{2} + \cdots + \left(-\frac{1}{2} 
            \right) ^{n}, 
            \; \forall n \in \mathbb{N} $ είναι απολύτως φραγμένη. Πράγματι,

            Πρόκειται για το άθροισμα $ n $ όρων γεωμετρικής προοδου. Έτσι
            \[ a_{n} = 1 + \left(- \frac{1}{2}\right) + \left(- \frac{1}{2} 
                \right)^{2} + \cdots + \left(- \frac{1}{2} \right)^{n} = 
                \frac{1 - (- \frac{1}{2} )^{n}}{1 - (- \frac{1}{2})} = 
            \frac{2}{3} \left[1 - \left(- \frac{1}{2} \right)^{n}\right] \]
            Επομένως
            \[
                \abs{a_{n}} = \abs{\frac{2}{3} \left[1-(- \frac{1}{2} )^{n}
                        \right]} = \frac{2}{3} \abs{\abs{1} - \left(- 
                \frac{1}{2}\right)^{n}} \leq 
                \frac{2}{3} \left(1 + \abs{-\frac{1}{2} }^{n} \right) = 
                \frac{2}{3} \left(1+ \frac{1}{2^{n}}\right) < \frac{2}{3}
                (1+1) = \frac{4}{3} 
            \] 

        \item $ a_{n}= 2n+5, \; \forall n \in \mathbb{N} $ είναι κάτω 
            φραγμένη.
            Πράγματι, $ 7 \leq 2n+5, \; \forall n \in \mathbb{N} $, άρα το 
            7 είναι κ.φ. της $ (a_{n} )_{n \in \mathbb{N}} $.

        \item $ a_{1}=2, \; a_{n+1}=2 - \frac{1}{a_{n}}, \forall n \in 
            \mathbb{N}$
            είναι φραγμένη. Πράγματι, επειδή η ακολουθία είναι αναδρομική, 
            με επαγωγή, έχουμε:
            \begin{itemize}
                \item Για $ n=1 $, $ a_{1}=2>1 $, ισχύει. 
                \item Έστω ότι ισχύει για $n$, δηλ. \inlineequation[eq:
                    anadepag1]{a_{n}>1}.
                \item Θ.δ.ο. ισχύει και για $ n+1 $. Πράγματι, από τη σχέση~
                    \eqref{eq: anadepag1}, έχουμε
                    \[
                        a_{n}>1 \Rightarrow \frac{1}{a_{n}} 
                        < 1 \Rightarrow - \frac{1}{a_{n}} > 
                        -1 \Rightarrow 2 - \frac{1}{a_{n}} 
                        > 2-1 \Rightarrow a_{n+1} > 1.
                    \] 
            \end{itemize}
    \end{enumerate}
\end{examples}

\section{Μονοτονία Ακολουθιών}

\begin{dfn}
    Μια ακολουθία $ (a_{n})_{n \in \mathbb{N}} $ λέγεται:
    \begin{enumerate}[i)]
        \item \textcolor{Col\thechapter}{γνησίως αύξουσα} 
            $ \overset{\text{ορ.}}{\Leftrightarrow} a_{n} 
            < a_{n+1}, \; n \in \mathbb{N}$
        \item \textcolor{Col\thechapter}{γνησίως φθίνουσα} 
            $ \overset{\text{ορ.}}{\Leftrightarrow} a_{n} 
            > a_{n+1}, \; n \in \mathbb{N}$
        \item \textcolor{Col\thechapter}{αύξουσα} 
            $ \overset{\text{ορ.}}{\Leftrightarrow} a_{n} \leq 
            a_{n+1}, \forall n \in \mathbb{N}  $.
        \item \textcolor{Col\thechapter}{φθίνουσα} 
            $ \overset{\text{ορ.}}{\Leftrightarrow} a_{n} \geq 
            a_{n+1}, \forall n \in \mathbb{N}  $.
    \end{enumerate}
\end{dfn}

\begin{rems}
\item {}
    \begin{enumerate}[i)]
        \item $ (a_{n})_{n \in \mathbb{N}} $ γνησίως αύξουσα (φθίνουσα) $ 
            \Rightarrow (a_{n})_{n \in \mathbb{N}} $ αύξουσα (φθίνουσα) 
        \item $ (a_{n})_{n \in \mathbb{N}} $ γνησίως φθίνουσα  $ 
            \Rightarrow (a_{n})_{n \in \mathbb{N}} $ ανω φραγμένη, με 
            α.φ. το $ a_{1} $  
        \item $ (a_{n})_{n \in \mathbb{N}} $ γνησίως αύξουσα  $ 
            \Rightarrow (a_{n})_{n \in \mathbb{N}} $ κάτω φραγμένη, με 
            κ.φ. το $ a_{1} $  
    \end{enumerate}
\end{rems}

\begin{dfn}\label{dfn:isodmono}(Ισοδύναμος ορισμός της μονοτονίας)
\item {}
    Αν μια ακολουθία $ (a_{n})_{n \in \mathbb{N}} $ \textcolor{Col1}
    {διατηρεί πρόσημο}
    $ \forall n \in \mathbb{N} $, τότε λέγεται:
    \begin{enumerate}[i)]
        \item \textcolor{Col\thechapter}{γνησίως αύξουσα} 
            $ \overset{\text{ορ.}}{\Leftrightarrow} 
            \frac{a_{n+1}}{a_{n}} > 1, \; \forall n \in \mathbb{N}$
        \item \textcolor{Col\thechapter}{γνησίως φθίνουσα} 
            $ \overset{\text{ορ.}}{\Leftrightarrow} 
            \frac{a_{n+1}}{a_{n}} < 1, \; \forall n \in \mathbb{N}$
        \item  \textcolor{Col\thechapter}{αύξουσα} 
            $ \overset{\text{ορ.}}{\Leftrightarrow} 
            \frac{a_{n+1}}{a_{n}} \geq 1, \; \forall n \in \mathbb{N} $
        \item  \textcolor{Col\thechapter}{φθίνουσα}
            $ \overset{\text{ορ.}}{\Leftrightarrow} 
            \frac{a_{n+1}}{a_{n}} \leq 1, \; \forall n \in \mathbb{N} $
    \end{enumerate}
\end{dfn}

\section{Μεθοδολογία εύρεσης μονοτονίας μιας ακολουθίας}
\begin{itemize}
    \item Σχηματίζουμε τη διαφορά $ a_{n+1} - a_n $ και ελέγχουμε το 
        πρόσημό της. Αν $ a_{n+1}-a_{n}>0, \; (<0), \; \forall n \in 
        \mathbb{N} $ τότε $ (a_{n})_{n \in \mathbb{N}}$ γνησίως 
        αύξουσα (γνησίως φθίνουσα). Αν για τουλάχιστον ένα 
        $ n \in \mathbb{N} $, στις παραπάνω ανισότητες, έχω ισότητα, 
        τότε  $ (a_{n})_{n \in \mathbb{N}} $ είναι αύξουσα (φθίνουσα).
    \item Αν οι όροι της ακολουϑίας διατηρούν πρόσημο, $ \; \forall n \in
        \mathbb{N} $ τότε συγκρίνουμε το πηλίκο δυο διαδοχικών όρων της 
        ακολουθίας με τη μονάδα, και βγάζουμε τα συμπεράσματά μας 
        σύμφωνα με τον ορισμό~\ref{dfn:isodmono}
    \item Αν η ακολουθία δίνεται με μη-αναδρομικό τύπο, και είναι 
        (αρκετά) σύνθετη, τότε μετατρέπω την ακολουθία στην αντίστοιχη 
        συνάρτηση και μελετάμε τη μονοτονία της αντίστοιχης συνάρτησης, 
        συνήθως με τη βοήθεια της παραγώγου.
    \item Αν η $ (a_{n})_{n \in \mathbb{N}} $ δίνεται με αναδρομικό 
        τύπο $ (a_{n+1}= f(a_{n}), \; \forall n \in \mathbb{N}) $ τότε 
        συνήθως η απόδειξή της γίνεται με Μαθηματική Επαγωγή.
\end{itemize}

\begin{examples}
\item {}
    \begin{enumerate}[i)]
        \item Η $ a_{n} = 2n-1, \; \forall n \in \mathbb{N} $ είναι γνησίως 
            αύξουσα. Πράγματι,

            \twocolumnsides{
                \begin{description}
                    \item[Α᾽ τρόπος]
                        \begin{align*}
                            n+1 &\geq n, \; \forall n \in \mathbb{N} \\
                            2(n+1) &\geq 2n, \; \forall n \in \mathbb{N} \\
                            2(n+1) -1 &\geq 2n-1, \; \forall n \in 
                            \mathbb{N} \\
                            a_{n+1} &\geq a_{n}, \; \forall n \in \mathbb{N}
                        \end{align*}
                \end{description}

                }{
                \begin{description}
                    \item[Β᾽ τρόπος] 
                        \begin{align*}
                            a_{n+1}-a_{n} &= 2(n+1)-1 - (2n-1) \\
                                          &= 2 >0, \; \forall n \in 
                                          \mathbb{N} 
                        \end{align*}
                \end{description}

            }

        \item Η $ a_{n} = \frac{(n-1)!}{n^{n}}, \; \forall n \in 
            \mathbb{N} $ 
            είναι γνησίως φθίνουσα. Πράγματι, επειδή όλοι οι όροι της 
            ακολουθίας 
            είναι θετικοί, επομένως διατηρεί πρόσημο, έχουμε:
            \[
                \frac{a_{n+1}}{a_n} =
                \frac{\frac{(n+1-1)!}{(n+1)^{n+1}}}{\frac{(n-1)!}{n^{n}}} =  
                \frac{n^{n}\cdot n!}{(n+1)^{n+1}\cdot (n-1)!} =
                \frac{n^{n+1}}{(n+1)^{n+1}} = \left(\frac{n}{n+1} 
                \right)^{n+1} < 1, \; \forall n
                \in \mathbb{N} 
            \] 

        \item Η $ a_{n}= \frac{4^{n}}{n^{2}}, \; \forall n \in \mathbb{N} $ 
            είναι αύξουσα. Πράγματι, επειδή όλοι οι όροι της ακολουθίας 
            είναι θετικοί, επομένως διατηρεί πρόσημο, έχουμε:
            \[
                \frac{a_{n+1}}{a_{n}} 
                = \frac{\frac{4^{n+1}}{(n+1)^{2}}}{\frac{4^{n}}{n^{2}}} 
                = \frac{4^{n+1}\cdot n^{2}}{4^{n}\cdot (n+1)^{2}} 
                = \frac{4n^{2}}{(n+1)^{2}}
                = \left( \frac{2n}{n+1} \right)^{2} \geq 1, 
                \; \forall n \in \mathbb{N} 
            \]
            Η ακολουθία δεν είναι γνησίως αύξουσα, γιατί $ 
            a_{1}= a_{2}=4$.

        \item Η $ a_{n+1}=2 - \frac{1}{a_{n}}, \; \forall n \in \mathbb{N}
            $ με $ a_{1} = 2 $ είναι γνησίως φθίνουσα. Πράγματι, 
            \begin{itemize}
                \item Για $ n=1 $, έχω: $ a_{1}= 2 >
                    \frac{3}{2} = 2 - \frac{1}{2} = a_{2}$, ισχύει.
                \item Έστω ότι ισχύει για $n$, δηλ.
                    \inlineequation[eq:epag]{a_{n+1}<a_{n}}.
                \item Θ.δ.ο. ισχύει για $ n+1 $. Πράγματι, 
                    \[
                        a_{(n+1)+1}=2- \frac{1}{a_{n+1}}
                        \overset{\eqref{eq:epag}}{<} 2 - 
                        \frac{1}{a_{n}} = a_{n+1}
                    \] 
            \end{itemize}
    \end{enumerate}
\end{examples}

\section{Σύγκλιση Ακολουθιας}

\begin{dfn}
    \textcolor{Col\thechapter}{Περιοχή} ένος πραγματικού αριθμού $ x_{0} $ 
    ονομάζεται κάθε ανοιχτό διάστημα $(a,b)$ που περιέχει το $ x_{0} $. 
\end{dfn}

\begin{rem}
\item {}
    \begin{enumerate}[i)]
        \item 
            Αν $ \varepsilon > 0 $, τότε περιοχές του $ x_{0} $ της μορφής 
            $ (x_{0}- \varepsilon , x_{0} + \varepsilon) $ έχουν 
            \textcolor{Col\thechapter}{ακτίνα} $ \varepsilon $
            και \textcolor{Col\thechapter}{κέντρο} το $ x_{0} $. 

        \item $ x \in (x_{0}- \varepsilon, x_{0} + \varepsilon) 
            \Leftrightarrow x_{0}- \varepsilon < x < x_{0}+ \varepsilon 
            \Leftrightarrow - \varepsilon < x - x_{0} < \varepsilon 
            \Leftrightarrow \abs{x- x_{0}} < \varepsilon  $ 
    \end{enumerate}
\end{rem}

\begin{dfn}
    Μια ακολουθία $ (a_{n})_{n \in \mathbb{N}} $ \textcolor{Col\thechapter}
    {συγκλίνει} στον πραγματικό 
    αριθμό $ a \in \mathbb{R} $ (έχει όριο το $ a \in \mathbb{R} $ ή 
    τείνει στο $ a \in \mathbb{R} $), και συμβολίζουμε 
    $ \lim\limits_{n\to \infty} a_{n}=a $ (ή $ a_{n} \xrightarrow{n \to 
    \infty} a $) αν 
    \[
        \forall \varepsilon >0, \; \exists n_{0} \in \mathbb{N} \; : 
        \; \forall n \in \mathbb{N} \; \text{με} \; n \geq n_{0} 
        \Rightarrow \abs{a_{n}-a} < \varepsilon
    \] 
\end{dfn}

\begin{rem}
    Γενικά το $ n_{0} $ εξαρτάται από το $ \varepsilon $ και ισχύει ότι
    $ n_{0} = n_{0}(\varepsilon) $.
\end{rem}

\begin{dfn}
    Η ακολουθία $ (a_{n})_{n \in \mathbb{N}}$ λέγεται
    \textcolor{Col\thechapter}{μηδενική ακολουθία}
    αν $ \lim\limits_{n\to \infty} = 0 $
\end{dfn}

\begin{examples}
\item {}
    \begin{enumerate}[i)]
        \item $ \lim\limits_{n \to \infty} \frac{1}{n} = 0 $.
            \begin{proof}
            \item {}
                \begin{description}
                    \item[Δοκιμή:] $ \abs{\frac{1}{n} -0} < \varepsilon
                        \Leftrightarrow \abs{\frac{1}{n}} < \varepsilon 
                        \Leftrightarrow \frac{1}{n} < \varepsilon 
                        \Leftrightarrow n > \frac{1}{\varepsilon}$
                \end{description}
                \begin{description}
                    \item[Α᾽ Τρόπος:] 
                        Έστω $ \varepsilon >0 $. Τότε $ \exists n_{0} \in
                        \mathbb{N} $ με \inlineequation[eq:1n1]{n_{0} > 
                        \frac{1}{\varepsilon}} (Αρχ. Ιδιοτ.) τέτοιο 
                        ώστε \inlineequation[eq:1n2]{\forall n \geq n_{0}}
                        \[
                            \abs{\frac{1}{n} -0} = \abs{\frac{1}{n}} =
                            \frac{1}{n} \overset{\eqref{eq:1n2}}{\leq}
                            \frac{1}{n_{0}} \overset{\eqref{eq:1n1}}{<} 
                            \frac{1}{\frac{1}{\varepsilon}} = \varepsilon 
                        \]

                    \item [Β᾽ Τρόπος:]
                        Έστω $ \varepsilon >0 $. Τότε $ \exists n_{0} \in
                        \mathbb{N} $ με \inlineequation[eq:1n3]
                        {\frac{1}{n_{0}} < \varepsilon} (Αρχ. Ιδιοτ.) 
                        τέτοιο ώστε \inlineequation[eq:1n4]{\forall n 
                        \geq n_{0}}
                        \[
                            \abs{\frac{1}{n} -0} = \abs{\frac{1}{n}} =
                            \frac{1}{n} \overset{\eqref{eq:1n4}}{\leq}
                            \frac{1}{n_{0}} \overset{\eqref{eq:1n3}}{<} 
                            \varepsilon 
                        \]
                \end{description}
            \end{proof}

        \item $ \lim\limits_{n \to \infty} \frac{1}{n^{4}} = 0 $. 
            \begin{proof}
            \item {}
                \begin{description}
                    \item[Δοκιμή:]$ \abs{\frac{1}{n^{4}} - 0} < \varepsilon 
                        \Leftrightarrow \abs{\frac{1}{n^{4}}} < \varepsilon 
                        \Leftrightarrow \frac{1}{n^{4}} < \varepsilon
                        \Leftrightarrow n^{4} > \frac{1}{\varepsilon}
                        \Leftrightarrow n > \sqrt[4]{\frac{1}{\varepsilon}}$
                \end{description}
                Έστω $ \varepsilon >0 $. Τότε $ \exists n_{0}  \in 
                \mathbb{N}$ με \inlineequation[eq:limexn41]{n_{0} >
                \sqrt[4]{\frac{1}{\varepsilon}}} (Αρχ. Ιδιοτ.)
                τέτοιο ώστε \inlineequation[eq:limexn42]{\forall n 
                \geq n_{0}} 
                \[
                    \abs{\frac{1}{n^{4}} - 0 } = \abs{\frac{1}{n^{4}}} 
                    = \frac{1}{n^{4}} \overset{\eqref{eq:limexn42}}\leq 
                    \frac{1}{n_{0}^{4}} \overset{\eqref{eq:limexn41}}{<}
                    \frac{1}{\left(\sqrt[4]{\frac{1}{\varepsilon}
                    }\right)^{4}} = \frac{1}{\frac{1}{\varepsilon}} =  
                    \varepsilon
                \] 
            \end{proof}

        \item $ \lim\limits_{n \to \infty} \frac{1}{\sqrt{n}} = 0$.
            \begin{proof}
            \item {}
                \begin{description}
                    \item[Δοκιμή:] $ \abs{\frac{1}{\sqrt{n}}-0 } 
                        < \varepsilon 
                        \Leftrightarrow \abs{\frac{1}{\sqrt{n}} } < 
                        \varepsilon 
                        \Leftrightarrow \frac{1}{\sqrt{n}} < 
                        \varepsilon \Leftrightarrow \sqrt{n} >
                        \frac{1}{\varepsilon} \Leftrightarrow n >
                        \left(\frac{1}{\varepsilon}\right)^{2}
                        $
                \end{description}
                Έστω $ \varepsilon > 0 $. Τότε $ \exists n_{0} \in 
                \mathbb{N} $
                με \inlineequation[eq:limexsqrt1]{n_{0} > \left(\frac{1}{
                \varepsilon}\right)^{2}} (Αρχ. Ιδιοτ.) τέτοιο ώστε 
                \inlineequation[eq:limexsqrt2]{\forall n \geq n_{0}}
\[
    \abs{\frac{1}{\sqrt{n}} -0} = \abs{\frac{1}{\sqrt{n}}} =
    \frac{1}{\sqrt{n}} \overset{\eqref{eq:limexsqrt2}}{\leq}
    \frac{1}{\sqrt{n_{0}}} \overset{\eqref{eq:limexsqrt1}}{<}
    \frac{1}{\sqrt{\left(\frac{1}{\varepsilon}\right)^{2}}} =
    \frac{1}{\frac{1}{\varepsilon}} = \varepsilon
 \] 
            \end{proof}

        \item $ \lim_{n \to \infty} \frac{\sin{n}}{n} = 0 $

            \begin{proof}
            \item {}
                \begin{description}
                    \item[Δοκιμή:] $ \abs{\frac{\sin{n}}{n} - 0} < 
                        \varepsilon \Leftrightarrow \abs{\frac{\sin{n}}{n}}
                        < \varepsilon \Leftrightarrow 
                        \inlineequation[eq:limexsin1]{\frac{\abs{\sin{n}}}
                        {n} < \varepsilon} $

                        Όμως
                        \inlineequation[eq:limexsin2]{\frac{\abs{\sin{n}}}
                            {n} \leq \frac{1}{n}, \; \forall n \in 
                        \mathbb{N}}

                        Οπότε από τις σχέσεις \eqref{eq:limexsin1} και 
                        \eqref{eq:limexsin2} αρκεί $ \frac{1}{n} < 
                        \varepsilon \Leftrightarrow n > \frac{1}{
                        \varepsilon} $
                \end{description}

                Έστω $ \varepsilon >0 $. Τότε $ \exists n_{0} \in \mathbb{N}
                $ με \inlineequation[eq:limexsin3]{n_{0} >
                \frac{1}{\varepsilon}} τέτοιο ώστε
                \inlineequation[eq:limexsin4]{\forall n \geq n_{0}}
                \[
                    \abs{\frac{\sin{n}}{n} - 0} =  \abs{\frac{\sin{n}}{n}} =
                    \frac{\abs{\sin{n}}}{n} \leq \frac{1}{n}
                    \overset{\eqref{eq:limexsin4}}{\leq}  \frac{1}{n_{0}}
                    \overset{\eqref{eq:limexsin3}}{<}
                    \frac{1}{\frac{1}{\varepsilon}
                } = \varepsilon 
                 \] 
            \end{proof}
    \end{enumerate}
\end{examples}

\begin{prop}
    Η ακολουθία $ \{ (-1)^{n} \}_{n \in \mathbb{N}} $ δεν συγκλίνει.
\end{prop}

\begin{proof}
\item {}
    Έστω ότι η $ a_{n}= (-1)^{n} $ συγκλίνει και έστω $ \lim_{n \to +
    \infty}(-1)^{n} = a $. 
    
    Τότε από τον ορισμό του ορίου, έχουμε ότι για 
    \[ 
        \varepsilon = 1, \; \exists n_{0} \in \mathbb{N} \; : \; \forall 
    n \geq n_{0} \quad \abs{(-1)^{n}-a} < 1 \Leftrightarrow -1 < (-1)^{n}
    -a < 1 \Leftrightarrow a-1 < (-1)^{n} < a+1
\]

Όμως για $ n_{1} \geq n_{0} $ με $ n_{1} $ άρτιος, έχουμε:

\[
    a-1 <  (-1)^{n_{1}} < a+1 \Leftrightarrow a-1 < 
    \underbrace{1 < a+1}_{a>0} 
 \] 

Όμως για $ n_{2} \geq n_{0} $ με $ n_{2} $ περιττός, έχουμε:

\[
    a-1 <  (-1)^{n_{2}} < a+1 \Leftrightarrow 
    \underbrace{a-1 < -1}_{a<0} < a+1
 \] 

 Οπότε καταλήγουμε σε άτοπο.
\end{proof}


\begin{prop}
    Έστω $ (a_{n})_{n \in \mathbb{N}} $ ακολουθία πραγματικών αριθμών και 
    $ l \in \mathbb{R} $. Τότε ισχύου οι ισοδυναμίες:

    \[ \lim_{n \to \infty} a_{n} = l \Leftrightarrow \lim_{n \to \infty} (
        a_{n}- l) = 0 
    \Leftrightarrow \lim_{n \to \infty} \abs{a_{n}-l} = 0\]
\end{prop}

\begin{proof}
\item {}
    \begin{align*} \lim_{n \to \infty} a_{n}= l & \Leftrightarrow \forall 
        \varepsilon > 0 \; \exists n_{0} \in \mathbb{N} \; : \; 
        \abs{a_{n}-l} < \varepsilon, \forall n \geq n_{0} \\ 
    \lim_{n \to \infty} (a_{n}-l) = 0 & \Leftrightarrow \forall 
    \varepsilon >0 \; \exists n_{0} \in \mathbb{N} \; : \; 
    \abs {(a_{n}-l) - 0} < \varepsilon , \forall n \geq n_{0} \\
    \lim_{n \to \infty} \abs{a_{n}-l}=0 < \varepsilon & \Leftrightarrow 
    \forall \varepsilon > 0 \; \exists n_{0} \in \mathbb{N} 
    \; : \; \abs {\abs{a_{n}- l } - 0} < \varepsilon, \forall n \geq n_{0} 
\end{align*}

και ισχύει ότι $ \abs{a_{n}-l} = \abs{(a_{n}-l)-0} = 
\abs{\abs{a_{n}- l} -0} $
\end{proof}

\begin{thm}[Μοναδικότητα του Ορίου]
    Το όριο μιας ακολουθίας, όταν υπάρχει είναι μοναδικό.
\end{thm}

\begin{proof}
    Έστω ότι μια ακολουθία $ (a_{n})_{n \in \mathbb{N}} $ συγκλίνει σε δυο 
    διαφορετικούς αριθμούς, $ a,b $. Τότε

    \[ 
        \lim_{n \to \infty} a_{n} = a \Leftrightarrow \forall \varepsilon >
        0, \; \exists n_{0} \in \mathbb{N} \; : \: \forall n \geq n_{0} 
        \quad \abs{a_{n}- a} \leq \varepsilon 
    \]

    Άρα και για $ \frac{\varepsilon}{2} \; \exists n_{1} \in \mathbb{N} \; 
    : \; \forall n \geq n_{1} \quad \abs{a_{n_{1}} - a} < 
    \frac{\varepsilon}{2}  $

    \[ 
        \lim_{n \to \infty} a_{n} = b \Leftrightarrow \forall \varepsilon 
        > 0, \; \exists n_{0} \in \mathbb{N} \; : \: \forall n \geq n_{0} 
        \quad \abs{a_{n}- b} \leq \varepsilon 
    \]

    Άρα και για $ \frac{\varepsilon}{2} \; \exists n_{2} \in \mathbb{N} \; 
    : \; \forall n \geq n_{2}  \quad \abs{a_{n_{2}} - b} < 
    \frac{\varepsilon}{2}  $

    Θέτουμε $ n_{0} = \max \{ n_{1}, n_{2} \} $, ώστε να ισχύουν οι κ οι 
    δύο παραπάνω ανισότητες ταυτόχρονα. 

    Επομένως, έχουμε 
    \[
        0 \leq \abs{a-b} = \abs{a- a_{n}+ a_{n}- b} \leq \abs{a- a_{n}} + 
        \abs{a_{n}- b} < \frac{\varepsilon}{2} + \frac{\varepsilon}{2} 
        = \varepsilon 
     \] 

     άρα από την πρόταση~\ref{prop:epsilon_prot} έχουμε ότι $ \abs{a-b} = 0 \Rightarrow a=b $.
\end{proof}

%TODO idiothtes oriwn

\begin{thm}
    Κάθε συγκλίνουσα ακολουθία είναι φραγμένη.
\end{thm}

\begin{proof}
    Έστω $ (a_{n})_{n \in \mathbb{Ν}} $ συγκλίνουσα $ \Rightarrow \exists 
    a \in \mathbb{R} $ τέτοιο ώστε $ \lim_{n \to +\infty} a_{n}=a $, άρα 
\[
    \forall \varepsilon > 0, \; \exists n_{0} \in \mathbb{N} \; : \; 
    \forall n \geq n_{0}\quad \abs{a_{n}-a} < \varepsilon  
 \] 
 αρα και για $ \varepsilon =1>0, \; \exists n_{0} \in \mathbb{N} \; : \; 
 \forall n \geq n_{0} \quad \inlineequation[eq:sygkfrag]{\abs{a_{n}-a} 
 < 1} $. 
 
 Δηλαδή $ \forall n \geq n_{0} $ έχουμε ότι 
\[
    \abs{a_{n}} = \abs{a_{n}-a + a} \leq \abs{a_{n} - a} + \abs{a} 
    \overset{\eqref{eq:sygkfrag}}{<} 1 + \abs{a}  
 \] 

 \begin{description}
     \item [Β᾽ τρόπος:] \[ \abs{\abs{a_{n}} - \abs{a}} \leq \abs{a_{n}- a} 
             < 1 \Leftrightarrow -1 < \abs{a_{n}} - \abs{a} < 1 
         \Leftrightarrow \abs{a} -1 < \abs{a_{n}} < \abs{a} + 1 \]
 \end{description}

 Δηλαδή, καταφέραμε να φράξουμε τους όρους της ακολουθίας με δείκτες 
 $n \geq n_{0} $.

 Τώρα, θέτουμε $ M = \max \{ \abs{a_{1}} , \abs{a_{2}} , \ldots, 
     \abs{a_{n-1}} , 1 + \abs{a}\} $ και έχουμε ότι $ \abs{a_{n}} < M, \; 
     \forall n \in \mathbb{N} $, επομένως η $ (a_{n})_{n \in \mathbb{N}} $ 
     είναι φραγμένη.
\end{proof}

\begin{rem}
    Προσοχή, το αντίστροφο της παραπάνω πρότασης, δεν ισχύει. Πράγματι, 
    για την  ακολουθία $ (a_{n})_{n \in \mathbb{N}} = (-1)^{n}, \; 
    \forall n \in \mathbb{N} $ έχουμε ότι είναι φραγμένη, όχι όμως και 
    συγκλίνουσα.
\end{rem}

\section{Υπακολουθίες}

\begin{dfn}
    Έστω $ (a_{n})_{n \in \mathbb{N}} $ ακολουθία, και έστω $ (k_{n})_{n \in \mathbb{N}} $ γνησίως
    αύξουσα ακολουθία φυσικών αριθμών $ (k_{1}<k_{2}<k_{3}<\cdots) $. Τότε η ακολουθία 
    $ b_{n} = (a_{k_{n}}), \; n \in \mathbb{N} $ ονομάζεται
    \textcolor{Col\thechapter}{υπακολουθία} της 
    $ (a_{n})_{n \in \mathbb{N}} $.
\end{dfn}

%TODO παραδειγματα υπακολουθιων

\begin{prop}
    Αν μια ακολουθία είναι φραγμένη τότε και κάθε υπακολουθία της είναι φραγμένη.
\end{prop}

%TODO υπολοιπες προτασεις εδω

\begin{lem}\label{lem:kn}
    Έστω $ (a_{n})_{n \in \mathbb{N}} $ ακολουθία και $ (a_{k_{n}})_
    {n \in \mathbb{N}} $ υπακολουθία της. Τότε $ k_{n} \geq n, \; \forall n \in 
    \mathbb{N} $.
\end{lem}

\begin{proof}
\item {}
    \begin{itemize}
        \item Για $ n=1 $, έχω: $ k_{n} \in \mathbb{N}, \; \forall n \in \mathbb{N} \Rightarrow 
            k_{n} \geq 1 $, ισχύει:
        \item Έστω ότι ισχύει για $ n $, δηλ. $ k_{n} \geq n $. 
        \item θ.δ.ο. ισχύει για $ n+1 $. Πράγματι, $ k_{n+1} > k_{n} \geq n \Rightarrow k_{n+1} > n 
            \Rightarrow k_{n+1} \geq n+1$.
    \end{itemize}
    \end{proof}

    \begin{prop}
        Αν μια ακολουθία $ (a_{n})_{n \in \mathbb{N}} $ συγκλίνει στο $ a \in \mathbb{R} $, τότε 
        και κάθε υπακολουθία της συγκλίνει επίσης στο $ a \in \mathbb{R} $.
    \end{prop}

    \begin{proof}
        Έστω $ (a_{k_{n}})_{n \in \mathbb{N}} $ υπακολουθία της $ (a_{n})_{n \in \mathbb{N}} $. 

        Έστω $ \varepsilon >0 $. 
        
        $ \lim_{a_{n}} = a \Leftrightarrow \forall \varepsilon >0, \; \exists n_{0} \in \mathbb{N} 
        \; : \; \forall n \geq n_{0} \quad \inlineequation[eq:ypaksygk]{\abs{a_{n}- a} < 
        \varepsilon} $.

        Άρα και για το $ \varepsilon $ που έχουμε επιλέξει, έχουμε ότι $ 
        \exists n_{0} \in \mathbb{N} \; : \; k_{n} \overset{\text{Λημ.} \; \ref{lem:kn}}{\geq} n 
        \geq n_{0}  $ 
        και άρα από τη σχέση~\eqref{eq:ypaksygk}, έχουμε 
        $\abs{a_{k_{n}} - a} < \varepsilon  $, δηλαδή $ \lim_{n \to +\infty} a_{k_{n}} = a$.
\end{proof}

\section{Ιδιότητες Ορίου Ακολουθίας}



\begin{prop}

    Έστω $ \lim_{n \to +\infty} a_{n} = a$ και $ \lim_{n \to +\infty} b_{n} = b $, όπου $ a,b \in
    \mathbb{R} $. Τότε:
    \begin{enumerate}[i)]
        \item $ \lim_{n \to +\infty} (a_{n} + b_{n}) = a+b = \lim_{n \to +\infty} a_{n} + 
            \lim_{n \to +\infty} b_{n} $
        \item $ \lim_{k a_{n}} = ka = k \lim_{n \to +\infty} a_{n}, \; k \in \mathbb{R} $
        \item $ \lim_{n \to +\infty} (a_{n}\cdot b_{n}) = a\cdot b = \lim_{n \to +\infty} a_{n}
            \cdot \lim_{n \to +\infty} b_{n}$
        \item $ \lim_{n \to +\infty} \frac{1}{a_{n}} = \frac{1}{a} = \frac{1}{\lim_{n \to +\infty}
            a_{n}}, \; a \neq 0, \; a_{n} \neq 0, \; \forall n \in \mathbb{N}  $
        \item $ \lim_{n \to +\infty} \frac{a_{n}}{b_{n}} = \frac{a}{b} = \frac{\lim_{n \to +\infty}
            a_{n}}{\lim_{n \to +\infty} b_{n}}, \; b \neq 0, \; n \neq 0, \forall n \in
            \mathbb{N} $
        \item $ \lim_{n \to +\infty} a_{n}^{k} = a^{k}, 
            \; k \geq 2, \; k \in \mathbb{N}  $  
        \item $ \lim_{n \to +\infty} a_{n}^{\frac{1}{k}}=a^{\frac{1}{k}}, \; \forall k \in 
            \mathbb{N}, a_{n} \geq 0, \; \forall n \in \mathbb{N} $
    \end{enumerate}
\end{prop}

\begin{proof}
\item {}
    \begin{enumerate}[i)]
        \item $ \lim_{n \to +\infty} a_{n} = a \Leftrightarrow \forall \varepsilon > 0, \; 
            \exists n_{1} \in \mathbb{N} \; : \; \forall n \geq n_{1} \quad \abs{a_{n}-a} <
            \varepsilon$.

            Αρα και για $ \varepsilon = \frac{\varepsilon}{ 2}, \; \exists n_{1} \in \mathbb{N} 
            \; : \; \forall n \geq n_{1} \quad \inlineequation[eq:idiot1]{\abs{a_{n}-a} <
            \frac{\varepsilon}{ 2}} $

            $ \lim_{n \to +\infty} b_{n} = b \Leftrightarrow \forall \varepsilon > 0, \; 
            \exists n_{2} \in \mathbb{N} \; : \; \forall n \geq n_{2} \quad \abs{b_{n}-b} <
            \varepsilon$.

            Αρα και για $ \varepsilon = \frac{\varepsilon}{ 2}, \; \exists n_{2} \in \mathbb{N} 
            \; : \; \forall n \geq n_{2} \quad \inlineequation[eq:idiot2]{\abs{b_{n}-b} <
            \frac{\varepsilon}{ 2}} $

            Θέτουμε $ n_{0}= \max \{ n1,n2 \} $.

            Έστω $ \varepsilon >0 $. Τότε για $ n \geq n_{0} $, έχουμε 
            \[
                \abs{(a_{n}+ b_{n}) - (a+b)} = \abs{(a_{n}- a) + (b_{n}-b)} \leq \abs{a_{n}- a} + 
                \abs{b_{n}-b} \overset{\eqref{eq:idiot1}}{\underset{\eqref{eq:idiot2}}{<}} 
                \frac{\varepsilon }{2} + \frac{\varepsilon}{ 2} = \varepsilon
             \] 

         \item 
            Αν $ k=0 $, τότε η σχέση είναι προφανής.

            Έστω $ k \neq 0 $. 

            $ \lim_{n \to +\infty} a_{n} = a \Leftrightarrow \forall \varepsilon >0, \; \exists 
            n_{0} \in \mathbb{N} \; : \; \forall n \geq n_{0} \quad \abs{a_{n}- a} < \varepsilon$. 

            Άρα και για $ \varepsilon = \frac{\varepsilon}{\abs{k}} >0 $, έχουμε ότι $ \exists 
            n_{0} \in \mathbb{N} \; : \; \forall n \geq n_{0} \quad \inlineequation[eq:idiot3]
            {\abs{a_{n}- a} < \frac{\varepsilon}{\abs{k}}}$. 

            Έστω $ \varepsilon >0 $. Τότε $ \forall n \geq n_{0} $, έχουμε ότι 
            \[
                \abs{k a_{n}- ka} = \abs{k(a_{n}- a)} = \abs{k} \cdot \abs{a_{n}- a}
                \overset{\eqref{eq:idiot3}}{<} \abs{k}
                \cdot \frac{\varepsilon}{\abs{k}} = \varepsilon 
             \] 

         \item Έχουμε ότι $ (a_{n})_{n \in \mathbb{N}} $ συγκλίνουσα $ \Rightarrow (a_{n})_{n \in
             \mathbb{N}} $ είναι φραγμένη. Άρα $ \exists M >0, \; M \in \mathbb{R} \; : \; 
             \abs{a_{n} \leq M, \; \forall n \in \mathbb{N}}$. Τότε 

             \begin{align*}
                 \abs{a_{n} b_{n} - ab} = \abs{a_{n}b_{n} - a_{n}b + a_{n}b -ab} = \abs{
                 a_{n}(b_{n}-b)+b(a_{n}-a) } &\leq \abs{a_{n}}\cdot \abs{b_{n}-b} + \abs{b} \cdot 
                 \abs{a_{n}-a} \\
                                             &\leq M \cdot  \abs{b_{n}-b} + \abs {b}\cdot 
                                             \abs{a_{n}-a}
             \end{align*}

             Απο τη σύγκλιση των $ a_{n} $ και $ b_{n} $, έχουμε:

             Θέτουμε $ \varepsilon_{1} = \frac{\varepsilon}{2M} >0 $ και $ \varepsilon_{2} =
             \frac{\varepsilon}{2\abs{b}} >0 $ και επιλέγοντας $ n_{0}= \max \{ n_{1}, n_{2}\} $.

             Έστω $ \varepsilon >0 $, τότε $ \forall n \geq n_{0}$, έχουμε ότι 
\[
    \abs{a_{n} b_{n} - ab} \leq M \cdot \abs{bn-b} + \abs{b} \cdot \abs{a_{n}- a} < 
    M \cdot \frac{\varepsilon}{2 M} + \abs {b} \cdot \frac{\varepsilon}{2 \abs{b}} =
    \frac{\varepsilon}{2} + \frac{\varepsilon}{2} = \varepsilon
 \]

 \begin{description}
     \item [Β᾽ Τρόπος:]
     \item {} 
         $ \lim_{n \to +\infty} a_{n}= 0, \; \lim_{n \to +\infty} b_{n} = 0
     \Rightarrow (a_{n})_{n \in \mathbb{N}}, \; (b_{n})_{n \in \mathbb{N}}  $ μηδενικές ακολουθίες, 
     άρα είναι φραγμένες. 

     Συνπώς (από μηδενική επι φραγμένη) έχουμε ότι \[ \lim_{n \to +\infty} [(a_{n}- a) 
         \cdot (b_{n} -b)] = 0 \Rightarrow \lim_{n \to +\infty} [a_{n} b_{n} - a_{n}b - b_{n} a +
     ab] = 0\] 

     Έχουμε ότι 
$
\left.
     \begin{aligned}
         \lim_{n \to +\infty} a_{n}b = ab \\
         \lim_{n \to +\infty} b_{n}a = ab
      \end{aligned} 
  \right\} $
  
  $ \overset{(+)}{\Rightarrow} \lim_{n \to +\infty} [a_{n} b_{n} + ab] = 2ab \Rightarrow 
  \lim_{n \to +\infty} (a_{n}b_{n}) = ab $
 \end{description}

 \item \begin{lem}
         $ b \neq 0 \Rightarrow \exists n_{0} \in \mathbb{N} \; : \; b_{n} \neq 0, \; n \geq n_{0} $ 
     \end{lem}

    \begin{proof}
    \item {}
        Έστω $ b \neq 0 \Rightarrow \frac{\abs{b}}{2} >0 $. 
        
        Για $ \varepsilon = \frac{\abs{b}}{2} >0 $ και λόγω ότι $ \lim_{n \to +\infty} b_{n} = b $
        έχουμε ότι $ \exists n_{1} \in \mathbb{N} \; : \; \forall n \geq n_{1} \quad \abs{b_{n}-b} 
        < \frac{\abs{b}}{2} $

        Άρα για $ n \geq n_{1} $, έχουμε 
        \[
            \abs{b} = \abs{b - b_{n} + b_{n}} \leq \abs{b - b_{n}} + \abs{b_{N}} \leq
            \frac{\abs{b}}{2} + \abs{b_{n}} \Leftrightarrow \abs{b_{n}} > \abs{b} -
            \frac{\abs{b}}{2} \Leftrightarrow \abs{b_{n}}  > \frac{\abs{b}}{2} \Rightarrow b_{n} 
            \neq 0, \; \forall n \geq n_{1}
        \] 

        \begin{description}
            \item [Β᾽ Τρόπος:] 
            \item {}
                Γενικά ισχύει ότι αν $ x >0 $ και $ \abs{y-x} < \frac{x}{2} $, 
                τότε \inlineequation[eq:lemidiot]{\abs{y} < \frac{x}{2}}.
                
                $ \lim_{n \to +\infty} b_{n} =b \Rightarrow \exists n_{0} \in \mathbb{N} \; : \; 
                \forall n \geq n_{0} \quad \abs{b_{n}-b} < \frac{\abs{b}}{2} = \varepsilon $

                Οπότε $ \forall n \geq n_{0} \; : \; \abs{\abs{b_{n}} - \abs{b}} \leq \abs{b_{n}-b}
                < \frac{\abs{b}}{2} \Rightarrow \abs{b_{n}} \overset{\eqref{eq:lemidiot}}{<} 
                \frac{\abs{b}}{2}   $ 
        \end{description}

        Τώρα αν $ n \geq n_{1} $ ισχύει ότι:

        $ \abs{\frac{1}{b_{n}} - \frac{1}{b}} = \frac{b-b_{n}}{b_{n}\cdot b} = \frac{\abs{b-b_{n}}
        }{\abs{b_{n}} \cdot \abs{b}} < \frac{2 \abs{b - b_{n}}}{\abs{b}^{2}} $

        Έστω $ \varepsilon >0 $, τότε
\[
    \lim_{n \to +\infty} b_{n} =b \Leftrightarrow \exists n_{2} \in \mathbb{N} \; : \; \forall n 
    \geq n_{2} \quad \abs{b - b_{n}} < \frac{\varepsilon \abs{b}^{2}}{2}
 \] 

 Επιλέγουμε $ n_{0} = \max \{ n_{1}, n_{2} \} $. Τότε $ \forall n \geq n_{0} $ ισχύει ότι 
 \[
     \abs{\frac{1}{b_{n}} - \frac{1}{b}} \leq \frac{2 \abs{b -b_{n}}}{\abs{b} ^{2}} < \varepsilon 
  \] 
    \end{proof}

\item $ \lim_{n \to +\infty} \frac{a_{n}}{b_{n}} = \lim_{n \to +\infty} 
    \left(a_{n}\cdot \frac{1}{b_{n}}
    \right) = \lim_{a_{n}} \cdot \lim_{n \to +\infty} \frac{1}{b_{n}} = a \cdot \frac{1}{b} =
    \frac{a}{b}$

\item 
Θα αποδείξουμε την πρόταση με Μαθηματική Επαγωγή.
    \begin{itemize}
        \item Για $ k=2 $, έχω: $ \lim_{n \to +\infty} a_{n}^{2} = \lim_{n \to
            +\infty} (a_{n} \cdot a_{n}) = \lim_{n \to +\infty} a_{n} \cdot \lim_{n \to +\infty}
            a_{n} =  a \cdot a = a^{2}  $, ισχύει.
        \item Έστω ότι ισχύει για $k$, δηλ. $ \lim_{n \to +\infty} a_{n}^{k} = a^{k}  $
        \item θ.δ.ο. ισχύει για $ k+1 $. Πράγματι, 
\[
    \lim_{n \to +\infty} a_{n}^{k+1}= \lim_{n \to +\infty} (a_{n}^{k} \cdot a_{n})  = \lim_{n \to
    +\infty} a_{n}^{k} \cdot \lim_{n \to +\infty} a_{n} = a^{k} \cdot a = a^{k+1}
 \] 
    \end{itemize}

\item 
Θα αποδείξουμε την πρόταση με Μαθηματική Επαγωγή.
\begin{itemize}
    %TODO αποδειξη
\item     
\end{itemize}

\end{enumerate}
\end{proof}

\begin{prop}[Μηδενική επί Φραγμένη έχει όριο 0]
\item {}
    Έστω $ (a_{n})_{n \in \mathbb{N}} $ μηδενική ακολουθία και $ (b_{n})_{n \in \mathbb{N}} $ 
    φραγμένη. Τότε, $ \lim_{n \to +\infty} (a_{n}\cdot b_{n}) = 0 $. 
\end{prop}

\begin{proof}
\item {}
    $ (b_{n})_{n \in \mathbb{N}} $ φραγμένη $ \Rightarrow \exists M>0 \; : \; \inlineequation[eq:
    mhdfrag]{\abs{b_{n}} \leq M}, \forall n \in \mathbb{N} $.

    $ (a_{n})_{n \in \mathbb{N}} $ μηδενική $ \Rightarrow \lim_{n \to +\infty} a_{n} = 0
    \Leftrightarrow  \forall \varepsilon >0, \; \exists n_{0} \in \mathbb{N} \; : \; \forall 
    n \geq n_{0} \quad \abs{a_{n}-0} < \varepsilon \Leftrightarrow \abs{a_{n}} < \varepsilon $ 

    Άρα και για $ \varepsilon = \frac{\varepsilon}{M} >0, \; \exists n_{0} \in \mathbb{N} \; : \; 
    \forall n \geq n_{0} \quad \inlineequation[eq:mhdfrag2]{\abs{a_{n}} < \frac{\varepsilon}{M}} $ 

    Έστω $ \varepsilon >0 $. Τότε $ \exists n_{0} \in \mathbb{N} $ ώστε $ \forall n \geq n_{0} $ 
    να έχουμε 
    \[
        \abs{a_{n}\cdot b_{n} - 0} = \abs{a_{n}\cdot b_{n} } = \abs{a_{n}} \cdot \abs{b_{n}} 
        \overset{\eqref{eq: mhdfrag}}{\leq} \abs{a_{n}} \cdot M \overset{\eqref{eq:mhdfrag2}}{<} 
        \frac{\varepsilon}{M} \cdot M = \varepsilon
     \]
\end{proof}

\begin{prop}[Κριτήριο Παρεμβολής]
\item {}
    Έστω $ (a_{n})_{n \in \mathbb{N}}, (b_{n})_{n \in \mathbb{N}} $ και 
    $ (c_{n})_{n \in \mathbb{N}} $, τρεις ακολουθίες, τέτοιες ώστε:

    \vspace{\baselineskip}

    \begin{minipage}{0.3\textwidth}
    \begin{enumerate}[i)]
        \item $ a_{n} \leq b_{n} \leq c_{n}, \; \forall n \in \mathbb{N} $ \hfill \tikzmark{a} 
        \item $ \lim_{n \to +\infty} a_{n} = \lim_{n \to +\infty} c_{n} = l $ \hfill \tikzmark{b}
    \end{enumerate}
\end{minipage}

    \mybrace{a}{b}[$ \lim_{n \to +\infty} b_{n} = l$]
\end{prop}

\begin{proof}
\item {}
    $ \lim_{n \to +\infty} a_{n} = l \Leftrightarrow \forall \varepsilon >0, \; \exists n_{1} 
    \in \mathbb{N} \; : \; \forall n \geq n_{1} \quad \abs{a_{n} - l} < \varepsilon \Leftrightarrow 
    \overbrace{l - \varepsilon < a_{n}} < l + \varepsilon $ 

    $ \lim_{n \to +\infty} c_{n} = l \Leftrightarrow \forall \varepsilon >0, \; \exists n_{2} 
    \in \mathbb{N} \; : \; \forall n \geq n_{2} \quad \abs{c_{n} - l} < \varepsilon \Leftrightarrow 
    l - \varepsilon < \underbrace{c_{n} < l + \varepsilon} $

    Επιλέγουμε $ n_{0} = \max \{ n_{1}, n_{2} \} $, οπότε 
    $ \forall \varepsilon > 0, \exists n_{0} \in \mathbb{N} \; : \; \forall n \geq n_{0} $
\begin{gather*}
        l - \varepsilon < a_{n} \leq b_{n} \leq c_{n} < l + \varepsilon \Leftrightarrow \\
        l - \varepsilon < b_{n} < l + \varepsilon \Leftrightarrow \\
        - \varepsilon < b_{n} - l < \varepsilon \Leftrightarrow \\
        \abs{b_{n}-l} < \varepsilon, \; \forall n \geq n_{0}
\end{gather*}
Άρα $ \lim_{n \to +\infty} b_{n} = l $.
\end{proof}

\begin{cor}
    Έστω $ (a_{n})_{n \in \mathbb{N}} $ και $ (b_{n})_{n \in \mathbb{N}} $ ακολουθίες, τέτοιες 
    ώστε: 

    \vspace{\baselineskip}

    \begin{minipage}{0.25\textwidth}
        \begin{enumerate}[i)]
            \item $ \abs{a_{n}} \leq b_{n}, \; \forall n \in \mathbb{N} $ \hfill \tikzmark{a}
            \item $ \lim_{n \to +\infty} b_{n} = 0$ \hfill \tikzmark{b}
        \end{enumerate}
    \end{minipage}

    \mybrace{a}{b}[$ \lim_{n \to +\infty} a_{n} = 0 $]
\end{cor}


\begin{prop}
    Έστω $ \lim_{n \to +\infty} a_{n} = a $ και $ \lim_{n \to +\infty} b_{n} = b $. Τότε
    $
        a_{n} \leq b_{n}, \; \forall n \in \mathbb{N} \Rightarrow a \leq b
     $ 
\end{prop}

\begin{proof}(Με άτοπο)
    Έστω $ a>b \Rightarrow a-b>0 $. Θέτουμε $ \varepsilon = \frac{a-b}{2} $.

   Από τον ορισμό του ορίου για τις δύο ακολουθίες, έχουμε ότι:
\begin{align}
    \exists n_{1} \in \mathbb{N} \; : \; \forall n \geq n_{1} \quad \abs{a_{n}-a} < \frac{a-b}{2} 
    \label{eq:a<b1} \\
    \exists n_{2} \in \mathbb{N} \; : \; \forall n \geq n_{2} \quad \abs{b_{n}-b} < \frac{a-b}{2}
    \label{eq:a<b2} \\
\end{align}

Απο την σχέση~\eqref{eq:a<b1} έχουμε ότι $ - \abs{a_{n}-a} > - \frac{a-b}{2} $. Όμως $ a_{n}- a \geq 
\abs{a_{n}-a} $, οπότε έχουμε $ a_{n}- a > - \frac{a-b}{2} \Rightarrow a_{n} > a -
\frac{a-b}{2} = \frac{a+b}{2}, \forall n \geq n_{1} $.

Απο την σχέση~\eqref{eq:a<b2} έχουμε ότι $  \abs{b_{n}-b} <  \frac{a-b}{2} $. Όμως $ b_{n}- b \leq 
\abs{b_{n}-b} $, οπότε έχουμε $ b_{n}- b <  \frac{a-b}{2} \Rightarrow b_{n} <  b +
\frac{a-b}{2} = \frac{a+b}{2}, \; \forall n \geq n_{2}  $.

Για $ n_{0} = \max \{ n_{1}, n_{2} \} $ έχουμε ότι $ a_{n} > \frac{a+b}{2}, \; \forall n \geq n_{0}$
και $ b_{n} < \frac{a+b}{2}, \; \forall n \geq n_{0} \Rightarrow a_{n} > b_{n}, \; \forall n 
\geq n_{0} $, άτοπο.

\end{proof}


\begin{prop}
    Έστω $ \lim_{n \to +\infty} a_{n} = a $ και $ \lim_{n \to +\infty} b_{n} = b $. Τότε
    $
        a_{n} < b_{n}, \; \forall n \in \mathbb{N} \Rightarrow a \leq b
     $ 
\end{prop}

\begin{proof}
    $ a_{n}< b_{n}, \; \forall n \in \mathbb{N} \Rightarrow a_{n} \leq b_{n}, \; \forall n \in
    \mathbb{N} \Rightarrow a \leq b $
\end{proof}

\begin{rem}
\item {}
    Αν $ a_{n}= \frac{1}{n}, \; \forall n \in \mathbb{N} $ και $ b_{n}=0, \; \forall n \in
    \mathbb{N} $, τότε έχουμε ότι $ \lim_{n \to +\infty} \frac{1}{n} = 0 = a $ και $ \lim_{n \to
    +\infty} b_{n} = 0 = b $, δηλαδή $ a=b=0  $.
\end{rem}

\begin{cor}
    Έστω $ (a_{n})_{n \in \mathbb{N}} $ ακολουθία, τέτοια ώστε $ \lim_{n \to +\infty} a_{n} = a$ 
    και $ a_{n} \geq 0, \; \forall n \in \mathbb{N} $, τότε $ a \geq 0 $.
\end{cor}

\begin{proof}
    Έστω $ b_{n} = 0, \; \forall n \in \mathbb{N} $. Τότε προφανώς $ a_{n} \geq b_{n}, \; \forall n
    \in \mathbb{N} \Rightarrow \lim_{n \to +\infty} a_{n} \geq \lim_{n \to +\infty} b_{n} = 0$.
\end{proof}

\begin{thm}
    Κάθε αύξουσα και άνω φραγμένη ακολουθία συγκλίνει στο supremum του συνόλου 
    των όρων της.
\end{thm}

\begin{proof}
    Έστω $ (a_{n})_{n \in \mathbb{N}} $ ακολουθία πραγματικών αριθμών, 
    τέτοια ώστε:
    \begin{enumerate}[i)]
        \item $ (a_{n})_{n \in \mathbb{N}} $ αύξουσα $ \Leftrightarrow 
            a_{n+1} \geq a_{n}, \; \forall n \in \mathbb{N}$ 
        \item $ (a_{n})_{n \in \mathbb{N}} $ άνω  φραγμένη $ \Leftrightarrow 
            a_{n} \leq M, \; \forall n \in \mathbb{N}$, με $ M \in
            \mathbb{R} $.
    \end{enumerate}

Έστω $ \varepsilon >0 $. Από τη χαρακτηριστική ιδιότητα του supremum, έχουμε
ότι $ \exists a_{n} \in A \; : \; s - \varepsilon < a_{n} $.

Ζητούμενο, είναι $ \lim_{n \to \infty} a_{n} = s \Leftrightarrow - 
\varepsilon < a_{n} -s < \varepsilon \Leftrightarrow s - \varepsilon < 
a_{n} < s + \varepsilon $. Πράγματι, 

$ a_{n} \leq s < s + \varepsilon, \; \forall n \in \mathbb{N} $ και 
$ a_{n} \overset{a_{n} \text{γν. αυξ.}}{\geq} a_{n_{0}} > s - \varepsilon, \forall n \geq n_{0} $

\end{proof}

\begin{thm}
    Κάθε φθίνουσα και κάτω φραγμένη ακολουθία συγκλίνει στο infimum του 
    συνόλου των όρων της.
\end{thm}

\begin{proof}
    Ομόίως με το προηγούμενο θεώρημα. Έστω $ (b_{n})_{n \in \mathbb{N}} $ 
    φθίνουσα και κάτω φραγμένη ακολουθία. Τότε η ακολουϑία $ -(b_{n})_{n \in
    \mathbb{N}} $ είναι αύξουσα και άνω φραγμένη, οπότε από το προηγούμενο 
    θεώρημα, η $ -(b_{n})_{n \in \mathbb{N}} $ συγκλίνει στο supremum του 
    συνόλου των όρων της, έστω $ \sup \{ - b_{1}, - b_{2}, -b_{3}, \ldots \}
    \Rightarrow (b_{n})_{n \in \mathbb{N}} $ συγκλίνει στο ... 
    %TODO   
\end{proof}

\begin{prop}
    $ \lim_{n \to \infty} x^{n} = 0, \; \abs{x} <1  $
\end{prop}

\begin{proof}
\item {}
    \begin{itemize}
        \item $ x = 0 \Rightarrow x^{n} = 0, \; \forall n \in \mathbb{N} $ 
                και άρα η ακολουθία είναι σταθερή και ίση με 0, επομένως 
                $ \lim_{n \to \infty} x^{n} = \lim_{n \to \infty} 0 = 0 $.

            \item $ x \neq 0 \Rightarrow 0 < \abs{x} < 1 \overset{x \neq 0}{
                \Rightarrow} \frac{1}{\abs{x}} > 1  $. 

                Οπότε $ a = \frac{1}{\abs{x}} - 1 > 0 $. Τότε έχουμε ότι 
                \begin{align*} 
                    \frac{1}{\abs{x}} = a+1 &\Rightarrow 
                0< \frac{1}{\abs{x}^{n}}  = (1+a)^{n}
                \overset{\text{Bernoulli}}{\geq} 1 + na \\ 
                & \Leftrightarrow 0 \leq \abs{x}^{n} \leq \frac{1}{1+na} \leq \frac{1}{na} =
                \frac{1}{a} \cdot \frac{1}{n} 
                \end{align*} 

                Επείδη, $ \lim_{n \to \infty} 0 = 0 $ και $ \lim_{n \to \infty}
                \left(\frac{1}{a} \cdot \frac{1}{n} \right) = 0$, από Κριτήριο 
                Παρεμβολής, έχουμε ότι $ \lim_{n \to \infty} \abs{x} ^{n} = 
                0 \Rightarrow \lim_{n \to \infty}x^{n} = 0$.
    \end{itemize}
\end{proof}


\section{Άπειρο Όριο Ακολουθίας}


\begin{dfn}
    Μια ακολουθία πραγματικών αριθμών $ (a_{n})_{n \in \mathbb{N}} $ 
    αποκλίνει στο $ +\infty $ (συμβ.: $ \lim_{n \to \infty} a_{n} = + 
    \infty $), αν $ \forall M>0, \; \exists n_{0} \in 
    \mathbb{N} \; : \; a_{n} > M, \; \forall n \geq n_{0}$
\end{dfn}

\begin{dfn}
    Μια ακολουθία πραγματικών αριθμών $ (a_{n})_{n \in \mathbb{N}} $ 
    αποκλίνει στο $ -\infty $ (συμβ.: $ \lim_{n \to \infty} a_{n} = - 
    \infty $), αν $ \forall M>0, \; \exists n_{0} \in 
    \mathbb{N} \; : \; a_{n} < -M, \; \forall n \geq n_{0}$
\end{dfn}

\begin{prop}
    Έστω $ (a_{n})_{n \in \mathbb{N}} $ ακολουθία θετικών 
    πραγματικών αριθμών. 
    Η $ (a_{n})_{n \in \mathbb{N}} $ αποκλίνει στο $ + \infty $ αν και 
    μονον αν η ακολουθία 
    $ \left(\frac{1}{a_{n}} \right)_{n \in \mathbb{N}} $ συγκλίνει στο 0.
\end{prop}


\begin{proof}
\item {}
    \begin{description}
        \item[$ (\Rightarrow) $] Έστω ότι $ (a_{n})_{n \in \mathbb{N}} $ 
            αποκλίνει στο $ + \infty $. Αφού $ (a_{n})_{n \in \mathbb{N}} $
            ακολουθία θετικών πραγματικών αριθμών, τότε $ a_{n} \neq 0, 
            \; \forall n \in \mathbb{N}$ και άρα, μπορούμε να θεωρήσουμε 
            την ακολουθία, $ b_{n} = \frac{1}{a_{n}}, \; \forall n \in
            \mathbb{N}  $. 

            Θα δείξουμε ότι $ \lim_{n \to \infty} b_{n} = 0 $. Πράγματι,

            Έστω $ \varepsilon >0 $. Θέτω $ M = \frac{1}{\varepsilon} >0 $.
            Τότε $ \exists n_{0} \in \mathbb{N} \; : \; \forall n \geq 
            n_{0} \quad a_{n}> M \Rightarrow a_{n} > \frac{1}{\varepsilon} 
            \Rightarrow \frac{1}{a_{n}} < \varepsilon \Rightarrow
            \abs{\frac{1}{a_{n}} - 0} < \varepsilon \Rightarrow 
            \abs{b_{n}-0} < \varepsilon, \forall n \geq n_{0} $

        \item [$ ( \Leftarrow) $]
            Έστω ότι για μια ακολουθία θετικών πραγματικών, ισχύει ότι 
            $ \lim_{n \to \infty} \frac{1}{a_{n}} = 0$. θ.δ.ο. η $ 
        (a_{n})_{n \in \mathbb{N}}$ αποκλίνει στο $ + \infty $. Πράγματι,

        Έστω $ M > 0 $. Θέτω $ \varepsilon = \frac{1}{M} > 0 $. Από 
        δεδομένο $ \exists n_{0} \in \mathbb{N} \; : \; \forall n \geq 
        n_{0} \quad \abs{\frac{1}{a_{n}} - 0} < \varepsilon \Rightarrow
        \frac{1}{a_{n}} > \frac{1}{\varepsilon}=M $
    \end{description}
\end{proof}

\begin{prop}
    $ \lim_{n \to \infty} x^{n} = +\infty, \; x >1 $.
\end{prop}

\begin{proof}
\item {}
    $ x >1 \Rightarrow x \neq 0 $. Θέτω  $a = \frac{1}{x} \Rightarrow 
    \frac{1}{a^{n}} = x^{n} $ και $ \lim_{n \to \infty} \frac{1}{a_{n}} = 
    \lim_{n \to \infty}x^{n}$.

    Προφανώς $ 0 < a <1 $, ως αντίστροφος του $x$. Σύμφωνα με γνωστή προταση
    $ \lim_{n \to \infty} a^{n} = 0 $. Η ακολουθία $ (a_{n})_{n \in \mathbb{N}}
    $ είναι μια ακολουθία θετικών όρων που συγκλίνει στο 0. Σύμφωνα με 
    την προηγούμενη πρόταση, η ακολουθία $ \left(\frac{1}{a_{n}}\right)_{n 
    \in \mathbb{N}} $ αποκλίνει στο $ + \infty $, και άρα έχουμε το ζητούμενο.
\end{proof}


\begin{prop}
    $ \lim_{n \to \infty} \sqrt[n]{n} = 1, \; \forall n \in \mathbb{N}  $
\end{prop}

\begin{proof}
    Έστω $ n \in \mathbb{N} \Rightarrow n \geq 1, \; \forall n \in \mathbb{N}
    \Rightarrow n ^{\frac{1}{2n}} \geq 1^{\frac{1}{2n}}, \; \forall n 
    \in \mathbb{N} \Rightarrow \sqrt[2n]{n} \geq 1, \; \forall n \in 
    \mathbb{N} $

    Θέτουμε $ a = \sqrt[2n]{n} -1 \geq 0 $. Τότε $ \sqrt[2n]{n} = 1 + a 
    \Rightarrow \sqrt{n} = (1+a)^{n} \overset{\text{Bernoulli}}{\geq} 1 
    + na \Rightarrow na \leq \sqrt{n} - 1  $. Οπότε $ 0 \leq a \leq
    \frac{\sqrt{n} -1}{n} $. Τώρα έχουμε, 

    $ \lim_{n \to \infty} \frac{\sqrt{n} -1}{n} = \lim_{n \to \infty} 
    \left( \frac{\sqrt{n}}{n} - \frac{1}{n}\right) = 0 - 0 = 0 $ και 
    άρα από το Κριτήριο Παρεμβολής, έχουμε ότι $ \lim_{n \to \infty} a = 0 
    \Rightarrow \lim_{n \to \infty} \sqrt[2n]{n} = 1 \Rightarrow \lim_{n \to
    \infty} (\sqrt[n]{n}) = \lim_{n \to \infty} (\sqrt[2n]{n})^{2} =  1^{2} = 1 \Rightarrow \lim_{n \to \infty}
    \sqrt[n]{n} =1 $.
\end{proof}

\begin{prop}
    Έστω $ a>0 $. Τότε $ \lim_{n \to \infty} \sqrt[n]{a}=1, \; \forall n \in
    \mathbb{N} $.
\end{prop}

\begin{proof}
\item {}
    \begin{itemize}
    \item $ a>1 \Rightarrow a^{\frac{1}{n}} > 1^{\frac{1}{n}}, \; \forall n
            \in \mathbb{N} \Rightarrow \sqrt[n]{n} >1, \; \forall n \in
            \mathbb{N} $

            Για κάθε $ n \in \mathbb{N} $ θέτουμε $ b_{n} = \sqrt[n]{n} -1 
            > 0 \Rightarrow \sqrt[n]{n} = b_{n} + 1 \Rightarrow a = (
            b_{n}+1)^{n} \geq 1 + n b_{n} \Rightarrow n b_{n} \leq a-1
            \Rightarrow 0 \leq b_{n} \leq \frac{a-1}{n}, \; \forall n \in
            \mathbb{N} $

            Επειδή $ \lim_{n \to \infty} \frac{a-1}{n} = \lim_{n \to \infty}
            (\frac{a}{n} - \frac{1}{n}) = 0 - 0 = 0 $, άρα από το 
            Κριτήριο Παρεμβολής, έχουμε ότι $ \lim_{n \to \infty} b_{n} = 
            \lim_{n \to \infty} \sqrt[n]{a}-1 = 0 \Rightarrow \lim_{n \to
        \infty} \sqrt[n]{a} =1 $.

    \item $ a< 1 \Rightarrow \frac{1}{a} > 1 $. Τότε από την προηγούμενη πρόταση
        έχουμε 
        $ \lim_{n \to \infty} \sqrt[n]{\frac{1}{a} } = 1 \Rightarrow \lim_{n \to
        \infty} \frac{1}{\sqrt[n]{\frac{1}{a}}} = \frac{1}{1} = 1 \Rightarrow
        \lim_{n \to \infty} \sqrt[n]{a} = 1$.
    \end{itemize}
\end{proof}

\begin{prop}
    $ \lim_{n \to \infty} \sqrt[n]{n!} = +\infty $ 
\end{prop}


\begin{proof}
\item {}
    \begin{itemize}
        \item 
    Έστω $ n \in \mathbb{N} $ με $ n $ άρτιος, δηλαδή ο $ \frac{n}{2} \in
    \mathbb{N} $. Τότε

    \begin{align*}
        \sqrt[n]{n!} = \sqrt[n]{1 \cdot 2 \cdot \frac{n+1}{2}
            \left(\frac{n+1}{2} +1\right) \cdots n} &\geq \sqrt[n]{1\cdot 2 \cdots
            \underbrace{\frac{n+1}{2} \cdot \frac{n+1}{2} \cdots \frac{n+1}{2}}_{
        \frac{n+1}{2} \; \text{φορές}}} \\
&\geq
\sqrt[n]{\left(\frac{n+1}{2}\right)^{\frac{n+1}{2}}} \\ 
&\geq \sqrt[n]{\left(\frac{n+1}{2}\right)^{\frac{n}{2}}} =
\sqrt{ \frac{n+1}{2} } \geq \sqrt{\frac{n}{2}} 
     \end{align*} 

 \item Έστω $ n \in \mathbb{N} $ με $ n $ περιττός, δηλαδή ο $
     \frac{n+1}{2} = \frac{n}{2} + \frac{1}{2} \in \mathbb{N}  $. Τότε

     \begin{align*}
         \sqrt[n]{n!} = \sqrt[n]{1\cdot 2 \cdots \frac{n+1}{2}
             \left(\frac{n+1}{2} +1\right) \cdots n}
             &\geq \sqrt[n]{1 \cdot 2 \cdots
     \frac{n+1}{2}  \cdot \frac{n+1}{2} \cdots \frac{n+1}{2} } \\ 
             &\geq 
     \sqrt[n]{\left(\frac{n+1}{2} \right)^{\frac{n+1}{2}}} \\ 
             & \geq
     \sqrt[n]{\left(\frac{n+1}{2} \right)^{\frac{n}{2}}} = \sqrt{\frac{n+1}{2}} \geq 
     \sqrt{\frac{n}{2}}
     \end{align*}
    \end{itemize}

    Από τα παραπάνω προκύπτει ότι $ \sqrt[n]{n!} \geq \sqrt{\frac{n}{2}
    }, \; \forall n \in \mathbb{N} $. 

    Θ.δ.ο. $ \lim_{n \to \infty} \sqrt[n]{n!} = + \infty $, δηλαδή ότι 
    η ακολουθία αποκλίνει.

    Έστω $ M >0 $. Από την Αρχιμήδεια Ιδιότητα υπάρχει φυσικός αριθμός $ 
    n_{0} \in  \mathbb{N} $ με $ n_{0}>2M^{2} $, ώστε 

    $ \forall n \geq n_{0} \quad \frac{n}{2} > M^{2} \Rightarrow
    \sqrt{\frac{n}{2}} > M \Rightarrow \sqrt[n]{n!} > M $. 
\end{proof}

\begin{thm}[Bolzano-Weierstrass]
    Κάθε φραγμένη ακολουθία πραγματικών αριθμών έχει συγκλίνουσα υπακολουθία
\end{thm}

%TODO apodeixh

%TODO
TODO η $ (1+ \frac{1}{n} )^{n} $ γν. αυξουσα κ ανω φραγμενη.

\section{Αναδρομικές Ακολουθίες}
\end{document}
