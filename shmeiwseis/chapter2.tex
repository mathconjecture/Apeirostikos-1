\documentclass[main.tex]{subfiles}


\begin{document}


\section{Όριο Ακολουθίας}

\begin{dfn}
    Ακολουθία πραγματικών αριθμών ονομάζεται κάθε συνάρτηση με πεδίο ορισμού 
    τους φυσικούς αριθμούς. 
    \begin{align*}
        a \colon &\mathbb{N} \to \mathbb{R} \\
                 &n \to a(n)=a_{n}
    \end{align*} 
    Οι ακολουθίες συμβολίζονται ως $ (a_{n})_{n \in \mathbb{N}} $  
    ή $ \{ a_{n} \} _{n \in \mathbb{N}} $  ή $ \{ a_{n} \} _{n=1}^{+\infty} $, κλπ.
\end{dfn}

\begin{dfn}
    Σύνολο Τιμών (Σ.Τ.) της ακολουθίας $ (a_{n})_{n \in \mathbb{N}} $, ονομάζουμε 
    το σύνολο των όρων της, δηλαδή το $ \{ a_{1}, a_{2}, \ldots, a_{n} \} $ το οποίο μπορεί 
    να είναι πεπερασμένο ή άπειρο.
\end{dfn}

\begin{examples}
\item {}
    \begin{enumerate}[i)]
        \item $ a_{n} = n, \; \forall n \in \mathbb{N} $. Έχει Σ.Τ. το σύνολο 
            $  \{ 1,2,3, \ldots \} $.
        \item $\left(\frac{1}{n}\right)_{n \in \mathbb{N}} $. Έχει Σ.Τ. το σύνολο 
            $  \left\{ 1, \frac{1}{2}, \frac{1}{3}, \ldots \right\} $.
        \item $ \{(-1)^{n}\}_{n=1}^{+ \infty}, $. Έχει Σ.Τ. 
            το σύνολο $ \{ -1,1 \} $.
        \item $ a_{n} = c, \; \forall n \in \mathbb{N}, c \in \mathbb{R} $.
            Έχει Σ.Τ. το σύνολο $ \{ c \} $ και ονομάζεται σταθερή ακολουθία.
        \item $ a_{n}=2n, \; \forall n \in \mathbb{N} $. Έχει Σ.Τ. το σύνολο 
            $ \{ 2,4,6, \ldots, 2n, \ldots \} $. Πρόκειται για την ακολουθία 
            των άρτιων φυσικών αριθμών.
        \item $ a_{n}= 2n-1, \; \forall n \in \mathbb{N} $. Έχει Σ.Τ. το 
            σύνολο $ \{ 1,3,5, \ldots, 2n+1, \ldots \} $. Πρόκεται για την 
            ακολουθία των περιττών φυσικών αριθμών.
        \item \label{ex:anadr} $ a_{1}= a_{2} = 1 $ και $ a_{n+2}=a_{n+1}+a_{n}, \; 
            \forall n \in \mathbb{N}$. Έχει Σ.Τ. το σύνολο $ \{ 1,1,2,3,5,8,
            13,21,34, \ldots,\} $. Πρόκειται για την ακολουθία Fibonacci. 
    \end{enumerate}
\end{examples}

\begin{rem}
\item {}
    \begin{enumerate}[i)]
        \item Ουσιαστικά οι ακολουθίες είναι λίστες πραγματικών αριθμών.
        \item Η ακολουθία \ref{ex:anadr}, όπου κάθε επόμενος όρος, ορίζεται με τη 
            βοήθεια του προηγούμενου, λέγεται αναδρομική ακολουθία. Προτάσεις που 
            αφορούν αναδρομικές ακολουθίες, αποδεικνύονται με Μαθηματική Επαγωγή.
    \end{enumerate}
\end{rem}


\begin{dfn}
    Δυο ακολουθίες, $(a_{n})_{n \in \mathbb{N}}$  και $ (b_{n})_{n \in \mathbb{N}} $
    ονομάζονται ίσες, αν $ a_{n} = b_{n}, \; \forall n \in \mathbb{N} $.
\end{dfn}

\begin{dfn}
    Οι πράξεις μεταξύ ακολουθιών, ορίζονται όπως ακριβώς και για τις συναρτήσεις.
\end{dfn}

\section{Φραγμένες Ακολουθίες}

\begin{dfn}
\item {}
    Μια ακολουθία $ (a_{n})_{n \in \mathbb{N}} $ ονομάζεται:
    \begin{enumerate}[i)]
        \item άνω φραγμένη $ \overset{\text{ορ.}}{\Leftrightarrow} \exists M \in 
            \mathbb{R} \; : \; a_{n} \leq M, \; \forall n \in \mathbb{N}$.
        \item κάτω φραγμένη $ \overset{\text{ορ.}}{\Leftrightarrow} \exists m \in 
            \mathbb{R} \; : \; m \leq a_{n}, \; \forall n \in \mathbb{N}  $
        \item φραγμένη $ \overset{\text{ορ.}}{\Leftrightarrow} \exists$ είναι άνω και 
            κάτω φραγμένη.
    \end{enumerate}

    \begin{prop}
        $ (a_{n})_{n \in \mathbb{N}} $ φραγμένη $ \Leftrightarrow \exists M>0, \; M \in 
        \mathbb{R} \; : \; \abs{a_{n}} \leq M, \; \forall n \in \mathbb{N} $
        (Απολύτως φραγμένη).
    \end{prop}
\end{dfn}

\begin{examples}
\item {}  
    \begin{enumerate}[i)]
        \item $ a_{n}= \frac{1}{n}, \; \forall n \in \mathbb{N} $, είναι φραγμένη.

            Πράγματι, $ 0 \leq \frac{1}{n} \leq 1, \; \forall n \in \mathbb{N} $. 

            Επίσης, $ \abs{\frac{1}{n}} = \frac{1}{n} \leq 1 $, άρα και 
            απολύτως φραγμένη.
        \item $ a_{n}=(-1)^{n} \frac{1}{n}, \; \forall n \in \mathbb{N} $ 
            είναι απολύτως φραγμένη. Πράγματι,

            \[
                \abs{a_{n}} = \abs{(-1)^{n} \frac{1}{n}} = \abs{(-1)^{n}} 
                \cdot \abs{\frac{1}{n}} = \abs{-1}^{n} \cdot \frac{1}{n}
                = 1 \cdot \frac{1}{n} = \frac{1}{n} \leq 1, \; \forall n \in 
                \mathbb{N}
            \] 

        \item $ a_{n}= \frac{(n-1)!}{n^{n}}, \; \forall n \in \mathbb{N} $
            είναι φραγμένη. Πράγματι, $ a_{n} > 0, \; \forall n \in 
            \mathbb{N}$, άρα 0 κ.φ. της $( a_{n})_{n \in \mathbb{N}} $. 
            Επίσης 
            \[
                a_{n}= \frac{(n-1)!}{n^{n}} = \frac{1 \cdot 2 \cdots
                    (n-1)}{n^{n}} < \frac{\overbrace{n \cdot n \cdots n}
                ^{n-1 \; \text{φορές}}}{n^{n}} = \frac{n^{n-1}}{n^{n}} =
                \frac{1}{n} \leq 1, \; \forall n \in \mathbb{N},
            \] άρα το 1 είναι α.φ. της $ (a_{n})_{n \in \mathbb{N}} $ 

        \item $ a_{n}= 1 + \left(- \frac{1}{2} \right) + \left(- 
            \frac{1}{2}\right)^{2} + \cdots + \left(-\frac{1}{2} \right) ^{n}, 
            \; \forall n \in \mathbb{N} $ είναι απολύτως φραγμένη. Πράγματι, 
            \[ a_{n} = 1 + \left(- \frac{1}{2}\right) + \left(- \frac{1}{2} 
                \right)^{2} + \cdots + \left(- \frac{1}{2} \right)^{n} = 
                \frac{1 - (- \frac{1}{2} )^{n}}{1 - (- \frac{1}{2})} = 
            \frac{2}{3} \left[1 - \left(- \frac{1}{2} \right)^{n}\right] \]
            Επομένως
            \[
                \abs{a_{n}} = \abs{\frac{2}{3} \left[1-(- \frac{1}{2} )^{n}\right]} = 
                \frac{2}{3} \abs{\abs{1} - \left(- \frac{1}{2}\right)^{n}} \leq 
                \frac{2}{3} \left(1 + \abs{-\frac{1}{2} }^{n} \right) = 
                \frac{2}{3} \left(1+ \frac{1}{2^{n}}\right) < \frac{2}{3}
                (1+1) = \frac{4}{3} 
            \] 

        \item $ a_{n}= 2n+5, \; \forall n \in \mathbb{N} $ είναι κάτω φραγμένη.
            Πράγματι, $ 7 \leq 2n+5, \; \forall n \in \mathbb{N} $, άρα το 
            7 είναι κ.φ. της $ (a_{n} )_{n \in \mathbb{N}} $.

        \item $ a_{1}=2, \; a_{n+1}=2 - \frac{1}{a_{n}}, \forall n \in \mathbb{N}$
            είναι φραγμένη. Πράγματι, με επαγωγή, έχουμε:
            \begin{itemize}
                \item Για $ n=1 $, $ a_{1}=2>1 $, ισχύει. 
                \item Έστω ότι ισχύει για $n$, δηλ. \inlineequation[eq:
                    anadepag1]{a_{n}>1}.
                \item Θ.δ.ο. ισχύει και για $ n+1 $. Πράγματι, από τη σχέση~
                    \eqref{eq: anadepag1}, έχουμε
                    \[
                        a_{n}>1 \Rightarrow \frac{1}{a_{n}} < 1 \Rightarrow - \frac{1}{a_{n}} > 
                        -1 \Rightarrow 2 - \frac{1}{a_{n}} > 2-1 \Rightarrow a_{n+1} > 1.
                    \] 
            \end{itemize}
    \end{enumerate}
\end{examples}

\section{Μονοτονία Ακολουθιών}

\begin{dfn}
    Μια ακολουθία $ (a_{n})_{n \in \mathbb{N}} $ λέγεται:
    \begin{enumerate}[i)]
        \item γνησίως αύξουσα $ \overset{\text{ορ.}}{\Leftrightarrow} a_{n} 
            < a_{n+1}, \; n \in \mathbb{N}$
        \item γνησίως φθίνουσα $ \overset{\text{ορ.}}{\Leftrightarrow} a_{n} 
            > a_{n+1}, \; n \in \mathbb{N}$
        \item άυξουσα $ \overset{\text{ορ.}}{\Leftrightarrow} a_{n} \leq 
            a_{n+1}, \forall n \in \mathbb{N}  $.
        \item φθίνουσα $ \overset{\text{ορ.}}{\Leftrightarrow} a_{n} \geq 
            a_{n+1}, \forall n \in \mathbb{N}  $.
    \end{enumerate}
\end{dfn}

\begin{rems}
\item {}
    \begin{enumerate}[i)]
        \item $ (a_{n})_{n \in \mathbb{N}} $ γνησίως αύξουσα (φθίνουσα) $ 
            \Rightarrow (a_{n})_{n \in \mathbb{N}} $ αύξουσα (φθίνουσα) 
        \item $ (a_{n})_{n \in \mathbb{N}} $ γνησίως φθίνουσα  $ 
            \Rightarrow (a_{n})_{n \in \mathbb{N}} $ ανω φραγμένη, με 
            α.φ. το $ a_{1} $  
        \item $ (a_{n})_{n \in \mathbb{N}} $ γνησίως αύξουσα  $ 
            \Rightarrow (a_{n})_{n \in \mathbb{N}} $ κάτω φραγμένη, με 
            κ.φ. το $ a_{1} $  
    \end{enumerate}
\end{rems}

\begin{dfn}\label{dfn:isodmono}[Ισοδύναμος ορισμός της μονοτονίας]
\item {}
    Αν μια ακολουθία $ (a_{n})_{n \in \mathbb{N}} $ διατηρεί πρόσημο 
    $ \forall n \in \mathbb{N} $, τότε λέγεται:
    \begin{enumerate}[i)]
        \item γνησίως αύξουσα $ \overset{\text{ορ.}}{\Leftrightarrow} 
            \frac{a_{n+1}}{a_{n}} > 1, \; \forall n \in \mathbb{N}$
        \item γνησίως φθίνουσα $ \overset{\text{ορ.}}{\Leftrightarrow} 
            \frac{a_{n+1}}{a_{n}} < 1, \; \forall n \in \mathbb{N}$
        \item  αύξουσα $ \overset{\text{ορ.}}{\Leftrightarrow} 
            \frac{a_{n+1}}{a_{n}} \geq 1 $, για τουλ. ένα $ n \in \mathbb{N} $
        \item  φθίνουσα $ \overset{\text{ορ.}}{\Leftrightarrow} 
            \frac{a_{n+1}}{a_{n}} \leq 1 $, για τουλ. ένα $ n \in \mathbb{N} $
    \end{enumerate}
\end{dfn}

\section{Μεθοδολογία εύρεσης μονοτονίας μιας ακολουθίας}
\begin{itemize}
    \item Σχηματίζουμε τη διαφορά $ a_{n+1} - a_n $ και ελέγχουμε το 
        πρόσημό της. Αν $ a_{n+1}-a_{n}>0, \; (<0), \; \forall n \in 
        \mathbb{N} $ τότε $ (a_{n})_{n \in \mathbb{N}}$ γνησίως 
        αύξουσα (γνησίως φθίνουσα). Αν για τουλάχιστον ένα 
        $ n \in \mathbb{N} $, στις παραπάνω ανισότητες, έχω ισότητα, 
        τότε  $ (a_{n})_{n \in \mathbb{N}} $ είναι αύξουσα (φθίνουσα).
    \item Αν οι όροι της ακολουϑίας διατηρούν πρόσημο, $ \; \forall n \in
        \mathbb{N} $ τότε συγκρίνουμε το πηλίκο δυο διαδοχικών όρων της 
        ακολουθίας με τη μονάδα, και βγάζουμε τα συμπεράσματά μας 
        σύμφωνα με τον ορισμό~\ref{dfn:isodmono}
    \item Αν η ακολουθία δίνεται με μη-αναδρομικό τύπο, και είναι 
        (αρκετά) σύνθετη, τότε μετατρέπω την ακολουθία στην αντίστοιχη 
        συνάρτηση και μελετάμε τη μονοτονία της αντίστοιχης συνάρτησης.
    \item Αν η $ (a_{n})_{n \in \mathbb{N}} $ δίνεται με αναδρομικό 
        τύπο $ (a_{n+1}= f(a_{n})) $ τότε συνήθως η απόδειξή της γίνεται 
        με Μαθηματική Επαγωγή.
\end{itemize}

\begin{examples}
\item {}
    \begin{enumerate}[i)]
        \item Η $ a_{n} = 2n-1, \; \forall n \in \mathbb{N} $ είναι γνησίως 
            αύξουσα. Πράγματι,

            \twocolumnsides{
                \begin{description}
                    \item[Α᾽ τρόπος]
                        \begin{align*}
                            n+1 &\geq n, \; \forall n \in \mathbb{N} \\
                            2(n+1) &\geq 2n, \; \forall n \in \mathbb{N} \\
                            2(n+1) -1 &\geq 2n-1, \; \forall n \in \mathbb{N} \\
                            a_{n+1} &\geq a_{n}, \; \forall n \in \mathbb{N}
                        \end{align*}
                \end{description}

                }{
                \begin{description}
                    \item[Β᾽ τρόπος] 
                        \begin{align*}
                            a_{n+1}-a_{n} &= 2(n+1)-1 - (2n-1) \\
                                          &= 2 >0, \; \forall n \in \mathbb{N} 
                        \end{align*}
                \end{description}

            }

        \item Η $ a_{n} = \frac{(n-1)!}{n^{n}}, \; \forall n \in \mathbb{N} $ 
            είναι γνησίως φθίνουσα. Πράγματι, επειδή όλοι οι όροι της ακολουθίας 
            είναι θετικοί, επομένως διατηρεί πρόσημο, έχουμε:
            \[
                \frac{a_{n+1}}{a_n} =
                \frac{\frac{(n+1-1)!}{(n+1)^{n+1}}}{\frac{(n-1)!}{n^{n}}} =  
                \frac{n^{n}\cdot n!}{(n+1)^{n+1}\cdot (n-1)!} =
                \frac{n^{n+1}}{(n+1)^{n+1}} = \left(\frac{n}{n+1} \right)^{n+1} < 1, \; \forall n
                \in \mathbb{N} 
            \] 

        \item Η $ a_{n}= \frac{4^{n}}{n^{2}}, \; \forall n \in \mathbb{N} $ είναι 
            αύξουσα. Πράγματι, επειδή όλοι οι όροι της ακολουθίας είναι θετικοί, 
            επομένως διατηρεί πρόσημο, έχουμε:
            \[
                \frac{a_{n+1}}{a_{n}} 
                = \frac{\frac{4^{n+1}}{(n+1)^{2}}}{\frac{4^{n}}{n^{2}}} 
                = \frac{4^{n+1}\cdot n^{2}}{4^{n}\cdot (n+1)^{2}} 
                = \frac{4n^{2}}{(n+1)^{2}}
                    = \left( \frac{2n}{n+1} \right)^{2} \geq 1, 
                    \; \forall n \in \mathbb{N} 
                \]
                Η ακολουθία δεν είναι γνησίως αύξουσα, γιατί $ 
                a_{1}= a_{2}=4$.
                
            \item Η $ a_{n+1}=2 - \frac{1}{a_{n}}, \; \forall n \in \mathbb{N}
                $ με $ a_{1} = 2 $ είναι γνησίως φθίνουσα. Πράγματι, 
                \begin{itemize}
                    \item Για $ n=1 $, έχω: $ a_{1}= 2 >
                        \frac{3}{2} = 2 - \frac{1}{2} = a_{2}$, ισχύει.
                    \item Έστω ότι ισχύει για $n$, δηλ.
                        \inlineequation[eq:epag]{a_{n+1}<a_{n}}.
                    \item Θ.δ.ο. ισχύει για $ n+1 $. Πράγματι, 
                        \[
                            a_{n+1+1}=2- \frac{1}{a_{n+1}}
                            \overset{\eqref{eq:epag}}{<} 2 - 
                            \frac{1}{a_{n}} = a_{n+1}
                         \] 
                \end{itemize}
   \end{enumerate}
\end{examples}

\section{Σύγκλιση Ακολουθιας}

\begin{dfn}
    Περιοχή ένος πραγματικού αριθμού $ x_{0} $ ονομάζεται κάθε ανοιχτό 
    διάστημα$ (a,b)$ που περιέχει το $ x_{0} $. 
\end{dfn}

\begin{rem}
\item {}
    \begin{enumerate}[i)]
        \item 
    Αν $ \varepsilon > 0 $, τότε περιοχές του $ x_{0} $ της μορφής $ 
    (x_{0}- \varepsilon , x_{0} + \varepsilon) $ έχουν ακτίνα $ \varepsilon $
    και κέντρο το $ x_{0} $. 

\item $ x \in (x_{0}- \varepsilon, x_{0} + \varepsilon) \Leftrightarrow
    x_{0}- \varepsilon < x < x_{0}+ \varepsilon \Leftrightarrow 
    - \varepsilon < x - x_{0} < \varepsilon \Leftrightarrow 
    \abs{x- x_{0}} < \varepsilon  $ 
    \end{enumerate}
\end{rem}

\begin{dfn}
    Μια ακολουθία $ (a_{n})_{n \in \mathbb{N}} $ συγκλίνει στον πραγματικό 
    αριθμό $ a \in \mathbb{R} $ (έχει όριο το $ a \in \mathbb{R} $ ή 
    τείνει στο $ a \in \mathbb{R} $), και συμβολίζουμε 
    $ \lim\limits_{n\to \infty} a_{n}=a $ (ή $ a_{n} \xrightarrow{n \to 
    \infty} a $) αν 
    \[
        \forall \varepsilon >0, \; \exists n_{0} \in \mathbb{N} \; : 
        \; \forall n \in \mathbb{N} \; \text{με} \; n > n_{0} \Rightarrow 
        \abs{a_{n}-a} < \varepsilon
     \] 
\end{dfn}

\begin{rem}
    Γενικά το $ n_{0} $ εξαρτάται από το $ \varepsilon $ και ισχύει ότι
    $ n_{0} = n_{0}(\varepsilon) $.
\end{rem}

\begin{dfn}
    Η ακολουθία $ (a_{n})_{n \in \mathbb{N}}$ λέγεται μηδενική ακολουθία 
    αν $ \lim\limits_{n\to \infty} = 0 $
\end{dfn}

\begin{examples}
\item {}
    \begin{enumerate}[i)]
        \item $ \lim\limits_{n \to \infty} \frac{1}{n} = 0 $.
            \begin{proof}
            \item {}
                \begin{description}
                    \item[Δοκιμή:] $ \abs{\frac{1}{n} -0} < \varepsilon
                        \Leftrightarrow \abs{\frac{1}{n}} < \varepsilon 
                        \Leftrightarrow \frac{1}{n} < \varepsilon 
                        \Leftrightarrow n > \frac{1}{\varepsilon}$
                \end{description}
                \begin{description}
                    \item[Α᾽ Τρόπος:] 
                Έστω $ \varepsilon >0 $. Τότε $ \exists n_{0} \in\mathbb{N} $
                με \inlineequation[eq:1n1]{n_{0} > \frac{1}{\varepsilon}} (Αρχ. Ιδιοτ.) τέτοιο 
                ώστε \inlineequation[eq:1n2]{\forall n \geq n_{0}}
                        \[
                            \abs{\frac{1}{n} -0} = \abs{\frac{1}{n}} =
                            \frac{1}{n} \overset{\eqref{eq:1n2}}{\leq}
                            \frac{1}{n_{0}} \overset{\eqref{eq:1n1}}{<} 
                            \frac{1}{\frac{1}{\varepsilon}} = \varepsilon 
                        \]

                    \item [Β᾽ Τρόπος:]
                Έστω $ \varepsilon >0 $. Τότε $ \exists n_{0} \in\mathbb{N} $
                με \inlineequation[eq:1n3]{\frac{1}{n_{0}} < \varepsilon} 
                (Αρχ. Ιδιοτ.) τέτοιο ώστε \inlineequation[eq:1n4]{\forall n 
                \geq n_{0}}
                        \[
                            \abs{\frac{1}{n} -0} = \abs{\frac{1}{n}} =
                            \frac{1}{n} \overset{\eqref{eq:1n4}}{\leq}
                         \frac{1}{n_{0}} \overset{\eqref{eq:1n3}}{<} 
                \varepsilon 
                         \]
                \end{description}
 
            \end{proof}

         \item $ \lim\limits_{n \to \infty} \frac{1}{n^{4}} = 0 $. 
             \begin{proof}
             \item {}
                 \begin{description}
                     \item[Δοκιμή:]$ \abs{\frac{1}{n^{4}} - 0} < \varepsilon 
                         \Leftrightarrow \abs{\frac{1}{n^{4}}} < \varepsilon 
                         \Leftrightarrow \frac{1}{n^{4}} < \varepsilon $
                 \end{description}
                 Έστω $ \varepsilon >0 $. Τότε $ \exists n_{0}  \in 
                 \mathbb{N}$ με $ \frac{1}{n_{0}^{4}} < \varepsilon $ 
                 τέτοιο ώστε $ \forall n \geq n_{0} $ 
                 \[
                     \abs{\frac{1}{n^{4}} - 0 } = \abs{\frac{1}{n^{4}}} 
                     = \frac{1}{n^{4}} \leq \frac{1}{n_{0}^{4}} < \varepsilon
                  \] 
             \end{proof}

         \item $ \lim\limits_{n \to \infty} \frac{1}{\sqrt{n}} = 0$.
             \begin{proof}
             \item {}
                 \begin{description}
                     \item[Δοκιμή:] $ \abs{\frac{1}{\sqrt{n}}-0 } 
                         < \varepsilon 
                         \Leftrightarrow \abs{\frac{1}{\sqrt{n}} } < 
                         \varepsilon 
                         \Leftrightarrow \frac{1}{\sqrt{n}} < 
                         \varepsilon $
                 \end{description}
                 Έστω $ \varepsilon > 0 $. Τότε $ \exists n_{0} \in \mathbb{N} $
                 με $ \frac{1}{\sqrt{n_{0}}} < \varepsilon $ τέτοιο ώστε 
             \end{proof}
     \end{enumerate}
 \end{examples}

%TODO
 TODO η $ (1+ \frac{1}{n} )^{n} $ γν. αυξουσα κ ανω φραγμενη.


\end{document}
