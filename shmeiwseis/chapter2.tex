\documentclass[main.tex]{subfiles}


\begin{document}


\section{Ορισμός Ακολουθίας}

\mydfn{\textcolor{Col\thechapter}{Ακολουθία} πραγματικών αριθμών ονομάζεται 
  κάθε συνάρτηση με πεδίο ορισμού τους φυσικούς αριθμούς. 
  \begin{align*}
    a \colon &\mathbb{N} \to \mathbb{R} \\
             &n \mapsto a(n)=a_{n}
  \end{align*} 
  Οι ακολουθίες συμβολίζονται ως $ (a_{n})_{n \in \mathbb{N}} $  ή 
$ \{ a_{n} \} _{n \in \mathbb{N}} $  ή $ \{ a_{n} \} _{n=1}^{+\infty}$ , κλπ.}

\mydfn{\textcolor{Col\thechapter}{Σύνολο Τιμών} (Σ.Τ.) της ακολουθίας 
  $ (a_{n})_{n \in \mathbb{N}} $, 
  ονομάζουμε το σύνολο των όρων της, δηλαδή το σύνολο 
$ \{ a_{1}, a_{2}, \ldots, a_{n} \} $ το οποίο μπορεί να είναι πεπερασμένο ή άπειρο.}

\begin{examples}
\item {}
  \begin{enumerate}[i)]
    \item Η ακολολουθία $ Η ακολουθία a_{n} = n, \; \forall n \in \mathbb{N} $. 
      Έχει Σ.Τ.  το σύνολο $  \{ 1,2,3, \ldots \} $.
    \item Η ακολουθία $ a_{n}=\left(\frac{1}{n}\right)_{n \in \mathbb{N}} $. 
      Έχει Σ.Τ. το σύνολο $  \left\{ 1, \frac{1}{2}, \frac{1}{3}, \ldots \right\} $.
    \item Η ακολουθία $ a_{n}= \{(-1)^{n}\}_{n=1}^{+ \infty}, $. Έχει Σ.Τ. 
      το σύνολο $ \{ -1,1 \} $.
    \item Η ακολουθία $ a_{n} = c, \; \forall n \in \mathbb{N}, c \in \mathbb{R} $.
      Έχει Σ.Τ. το σύνολο $ \{ c \} $ και ονομάζεται 
      \textcolor{Col\thechapter}{σταθερή ακολουθία}.
    \item Η ακολουθία $ a_{n}=2n, \; \forall n \in \mathbb{N} $. Έχει Σ.Τ. το 
      σύνολο $ \{ 2,4,6, \ldots, 2n, \ldots \} $. Πρόκειται για την 
      \textcolor{Col\thechapter}{ακολουθία των άρτιων} φυσικών αριθμών.
    \item Η ακολουθία $ a_{n}= 2n-1, \; \forall n \in \mathbb{N} $. Έχει Σ.Τ. το 
      σύνολο $ \{ 1,3,5, \ldots, 2n-1, \ldots \} $. Πρόκεται για την 
      \textcolor{Col\thechapter}{ακολουθία των περιττών} φυσικών αριθμών.
    \item \label{ex:anadr} Η ακολουθία $ a_{1}= a_{2} = 1 $ και $ a_{n+2}=a_{n+1}
      +a_{n}, \; \forall n \in \mathbb{N}$. Έχει Σ.Τ. το σύνολο 
      $ \{ 1,1,2,3,5,8, 13,21,34, \ldots\} $.  Πρόκειται για την 
      \textcolor{Col\thechapter}{ακολουθία Fibonacci}. 
  \end{enumerate}
\end{examples}

\begin{rem}
\item {}
  \begin{enumerate}[i)]
    \item Ουσιαστικά οι ακολουθίες είναι \textbf{λίστες} πραγματικών αριθμών.
    \item Η ακολουθία~\ref{ex:anadr}, όπου κάθε επόμενος όρος,
      ορίζεται με τη βοήθεια του προηγούμενου, λέγεται
      \textcolor{Col\thechapter}{αναδρομική ακολουθία}. 
      Προτάσεις που αφορούν αναδρομικές ακολουθίες, 
      αποδεικνύονται με \textbf{Μαθηματική Επαγωγή}.
  \end{enumerate}
\end{rem}

\mydfn{Δυο ακολουθίες, $(a_{n})_{n \in \mathbb{N}}$  και $ (b_{n})_{n \in 
  \mathbb{N}} $ ονομάζονται \textcolor{Col\thechapter}{ίσες}, αν $ a_{n} 
= b_{n}, \; \forall n \in \mathbb{N} $.}

\mydfn{Οι πράξεις μεταξύ ακολουθιών, ορίζονται όπως ακριβώς και για τις 
συναρτήσεις.}


\section{Μονοτονία Ακολουθιών}

\mydfn{Μια ακολουθία $ (a_{n})_{n \in \mathbb{N}} $ λέγεται:
  \begin{enumerate}[i)]
    \item \textcolor{Col\thechapter}{γνησίως αύξουσα} 
      $ \overset{\text{ορ.}}{\Leftrightarrow} a_{n} 
      < a_{n+1}, \; n \in \mathbb{N}$
    \item \textcolor{Col\thechapter}{γνησίως φθίνουσα} 
      $ \overset{\text{ορ.}}{\Leftrightarrow} a_{n} 
      > a_{n+1}, \; n \in \mathbb{N}$
    \item \textcolor{Col\thechapter}{αύξουσα} 
      $ \overset{\text{ορ.}}{\Leftrightarrow} a_{n} \leq 
      a_{n+1}, \forall n \in \mathbb{N}  $.
    \item \textcolor{Col\thechapter}{φθίνουσα} 
      $ \overset{\text{ορ.}}{\Leftrightarrow} a_{n} \geq 
      a_{n+1}, \forall n \in \mathbb{N}  $.
\end{enumerate}}

\begin{rems}
\item {}
  \begin{enumerate}[i)]
    \item $ (a_{n})_{n \in \mathbb{N}} $ γνησίως αύξουσα (φθίνουσα) $ 
      \Rightarrow (a_{n})_{n \in \mathbb{N}} $ αύξουσα (φθίνουσα) 
    \item $ (a_{n})_{n \in \mathbb{N}} $ γνησίως φθίνουσα  $ 
      \Rightarrow (a_{n})_{n \in \mathbb{N}} $ ανω φραγμένη, με 
      α.φ. το $ a_{1} $  
    \item $ (a_{n})_{n \in \mathbb{N}} $ γνησίως αύξουσα  $ 
      \Rightarrow (a_{n})_{n \in \mathbb{N}} $ κάτω φραγμένη, με 
      κ.φ. το $ a_{1} $  
  \end{enumerate}
\end{rems}

\begin{dfn}\label{dfn:isodmono}(Ισοδύναμος ορισμός της μονοτονίας)
\item {}
  Αν μια ακολουθία $ (a_{n})_{n \in \mathbb{N}} $ \textcolor{Col1}
  {διατηρεί πρόσημο}
  $ \forall n \in \mathbb{N} $, τότε λέγεται:
  \begin{enumerate}[i)]
    \item \textcolor{Col\thechapter}{γνησίως αύξουσα} 
      $ \overset{\text{ορ.}}{\Leftrightarrow} 
      \frac{a_{n+1}}{a_{n}} > 1, \; \forall n \in \mathbb{N}$
    \item \textcolor{Col\thechapter}{γνησίως φθίνουσα} 
      $ \overset{\text{ορ.}}{\Leftrightarrow} 
      \frac{a_{n+1}}{a_{n}} < 1, \; \forall n \in \mathbb{N}$
    \item  \textcolor{Col\thechapter}{αύξουσα} 
      $ \overset{\text{ορ.}}{\Leftrightarrow} 
      \frac{a_{n+1}}{a_{n}} \geq 1, \; \forall n \in \mathbb{N} $
    \item  \textcolor{Col\thechapter}{φθίνουσα}
      $ \overset{\text{ορ.}}{\Leftrightarrow} 
      \frac{a_{n+1}}{a_{n}} \leq 1, \; \forall n \in \mathbb{N} $
  \end{enumerate}
\end{dfn}

\section{Μεθοδολογία εύρεσης μονοτονίας μιας ακολουθίας}
\begin{itemize}
  \item Σχηματίζουμε τη διαφορά $ a_{n+1} - a_n $ και ελέγχουμε το 
    πρόσημό της. Αν $ a_{n+1}-a_{n}>0, \; (<0), \; \forall n \in 
    \mathbb{N} $ τότε $ (a_{n})_{n \in \mathbb{N}}$ γνησίως 
    αύξουσα (γνησίως φθίνουσα). Αν για τουλάχιστον ένα 
    $ n \in \mathbb{N} $, στις παραπάνω ανισότητες, έχω ισότητα, 
    τότε  $ (a_{n})_{n \in \mathbb{N}} $ είναι αύξουσα (φθίνουσα).
  \item Αν οι όροι της ακολουϑίας διατηρούν πρόσημο, $ \; \forall n \in
    \mathbb{N} $ τότε συγκρίνουμε το πηλίκο δυο διαδοχικών όρων της 
    ακολουθίας με τη μονάδα, και βγάζουμε τα συμπεράσματά μας 
    σύμφωνα με τον ορισμό~\ref{dfn:isodmono}
  \item Αν η ακολουθία δίνεται με μη-αναδρομικό τύπο, και είναι 
    (αρκετά) σύνθετη, τότε μετατρέπω την ακολουθία στην αντίστοιχη 
    συνάρτηση και μελετάμε τη μονοτονία της αντίστοιχης συνάρτησης, 
    συνήθως με τη βοήθεια της παραγώγου.
  \item Αν η $ (a_{n})_{n \in \mathbb{N}} $ δίνεται με αναδρομικό 
    τύπο $ (a_{n+1}= f(a_{n}), \; \forall n \in \mathbb{N}) $ τότε 
    συνήθως η απόδειξή της γίνεται με Μαθηματική Επαγωγή.
\end{itemize}

\begin{examples}
\item {}
  \begin{enumerate}[i)]
    \item Η $ a_{n} = 2n-1, \; \forall n \in \mathbb{N} $ είναι γνησίως 
      αύξουσα. Πράγματι,

      \twocolumnsides{%
        \begin{description}
          \item[Α᾽ τρόπος]
            \begin{align*}
              n+1 &\geq n, \; \forall n \in \mathbb{N} \\
              2(n+1) &\geq 2n, \; \forall n \in \mathbb{N} \\
              2(n+1) -1 &\geq 2n-1, \; \forall n \in 
              \mathbb{N} \\
              a_{n+1} &\geq a_{n}, \; \forall n \in \mathbb{N}
            \end{align*}
        \end{description}

        }{%
        \begin{description}
          \item[Β᾽ τρόπος] 
            \begin{align*}
              a_{n+1}-a_{n} &= 2(n+1)-1 - (2n-1) \\
                            &= 2 >0, \; \forall n \in 
                            \mathbb{N} 
            \end{align*}
        \end{description}

      }

    \item Η $ a_{n} = \frac{(n-1)!}{n^{n}}, \; \forall n \in 
      \mathbb{N} $ 
      είναι γνησίως φθίνουσα. Πράγματι, επειδή όλοι οι όροι της 
      ακολουθίας 
      είναι θετικοί, επομένως διατηρεί πρόσημο, έχουμε:
      \[
        \frac{a_{n+1}}{a_n} =
        \frac{\frac{(n+1-1)!}{(n+1)^{n+1}}}{\frac{(n-1)!}{n^{n}}} 
        =  \frac{n^{n}\cdot n!}{(n+1)^{n+1}\cdot (n-1)!} =
        \frac{n^{n+1}}{(n+1)^{n+1}} = \left(\frac{n}{n+1} 
        \right)^{n+1} < 1, \; \forall n
        \in \mathbb{N} 
      \] 

    \item Η $ a_{n}= \frac{4^{n}}{n^{2}}, \; \forall n \in \mathbb{N} $ 
      είναι αύξουσα. Πράγματι, επειδή όλοι οι όροι της ακολουθίας 
      είναι θετικοί, επομένως διατηρεί πρόσημο, έχουμε:
      \[
        \frac{a_{n+1}}{a_{n}} 
        = \frac{\frac{4^{n+1}}{(n+1)^{2}}}{\frac{4^{n}}{n^{2}}} 
        = \frac{4^{n+1}\cdot n^{2}}{4^{n}\cdot (n+1)^{2}} 
        = \frac{4n^{2}}{(n+1)^{2}}
        = \left( \frac{2n}{n+1} \right)^{2} \geq 1, 
        \; \forall n \in \mathbb{N} 
      \]
      Η ακολουθία δεν είναι γνησίως αύξουσα, γιατί $ 
      a_{1}= a_{2}=4$.

    \item Η $ a_{n+1}=2 - \frac{1}{a_{n}}, \; \forall n \in \mathbb{N}
      $ με $ a_{1} = 2 $ είναι γνησίως φθίνουσα. Πράγματι, 
      \begin{itemize}
        \item Για $ n=1 $, έχω: $ a_{1}= 2 >
          \frac{3}{2} = 2 - \frac{1}{2} = a_{2}$, ισχύει.
        \item Έστω ότι ισχύει για $n$, δηλ.
          \inlineequation[eq:epag]{a_{n+1}<a_{n}}.
        \item Θ.δ.ο. ισχύει για $ n+1 $. Πράγματι, 
          \[
            a_{(n+1)+1}=2- \frac{1}{a_{n+1}}
            \overset{\eqref{eq:epag}}{<} 2 - 
            \frac{1}{a_{n}} = a_{n+1}
          \] 
      \end{itemize}

    \item Να δείξετε ότι ακολουθία $ a_{n} = (-1)^{n} \frac{1}{n^{2}}, 
      \; \forall n \in \mathbb{N} $ 
      δεν είναι μονότονη.

      \begin{proof}
      \item {}
        Θα δείξουμε ότι η ακολουθία δεν διατηρεί πρόσημο. Πράγματι, 
        $ \forall n \in \mathbb{N} $
        \begin{align*}
          a_{n+1}- a_{n} = \frac{(-1)^{n+1}}{(n+1)^{2}} - 
          \frac{(-1)^{n}} {n^{2}} 
                        &= \frac{(-1)^{n+1}}{(n+1)^{2}} + 
                        \frac{(-1)^{n+1}}{n^{2}}= \\
                        &= (-1)^{n+1}\Biggl[\underbrace{\frac{1}{(n+1)^{2}} + 
                            \frac{1}{n^{2}}}_{b_{n} > 0, \; \forall n \in 
                        \mathbb{N}}\Biggr] 
                        = \begin{cases}
                          -b_{n}, & n \; \text{περιττός} \\
                          b_{n}, & n \; \text{άρτιος} 
                        \end{cases}
        \end{align*} 
      \end{proof}
  \end{enumerate}
\end{examples}

\myprop{Η ακολουθία $ \left(1+ \frac{1}{n}\right)^{n} $ είναι γνησίως αύξουσα.}

\begin{proof}
\item {}
  Έστω $ n_{0} \in \mathbb{N} $. Τότε 
  \begin{align*}
    \left(1+ \frac{1}{n_{0}+1} \right)^{n_{0}+1} > 
    \left(1+ \frac{1}{n_{0}} \right)^{n_{0}} 
        &\Leftrightarrow \left(1+ \frac{1}{n_{0}+1} \right)^{n_{0}} 
        \cdot \left(1 + \frac{1}{n_{0} +1} \right) > \left(1+ \frac{1}{n_{0}+1} 
        \right) \\
        & \Leftrightarrow \frac{(n_{0}+2)^{n_{0}}}{(n_{0}+1)^{n_{0}}} \cdot 
        \frac{n_{0}+2}{n_{0}+1} > \frac{(n_{0}+1)^{n_{0}}}{n_{0}^{n_{0}}} \\
        & \Leftrightarrow \frac{n_{0}^{n_{0}}(n_{0}+2)^{n_{0}}}{(n_{0}+1)^{2n0}} > 
        \frac{n_{0}+1}{n_{0}+2} \\
        & \Leftrightarrow \left(\frac{n_{0}(n_{0}+2)}{(n_{0}+1)^{2}}\right)^{n_{0}} > 
        \frac{n_{0}+2-2+1}{n_{0}+2} \\
        & \Leftrightarrow \left(\frac{n_{0}^{2}+2 n_{0}+1-1}{(n_{0}+1)^{2}}\right)
        ^{n_{0}} > 1-\frac{1}{n_{0}+2} \\
        & \Leftrightarrow \left(\frac{(n_{0}+1)^{2}-1}{(n_{0}+1)^{2}} \right)^{n_{0}}
        > 1 - \frac{1}{n_{0}+2} \\
        & \Leftrightarrow \left(1 - \frac{1}{(n_{0}+1)^{2}} \right)^{n_{0} } > 1 - 
        \frac{1}{n_{0}+2} \\
  \end{align*} 
  Άρα αρκεί να δείξουμε αυτή την ανισότητα. Πράγματι,
  για $ a = - \frac{1}{(n_{0}+1)^{2}} > -1 $ απο ανισότητα Bernoulli: 
  \begin{align*}
    \left(1- \frac{1}{(n_{0}+1)^{2}}\right)^{n_{0}} > 1 - n_{0}\cdot 
    \frac{1}{(n_{0}+1)^{2}} > 1 - \frac{1}{n_{0}+2} 
  \end{align*}
  όπου χρησιμοποιήσαμε το γεγονός ότι $ \frac{n_{0}}{(n_{0}+1)^{2}} < 
  \frac{1}{n_{0}+2}  $, το οποίο ισχύει, γιατί 
  \[
    \frac{n_{0}}{(n_{0}+1)^{2}} < \frac{1}{n_{0}+2} 
    \Leftrightarrow n_{0}(n_{0}+2) < (n_{0}+1)^{2} 
    \Leftrightarrow n_{0}^{2}+2 n_{0} < n_{0}^{2} + 2 n_{0}+1 
    \Leftrightarrow 0 < 1
  \]
\end{proof}

\section{Φραγμένες Ακολουθίες}

\mydfn{Μια ακολουθία $ (a_{n})_{n \in \mathbb{N}} $ ονομάζεται:
  \begin{enumerate}[i)]
    \item \textcolor{Col\thechapter}{άνω φραγμένη} 
      $ \overset{\text{ορ.}}{\Leftrightarrow} \exists M \in 
      \mathbb{R} \; : \; a_{n} \leq M, \; \forall n \in \mathbb{N}$.
    \item \textcolor{Col\thechapter}{κάτω φραγμένη} 
      $ \overset{\text{ορ.}}{\Leftrightarrow} \exists m \in 
      \mathbb{R} \; : \; m \leq a_{n}, \; \forall n \in \mathbb{N}  $
    \item \textcolor{Col\thechapter}{φραγμένη} 
      $ \overset{\text{ορ.}}{\Leftrightarrow} $ είναι άνω και κάτω φραγμένη.
\end{enumerate}}

\myprop{$ (a_{n})_{n \in \mathbb{N}} $ φραγμένη $ \Leftrightarrow \exists M>0, 
  \; M \in \mathbb{R} \; : \; \abs{a_{n}} \leq M, \; \forall n \in \mathbb{N} $ 
(\textbf{απολύτως φραγμένη}).}
\begin{proof}
\item {}
  \begin{description}
    \item [($ \Rightarrow $)] Έστω $ (a_{n})_{n \in \mathbb{N}} $ φραγμένη. Τότε η 
      $ (a_{n})_{n \in \mathbb{N}} $ είναι άνω και κάτω φραγμένη, δηλαδή υπάρχουν 
      $ m,M \in \mathbb{R} $ ώστε $ m \leq a_{n} \leq M, \; \forall n \in \mathbb{N} $.
      Θεωρούμε $ \mu = \max \{ \abs{m} , \abs{M} \} $. Τότε, η παραπάνω διπλή 
      ανισότητα, λαμβάνοντας υπόψιν και τις γνωστές ιδιότητες της απόλυτης τιμής, 
      γίνεται
      \[
        - \mu \leq - \abs{m} \leq m \leq a_{n} \leq M \leq \abs{M} \leq \mu , 
        \quad \forall n \in \mathbb{N}
      \]
      Επομένως $ \exists \mu > 0 $ ώστε 
      $ \abs{a_{n}} \leq \mu, \; \forall n \in \mathbb{N} $.
    \item [($ \Leftarrow$)] Προφανώς, αν $( a_{n})_{n \in \mathbb{N}} $ είναι 
      απολύτως φραγμένη, τότε $ \exists M>0 $ ωστε $ -M \leq a_{n} \leq M, \; \forall n
      \in \mathbb{N} $, και άρα η ακολουθία είναι φραγμένη. \qedhere
  \end{description}
\end{proof}

\begin{rem}
  Πολλές φορές στη βιβλιογραφία, η παραπάνω πρόταση, δίνεται και ως ορισμός της 
  φραγμένης ακολουθίας.
\end{rem}

\begin{examples}
\item {}  
  \begin{enumerate}[i)]
    \item Η ακολουθία $ a_{n}= \frac{1}{n}, \; \forall n \in \mathbb{N} $, είναι 
      φραγμένη.

      Πράγματι, $ 0 \leq \frac{1}{n} \leq 1, \; \forall n \in 
      \mathbb{N} $. 

      Επίσης, $ \abs{\frac{1}{n}} = \frac{1}{n} \leq 1 $, άρα και 
      απολύτως φραγμένη.
    \item Η ακολουθία $ a_{n}=(-1)^{n} \frac{1}{n}, \; \forall n \in \mathbb{N} $ 
      είναι απολύτως φραγμένη. Πράγματι,
      \[
        \abs{a_{n}} = \abs{(-1)^{n} \frac{1}{n}} = \abs{(-1)^{n}} 
        \cdot \abs{\frac{1}{n}} = \abs{-1}^{n} \cdot \frac{1}{n}
        = 1 \cdot \frac{1}{n} = \frac{1}{n} \leq 1, \; \forall n 
        \in \mathbb{N}
      \] 

    \item Η ακολουθία $ a_{n}= \frac{(n-1)!}{n^{n}}, \; \forall n \in \mathbb{N} $
      είναι φραγμένη. Πράγματι, 

      $ a_{n} > 0, \; \forall n \in 
      \mathbb{N}$, άρα 0 κ.φ. της $( a_{n})_{n \in 
      \mathbb{N}} $. 
      Επίσης 
      \[
        a_{n}= \frac{(n-1)!}{n^{n}} = \frac{1 \cdot 2 
          \cdots (n-1)}{n^{n}} < \frac{\smash{\overbrace{n 
              \cdot n \cdots n} ^{n-1 \; 
        \text{φορές}}}}{n^{n}} = \frac{n^{n-1}}{n^{n}} =
        \frac{1}{n} \leq 1, \; \forall n \in \mathbb{N},
      \]
      άρα το 1 είναι α.φ. της $(a_{n})_{n \in \mathbb{N}}$. 

    \item Η ακολουθία $ a_{n}= 1 + \left(- \frac{1}{2} \right) + \left(- 
        \frac{1}{2}\right)^{2} + \cdots + \left(-\frac{1}{2} 
      \right) ^{n}, 
      \; \forall n \in \mathbb{N} $ είναι απολύτως φραγμένη. Πράγματι,

      Πρόκειται για το άθροισμα των $ n $ πρώτων όρων \textbf{γεωμετρικής προοδου} με 
      λόγο $ -\frac{1}{2} $. Έτσι
      \[ a_{n} = 1 + \left(- \frac{1}{2}\right) + \left(- \frac{1}{2} 
        \right)^{2} + \cdots + \left(- \frac{1}{2} \right)^{n} = 
        \frac{1 - (- \frac{1}{2} )^{n}}{1 - (- \frac{1}{2})} = 
      \frac{2}{3} \left[1 - \left(- \frac{1}{2} \right)^{n}\right] \]
      Επομένως
      \[
        \abs{a_{n}} = \abs{\frac{2}{3} \left[1-\left(- \frac{1}{2} \right)^{n}
            \right]} = \frac{2}{3} \abs{1 - \left(- 
        \frac{1}{2}\right)^{n}} \leq 
        \frac{2}{3} \left(1 + \abs{-\frac{1}{2} }^{n} \right) = 
        \frac{2}{3} \left(1+ \frac{1}{2^{n}}\right) < \frac{2}{3}
        (1+1) = \frac{4}{3} 
      \] 

    \item Η ακολουθία $ a_{n}= 2n+5, \; \forall n \in \mathbb{N} $ είναι κάτω 
      φραγμένη.
      Πράγματι, $ 7 \leq 2n+5, \; \forall n \in \mathbb{N} $, άρα το 
      7 είναι κ.φ. της $ (a_{n} )_{n \in \mathbb{N}} $.

      Προσοχή, η ακολουθία $ a_{n}= 2n+5, \; \forall n \in \mathbb{N} $ δεν είναι 
      άνω φραγμένη, γιατί αν υποθέσουμε ότι είναι, τότε $ \exists M>0 $ ώστε 
      $ a_{n} \leq M \Leftrightarrow 2n+5 \leq M \Leftrightarrow 2n \leq M-5
      \Leftrightarrow n \leq \frac{M-5}{2}, \; \forall n \in \mathbb{N} $, άτοπο, 
      γιατί το $ \mathbb{N} $ δεν είναι άνω φραγμένο.

    \item Η ακολουθία $ a_{1}=2, \; a_{n+1}=2 - \frac{1}{a_{n}}, \forall n \in 
      \mathbb{N}$
      είναι φραγμένη. Πράγματι, επειδή η ακολουθία είναι αναδρομική, 
      με επαγωγή, έχουμε:
      \begin{itemize}
        \item Για $ n=1 $, $ a_{1}=2>1 $, ισχύει. 
        \item Έστω ότι ισχύει για $n$, δηλ. \inlineequation[eq:
          anadepag1]{a_{n}>1}.
        \item Θ.δ.ο. ισχύει και για $ n+1 $. Πράγματι, από τη 
          σχέση~\eqref{eq: anadepag1}, έχουμε
          \[
            a_{n}>1 \Rightarrow \frac{1}{a_{n}} 
            < 1 \Rightarrow - \frac{1}{a_{n}} > 
            -1 \Rightarrow 2 - \frac{1}{a_{n}} 
            > 2-1 \Rightarrow a_{n+1} > 1.
          \] 
      \end{itemize}

    \item Να δείξετε ότι η ακολουθία $ a_{n} = 
      \frac{n^{2}+1}{3n+ \sin^{3}{n}} $ δεν είναι άνω φραγμένη. 

      \begin{proof}
      \item {}
        Έστω ότι η $ a_{n} = \frac{n^{2}+1}{3n+ \sin^{3}{n}} $ είναι άνω 
        φραγμένη. Τότε επειδή είναι και κάτω φραγμένη, από το 0, έχουμε ότι
        $ \exists M>0 \; : \; \abs{a_{n}} \leq M, \; \forall n \in 
        \mathbb{N} $, δηλαδή
        \begin{align*}
          \abs{\frac{n^{2}+1}{3n + \sin^{3}{n}}} \leq M \Leftrightarrow 
          \frac{n^{2}+1}{\abs{3n + \sin^{3}{n}}} \leq M \Leftrightarrow 
          n^{2}+1
               &\leq M \cdot \abs{3n + \sin^{3}{n}} \\ 
               &\leq 3nM + M \cdot \abs{\sin{n}} ^{3} \\
               &\leq 3n M +M, \; \forall n \in \mathbb{N}
        \end{align*} 
        Δηλαδή, 
        \[
          n^{2}-3nM \leq M-1 \Rightarrow n^{2}-3nM < M, \; \forall n \in 
          \mathbb{N}
        \] 
        και συμπληρώνοντας το τετράγωνο έχουμε
        \[
          \left(n - \frac{3}{2} M\right)^{2} < M + \frac{9}{4} M^{2}
          \Rightarrow \abs{n - \frac{3}{2} M} < 
          \sqrt{M + \frac{9}{4} M^{2}}, \; \forall n \in \mathbb{N}
        \]
        οπότε
        \[
          - \sqrt{M + \frac{9}{4} M^{2}}< \underbrace{n- \frac{3}{2} M 
          < \sqrt{M + \frac{9}{4} M^{2}}}, \; \forall n \in \mathbb{N} 
          \Rightarrow n < \frac{3}{2} M + \sqrt{M + \frac{9}{4} M^{2}}, 
          \; \forall n \in \mathbb{N} 
        \] 
        άτοπο, γιατί $ \mathbb{N} $ όχι άνω φραγμένο.
      \end{proof}
  \end{enumerate}
\end{examples}

\myprop{Η ακολουθία $ a_{n} = \left(1 + \frac{1}{n} \right)^{n} $ είναι φραγμένη.}

\begin{proof}
\item {}
  Θεωρούμε την (βοηθητική) ακολουθία $ b_{n} = \left(1+ \frac{1}{n} \right)^{n+1}, 
  \; \forall n \in \mathbb{N}$. Θα δείξουμε ότι η $ (b_{n})_{n \in \mathbb{N}} $ 
  είναι \textbf{γνησίως φθίνουσα}. Πράγματι, 
  \begin{align*}
    b_{n+1}< b_{n} 
        &\Leftrightarrow \left(1+ \frac{1}{n_{0}+1} \right)^{n_{0}+2} < \left(1 + 
        \frac{1}{n_{0}} \right)^{n_{0} +1} \\
        & \Leftrightarrow \left(1+ \frac{1}{n_{0}+1} \right)^{n_{0}+1} \cdot 
        \left(1 + \frac{1}{n_{0}+1}\right) < \left(1+ \frac{1}{n_{0}} \right)^{n_{0}+1}
        \\
        & \Leftrightarrow \left(\frac{n_{0}+2}{n_{0}+1} \right)^{n_{0}+1} 
        \cdot \left(1 + \frac{1}{n_{0}+1}\right) < \left(\frac{n_{0}+1}{n_{0}} 
        \right)^{n_{0}+1} \\
        & \Leftrightarrow \left(1 + \frac{1}{n_{0}+1} \right) < 
        \left(\frac{(n_{0}+1)^{2}}{n_{0}(n_{0} +2)} \right)^{n_{0}+1} \\
        & \Leftrightarrow \left(1 + \frac{1}{n_{0}+1} \right) < \left(\frac{n_{0}^{2}+
        2 n_{0}+1}{n_{0}^{2}+2 n_{0}} \right)^{n_{0}+1} \\
        & \Leftrightarrow \left(1 + \frac{1}{n_{0}+1} \right) < 
        \left(1+ \frac{1}{n_{0}(n_{0}+2)} \right)^{n_{0}+1}
  \end{align*} 

  Για $ a= \frac{1}{n_{0}(n_{0}+2)} \geq 0 $, η ανισότητα Bernoulli, δίνει:
  \[
    \left(1+ \frac{1}{n_{0}(n_{0}+2)} \right)^{n_{0}+1} > 1 + (n_{0}+1)\cdot 
    \frac{1}{n_{0}(n_{0}+2)} = 1 + \frac{n_{0}+1}{n_{0}^{2}+ 2 n_{0}} 
  \] 
  Άρα αρκεί να δείξουμε ότι 
  \begin{align*}
    1 + \frac{n_{0}+1}{n_{0}^{2}+ 2 n_{0}} &> 1 + \frac{1}{n_{0}+1} 
    \Leftrightarrow  \\
    (n_{0}+1)^{2} &> n_{0}^{2} + 2 n_{0} \Leftrightarrow \\
    1 &> 0 \quad \text{που ισχύει.}
  \end{align*}

  Άρα τελικά,
  \[
    b_{n} = \left(1+ \frac{1}{n} \right)^{n+1} = \left(1+ \frac{1}{n} \right)^{n} 
    \cdot \left(1 + \frac{1}{n} \right) = a_{n}\cdot \left(1+ \frac{1}{n} \right) 
    > a_{n}, \; \forall n \in \mathbb{N} 
  \]
  Οπότε, λαμβάνοντας υπόψιν ότι η ακολουθία 
  $ a_{n} = \left(1 + \frac{1}{n}\right)^{n} $ είναι \textbf{γνησίως αύξουσα}, έχουμε
  \begin{align*}
    a_{1} \leq a_{n} &< b_{n} \leq b_{1}, \; \forall n \in \mathbb{N} 
    \Leftrightarrow \\
    2 \leq a_{n} &\leq 4, \; \forall n \in \mathbb{N}
  \end{align*}
  άρα η $ a_{n} = \left(1 + \frac{1}{n} \right)^{n} $ είναι φραγμένη.
\end{proof}


\section{Σύγκλιση Ακολουθιας}

\mydfn{\textcolor{Col\thechapter}{Περιοχή} ένος πραγματικού αριθμού $ x_{0} $ 
ονομάζεται κάθε ανοιχτό διάστημα $(a,b)$ που περιέχει το $ x_{0} $. }

\begin{rem}
\item {}
  \begin{enumerate}[i)]
    \item 
      Αν $ \varepsilon > 0 $, τότε περιοχές του $ x_{0} $ της μορφής 
      $ (x_{0}- \varepsilon , x_{0} + \varepsilon) $ έχουν 
      \textcolor{Col\thechapter}{ακτίνα} $ \varepsilon $
      και \textcolor{Col\thechapter}{κέντρο} το $ x_{0} $. 

    \item $ x \in (x_{0}- \varepsilon, x_{0} + \varepsilon) 
      \Leftrightarrow x_{0}- \varepsilon < x < x_{0}+ \varepsilon 
      \Leftrightarrow - \varepsilon < x - x_{0} < \varepsilon 
      \Leftrightarrow \abs{x- x_{0}} < \varepsilon  $ 
  \end{enumerate}
\end{rem}

\mydfn{Μια ακολουθία $ (a_{n})_{n \in \mathbb{N}} $ \textcolor{Col\thechapter}
  {συγκλίνει} στον πραγματικό 
  αριθμό $ a \in \mathbb{R} $ (έχει όριο το $ a \in \mathbb{R} $ ή 
  τείνει στο $ a \in \mathbb{R} $), και συμβολίζουμε 
  $ \lim\limits_{n\to \infty} a_{n}=a $ (ή $ a_{n} \xrightarrow{n \to 
  \infty} a $) αν 
  \[
    \forall \varepsilon >0, \; \exists n_{0} \in \mathbb{N} \; : 
    \; \forall n \in \mathbb{N} \; \text{με} \; n \geq n_{0} 
    \Rightarrow \abs{a_{n}-a} < \varepsilon
\] }

\begin{rem}
  Γενικά το $ n_{0} $ εξαρτάται από το $ \varepsilon $ και ισχύει ότι
  $ n_{0} = n_{0}(\varepsilon) $.
\end{rem}

\mydfn{Μια ακολουθία λέμε ότι \textcolor{Col\thechapter}{αποκλίνει} 
  (ή είναι αποκλίνουσα), αν 
  \begin{itemize}
    \item δεν υπάρχει το όριό της, για παράδειγμα αν η ακολουθία 
      ταλαντεύεται
    \item απειρίζεται, θετικά ή αρνητικά.
\end{itemize}}

\begin{rem}[Άρνηση ορισμού του ορίου]
  \[
    \lim_{n \to \infty} a_{n} \neq a \Leftrightarrow \exists 
    \varepsilon >0, \; \forall n \in \mathbb{N}, \; \exists n_{0} 
    \geq n \quad \abs{a_{n_{0}}-a} \geq \varepsilon 
  \] 
\end{rem}

\mydfn{Η ακολουθία $ (a_{n})_{n \in \mathbb{N}}$ λέγεται
  \textcolor{Col\thechapter}{μηδενική ακολουθία}
αν $ \lim\limits_{n\to \infty} = 0 $}

\begin{examples}
\item {}
  \begin{enumerate}[i)]
    \item $ \lim\limits_{n \to \infty} \frac{1}{n} = 0 $.
      \begin{proof}
      \item {}
        \begin{description}
          \item[Δοκιμή:] $ \abs{\frac{1}{n} -0} < \varepsilon
            \Leftrightarrow \abs{\frac{1}{n}} < \varepsilon 
            \Leftrightarrow \frac{1}{n} < \varepsilon 
            \Leftrightarrow n > \frac{1}{\varepsilon}$
        \end{description}
        \begin{description}
          \item[Α᾽ Τρόπος:] 
            Έστω $ \varepsilon >0 $. Τότε $ \exists n_{0} \in
            \mathbb{N} $ με \inlineequation[eq:1n1]{n_{0} > 
            \frac{1}{\varepsilon}} (Αρχ. Ιδιοτ.) τέτοιο 
            ώστε \inlineequation[eq:1n2]{\forall n \geq n_{0}}
            \[
              \abs{\frac{1}{n} -0} = \abs{\frac{1}{n}} =
              \frac{1}{n} \overset{\eqref{eq:1n2}}{\leq}
              \frac{1}{n_{0}} \overset{\eqref{eq:1n1}}{<} 
              \frac{1}{\frac{1}{\varepsilon}} = \varepsilon 
            \]

          \item [Β᾽ Τρόπος:]
            Έστω $ \varepsilon >0 $. Τότε $ \exists n_{0} \in
            \mathbb{N} $ με \inlineequation[eq:1n3]
            {\frac{1}{n_{0}} < \varepsilon} (Αρχ. Ιδιοτ.) 
            τέτοιο ώστε \inlineequation[eq:1n4]{\forall n 
            \geq n_{0}}
            \[
              \abs{\frac{1}{n} -0} = \abs{\frac{1}{n}} =
              \frac{1}{n} \overset{\eqref{eq:1n4}}{\leq}
              \frac{1}{n_{0}} \overset{\eqref{eq:1n3}}{<} 
              \varepsilon 
            \]
        \end{description}
      \end{proof}

    \item $ \lim\limits_{n \to \infty} \frac{1}{n^{4}} = 0 $. 
      \begin{proof}
      \item {}
        \begin{description}
          \item[Δοκιμή:]$ \abs{\frac{1}{n^{4}} - 0} < \varepsilon 
            \Leftrightarrow \abs{\frac{1}{n^{4}}} < \varepsilon 
            \Leftrightarrow \frac{1}{n^{4}} < \varepsilon
            \Leftrightarrow n^{4} > \frac{1}{\varepsilon}
            \Leftrightarrow n > \sqrt[4]{\frac{1}{\varepsilon}}$
        \end{description}
        Έστω $ \varepsilon >0 $. Τότε $ \exists n_{0}  \in 
        \mathbb{N}$ με \inlineequation[eq:limexn41]{n_{0} >
        \sqrt[4]{\frac{1}{\varepsilon}}} (Αρχ. Ιδιοτ.)
        τέτοιο ώστε \inlineequation[eq:limexn42]{\forall n 
        \geq n_{0}} 
        \[
          \abs{\frac{1}{n^{4}} - 0 } = \abs{\frac{1}{n^{4}}} 
          = \frac{1}{n^{4}} \overset{\eqref{eq:limexn42}}\leq 
          \frac{1}{n_{0}^{4}} \overset{\eqref{eq:limexn41}}{<}
          \frac{1}{\left(\sqrt[4]{\frac{1}{\varepsilon}
          }\right)^{4}} = \frac{1}{\frac{1}{\varepsilon}} =  
          \varepsilon
        \] 
      \end{proof}

    \item $ \lim\limits_{n \to \infty} \frac{1}{\sqrt{n}} = 0$.
      \begin{proof}
      \item {}
        \begin{description}
          \item[Δοκιμή:] $ \abs{\frac{1}{\sqrt{n}}-0 } 
            < \varepsilon 
            \Leftrightarrow \abs{\frac{1}{\sqrt{n}} } < 
            \varepsilon 
            \Leftrightarrow \frac{1}{\sqrt{n}} < 
            \varepsilon \Leftrightarrow \sqrt{n} >
            \frac{1}{\varepsilon} \Leftrightarrow n >
            \left(\frac{1}{\varepsilon}\right)^{2}
            $
        \end{description}
        Έστω $ \varepsilon > 0 $. Τότε $ \exists n_{0} \in 
        \mathbb{N} $
        με \inlineequation[eq:limexsqrt1]{n_{0} > \left(\frac{1}
        {\varepsilon}\right)^{2}} (Αρχ. Ιδιοτ.) τέτοιο ώστε 
        \inlineequation[eq:limexsqrt2]{\forall n \geq n_{0}}
        \[
          \abs{\frac{1}{\sqrt{n}} -0} = \abs{\frac{1}{\sqrt{n}}} =
          \frac{1}{\sqrt{n}} \overset{\eqref{eq:limexsqrt2}}{\leq}
          \frac{1}{\sqrt{n_{0}}} \overset{\eqref{eq:limexsqrt1}}
          {<} \frac{1}{\sqrt{\left(\frac{1}{\varepsilon}\right)
          ^{2}}} = \frac{1}{\frac{1}{\varepsilon}} = \varepsilon
        \] 
      \end{proof}

    \item $ \lim_{n \to \infty} \frac{\sin{n}}{n} = 0 $

      \begin{proof}
      \item {}
        \begin{description}
          \item[Δοκιμή:] $ \abs{\frac{\sin{n}}{n} - 0} < 
            \varepsilon \Leftrightarrow \abs{\frac{\sin{n}}{n}}
            < \varepsilon \Leftrightarrow 
            \inlineequation[eq:limexsin1]{\frac{\abs{\sin{n}}}
            {n} < \varepsilon} $

            Όμως
            \inlineequation[eq:limexsin2]{\frac{\abs{\sin{n}}}
              {n} \leq \frac{1}{n}, \; \forall n \in 
            \mathbb{N}}

            Οπότε από τις σχέσεις \eqref{eq:limexsin1} και 
            \eqref{eq:limexsin2} αρκεί $ \frac{1}{n} < 
            \varepsilon \Leftrightarrow n > \frac{1}{
            \varepsilon} $
        \end{description}

        Έστω $ \varepsilon >0 $. Τότε $ \exists n_{0} \in \mathbb{N}
        $ με \inlineequation[eq:limexsin3]{n_{0} >
        \frac{1}{\varepsilon}} τέτοιο ώστε
        \inlineequation[eq:limexsin4]{\forall n \geq n_{0}}
        \[
          \abs{\frac{\sin{n}}{n} - 0} =  \abs{\frac{\sin{n}}{n}} =
          \frac{\abs{\sin{n}}}{n} \leq \frac{1}{n}
          \overset{\eqref{eq:limexsin4}}{\leq}  \frac{1}{n_{0}}
          \overset{\eqref{eq:limexsin3}}{<}
          \frac{1}{\frac{1}{\varepsilon}
          } = \varepsilon 
        \] 
      \end{proof}
  \end{enumerate}
\end{examples}

\myprop{Η ακολουθία $ \{ (-1)^{n} \}_{n \in \mathbb{N}} $ δεν συγκλίνει.}

\begin{proof}
\item {}
  Έστω ότι η $ a_{n}= (-1)^{n} $ συγκλίνει και έστω $ \lim_{n \to +
  \infty}(-1)^{n} = a $. 

  Τότε από τον ορισμό του ορίου, έχουμε ότι για 
  \[ 
    \varepsilon = 1, \; \exists n_{0} \in \mathbb{N} \; : \; \forall 
    n \geq n_{0} \quad \abs{(-1)^{n}-a} < 1 \Leftrightarrow -1 < 
    (-1)^{n} -a < 1 \Leftrightarrow a-1 < (-1)^{n} < a+1
  \]

  Όμως για $ n_{1} \geq n_{0} $ με $ n_{1} $ άρτιος, έχουμε:

  \[
    a-1 <  (-1)^{n_{1}} < a+1 \Leftrightarrow a-1 < 
    \underbrace{1 < a+1}_{a>0} 
  \] 

  Όμως για $ n_{2} \geq n_{0} $ με $ n_{2} $ περιττός, έχουμε:

  \[
    a-1 <  (-1)^{n_{2}} < a+1 \Leftrightarrow 
    \underbrace{a-1 < -1}_{a<0} < a+1
  \] 

  Οπότε καταλήγουμε σε άτοπο.
\end{proof}


\myprop{Έστω $ (a_{n})_{n \in \mathbb{N}} $ ακολουθία πραγματικών αριθμών και 
  $ l \in \mathbb{R} $. Τότε ισχύου οι ισοδυναμίες:

  \[ \lim_{n \to \infty} a_{n} = l \Leftrightarrow \lim_{n \to \infty} (
    a_{n}- l) = 0 
\Leftrightarrow \lim_{n \to \infty} \abs{a_{n}-l} = 0\]}

\begin{proof}
\item {}
  \begin{align*} 
    \lim_{n \to \infty} a_{n}= l & \Leftrightarrow \forall 
    \varepsilon > 0 \; \exists n_{0} \in \mathbb{N} \; : \; 
    \abs{a_{n}-l} < \varepsilon, \forall n \geq n_{0} \\ 
    \lim_{n \to \infty} (a_{n}-l) = 0 & \Leftrightarrow \forall 
    \varepsilon >0 \; \exists n_{0} \in \mathbb{N} \; : \; 
    \abs {(a_{n}-l) - 0} < \varepsilon , \forall n \geq n_{0} \\
    \lim_{n \to \infty} \abs{a_{n}-l}=0 < \varepsilon & \Leftrightarrow 
    \forall \varepsilon > 0 \; \exists n_{0} \in \mathbb{N} 
    \; : \; \abs {\abs{a_{n}- l } - 0} < \varepsilon, \forall n \geq 
    n_{0} 
  \end{align*}

  και ισχύει ότι $ \abs{a_{n}-l} = \abs{(a_{n}-l)-0} = 
  \abs{\abs{a_{n}- l} -0} $
\end{proof}

\mythm{Το όριο μιας ακολουθίας, όταν υπάρχει είναι μοναδικό.}

\begin{proof}
\item {}
  Έστω ότι μια ακολουθία $ (a_{n})_{n \in \mathbb{N}} $ συγκλίνει σε 
  δυο διαφορετικούς αριθμούς, $ a,b $. Έστω $ \varepsilon >0 $, τότε

  \[ 
    \lim_{n \to \infty} a_{n} = a \Leftrightarrow \forall 
    \varepsilon > 0, \; \exists n_{0} \in \mathbb{N} \; : \: 
    \forall n \geq n_{0} \quad \abs{a_{n}- a} \leq \varepsilon 
  \]

  Άρα και για $ \varepsilon = \frac{\varepsilon}{2}, \; \exists n_{1} \in 
  \mathbb{N} \; : \; \forall n \geq n_{1} \quad \abs{a_{n_{1}} - a} 
  < \frac{\varepsilon}{2}  $

  \[ 
    \lim_{n \to \infty} a_{n} = b \Leftrightarrow \forall 
    \varepsilon > 0, \; \exists n_{0} \in \mathbb{N} \; : \: 
    \forall n \geq n_{0} \quad \abs{a_{n}- b} \leq \varepsilon 
  \]

  Άρα και για $ \varepsilon = \frac{\varepsilon}{2}, \; \exists n_{2} \in 
  \mathbb{N} \; : \; \forall n \geq n_{2}  \quad 
  \abs{a_{n_{2}} - b} < \frac{\varepsilon}{2}  $

  Θέτουμε $ n_{0} = \max \{ n_{1}, n_{2} \} $, ώστε να ισχύουν οι 
  κ οι δύο παραπάνω ανισότητες ταυτόχρονα. 

  Επομένως, έχουμε 
  \[
    0 \leq \abs{a-b} = \abs{a- a_{n}+ a_{n}- b} \leq 
    \abs{a- a_{n}} + \abs{a_{n}- b} < \frac{\varepsilon}{2} + 
    \frac{\varepsilon}{2} = \varepsilon 
  \] 

  άρα από την πρόταση~\ref{prop:epsilon_prot} έχουμε ότι 
  $ \abs{a-b} = 0 \Rightarrow a=b $, άτοπο, γιατί $ a \neq b $.
\end{proof}

%TODO idiothtes oriwn

\mythm{Κάθε συγκλίνουσα ακολουθία είναι φραγμένη.}

\begin{proof}
  Έστω $ (a_{n})_{n \in \mathbb{Ν}} $ συγκλίνουσα $ \Rightarrow 
  \exists a \in \mathbb{R} $ τέτοιο ώστε $ \lim_{n \to +\infty} a_{n}
  =a $, άρα 
  \[
    \forall \varepsilon > 0, \; \exists n_{0} \in \mathbb{N} \; 
    : \; \forall n \geq n_{0}\quad \abs{a_{n}-a} < \varepsilon  
  \] 
  αρα και για $ \varepsilon =1>0, \; \exists n_{0} \in \mathbb{N} \; 
  : \; \forall n \geq n_{0} \quad 
  \inlineequation[eq:sygkfrag]{\abs{a_{n}-a} < 1} $. 

  Δηλαδή $ \forall n \geq n_{0} $ έχουμε ότι 
  \[
    \abs{a_{n}} = \abs{a_{n}-a + a} \leq \abs{a_{n} - a} + \abs{a} 
    \overset{\eqref{eq:sygkfrag}}{<} 1 + \abs{a}  
  \] 

  Δηλαδή, καταφέραμε να φράξουμε τους όρους της ακολουθίας με δείκτες 
  $n \geq n_{0} $.

  Τώρα, θέτουμε $ M = \max \{ \abs{a_{1}} , \abs{a_{2}} , \ldots, 
  \abs{a_{n-1}} , 1 + \abs{a}\} $ και έχουμε ότι $ \abs{a_{n}} 
  < M, \; \forall n \in \mathbb{N} $, επομένως η 
  $ (a_{n})_{n \in \mathbb{N}} $ είναι φραγμένη.
\end{proof}

\begin{rem}
  Προσοχή, το αντίστροφο της παραπάνω πρότασης, δεν ισχύει. Πράγματι, 
  για την  ακολουθία $ (a_{n})_{n \in \mathbb{N}} = (-1)^{n}, \; 
  \forall n \in \mathbb{N} $ έχουμε ότι είναι φραγμένη, όχι όμως και 
  συγκλίνουσα.
\end{rem}

\section{Υπακολουθίες}

\mydfn{Έστω $ (a_{n})_{n \in \mathbb{N}} $ ακολουθία, και έστω $ 
  (k_{n})_{n \in \mathbb{N}} $ γνησίως
  αύξουσα ακολουθία φυσικών αριθμών $ (k_{1}<k_{2}<k_{3}<\cdots) $. 
  Τότε η ακολουθία $ b_{n} = (a_{k_{n}}), \; n \in \mathbb{N} $ 
  ονομάζεται \textcolor{Col\thechapter}{υπακολουθία} της 
$ (a_{n})_{n \in \mathbb{N}} $.}

%TODO παραδειγματα υπακολουθιων

\myprop{Αν μια ακολουθία είναι φραγμένη τότε και κάθε υπακολουθία της 
είναι φραγμένη.}

\begin{rem}
\item {}
  \begin{enumerate}[i)]
    \item Ανάλογες προτάσεις ισχύουν και για την περίπτωση όπου η 
      ακολουθία είναι άνω, ή αντίστοιχα κάτω φραγμένη.
    \item 
      Ενδιαφέρον παρουσιάζει το αντιθετοάντίστροφο της παραπάνω 
      πρότασης, όπου λέει ότι αν τουλάχιστον μία υπακολουθία μιας 
      ακολουθίας δεν είναι φραγμένη, τότε κ η ίδια η ακολουθία 
      δεν θα είναι φραγμένη.
  \end{enumerate}
\end{rem}

\myprop{Αν μια ακολουθία έιναι γνησίως αύξουσα (αντίστοιχα γνησίως φθίνουσα) 
  τότε και κάθε υπακολουθία της θα είναι γνησίως αύξουσα (αντίστοιχα 
γνησίως φθίνουσα).}
%TODO υπολοιπες προτασεις εδω

\begin{lem}\label{lem:kn}
  Έστω $ (a_{n})_{n \in \mathbb{N}} $ ακολουθία και $ (a_{k_{n}})_
  {n \in \mathbb{N}} $ υπακολουθία της. Τότε $ k_{n} \geq n, \; 
  \forall n \in \mathbb{N} $.
\end{lem}

\begin{proof}
\item {}
  \begin{itemize}
    \item Για $ n=1 $, έχω: $ k_{n} \in \mathbb{N}, \; \forall n 
      \in \mathbb{N} \Rightarrow k_{n} \geq 1 $, ισχύει:
    \item Έστω ότι ισχύει για $ n $, δηλ. $ k_{n} \geq n $. 
    \item θ.δ.ο. ισχύει για $ n+1 $. Πράγματι, 
      \[ k_{n+1} 
        \overset{k_{n} \; \text{γν.αυξ.}}{>} k_{n} 
        \geq n \Rightarrow k_{n+1} > n \Rightarrow k_{n+1} 
      \geq n+1\]
  \end{itemize}
\end{proof}

\myprop{Αν μια ακολουθία $ (a_{n})_{n \in \mathbb{N}} $ συγκλίνει στο $ a 
  \in \mathbb{R} $, τότε και κάθε υπακολουθία της συγκλίνει επίσης 
στο $ a \in \mathbb{R} $.}

\begin{proof}
\item {}
  Έστω $ (a_{k_{n}})_{n \in \mathbb{N}} $ υπακολουθία της 
  $ (a_{n})_{n \in \mathbb{N}} $. 

  Έστω $ \varepsilon >0 $. Τότε

  \[ \lim_{a_{n}} = a \Leftrightarrow \forall \varepsilon >0, \; 
    \exists n_{0} \in \mathbb{N} \; : \; \forall n \geq n_{0} 
  \quad \inlineequation[eq:ypaksygk]{\abs{a_{n}- a} < \varepsilon}. \]

  Άρα και για το $ \varepsilon $ που έχουμε επιλέξει, έχουμε ότι $ 
  \exists n_{0} \in \mathbb{N} \; : \; k_{n} \overset{\text{Λημ.} 
  \;~\ref{lem:kn}}{\geq} n \geq n_{0}  $ 
  και άρα από τη σχέση~\eqref{eq:ypaksygk}, έχουμε 
  $\abs{a_{k_{n}} - a} < \varepsilon  $, δηλαδή 
  $ \lim_{n \to +\infty} a_{k_{n}} = a$.
\end{proof}

\begin{rem}
  Αν δυο υπακολουθίες μιας ακολουθίας $ (a_{n})_{n \in \mathbb{N}} $ συγκλίνουν 
  σε διαφορετικά όρια, τότε δεν υπάρχει το όριο της ακολουθίας 
  $ (a_{n})_{n \in \mathbb{N}} $.
\end{rem}

\begin{example}
  Να δείξετε ότι η ακολουθία $ a_{n} = (-1)^{n}\frac{n}{n+1} $ δεν συγκλίνει. 
\end{example}

\begin{proof}
\item {}
  Θεωρούμε τις υπακολουθίες:
  \begin{itemize}
    \item $ a_{2n} = (-1)^{2n} \frac{2n}{2n+1} = \frac{2n}{2n+1} = \frac{2n}{n(2+
      \frac{1}{n})} = \frac{2}{2+ \frac{1}{n}} \xrightarrow{n \to \infty} 1 $
    \item $ a_{2n-1} = (-1)^{2n-1} \frac{2n-1}{2n-1+1} = -\frac{2n-1}{2n} = 
      -\left(1 - \frac{1}{2n}\right) \xrightarrow{n \to \infty} -1  $
  \end{itemize}
  Άρα, η ακολουθία $ a_{n}= (-1)^{n} \frac{n}{n+1} $ δεν συγκλίνει.
\end{proof}



\section{Ιδιότητες Ορίου Ακολουθίας}

\myprop{Έστω $ \lim_{n \to +\infty} a_{n} = a$ και $ \lim_{n \to +\infty} 
  b_{n} = b $, όπου $ a,b \in \mathbb{R} $. Τότε:
  \begin{enumerate}[i)]
    \item $ \lim_{n \to +\infty} (a_{n} + b_{n}) = a+b = 
      \lim_{n \to +\infty} a_{n} + \lim_{n \to +\infty} b_{n} $
    \item $ \lim_{k a_{n}} = ka = k \lim_{n \to +\infty} a_{n}, \; 
      k \in \mathbb{R} $
    \item $ \lim_{n \to +\infty} (a_{n}\cdot b_{n}) = a\cdot b = 
      \lim_{n \to +\infty} a_{n}\cdot \lim_{n \to +\infty} b_{n}$
    \item $ \lim_{n \to +\infty} \frac{1}{a_{n}} = \frac{1}{a} = 
      \frac{1}{\lim\limits_{n \to +\infty}
      a_{n}}, \; a \neq 0, \; a_{n} \neq 0, \; \forall n \in 
      \mathbb{N}  $
    \item $ \lim_{n \to +\infty} \frac{a_{n}}{b_{n}} = \frac{a}{b} 
      = \frac{\lim\limits_{n \to +\infty} a_{n}}
      {\lim\limits_{n \to +\infty} 
      b_{n}}, \; b \neq 0, \; n \neq 0, \forall n \in \mathbb{N} $
    \item $ \lim_{n \to +\infty} {a_{n}}^{k} = a^{k}, 
      \; k \geq 2, \; k \in \mathbb{N}  $  
    \item $ \lim_{n \to +\infty} {a_{n}}^{\frac{1}{k}}
      =a^{\frac{1}{k}}, \; \forall k \in 
      \mathbb{N}, a_{n} \geq 0, \; \forall n \in \mathbb{N} $
\end{enumerate}}

\begin{proof}
\item {}
  \begin{enumerate}[i)]
    \item $ \lim_{n \to +\infty} a_{n} = a \Leftrightarrow \forall 
      \varepsilon > 0, \; \exists n_{1} \in \mathbb{N} \; : \; 
      \forall n \geq n_{1} \quad \abs{a_{n}-a} < \varepsilon$.

      Αρα και για $ \varepsilon = \frac{\varepsilon}{ 2}, \; 
      \exists n_{1} \in \mathbb{N} \; : \; \forall n \geq n_{1} 
      \quad \inlineequation[eq:idiot1]{\abs{a_{n}-a} <
      \frac{\varepsilon}{2}} $

      $ \lim_{n \to +\infty} b_{n} = b \Leftrightarrow \forall 
      \varepsilon > 0, \; \exists n_{2} \in \mathbb{N} \; : \; 
      \forall n \geq n_{2} \quad \abs{b_{n}-b} < \varepsilon$.

      Αρα και για $ \varepsilon = \frac{\varepsilon}{ 2}, \; 
      \exists n_{2} \in \mathbb{N} \; : \; \forall n \geq n_{2} 
      \quad \inlineequation[eq:idiot2]{\abs{b_{n}-b} <
      \frac{\varepsilon}{2}} $

      Θέτουμε $ n_{0}= \max \{ n_{1}, n_{2} \} $, ώστε να ισχύουν 
      και οι δύο παραπάνω ανισότητες ταυτόχρονα.

      Έστω $ \varepsilon >0 $. Τότε για $ n \geq n_{0} $, έχουμε 
      \[
        \abs{(a_{n}+ b_{n}) - (a+b)} = \abs{(a_{n}- a) + 
        (b_{n}-b)} \leq \abs{a_{n}- a} + \abs{b_{n}-b} 
        \overset{\eqref{eq:idiot1}}
        {\underset{\eqref{eq:idiot2}}{<}} 
        \frac{\varepsilon }{2} + \frac{\varepsilon}{ 2} = 
        \varepsilon
      \] 

    \item 
      Αν $ k=0 $, τότε η σχέση είναι προφανής.

      Έστω $ k \neq 0 $. 

      $ \lim_{n \to +\infty} a_{n} = a \Leftrightarrow \forall 
      \varepsilon >0, \; \exists 
      n_{0} \in \mathbb{N} \; : \; \forall n \geq n_{0} \quad 
      \abs{a_{n}- a} < \varepsilon$. 

      Άρα και για $ \varepsilon = \frac{\varepsilon}{\abs{k}} >0$,
      έχουμε ότι $ \exists n_{0} \in \mathbb{N} \; : \; 
      \forall n \geq n_{0} \quad \inlineequation[eq:idiot3]
      {\abs{a_{n}- a} < \frac{\varepsilon}{\abs{k}}}$. 

      Έστω $ \varepsilon >0 $. Τότε $ \forall n \geq n_{0} $, 
      έχουμε ότι 
      \[
        \abs{k a_{n}- ka} = \abs{k(a_{n}- a)} = \abs{k} \cdot 
        \abs{a_{n}- a} \overset{\eqref{eq:idiot3}}{<} \abs{k}
        \cdot \frac{\varepsilon}{\abs{k}} = \varepsilon 
      \] 

    \item Έχουμε ότι $ (a_{n})_{n \in \mathbb{N}} $ συγκλίνουσα $ 
      \Rightarrow (a_{n})_{n \in \mathbb{N}} $ είναι φραγμένη. 
      Άρα 
      \[ 
        \exists M >0, \; M \in \mathbb{R} \; : \; \forall n \in 
        \mathbb{N} \quad \inlineequation[eq:idiot4]{\abs{a_{n}} 
        \leq M}
      \] 
      Τότε 
      \begin{align*}
        \abs{a_{n} b_{n} - ab} = \abs{a_{n}b_{n} - a_{n}b + 
        a_{n}b -ab} &= \abs{a_{n}(b_{n}-b)+b(a_{n}-a) } \\
                    &\leq \abs{a_{n}}\cdot \abs{b_{n}-b} + \abs{b} 
                    \cdot \abs{a_{n}-a} \\
                    &\overset{\eqref{eq:idiot4}}{\leq} 
                    M \cdot  \abs{b_{n}-b} + \abs {b}\cdot 
                    \abs{a_{n}-a}
      \end{align*}

      Απο τη σύγκλιση των $ a_{n} $ και $ b_{n} $, έχουμε:

      Θέτουμε $ \varepsilon_{1} = \frac{\varepsilon}{2M} >0 $ 
      και $ \varepsilon_{2} = \frac{\varepsilon}{2\abs{b}} >0 $ 
      και επιλέγουμε $ n_{0}= \max \{ n_{1}, n_{2}\} $.

      Έστω $ \varepsilon >0 $, τότε $ \forall n \geq n_{0}$, 
      έχουμε ότι 
      \[
        \abs{a_{n} b_{n} - ab} \leq M \cdot \abs{b_{n}-b} + 
        \abs{b} \cdot \abs{a_{n}- a} < M \cdot 
        \frac{\varepsilon}{2 M} + \abs {b} \cdot 
        \frac{\varepsilon}{2 \abs{b}} =
        \frac{\varepsilon}{2} + \frac{\varepsilon}{2} = 
        \varepsilon
      \]

      \begin{description}
        \item [Β᾽ Τρόπος:]
        \item {} 
          \begin{minipage}{0.45\textwidth}
            \begin{itemize}
              \item $ \lim_{n \to \infty} a_{n} 
                = a \Leftrightarrow 
                \lim_{n \to +\infty} (a_{n}-a) 
                = 0 $ \hfill \tikzmark{a}
              \item $ \lim_{n \to \infty} b_{n} 
                = b \Leftrightarrow 
                \lim_{n \to +\infty} (b_{n}-b) 
                = 0 $ \hfill \tikzmark{b}
            \end{itemize}
          \end{minipage}

          \mybrace{a}{b}[$(a_{n})_{n \in \mathbb{N}}, \; 
          (b_{n})_{n \in \mathbb{N}}$ μηδενικές ακολουθίες, 
          άρα φραγμένες.]

          Συνπώς (από μηδενική επι φραγμένη) έχουμε ότι 
          \[ \lim_{n \to +\infty} [(a_{n}- a) 
            \cdot (b_{n} -b)] = 0 \Rightarrow 
            \lim_{n \to +\infty} [a_{n} b_{n} - a_{n}b -
          b_{n} a + ab] = 0\] 

          Έχουμε ότι 
          $
          \left.
            \begin{aligned}
              \lim_{n \to +\infty} a_{n}b = ab \\
              \lim_{n \to +\infty} b_{n}a = ab
            \end{aligned} 
          \right\}  
          \overset{(+)}{\Rightarrow} \lim_{n \to +\infty} 
          [a_{n} b_{n} + ab] = 2ab \Rightarrow 
          \lim_{n \to +\infty} (a_{n}b_{n}) = ab $
      \end{description}

    \item (Χωρίς Απόδειξη)

      % \begin{lem}
      %     $ b \neq 0 \Rightarrow \exists n_{0} \in \mathbb{N} \; 
      %     : \; b_{n} \neq 0, \; n \geq n_{0} $ 
      % \end{lem}

      % \begin{proof}
      % \item {}
      %     Έστω $ b \neq 0 \Rightarrow \frac{\abs{b}}{2} >0 $. 

      % Για $ \varepsilon = \frac{\abs{b}}{2} >0 $ και λόγω 
      % ότι $ \lim_{n \to +\infty} b_{n} = b $
      % έχουμε ότι $ \exists n_{1} \in \mathbb{N} \; : \; 
      % \forall n \geq n_{1} \quad \abs{b_{n}-b} 
      % < \frac{\abs{b}}{2} $

      % Άρα για $ n \geq n_{1} $, έχουμε 
      % \[
      %     \abs{b} = \abs{b - b_{n} + b_{n}} \leq 
      %     \abs{b - b_{n}} + \abs{b_{N}} \leq
      %     \frac{\abs{b}}{2} + \abs{b_{n}} \Leftrightarrow 
      %     \abs{b_{n}} > \abs{b} -
      %     \frac{\abs{b}}{2} \Leftrightarrow \abs{b_{n}}  > 
      %     \frac{\abs{b}}{2} \Rightarrow b_{n} 
      %     \neq 0, \; \forall n \geq n_{1}
      % \] 

      % \begin{description}
      %     \item [Β᾽ Τρόπος:] 
      %     \item {}
      %         Γενικά ισχύει ότι αν $ x >0 $ και $ 
      %         \abs{y-x} < \frac{x}{2} $, 
      %         τότε \inlineequation[eq:lemidiot]{\abs{y} < 
      %         \frac{x}{2}}.

      % $ \lim_{n \to +\infty} b_{n} =b \Rightarrow 
      % \exists n_{0} \in \mathbb{N} \; : \; 
      % \forall n \geq n_{0} \quad \abs{b_{n}-b} < 
      % \frac{\abs{b}}{2} = \varepsilon $

      % Οπότε $ \forall n \geq n_{0} \; : \; 
      % \abs{\abs{b_{n}} - \abs{b}} \leq \abs{b_{n}-b}
      % < \frac{\abs{b}}{2} \Rightarrow \abs{b_{n}} 
      % \overset{\eqref{eq:lemidiot}}{<} 
      % \frac{\abs{b}}{2}   $ 
      % \end{description}

      % Τώρα αν $ n \geq n_{1} $ ισχύει ότι:

      %                 $ \abs{\frac{1}{b_{n}} - \frac{1}{b}} = 
      %                 \frac{b-b_{n}}{b_{n}\cdot b} = \frac{\abs{b-b_{n}}
      %                     }{\abs{b_{n}} \cdot \abs{b}} < \frac{2 
      %                 \abs{b - b_{n}}}{\abs{b}^{2}} $

      % Έστω $ \varepsilon >0 $, τότε
      % \[
      %     \lim_{n \to +\infty} b_{n} =b \Leftrightarrow 
      %     \exists n_{2} \in \mathbb{N} \; : \; \forall n 
      %     \geq n_{2} \quad \abs{b - b_{n}} < 
      %     \frac{\varepsilon \abs{b}^{2}}{2}
      % \] 

      % Επιλέγουμε $ n_{0} = \max \{ n_{1}, n_{2} \} $. 
      % Τότε $ \forall n \geq n_{0} $ ισχύει ότι 
      % \[
      %     \abs{\frac{1}{b_{n}} - \frac{1}{b}} \leq 
      %     \frac{2 \abs{b -b_{n}}}{\abs{b} ^{2}} < \varepsilon 
      % \] 
      % \end{proof}

    \item 
      $ \lim_{n \to +\infty} \frac{a_{n}}{b_{n}} = 
      \lim_{n \to +\infty} \left(a_{n}\cdot \frac{1}{b_{n}}
      \right) = \lim_{n \to +\infty} a_{n} \cdot \lim_{n \to +\infty} 
      \frac{1}{b_{n}} = a \cdot \frac{1}{b} = \frac{a}{b}$

    \item 
      Θα αποδείξουμε την πρόταση με Μαθηματική Επαγωγή.
      \begin{itemize}
        \item Για $ k=2 $, έχω: $ \lim_{n \to +\infty} 
          a_{n}^{2} = \lim_{n \to
          +\infty} (a_{n} \cdot a_{n}) = 
          \lim_{n \to +\infty} a_{n} \cdot 
          \lim_{n \to +\infty}
          a_{n} =  a \cdot a = a^{2}  $, ισχύει.
        \item Έστω ότι ισχύει για $k$, δηλ. $ 
          \lim_{n \to +\infty} {a_{n}}^{k} = a^{k}  $
        \item θ.δ.ο. ισχύει για $ k+1 $. Πράγματι, 
          \[
            \lim_{n \to +\infty} {a_{n}}^{k+1}= 
            \lim_{n \to +\infty} ({a_{n}}^{k} \cdot a_{n})  
            = \lim_{n \to +\infty} {a_{n}}^{k} \cdot 
            \lim_{n \to +\infty} a_{n} = a^{k} \cdot a = 
            a^{k+1}
          \] 
      \end{itemize}

    \item (Χωρίς Απόδειξη)
  \end{enumerate}
\end{proof}

\myprop{Έστω $ (a_{n})_{n \in \mathbb{N}} $ μηδενική ακολουθία και 
  $ (b_{n})_{n \in \mathbb{N}} $ 
φραγμένη. Τότε, $ \lim_{n \to +\infty} (a_{n}\cdot b_{n}) = 0 $.}

\begin{proof}
\item {}
  Έστω $ \varepsilon >0$, τότε
  \[ 
    (b_{n})_{n \in \mathbb{N}} \;\; \text{φραγμένη} \; \Rightarrow 
    \exists M>0 
    \; : \; \inlineequation[eq:qq1]{\abs{b_{n}} \leq M}, 
    \forall n \in \mathbb{N} 
  \]
  \[ 
    (a_{n})_{n \in \mathbb{N}}  \;\; \text{μηδενική} \; \Rightarrow  
    \text{για} \;\; \frac{\varepsilon}{M} > 0, \;  \exists n_{0} \in 
    \mathbb{N} \; : \; \forall n \geq n_{0} \quad \abs{a_{n}-0} 
    < \frac{\varepsilon}{M} \Leftrightarrow     
    \inlineequation[eq:qq2]{\abs{a_{n}} < \frac{\varepsilon}{M}}
  \]
  Άρα για κάθε $ n \geq n_{0} $ έχουμε
  \[
    \abs{a_{n}\cdot b_{n} - 0} = \abs{a_{n}\cdot b_{n} } 
    = \abs{a_{n}} \cdot \abs{b_{n}} 
    \overset{\eqref{eq:qq1}}{\leq} \abs{a_{n}} \cdot M 
    \overset{\eqref{eq:qq2}}{<} 
    \frac{\varepsilon}{M} \cdot M = \varepsilon
  \]
  Άρα $ \lim_{n \to +\infty} a_{n} \cdot b_{n} = 0 $.
\end{proof}

%TODO παραδείγματα (μηδενικη επι φραγμενη)

\myprop{Έστω $ (a_{n})_{n \in \mathbb{N}}, (b_{n})_{n \in \mathbb{N}} $ και 
  $ (c_{n})_{n \in \mathbb{N}} $, τρεις ακολουθίες, τέτοιες ώστε:

  \vspace{\baselineskip}

  \begin{minipage}{0.35\textwidth}
    \begin{enumerate}[i)]
      \item $ a_{n} \leq b_{n} \leq c_{n}, \; \forall n \in 
        \mathbb{N} $ \hfill \tikzmark{a} 
      \item $ \lim_{n \to +\infty} a_{n} = \lim_{n \to +\infty} 
        c_{n} = l $ \hfill \tikzmark{b}
    \end{enumerate}
  \end{minipage}

\mybrace{a}{b}[$ \lim_{n \to +\infty} b_{n} = l$]}

\begin{proof}
\item {}
  Έστω $ \lim_{n \to +\infty} a_{n} = \lim_{n \to +\infty} c_{n} = l $ και 
  έστω $ \varepsilon >0 $, τότε
  \[ \lim_{n \to +\infty} a_{n} = l \Leftrightarrow \forall 
    \varepsilon >0, \; \exists n_{1} \in \mathbb{N} \; : \; \forall n 
    \geq n_{1} \quad \abs{a_{n} - l} < \varepsilon \Leftrightarrow 
  \overbrace{l - \varepsilon < a_{n}} < l + \varepsilon \] 

  \[ \lim_{n \to +\infty} c_{n} = l \Leftrightarrow \forall 
    \varepsilon >0, \; \exists n_{2} 
    \in \mathbb{N} \; : \; \forall n \geq n_{2} \quad \abs{c_{n} - l} 
    < \varepsilon \Leftrightarrow 
  l - \varepsilon < \underbrace{c_{n} < l + \varepsilon} \]

  Επιλέγουμε $ n_{0} = \max \{ n_{1}, n_{2} \} $, οπότε 
  $ \exists n_{0} \in \mathbb{N} \; : \; 
  \forall n \geq n_{0} $
  \begin{gather*}
    l - \varepsilon < a_{n} \leq b_{n} \leq c_{n} < l + 
    \varepsilon \Leftrightarrow \\
    l - \varepsilon < b_{n} < l + \varepsilon \Leftrightarrow \\
    - \varepsilon < b_{n} - l < \varepsilon \Leftrightarrow \\
    \abs{b_{n}-l} < \varepsilon, \; \forall n \geq n_{0}
  \end{gather*}
  Άρα $ \lim_{n \to +\infty} b_{n} = l $.
\end{proof}

\begin{cor}
  Έστω $ (a_{n})_{n \in \mathbb{N}} $ και $ 
  (b_{n})_{n \in \mathbb{N}} $ ακολουθίες, τέτοιες ώστε: 

  \vspace{\baselineskip}

  \begin{minipage}{0.25\textwidth}
    \begin{enumerate}[i)]
      \item $ \abs{a_{n}} \leq \abs{b_{n}}, \; \forall n \in 
        \mathbb{N} $ \hfill \tikzmark{a}
      \item $ \lim_{n \to +\infty} b_{n} = 0$ \hfill \tikzmark{b}
    \end{enumerate}
  \end{minipage}

  \mybrace{a}{b}[$ \lim_{n \to +\infty} a_{n} = 0 $]
\end{cor}

\myprop{Έστω $ \lim_{n \to +\infty} a_{n} = a $ και $ 
  \lim_{n \to +\infty} b_{n} = b $. Τότε
$ a_{n} \leq b_{n}, \; \forall n \in \mathbb{N} \Rightarrow a \leq b $ }

\begin{proof}(Με άτοπο)
\item {}
  Έστω $ a>b \Rightarrow a-b>0 $. Θέτουμε $ \varepsilon = 
  \frac{a-b}{2} $.

  Από τον ορισμό του ορίου για τις δύο ακολουθίες, έχουμε ότι:
  \begin{align}
    \exists n_{1} \in \mathbb{N} \; : \; \forall n \geq n_{1} 
    \quad \abs{a_{n}-a} < \frac{a-b}{2} \label{eq:aleqb1} \\
    \exists n_{2} \in \mathbb{N} \; : \; \forall n \geq n_{2} 
    \quad \abs{b_{n}-b} < \frac{a-b}{2} \label{eq:aleqb2} 
  \end{align}

  Έχουμε ότι

  \begin{minipage}{0.5\textwidth}
    \begin{itemize}
      \item $ -\abs{a_{n}-a} \leq a_{n}-a \leq \abs{a_{n}-a} $ \hfill 
        \tikzmark{a}
      \item Από την~\eqref{eq:aleqb1} έχουμε $ - \abs{a_{n}-a} > - 
        \frac{a-b}{2} $ \hfill \tikzmark{b}
    \end{itemize}    
  \end{minipage}

  \mybrace{a}{b}[$ a_{n}- a > - \frac{a-b}{2}$]

  Άρα \[ a_{n} > a - \frac{a-b}{2} 
  = \frac{a+b}{2}, \forall n \geq n_{1} \]

  Έχουμε ότι

  \begin{minipage}{0.45\textwidth}
    \begin{itemize}
      \item $ -\abs{b_{n}-b} \leq b_{n}-b \leq \abs{b_{n}-b} $ \hfill 
        \tikzmark{a}
      \item Από την~\eqref{eq:aleqb2} έχουμε $ \abs{b_{n}-b} < 
        \frac{a-b}{2} $ \hfill \tikzmark{b}
    \end{itemize}    
  \end{minipage}

  \mybrace{a}{b}[$ b_{n}- b < \frac{a-b}{2}$]

  Άρα \[ b_{n} < b + \frac{a-b}{2} 
  = \frac{a+b}{2}, \forall n \geq n_{1} \]


  Για $ n_{0} = \max \{ n_{1}, n_{2} \} $ έχουμε ότι $ a_{n} > 
  \frac{a+b}{2}, \; \forall n \geq n_{0}$
  και $ b_{n} < \frac{a+b}{2}, \; \forall n \geq n_{0}$, οπότε

  $a_{n} > b_{n}, \; \forall n \geq n_{0} $, άτοπο.
\end{proof}


\myprop{Έστω $ \lim_{n \to +\infty} a_{n} = a $ και $ \lim_{n \to +\infty} 
  b_{n} = b $. Τότε
$ a_{n} < b_{n}, \; \forall n \in \mathbb{N} \Rightarrow a \leq b $ }

\begin{proof}
\item {}
  \[ a_{n}< b_{n}, \; \forall n \in \mathbb{N} \Rightarrow a_{n} \leq 
  b_{n}, \; \forall n \in \mathbb{N} \Rightarrow a \leq b \]
\end{proof}

\begin{rem}
\item {}
  Αν $ a_{n}= \frac{1}{n}, \; \forall n \in \mathbb{N} $ 
  και $ b_{n}=0, \; \forall n \in \mathbb{N} $, τότε έχουμε ότι 
  $ \lim_{n \to +\infty} \frac{1}{n} = 0 = a $ και $ \lim_{n \to
  +\infty} b_{n} = 0 = b $, δηλαδή $ a=b=0  $.
\end{rem}

\begin{cor}
  Έστω $ (a_{n})_{n \in \mathbb{N}} $ ακολουθία, τέτοια ώστε 
  $ \lim_{n \to +\infty} a_{n} = a$ και $ a_{n} \geq 0, \; \forall 
  n \in \mathbb{N} $, τότε $ a \geq 0 $.
\end{cor}

\begin{proof}
  Έστω $ b_{n} = 0, \; \forall n \in \mathbb{N} $. Τότε προφανώς 
  $ a_{n} \geq b_{n}, \; \forall n
  \in \mathbb{N} \Rightarrow \lim_{n \to +\infty} a_{n} \geq 
  \lim_{n \to +\infty} b_{n} = 0$.
\end{proof}

\mythm{Κάθε αύξουσα και άνω φραγμένη ακολουθία συγκλίνει στο 
supremum του συνόλου των όρων της.}

\begin{proof}
  Έστω $ (a_{n})_{n \in \mathbb{N}} $ ακολουθία πραγματικών αριθμών, 
  τέτοια ώστε:
  \begin{enumerate}[i)]
    \item $ (a_{n})_{n \in \mathbb{N}} $ αύξουσα $ \Leftrightarrow 
      a_{n+1} \geq a_{n}, \; \forall n \in \mathbb{N}$ 
    \item $ (a_{n})_{n \in \mathbb{N}} $ άνω  φραγμένη $ 
      \Leftrightarrow a_{n} \leq M, \; \forall n \in 
      \mathbb{N}$, με $ M \in \mathbb{R} $.
  \end{enumerate}

  Έστω $ \varepsilon >0 $. Από τη χαρακτηριστική ιδιότητα του 
  supremum, έχουμε ότι $ \exists a_{n_{0}} \in A \; : \; s - \varepsilon 
  < a_{n_{0}} $.

  Ζητούμενο, είναι $ \lim_{n \to \infty} a_{n} = s \Leftrightarrow 
  - \varepsilon < a_{n} -s < \varepsilon \Leftrightarrow s - 
  \varepsilon < a_{n} < s + \varepsilon $. Πράγματι, 

  \[ a_{n} \leq s < s + \varepsilon, \; \forall n \in \mathbb{N} \] και 
  \[ a_{n} \overset{a_{n} \text{γν. αυξ.}}{\geq} a_{n_{0}} > s - 
  \varepsilon, \forall n \geq n_{0} \]
\end{proof}

\mythm{Κάθε φθίνουσα και κάτω φραγμένη ακολουθία συγκλίνει στο infimum του 
συνόλου των όρων της.}

\begin{proof}
  Ομόίως με το προηγούμενο θεώρημα. Έστω 
  $ (b_{n})_{n \in \mathbb{N}} $ 
  φθίνουσα και κάτω φραγμένη ακολουθία. Τότε η ακολουϑία 
  $ -(b_{n})_{n \in \mathbb{N}} $ είναι αύξουσα και άνω φραγμένη, 
  οπότε από το προηγούμενο θεώρημα, η $ -(b_{n})_{n \in \mathbb{N}} $
  συγκλίνει στο supremum του συνόλου των όρων της, δηλαδή στο  
  $ \sup \{ - b_{1}, - b_{2}, -b_{3}, \ldots \}$. Οπότε η 
  $ (b_{n})_{n \in \mathbb{N}} $ 
  %TODO   
\end{proof}

\myprop{$ \lim_{n \to \infty} x^{n} = 0, \; \abs{x} <1  $}

\begin{proof}
\item {}
  \begin{itemize}
    \item $ x = 0 \Rightarrow x^{n} = 0, \; \forall n \in 
      \mathbb{N} $ και άρα η ακολουθία είναι σταθερή και ίση με 
      0, επομένως 
      $ \lim_{n \to \infty} x^{n} = \lim_{n \to \infty} 0 = 0 $.

    \item $ x \neq 0 \Rightarrow 0 < \abs{x} < 1 
      \overset{x \neq 0}{\Rightarrow} \frac{1}{\abs{x}} > 1  $. 

      Οπότε θέτουμε $ a = \frac{1}{\abs{x}} - 1 > 0 $. Τότε έχουμε ότι 
      \begin{align*} 
        \frac{1}{\abs{x}} = a+1 
                    &\Rightarrow 0< \frac{1}{\abs{x}^{n}}  = (1+a)^{n}
                    \overset{\text{Bernoulli}}{\geq} 1 + na \\ 
                    & \Leftrightarrow 0 \leq \abs{x}^{n} \leq 
                    \frac{1}{1+na} \leq \frac{1}{na} = \frac{1}{a} 
                    \cdot \frac{1}{n} 
      \end{align*} 

      Επείδη, $ \lim_{n \to \infty} 0 = 0 $ και 
      $ \lim_{n \to \infty} \left(\frac{1}{a} \cdot \frac{1}{n} 
      \right) = 0$, από Κριτήριο Παρεμβολής, έχουμε ότι 
      $ \lim_{n \to \infty} \abs{x} ^{n} = 
      0 \Rightarrow \lim_{n \to \infty}x^{n} = 0$.
  \end{itemize}
\end{proof}


\section{Άπειρο Όριο Ακολουθίας}


\mydfn{Μια ακολουθία πραγματικών αριθμών $ (a_{n})_{n \in \mathbb{N}} $ 
  \textcolor{Col\thechapter}{αποκλίνει} στο $ +\infty $ (συμβ.: 
  $ \lim_{n \to \infty} a_{n} = + 
  \infty $), αν $ \forall M>0, \; \exists n_{0} \in 
\mathbb{N} \; : \; a_{n} > M, \; \forall n \geq n_{0}$}

\mydfn{Μια ακολουθία πραγματικών αριθμών $ (a_{n})_{n \in \mathbb{N}} $ 
  \textcolor{Col\thechapter}{αποκλίνει} στο $ -\infty $ (συμβ.: 
  $ \lim_{n \to \infty} a_{n} = - 
  \infty $), αν $ \forall M>0, \; \exists n_{0} \in 
\mathbb{N} \; : \; a_{n} < -M, \; \forall n \geq n_{0}$}

\myprop{Έστω $ (a_{n})_{n \in \mathbb{N}} $ γνησίως αύξουσα και μη-άνω φραγμένη 
  ακολουϑία, τότε $ \lim_{n \to +\infty} a_{n} = + \infty $.
}

\myprop{Έστω $ (a_{n})_{n \in \mathbb{N}} $ γνησίως φθίνουσακ  και μη-κάτω φραγμένη 
  ακολουϑία, τότε $ \lim_{n \to +\infty} a_{n} = - \infty $.
}

\myprop{Έστω $ (a_{n})_{n \in \mathbb{N}} $ ακολουθία θετικών 
  πραγματικών αριθμών. 
  Η $ (a_{n})_{n \in \mathbb{N}} $ αποκλίνει στο $ + \infty $ αν και 
  μονον αν η ακολουθία 
  $ \left(\frac{1}{a_{n}} \right)_{n \in \mathbb{N}} $ συγκλίνει 
στο 0.}


\begin{proof}
\item {}
  \begin{description}
    \item[$ (\Rightarrow) $] Έστω ότι $ (a_{n})_{n \in \mathbb{N}} $ 
      αποκλίνει στο $ + \infty $. Αφού $ (a_{n})_{n \in \mathbb{N}} $
      ακολουθία θετικών πραγματικών αριθμών, τότε $ a_{n} \neq 0, 
      \; \forall n \in \mathbb{N}$ και άρα, μπορούμε να θεωρήσουμε 
      την ακολουθία, $ b_{n} = \frac{1}{a_{n}}, \; \forall n \in
      \mathbb{N}  $. 

      Θα δείξουμε ότι $ \lim_{n \to \infty} b_{n} = 0 $. Πράγματι,

      Έστω $ \varepsilon >0 $. Θέτω $ M = \frac{1}{\varepsilon} >0 $.
      Τότε $ \exists n_{0} \in \mathbb{N} \; : \; \forall n \geq 
      n_{0} \quad a_{n}> M \Rightarrow a_{n} > \frac{1}{\varepsilon} 
      \Rightarrow \frac{1}{a_{n}} < \varepsilon \Rightarrow
      \abs{\frac{1}{a_{n}} - 0} < \varepsilon \Rightarrow 
      \abs{b_{n}-0} < \varepsilon, \forall n \geq n_{0} $

    \item [$ ( \Leftarrow) $]
      Έστω ότι για μια ακολουθία θετικών πραγματικών, ισχύει ότι 
      $ \lim_{n \to \infty} \frac{1}{a_{n}} = 0$. θ.δ.ο. η $ 
      (a_{n})_{n \in \mathbb{N}}$ αποκλίνει στο $ + \infty $. Πράγματι,

      Έστω $ M > 0 $. Θέτω $ \varepsilon = \frac{1}{M} > 0 $. Από 
      δεδομένο $ \exists n_{0} \in \mathbb{N} \; : \; \forall n \geq 
      n_{0} \quad \abs{\frac{1}{a_{n}} - 0} < \varepsilon \Rightarrow
      a_{n} > \frac{1}{\varepsilon}=M $
  \end{description}
\end{proof}

\myprop{Αν $ x>1 $ τότε $ \lim_{n \to \infty} x^{n} = +\infty $}

\begin{proof}
\item {}
  \begin{description}
    \item [Α᾽ Τρόπος:]
      $ x >1 \Rightarrow x \neq 0 $. Θέτω  $a = \frac{1}{x} \Rightarrow 
      \frac{1}{a^{n}} = x^{n} $ και $ \lim_{n \to \infty} \frac{1}{a_{n}} = 
      \lim_{n \to \infty}x^{n}$. 


      Προφανώς $ 0 < a <1 $, ως αντίστροφος του $x$. Σύμφωνα με γνωστή προταση
      $ \lim_{n \to \infty} a^{n} = 0 $. Η ακολουθία $ (a_{n})_{n \in \mathbb{N}}
      $ είναι μια ακολουθία θετικών όρων που συγκλίνει στο 0. Σύμφωνα με 
      την προηγούμενη πρόταση, η ακολουθία $ \left(\frac{1}{a_{n}}\right)_{n 
      \in \mathbb{N}} $ αποκλίνει στο $ + \infty $, και άρα έχουμε το ζητούμενο.
    \item [Β᾽ Τρόπος:]
      $ x>1 \Rightarrow x-1>0 $. Θέτουμε $ a = x-1>0 $, άρα $ x = 1+a 
      \Rightarrow x^{n} = (1+a)^{n} \geq 1+na, \; \; \forall n \in \mathbb{N} $ 
      Επομένως 
      \[
        \lim_{n \to +\infty} x^{n} = \lim_{n \to +\infty} (1+na) 
        \overset{a>0}{=} +\infty 
      \]
  \end{description}
\end{proof}


\myprop{$ \lim_{n \to \infty} \sqrt[n]{n} = 1, \; \forall n \in \mathbb{N}  $}

\begin{proof}
  Έστω $ n \in \mathbb{N} \Rightarrow n \geq 1, \; \forall n \in 
  \mathbb{N} \Rightarrow n ^{\frac{1}{2n}} \geq 1^{\frac{1}{2n}}, \; 
  \forall n \in \mathbb{N} \Rightarrow \sqrt[2n]{n} \geq 1, \; \forall n 
  \in \mathbb{N} $

  Θέτουμε $ a = \sqrt[2n]{n} -1 \geq 0 $. Τότε $ \sqrt[2n]{n} = 1 + a 
  \Rightarrow \sqrt{n} = (1+a)^{n} \overset{\text{Bernoulli}}{\geq} 1 
  + na \Rightarrow na \leq \sqrt{n} - 1  $. Οπότε $ 0 \leq a \leq
  \frac{\sqrt{n} -1}{n} $. Τώρα έχουμε, 

  $ \lim_{n \to \infty} \frac{\sqrt{n} -1}{n} = \lim_{n \to \infty} 
  \left( \frac{\sqrt{n}}{n} - \frac{1}{n}\right) = 0 - 0 = 0 $ και 
  άρα από το Κριτήριο Παρεμβολής, έχουμε ότι \[ \lim_{n \to \infty} a = 0 
    \Rightarrow \lim_{n \to \infty} \sqrt[2n]{n} = 1 \Rightarrow \lim_{n \to
    \infty} (\sqrt[n]{n}) = \lim_{n \to \infty} (\sqrt[2n]{n})^{2} =  
  1^{2} = 1 \Rightarrow \lim_{n \to \infty} \sqrt[n]{n} =1 \]
\end{proof}

\myprop{Έστω $ a>0 $. Τότε $ \lim_{n \to \infty} \sqrt[n]{a}=1, \; \forall n \in
\mathbb{N} $.}

\begin{proof}
\item {}
  \begin{itemize}
    \item $ a>1 \Rightarrow a^{\frac{1}{n}} > 1^{\frac{1}{n}}, \; 
      \forall n \in \mathbb{N} \Rightarrow \sqrt[n]{a} >1, \; 
      \forall n \in \mathbb{N} $

      Για κάθε $ n \in \mathbb{N} $ θέτουμε $ b_{n} = \sqrt[n]{a} -1 
      > 0 \Rightarrow \sqrt[n]{a} = b_{n} + 1 \Rightarrow a = (
      b_{n}+1)^{n} \geq 1 + n b_{n} \Rightarrow n b_{n} \leq a-1
      \Rightarrow 0 \leq b_{n} \leq \frac{a-1}{n}, \; \forall n \in
      \mathbb{N} $

      Επειδή $ \lim_{n \to \infty} \frac{a-1}{n} = \lim_{n \to \infty}
      \left(\frac{a}{n} - \frac{1}{n}\right) = 0 - 0 = 0 $, άρα από το 
      Κριτήριο Παρεμβολής, έχουμε ότι $ \lim_{n \to \infty} b_{n} = 
      \lim_{n \to \infty} (\sqrt[n]{a}-1) = 0 \Rightarrow \lim_{n \to
      \infty} \sqrt[n]{a} =1 $.

    \item $ a< 1 \Rightarrow \frac{1}{a} > 1 $. Τότε από την 
      προηγούμενη πρόταση έχουμε 
      $ \lim_{n \to \infty} \sqrt[n]{\frac{1}{a} } = 1 \Rightarrow 
      \lim_{n \to \infty} \frac{1}{\sqrt[n]{\frac{1}{a}}} = 
      \frac{1}{1} = 1 \Rightarrow \lim_{n \to \infty} \sqrt[n]{a} 
      = 1$.
  \end{itemize}
\end{proof}

\myprop{$ \lim_{n \to \infty} \sqrt[n]{n!} = +\infty $ }


\begin{proof}
\item {}
  \begin{itemize}
    \item 
      Έστω $ n \in \mathbb{N} $ με $ n $ άρτιος, δηλαδή ο 
      $ \frac{n}{2} \in \mathbb{N} $. Τότε

      \begin{align*}
        \sqrt[n]{n!} = \sqrt[n]{1 \cdot 2 \cdot \frac{n+1}{2}
        \left(\frac{n+1}{2} +1\right) \cdots n} 
                    &\geq \sqrt[n]{1\cdot 2 \cdots
                      \underbrace{\frac{n+1}{2} \cdot \frac{n+1}{2} 
                        \cdots \frac{n+1}{2}}_{\frac{n+1}{2} \; 
                    \text{φορές}}} \\
                    &\geq \sqrt[n]{\left(\frac{n+1}{2}\right)^
                    {\frac{n+1}{2}}} \\ 
                    &\geq \sqrt[n]{\left(\frac{n+1}{2}\right)^
                    {\frac{n}{2}}} = \sqrt{ \frac{n+1}{2} } \geq 
                    \sqrt{\frac{n}{2}} 
      \end{align*} 

    \item Έστω $ n \in \mathbb{N} $ με $ n $ περιττός, δηλαδή ο $
      \frac{n+1}{2} = \frac{n}{2} + \frac{1}{2} \in \mathbb{N} $.
      Τότε

      \begin{align*}
        \sqrt[n]{n!} = \sqrt[n]{1\cdot 2 \cdots \frac{n+1}{2}
        \left(\frac{n+1}{2} +1\right) \cdots n}
             &\geq \sqrt[n]{1 \cdot 2 \cdots
             \frac{n+1}{2}  \cdot \frac{n+1}{2} \cdots \frac{n+1}{2} } \\ 
             &\geq 
             \sqrt[n]{\left(\frac{n+1}{2} \right)^{\frac{n+1}{2}}} \\ 
             & \geq
             \sqrt[n]{\left(\frac{n+1}{2} \right)^{\frac{n}{2}}} = 
             \sqrt{\frac{n+1}{2}} \geq 
             \sqrt{\frac{n}{2}}
      \end{align*}
  \end{itemize}

  Από τα παραπάνω προκύπτει ότι $ \sqrt[n]{n!} \geq \sqrt{\frac{n}{2}
  }, \; \forall n \in \mathbb{N} $. 

  Θ.δ.ο. $ \lim_{n \to \infty} \sqrt[n]{n!} = + \infty $, 
  δηλαδή ότι η ακολουθία αποκλίνει.

  Έστω $ M >0 $. Από την Αρχιμήδεια Ιδιότητα υπάρχει 
  φυσικός αριθμός $ 
  n_{0} \in  \mathbb{N} $ με $ n_{0}>2M^{2} $, ώστε 

  $ \forall n \geq n_{0} \quad \frac{n}{2} > M^{2} \Rightarrow
  \sqrt{\frac{n}{2}} > M \Rightarrow \sqrt[n]{n!} > M $. 
\end{proof}


\mythmm{Bolzano-Weierstrass}{Κάθε φραγμένη ακολουθία πραγματικών αριθμών 
έχει συγκλίνουσα υπακολουθία.}


\begin{proof}[Χωρίς Απόδειξη]

\end{proof}








\end{document}
