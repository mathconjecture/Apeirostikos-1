\documentclass[main.tex]{subfiles}


\begin{document}


\section{Όριο Ακολουθίας}

\begin{dfn}
    Ακολουθία πραγματικών αριθμών ονομάζεται κάθε συνάρτηση με πεδίο ορισμού 
    τους φυσικούς αριθμούς. 
    \begin{align*}
        a \colon &\mathbb{N} \to \mathbb{R} \\
                 &n \to a(n)=a_{n}
     \end{align*} 
     Οι ακολουθίες συμβολίζονται ως $ (a_{n})_{n \in \mathbb{N}} $  
      ή $ \{ a_{n} \} _{n \in \mathbb{N}} $  ή $ \{ a_{n} \} _{n=1}^{+\infty} $, κλπ.
\end{dfn}

\begin{dfn}
    Σύνολο Τιμών (Σ.Τ.) της ακολουθίας $ (a_{n})_{n \in \mathbb{N}} $, ονομάζουμε 
    το σύνολο των όρων της, $ \{ a_{1}, a_{2}, \ldots, a_{n} \} $ το οποίο μπορεί 
    να είναι πεπερασμένο ή άπειρο.
\end{dfn}

\begin{examples}
        \item {}
    \begin{enumerate}[i)]
        \item $ a_{n} = n, \; \forall n \in \mathbb{N} $. Έχει Σ.Τ. το σύνολο 
            $  \{ 1,2,3, \ldots \} $.
        \item $\left(\frac{1}{n}\right)_{n \in \mathbb{N}} $. Έχει Σ.Τ. το σύνολο 
            $  \left\{ 1, \frac{1}{2}, \frac{1}{3}, \ldots \right\} $.
        \item $ \{(-1)^{n}\}_{n=1}^{+ \infty}, $. Έχει Σ.Τ. 
            το σύνολο $ \{ -1,1 \} $.
        \item $ a_{n} = c, \; \forall n \in \mathbb{N}, c \in \mathbb{R} $.
            Έχει Σ.Τ. το σύνολο $ \{ c \} $ και ονομάζεται σταθερή ακολουθία.
        \item $ a_{n}=2n, \; \forall n \in \mathbb{N} $. Έχει Σ.Τ. το σύνολο 
            $ \{ 2,4,6, \ldots, 2n, \ldots \} $. Πρόκειται για την ακολουθία 
            των άρτιων φυσικών αριθμών.
        \item $ a_{n}= 2n-1, \; \forall n \in \mathbb{N} $. Έχει Σ.Τ. το 
            σύνολο $ \{ 1,3,5, \ldots, 2n+1, \ldots \} $. Πρόκεται για την 
            ακολουθία των περιττών φυσικών αριθμών.
        \item \label{ex:anadr} $ a_{1}= a_{2} = 1 $ και $ a_{n+2}=a_{n+1}+a_{n}, \; 
            \forall n \in \mathbb{N}$. Έχει Σ.Τ. το σύνολο $ \{ 1,1,2,3,5,8,
            13,21,34, \ldots,\} $. Πρόκειται για την ακολουθία Fibonacci. 
    \end{enumerate}
\end{examples}

\begin{rem}
\item {}
    \begin{enumerate}[i)]
        \item Ουσιαστικά οι ακολουθίες είναι λίστες πραγματικών αριθμών.
        \item Η ακολουθία \ref{ex:anadr}, όπου κάθε επόμενος όρος, ορίζεται με τη 
            βοήθεια του προηγούμενου, λέγεται αναδρομική ακολουθία. Προτάσεις που 
            αφορούν αναδρομικές ακολουθίες, αποδεικνύονται με Μαθηματική Επαγωγή.
    \end{enumerate}
\end{rem}


\begin{dfn}
    Δυο ακολουθίες, $(a_{n})_{n \in \mathbb{N}}$  και $ (b_{n})_{n \in \mathbb{N}} $
    ονομάζονται ίσες, αν $ a_{n} = b_{n}, \; \forall n \in \mathbb{N} $.
\end{dfn}

\begin{dfn}
    Οι πράξεις μεταξύ ακολουθιών, ορίζονται όπως ακριβώς και για τις συναρτήσεις.
\end{dfn}

\section{Φραγμένες Ακολουθίες}

\begin{dfn}
\item {}
    Μια ακολουθία $ (a_{n})_{n \in \mathbb{N}} $ ονομάζεται:
    \begin{enumerate}[i)]
        \item άνω φραγμένη $ \overset{\text{ορ.}}{\Leftrightarrow} \exists M \in 
            \mathbb{R} \; : \; a_{n} \leq M, \; \forall n \in \mathbb{N}$.
        \item κάτω φραγμένη $ \overset{\text{ορ.}}{\Leftrightarrow} \exists m \in 
            \mathbb{R} \; : \; m \leq a_{n}, \; \forall n \in \mathbb{N}  $
        \item φραγμένη $ \overset{\text{ορ.}}{\Leftrightarrow} \exists$ είναι άνω και 
            κάτω φραγμένη.
    \end{enumerate}

    \begin{prop}
        $ (a_{n})_{n \in \mathbb{N}} $ φραγμένη $ \Leftrightarrow \exists M>0, \; M \in 
        \mathbb{R} \; : \; \abs{a_{n}} \leq M, \; \forall n \in \mathbb{N} $
        (Απολύτως φραγμένη).
    \end{prop}
\end{dfn}

\begin{examples}
\item {}  
    \begin{enumerate}[i)]
        \item $ a_{n}= \frac{1}{n}, \; \forall n \in \mathbb{N} $, είναι φραγμένη.
            
            Πράγματι, $ 0 \leq \frac{1}{n} \leq 1, \; \forall n \in \mathbb{N} $. 

            Επίσης, $ \abs{\frac{1}{n}} = \frac{1}{n} \leq 1 $, άρα και 
            απολύτως φραγμένη.
        \item $ a_{n}=(-1)^{n} \frac{1}{n}, \; \forall n \in \mathbb{N} $ 
            είναι απολύτως φραγμένη. Πράγματι,

            \[
                \abs{a_{n}} = \abs{(-1)^{n} \frac{1}{n}} = \abs{(-1)^{n}} 
                \cdot \abs{\frac{1}{n}} = \abs{-1}^{n} \cdot \frac{1}{n}
                = 1 \cdot \frac{1}{n} = \frac{1}{n} \leq 1, \; \forall n \in 
                \mathbb{N}
             \] 

         \item $ a_{n}= \frac{(n-1)!}{n^{n}}, \; \forall n \in \mathbb{N} $
             είναι φραγμένη. Πράγματι, $ a_{n} > 0, \; \forall n \in 
             \mathbb{N}$, άρα 0 κ.φ. της $( a_{n})_{n \in \mathbb{N}} $. 
             Επίσης 
             \[
                 a_{n}= \frac{(n-1)!}{n^{n}} = \frac{1 \cdot 2 \cdots
                 (n-1)}{n^{n}} < \frac{\overbrace{n \cdot n \cdots n}
             ^{n-1 \; \text{φορές}}}{n^{n}} = \frac{n^{n-1}}{n^{n}} =
             \frac{1}{n} \leq 1, \; \forall n \in \mathbb{N},
         \] άρα το 1 είναι α.φ. της $ (a_{n})_{n \in \mathbb{N}} $ 

     \item $ a_{n}= 1 + \left(- \frac{1}{2} \right) + \left(- 
         \frac{1}{2}\right)^{2} + \cdots + \left(-\frac{1}{2} \right) ^{n}, 
         \; \forall n \in \mathbb{N} $ είναι απολύτως φραγμένη. Πράγματι, 
         \[ a_{n} = 1 + \left(- \frac{1}{2}\right) + \left(- \frac{1}{2} 
             \right)^{2} + \cdots + \left(- \frac{1}{2} \right)^{n} = 
             \frac{1 - (- \frac{1}{2} )^{n}}{1 - (- \frac{1}{2})} = 
         \frac{2}{3} \left[1 - \left(- \frac{1}{2} \right)^{n}\right] \]
         Επομένως
         \[
             \abs{a_{n}} = \abs{\frac{2}{3} \left[1-(- \frac{1}{2} )^{n}\right]} = 
             \frac{2}{3} \abs{\abs{1} - \left(- \frac{1}{2}\right)^{n}} \leq 
             \frac{2}{3} \left(1 + \abs{-\frac{1}{2} }^{n} \right) = 
             \frac{2}{3} \left(1+ \frac{1}{2^{n}}\right) < \frac{2}{3}
             (1+1) = \frac{4}{3} 
          \] 

      \item $ a_{n}= 2n+5, \; \forall n \in \mathbb{N} $ είναι κάτω φραγμένη.
          Πράγματι, $ 7 \leq 2n+5, \; \forall n \in \mathbb{N} $, άρα το 
          7 είναι κ.φ. της $ (a_{n} )_{n \in \mathbb{N}} $.

      \item $ a_{1}=2, \; a_{n+1}=2 - \frac{1}{a_{n}}, \forall n \in \mathbb{N}$
          είναι φραγμένη. Πράγματι, με επαγωγή, έχουμε:
          \begin{itemize}
              \item Για $ n=1 $, $ a_{1}=2>1 $, ισχύει. 
              \item Έστω ότι ισχύει για $n$, δηλ. \inlineequation[eq:
                  anadepag1]{a_{n}>1}.
              \item Θ.δ.ο. ισχύει και για $ n+1 $. Πράγματι, από τη σχέση~
                  \eqref{eq: anadepag1}, έχουμε
\[
    a_{n}>1 \Rightarrow \frac{1}{a_{n}} < 1 \Rightarrow - \frac{1}{a_{n}} > 
    -1 \Rightarrow 2 - \frac{1}{a_{n}} > 2-1 \Rightarrow a_{n+1} > 1.
 \] 
          \end{itemize}
    \end{enumerate}
\end{examples}

\section{Μονοτονία Ακολουθιών}

\begin{dfn}
    Μια ακολουθία $ (a_{n})_{n \in \mathbb{N}} $ λέγεται:
    \begin{enumerate}[i)]
        \item άυξουσα $ \overset{\text{ορ.}}{\Leftrightarrow} a_{n} \leq 
            a_{n+1}, \forall n \in \mathbb{N}  $.
        \item φθίνουσα $ \overset{\text{ορ.}}{\Leftrightarrow} a_{n} \geq 
            a_{n+1}, \forall n \in \mathbb{N}  $.
        \item γνησίως αύξουσα $ \overset{\text{ορ.}}{\Leftrightarrow} a_{n} 
            < a_{n+1}$
        \item γνησίως φθίνουσα $ \overset{\text{ορ.}}{\Leftrightarrow} a_{n} 
            > a_{n+1}$
    \end{enumerate}
\end{dfn}

\begin{rems}
\item {}
    \begin{enumerate}[i)]
        \item $ (a_{n})_{n \in \mathbb{N}} $ γνησίως αύξουσα (φθίνουσα) $ 
            \Rightarrow (a_{n})_{n \in \mathbb{N}} $ αύξουσα (φθίνουσα) 
        \item $ (a_{n})_{n \in \mathbb{N}} $ γνησίως φθίνουσα  $ 
            \Rightarrow (a_{n})_{n \in \mathbb{N}} $ ανω φραγμένη, με 
            α.φ. το $ a_{1} $  
        \item $ (a_{n})_{n \in \mathbb{N}} $ γνησίως αύξουσα  $ 
            \Rightarrow (a_{n})_{n \in \mathbb{N}} $ κάτω φραγμένη, με 
            κ.φ. το $ a_{1} $  
    \end{enumerate}
\end{rems}

\begin{dfn}[Ισοδύναμος ορισμός της μονοτονίας]
    Αν μια ακολουθία $ (a_{n})_{n \in \mathbb{N}} $ διατηρεί πρόσημο
    
\end{dfn}


\end{document}
