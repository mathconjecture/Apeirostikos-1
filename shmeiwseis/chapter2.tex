\documentclass[main.tex]{subfiles}


\begin{document}


\section{Όριο Ακολουθίας}

\begin{dfn}
    Ακολουθία πραγματικών αριθμών ονομάζεται κάθε συνάρτηση με πεδίο ορισμού 
    τους φυσικούς αριθμούς. 
    \begin{align*}
        a \colon &\mathbb{N} \to \mathbb{R} \\
                 &n \to a(n)=a_{n}
     \end{align*} 
     Οι ακολουθίες συμβολίζονται ως $ (a_{n})_{n \in \mathbb{N}} $ ή 
     $ \{ a_{n} \} _{n=1}^{+\infty} $ ή $ \{ a_{n} \} _{n \in \mathbb{N}} 
     $, κλπ.
\end{dfn}

\begin{dfn}
    Πεδίο Ορισμού της ακολουθίας, ονομάζουμε το σύνολο των όρων της, 
    $ \{ a_{1}, a_{2}, \ldots, a_{n} \} $ το οποίο μπορεί να είναι 
    πεπερασμένο ή άπειρο.
\end{dfn}

\begin{examples}
        \item {}
    \begin{enumerate}[i)]
        \item $ a_{n}, \; \forall n \in \mathbb{N} $. Έχει Π.Ο. το σύνολο 
            $  \{ 1,2,3, \ldots \} $.
        \item $ a_{n} = (-1)^{n}, \; \forall n \in \mathbb{N} $. Έχει Π.Ο. 
            το σύνολο $ \{ -1,1 \} $.
        \item $ a_{n} = c, \; \forall n \in \mathbb{N}, c \in \mathbb{R} $.
            Έχει Π.Ο. το σύνολο $ \{ c \} $ και ονομάζεται σταθερή ακολουθία.
        \item $ a_{n}=2n, \; \forall n \in \mathbb{N} $. Έχει Π.Ο. το σύνολο 
            $ \{ 2,4,6, \ldots, 2n, \ldots \} $. Πρόκειτα για την ακολουθία 
            των άρτιων φυσικών αριθμών.
        \item $ a_{n}= 2n-1, \; \forall n \in \mathbb{N} $. Έχει Π.Ο. το 
            σύνολο $ \{ 1,3,5, \ldots, 2n+1, \ldots \} $. Πρόκεται για την 
            ακολουθία των περιττών φυσικών αριθμών.
        \item $ a_{1}= a_{2} = 1 $ και $ a_{n+2}=a_{n+1}+a_{n}, \; 
            \forall n \mathbb{N}$. Έχει Σ.Τ. το σύνολο $ \{ 1,1,2,3,5,8,
            13,21,34, \ldots,\} $. Πρόκειται για την ακολουθία Fibonacci. 
    \end{enumerate}
\end{examples}

\begin{rem}
    Ουσιαστικά οι ακολουθίες είναι λίστες πραγματικών αριθμών.
\end{rem}


\end{document}
