\documentclass[main.tex]{subfiles}


\begin{document}


\section{Όριο Ακολουθίας}

\begin{dfn}
    Ακολουθία πραγματικών αριθμών ονομάζεται κάθε συνάρτηση με πεδίο ορισμού 
    τους φυσικούς αριθμούς. 
    \begin{align*}
        a \colon &\mathbb{N} \to \mathbb{R} \\
                 &n \to a(n)=a_{n}
     \end{align*} 
     Οι ακολουθίες συμβολίζονται ως $ (a_{n})_{n \in \mathbb{N}} $  
      ή $ \{ a_{n} \} _{n \in \mathbb{N}} $  ή $ \{ a_{n} \} _{n=1}^{+\infty} $, κλπ.
\end{dfn}

\begin{dfn}
    Σύνολο Τιμών (Σ.Τ.) της ακολουθίας $ (a_{n})_{n \in \mathbb{N}} $, ονομάζουμε 
    το σύνολο των όρων της, $ \{ a_{1}, a_{2}, \ldots, a_{n} \} $ το οποίο μπορεί 
    να είναι πεπερασμένο ή άπειρο.
\end{dfn}

\begin{examples}
        \item {}
    \begin{enumerate}[i)]
        \item $ a_{n} = n, \; \forall n \in \mathbb{N} $. Έχει Σ.Τ. το σύνολο 
            $  \{ 1,2,3, \ldots \} $.
        \item $\left(\frac{1}{n}\right)_{n \in \mathbb{N}} $. Έχει Σ.Τ. το σύνολο 
            $  \left\{ 1, \frac{1}{2}, \frac{1}{3}, \ldots \right\} $.
        \item $ \{(-1)^{n}\}_{n=1}^{+ \infty}, $. Έχει Σ.Τ. 
            το σύνολο $ \{ -1,1 \} $.
        \item $ a_{n} = c, \; \forall n \in \mathbb{N}, c \in \mathbb{R} $.
            Έχει Σ.Τ. το σύνολο $ \{ c \} $ και ονομάζεται σταθερή ακολουθία.
        \item $ a_{n}=2n, \; \forall n \in \mathbb{N} $. Έχει Σ.Τ. το σύνολο 
            $ \{ 2,4,6, \ldots, 2n, \ldots \} $. Πρόκειται για την ακολουθία 
            των άρτιων φυσικών αριθμών.
        \item $ a_{n}= 2n-1, \; \forall n \in \mathbb{N} $. Έχει Σ.Τ. το 
            σύνολο $ \{ 1,3,5, \ldots, 2n+1, \ldots \} $. Πρόκεται για την 
            ακολουθία των περιττών φυσικών αριθμών.
        \item \label{ex:anadr} $ a_{1}= a_{2} = 1 $ και $ a_{n+2}=a_{n+1}+a_{n}, \; 
            \forall n \in \mathbb{N}$. Έχει Σ.Τ. το σύνολο $ \{ 1,1,2,3,5,8,
            13,21,34, \ldots,\} $. Πρόκειται για την ακολουθία Fibonacci. 
    \end{enumerate}
\end{examples}

\begin{rem}
\item {}
    \begin{enumerate}[i)]
        \item Ουσιαστικά οι ακολουθίες είναι λίστες πραγματικών αριθμών.
        \item Η ακολουθία \ref{ex:anadr}, όπου κάθε επόμενος όρος, ορίζεται με τη 
            βοήθεια του προηγούμενου, λέγεται αναδρομική ακολουθία. Προτάσεις που 
            αφορούν αναδρομικές ακολουθίες, αποδεικνύονται με Μαθηματική Επαγωγή.
    \end{enumerate}
\end{rem}


\begin{dfn}
    Δυο ακολουθίες, $(a_{n})_{n \in \mathbb{N}}$  και $ (b_{n})_{n \in \mathbb{N}} $
    ονομάζονται ίσες, αν $ a_{n} = b_{n}, \; \forall n \in \mathbb{N} $.
\end{dfn}

\begin{dfn}
    Οι πράξεις μεταξύ ακολουθιών, ορίζονται όπως ακριβώς και για τις συναρτήσεις.
\end{dfn}

\section{Φραγμένες Ακολουθίες}

\begin{dfn}
\item {}
    Μια ακολουθία $ (a_{n})_{n \in \mathbb{N}} $ ονομάζεται:
    \begin{enumerate}[i)]
        \item άνω φραγμένη $ \overset{\text{ορ.}}{\Leftrightarrow} \exists M \in 
            \mathbb{R} \; : \; a_{n} \leq M, \; \forall n \in \mathbb{N}$.
        \item κάτω φραγμένη $ \overset{\text{ορ.}}{\Leftrightarrow} \exists m \in 
            \mathbb{R} \; : \; m \leq a_{n}, \; \forall n \in \mathbb{N}  $
        \item φραγμένη $ \overset{\text{ορ.}}{\Leftrightarrow} \exists$ είναι άνω και 
            κάτω φραγμένη.
    \end{enumerate}

    \begin{prop}
        $ (a_{n})_{n \in \mathbb{N}} $ φραγμένη $ \Leftrightarrow \exists M>0, \; M \in 
        \mathbb{R} \; : \; \abs{a_{n}} \leq M, \; \forall n \in \mathbb{N} $.
    \end{prop}
\end{dfn}

\begin{examples}
\item {}  
    \begin{enumerate}[i)]
        \item $ a_{n}= \frac{1}{n}, \; \forall n \in \mathbb{N} $, είναι φραγμένη.
            
            Πράγματι, $ 0 \leq \frac{1}{n} \leq 1, \; \forall n \in \mathbb{N} $. 

            Επίσης, $ \abs{\frac{1}{n}} = \frac{1}{n} \leq 1. $
        \item 
    \end{enumerate}
\end{examples}

\end{document}
