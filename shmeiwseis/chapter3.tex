\documentclass[main.tex]{subfiles}


\begin{document}



\section{Ορισμός}


Έστω $ {(a_{n})}_{n \in \mathbb{N}}$ ακολουθία. Τότε ορίζουμε την ακολουϑία 
\begin{align*}
    S_{1} &= a_{1} \\
    S_{2} &= a_{1}+ a_{2} \\
    \vdots \\
    S_{N} &= a_{1}+ a_{2}+ \cdots + a_{N} \\
    \vdots 
\end{align*}

Η ακολουθία $ {(S_{n})}_{n \in \mathbb{N}} $ ονομάζεται σειρά και συμβολίζεται
\[
    \sum_{n=1}^{\infty} a_{n} = a_{1}+ a_{2}+ \cdots + a_{n} + \cdots   
\] 
Οι όροι της ακολουθίας $ {(S_{n})}_{n \in \mathbb{N}} $ ονομάζονται μερικά αθροίσματα 
της σειράς.

\section{Παραδείγματα}

\begin{enumerate}
    \item $ \sum_{n=1}^{\infty} \frac{1}{n} $, είναι η ακολουθία 
        $ {(S_{n})}_{n \in \mathbb{N}} $, όπου $ S_{1}=1, \; S_{2}=1+ \frac{1}{2}, \; 
        S_{3}= 1 + \frac{1}{2} + \frac{1}{3}, \ldots  $

    \item $ \sum_{n=1}^{\infty} n  $, είναι η ακολουθία ${(S_{n})}_{n \in \mathbb{N}}$,
        όπου $ S_{1}=1, \; S_{2}=1+2, \; S_{3}=1+2+3, \ldots $

    \item $ \sum_{n=1}^{\infty} 1  $, είναι η ακολουθία ${(S_{n})}_{n \in \mathbb{N}}$,
        όπου $ S_{1}=1, \; S_{2}=1+1, \; S_{3}=1+1+1, \ldots $ η οποία προφανώς 
        απειρίζεται θετικά.

    \item $ \sum_{n=1}^{\infty} {(-1)}^{n}  $, είναι η ακολουθία 
        $ {(S_{n})}_{n \in \mathbb{N}} $, όπου $ S_{1}=-1, \; S_{2}=0, \; S_{3}=-1, 
        \ldots $, η οποία προφανώς δεν συγκλίνει, γιατί έχει δύο διαφορετικές, σταθερές 
        συγκλίνουσες υπακολουθίες.
\end{enumerate}

\begin{rems}
\item {}
    \begin{myitemize}
    \item Η σειρά $ \sum_{n=1}^{\infty} \frac{1}{n}  $, λέγεται αρμονική σειρά 
        και αποκλίνει. (απειρίζεται θετικά)
    \item Η σειρά $ \sum_{n=1}^{\infty} \frac{1}{n^{\rho}}  $, λέγεται γενικευμένη 
        αρμονική σειρά και συγκλίνει αν και μόνον αν $ \rho > 1 $.
    \end{myitemize}
\end{rems}

\section{Σύγκλιση Σειρών}

Έστω $ \sum_{n=1}^{\infty} a_{n}  $ σειρά και έστω $ {(S_{n})}_{n \in \mathbb{N}} $ 
η ακολουθία των μερικών αθροισμάτών της. 

Λέμε ότι η σειρά συγκλίνει στον πραγματικό αριθμό $ S \in \mathbb{R} $, ο οποίος 
τότε καλείται άθροισμα της σειράς, αν η ακολουθία των μερικών αθροισμάτων 
της συγκλίνει στον $ S $. Δηλαδή
\[
    \sum_{n=1}^{\infty} a_{n} = S \Leftrightarrow \lim_{n \to \infty} S_{n} = S  
\] 

\section{Τηλεσκοπικές Σειρές}

\begin{dfn}
    Οι σειρές $ \sum_{n=1}^{\infty} a_{n} $ των οποίων 
    οι όροι γράφονται στη μορφή 
    $ a_{n} = b_{n} - b_{n+1}, \; \forall n \in \mathbb{N} $ όπου για την ακολουθία
    $ {(b_{n})}_{n \in \mathbb{N}}$, μπορεί να υπολογιστεί το όριο της, λέγονται 
    τηλεσκοπικές.
\end{dfn}

\begin{examples}
\item {}
    \begin{enumerate}
        \item Να υπολογιστεί το άθροισμα της σειράς $ \sum_{n=1}^{\infty} 
            \frac{1}{n(n+1)} $.
            \begin{proof}
                Έχουμε
                \[
                    a_{n} = \frac{1}{n(n+1)} = \frac{A}{n} + \frac{B}{n+1} = 
                    \frac{1}{n} - \frac{1}{n+1} 
                \]
                επομένως η σειρά γράφεται
                \[
                    \sum_{n=1}^{\infty} \frac{1}{n(n+1)} = 
                    \sum_{n=1}^{\infty} \left(\frac{1}{n} - \frac{1}{n+1}\right) 
                \] 
                Υπολογίζουμε την ακολουθία των μερικών αθροισμάτων και έχουμε
                \begin{align*}
                    S_{1} &= 1- \frac{1}{2} \\
                    S_{2} &= 1- \frac{1}{2} + \frac{1}{2} - \frac{1}{3} = 
                    1 - \frac{1}{3}  \\
                    S_{3} &=  1- \frac{1}{2} + \frac{1}{2} - \frac{1}{3} + 
                    \frac{1}{3} - \frac{1}{4} = 1 - \frac{1}{4}  \\
                    \vdots \\
                    S_{n} &= 1 - \frac{1}{n+1} \\
                    \vdots
                \end{align*}
                Επομένως $ \lim_{n \to \infty} {(S_{n})}_{n \in \mathbb{N}} = 
                \lim_{n \to \infty} \left(1 - \frac{1}{n+1}\right) = 1 - 0 = 1 $
            \end{proof}

        \item Να υπολογιστεί το άθροισμα της σειράς $ \sum_{n=1}^{\infty} 
            \ln{\left(\frac{n}{n+1}\right)} $
            \begin{proof}
                Έχουμε
                \[
                    a_{n} = \ln{\left(\frac{n}{n+1}\right)} = \ln{n} - \ln{(n+1)}  
                \]
                επομένως η σειρά γράφεται
                \[
                    \sum_{n=1}^{\infty} \ln{\left(\frac{n}{n+1} \right)} = 
                    \sum_{n=1}^{\infty} [ \ln{n} - \ln{(n+1)} ]
                \] 
                Υπολογίζουμε την ακολουθία των μερικών αθροισμάτων και έχουμε
                \begin{align*}
                    S_{1} &= \ln{1} - \ln{2} = - \ln{2}  \\
                    S_{2} &= \ln{1} - \ln{2} + \ln{2} - \ln{3} = - \ln{3}  \\
                    S_{3} &= \ln{1} - \ln{2} + \ln{2} - \ln{3} + \ln{3} - \ln{4} = - 
                    \ln{4}\\
                    \vdots \\
                    S_{n} &= - \ln{(n+1)} \\
                    \vdots
                \end{align*}
                επομένως $ \lim_{n \to \infty} {(S_{n})}_{n \in \mathbb{N}} = 
                \lim_{n \to \infty} (- \ln{(n+1)}) = - \infty $ οπότε η σειρά αποκλίνει.
            \end{proof}
    \end{enumerate}
\end{examples}




\begin{prop}\label{prop:diff}
\item {}
    \begin{enumerate}[i)]
        \item $ \sum_{n=1}^{\infty} a_{n}  $ συγκλίνει 
            $ \Leftrightarrow \sum_{n= n_{0}}^{\infty} a_{n}  $ συγκίνει.
        \item\label{prop:diff2} $ \sum_{n=1}^{\infty} a_{n}  $ συγκλίνει 
            $ \Rightarrow \sum_{n=1}^{\infty} a_{n} - \sum_{n= n_{0}}^{\infty} a_{n}  
            = \sum_{n=1}^{n_{0}-1} a_{n} $
    \end{enumerate}
\end{prop}
\begin{proof}

\end{proof}

\begin{example}
    Να υπολογιστεί το άθροισμα της σειράς $ \sum_{n=3}^{\infty} \frac{1}{n(n+1)} $.
\end{example}
\begin{proof}
    \[
        \sum_{n=3}^{\infty} \frac{1}{n(n+1)} = \sum_{n=1}^{\infty} 
        \frac{1}{n(n+1)} - \sum_{n=1}^{2} \frac{1}{n(n+1)} = 1 - \frac{1}{2} - 
        \frac{1}{6} = \frac{2}{6} = \frac{1}{3}
    \] 
\end{proof}


\section{Γεωμετρική Σειρά}

Η σειρά $ \sum_{n=0}^{\infty} x^{n} $ λέγεται Γεωμετρική και ο αριθμός $x$ λέγεται 
λόγος της.

\begin{prop}
    Η σειρά $ \sum_{n=0}^{\infty} x^{n} $ συγκλίνει $ \Leftrightarrow \abs{x} < 1 $
    και ισχύει ότι $ \sum_{n=0}^{\infty} x^{n} = \frac{1}{1-x} $
\end{prop}
\begin{proof}
\item {}
    Ισχύει ότι $ S_{N} = \sum_{n=0}^{N} x^{n} =  1 + x + x^{2} + \cdots x^{N} = 
    \frac{1- x^{n+1}}{1-x} $.

    Όμως $ \lim_{n \to \infty} x^{n} = 0 \Leftrightarrow \abs{x} < 1 $, οπότε 
    $ \lim_{n \to \infty} S_{N} = \frac{1}{1-x} $.
\end{proof}

\begin{cor}
    Η σειρά $ \sum_{n= n_{0}}^{\infty} x^{n} $ συγκλίνει $ \Leftrightarrow \abs{x} <1 $
    και ισχύει ότι $ \sum_{n= n_{0}}^{\infty} x^{n} = \frac{x^{n_{0}}}{1-x} $.
\end{cor}
\begin{proof}
\item {}
    Αν $ \abs{x} < 1 $ τότε η σειρά $ \sum_{n=0}^{\infty} x^{n} $ συγκλίνει ως 
    γεωμετρική και ισχύει $ \sum_{n=0}^{\infty} x^{n} = \frac{1}{1-x} $. 

    Οπότε έχουμε
    \begin{align*}
        \sum_{n=0}^{\infty} x^{n} - \sum_{n= n_{0}}^{\infty} x^{n} = 
        \sum_{n= 1}^{n_{0} -1} x^{n} = 1 + x + x^{2} + \cdots + x^{n_{0}-1} 
        \Rightarrow \\
        \sum_{n= n_{0}}^{\infty} x^{n} = \sum_{n=0}^{\infty} x^{n} - (1 + x + x^{2} + 
        \cdots + x^{n_{0}-1}) = \frac{1}{1-x} - \frac{1- x^{n_{0}}}{1-x} = 
        \frac{x^{n_{0}}}{1-x}  
    \end{align*} 
\end{proof}

\begin{examples}
\item Η σειρά $ \sum_{n=0}^{\infty} \frac{1}{2^{n}} = \sum_{n=0}^{\infty} 
    {\left(\frac{1}{2} \right)}^{n} $ είναι Γεωμετρική με λόγο $ x = \frac{1}{2} $ 
    και επειδή 
    $ \abs{x} = \abs{\frac{1}{2}} = \frac{1}{2} < 1  $, ισχύει ότι η σειρά συγκλίνει.
    Έχουμε
    \[
        \sum_{n=0}^{\infty} {\left(\frac{1}{2} \right)}^{n} = 
        \frac{1}{1 - \frac{1}{2}} = 1
    \] 

\item Η σειρά $ \sum_{n=0}^{\infty} \frac{9^{n}}{10^{n+1}} = 
    \sum_{n=0}^{\infty} \frac{9^{n}}{10 \cdot 10^{n}} = 
    \frac{1}{10} \sum_{n=0}^{\infty} {\left(\frac{9}{10} \right)}^{n} $ είναι
    Γεωμετρική με λόγο $ x = \frac{9}{10} $ και επειδή $ \abs{x} = 
    \abs{\frac{9}{10}} = \frac{9}{10} < 1$, ισχύει ότι η σειρά συγκλίνει.
    Έχουμε 
    \[
        \sum_{n=0}^{\infty}  \frac{9^{n}}{10^{n+1}} = 
        \frac{1}{10} \sum_{n=0}^{\infty} {\left(\frac{9}{10} \right)}^{n} = 
        \frac{1}{10} \cdot \frac{1}{1 - \frac{9}{10}} = \frac{1}{10} \cdot 10 = 1
    \]
\end{examples}

\begin{prop}
    Έστω $ \sum_{n=1}^{\infty} a_{n} = a $ και $ \sum_{n=1}^{\infty} b_{n} = b $ 
    συγκλίνουσες σειρές. Τότε:
    \begin{enumerate}[i)]
        \item $ \sum_{n=1}^{\infty} (a_{n}+b_{n}) = a+b $
        \item $ \sum_{n=1}^{\infty} (k\cdot a_{n}) = k\cdot a $ 
    \end{enumerate}
\end{prop}

\begin{example}
    \begin{enumerate}
        \item Να βρεθεί το άθροισμα της σειράς $ \sum_{n=5}^{\infty} 
            \frac{2^{2n}+5^{n}}{3^{3n}} $
            \begin{proof}
            \item {} 
                Έχουμε
                \begin{align*}
                    \sum_{n=5}^{\infty} \frac{2^{2n}+5^{n}}{3^{3n}} = 
                    \sum_{n=5}^{\infty} \left( \frac{2^{2n}}{3^{3n}} + 
                    \frac{5^{n}}{3^{3n}}\right) = \sum_{n=5}^{\infty} 
                    \left[{\left(\frac{4}{27} \right)}^{n} +
                    {\left(\frac{5}{27} \right)}^{n}\right]
                \end{align*}
                Όμως
                \begin{align*}
                    \sum_{n=5}^{\infty} {\left(\frac{4}{27} \right)}^{n} 
                    &= \frac{4^{5}}{27^{5}} \cdot 
                    \frac{1}{1- \frac{4}{27}} = \frac{4^{5}}{27^{5}} \cdot 
                    \frac{1}{27-4} = \frac{4^{5}}{27^{4}\cdot 23} 
                    \intertext{και}
                    \sum_{n=5}^{\infty} \left(\frac{5}{27} \right)^{n} 
                    &= \frac{5^{5}}{27^{5}} \cdot \frac{1}{1 - \frac{5}{27}} = 
                    \frac{5^{5}}{27^{4}} \cdot \frac{1}{27-5} = 
                    \frac{5^{5}}{27^{4}\cdot 22} 
                \end{align*}
                Επειδή και οι δύο σειρές συγκλίνουν τότε από την προηγούμενη 
                πρόταση έχουμε ότι 
                \[
                    \sum_{n=5}^{\infty} \frac{2^{2n}+5^{n}}{3^{3n}} =  
                    \frac{4^{5}}{27^{4}\cdot 23} + \frac{5^{5}}{27^{4}\cdot 22} 
                \] 
            \end{proof}

        \item Να υπολογιστεί το άθροισμα της σειράς $ \sum_{n=1}^{\infty} 
            \frac{3^{n-1}-1}{6^{n-1}} $.
            \begin{proof}
            \item {}
                Έχουμε 
                \[ 
                    \sum_{n=1}^{\infty} \frac{3^{n-1}-1}{6^{n-1}} = \sum_{n=1}^{\infty} 
                    \left(\frac{3^{n-1}}{6^{n-1}} - \frac{1}{6^{n-1}}\right) = 
                    \sum_{n=1}^{\infty} \left[\left(\frac{3}{6} \right)^{n-1} -
                    \left( \frac{1}{6}\right) ^{n-1}\right] = \sum_{n=1}^{\infty} 
                    \left[\left(\frac{1}{2} \right)^{n-1} - \left(\frac{1}{3} 
                    \right)^{n-1}\right]
                \]
                Όμως 
                \begin{align*}
                    \sum_{n=1}^{\infty} \left(\frac{1}{2} \right)^{n-1} 
                    &= \frac{1}{1- \frac{1}{2}} = 2
                    \intertext{και}
                    \sum_{n=1}^{\infty} \left(\frac{1}{6} \right)^{n-1} 
                    &= \frac{1}{1- \frac{1}{6}} = \frac{6}{5} 
                \end{align*} 
                Επειδή και οι δύο σειρές συγκλίνουν τότε από την προηγούμενη 
                πρόταση έχουμε ότι 
                \[
                    \sum_{n=1}^{\infty} \frac{3^{n_1}-1}{6^{n-1}} =  
                    \sum_{n=1}^{\infty} \left[\left(\frac{1}{2} \right)^{n_1} - 
                    \left(\frac{1}{6} \right)^{n-1}\right] = 2 - \frac{6}{5} = 
                    \frac{4}{5} 
                \] 
            \end{proof}
    \end{enumerate}
\end{example}

\begin{rem}
    Έστω ότι $ \sum_{n=1}^{\infty} a_{n} = a $ και $ \sum_{n=1}^{\infty} b_{n} = b $, 
    με $ b_{n} \neq 0, \; \forall n \in \mathbb{N} $ και $ b \neq 0 $. Τότε η σειρά 
    $ \sum_{n=1}^{\infty} \frac{a_{n}}{b_{n}} $ μπορεί 
    \begin{enumerate}[i)]
        \item Να αποκλίνει 
        \item Να συγκλίνει, και μάλιστα $ \sum_{n=1}^{\infty} \frac{a_{n}}{b_{n}} \neq 
            \frac{a}{b} $
    \end{enumerate}
    \begin{proof}[Παράδειγματα]
    \item {}
        \begin{enumerate}[i)]
            \item Έστω η σειρά $ \sum_{n=1}^{\infty} 
                \underbrace{\frac{1}{n(n+1)}}_{a_{n}} = 1 $ και 
                $ \sum_{n=1}^{\infty} \underbrace{\frac{1}{(n+1)(n+2)}}_{b_{n}} = 
                \frac{1}{2} $. Τότε η σειρά 
                \[
                    \sum_{n=1}^{\infty} \frac{a_{n}}{b_{n}} = \sum_{n=1}^{\infty} 
                    \frac{\frac{1}{n(n+1)}}{\frac{1}{(n+1)(n+2)}} = 
                    \sum_{n=1}^{\infty} \frac{n+2}{n} 
                \]
                αποκλίνει γιατί $ \lim_{n \to \infty} \frac{n+2}{n} = 1 \neq 0 $

            \item Έστω οι σειρές 

                \begin{minipage}{0.75\textwidth}
                    \begin{myitemize}
                    \item 
                        $ \sum_{n=1}^{\infty} \underbrace{\frac{1}{2^{2n-1}}}_{a_{n}} = 2 
                        \sum_{n=1}^{\infty} \frac{1}{4^{n}} = 2 \cdot \sum_{n=1}^{\infty} 
                        \left(\frac{1}{4} \right)^{n} = 2 \cdot \frac{1}{4} \cdot  
                        \frac{1}{1 - \frac{1}{4}} = \frac{1}{2} \cdot \frac{4}{3} = 
                        \frac{2}{3} = a$ \hfill \tikzmark{a}

                    \item $ \sum_{n=1}^{\infty} \underbrace{\frac{3}{2^{n-1}}}_{b_{n}} = 3 
                        \sum_{n=1}^{\infty} \frac{1}{2^{n-1}} = 3 \cdot \sum_{n=1}^{\infty} 
                        \left(\frac{1}{2} \right)^{n-1} = 3 \cdot \frac{1}{1 - 
                        \frac{1}{2}} = 3 \cdot 2 = 6 = b$ \hfill \tikzmark{b}
                    \end{myitemize}
                \end{minipage}

                \mybrace{a}{b}[$ \frac{a}{b} = \frac{\frac{2}{3}}{6} = \frac{1}{9} $]

                Όμως 
                \[
                    \sum_{n=1}^{\infty} \frac{a_{n}}{b_{n}} = \sum_{n=1}^{\infty} 
                    \frac{\frac{1}{2^{2n-1}}}{\frac{3}{2^{n-1}}} = 
                    \frac{1}{3} \sum_{n=1}^{\infty} \frac{2^{n-1}}{2^{2n-1}} = 
                    \frac{1}{3} \sum_{n=1}^{\infty} \frac{2^{n}}{2^{2n}} = 
                    \frac{1}{3} \sum_{n=1}^{\infty} \frac{1}{2^{n}} = \frac{1}{3} \cdot 
                    \frac{\frac{1}{2}}{1 - \frac{1}{2}} = \frac{1}{3} \neq \frac{1}{9} 
                    = \frac{a}{b}
                \] 
        \end{enumerate}

    \end{proof}
\end{rem}

\begin{prop}
    Έστω ότι η σειρά $ \sum_{n=1}^{\infty} a_{n} $ συγκλίνει. Τότε 
    $ \forall \varepsilon >0, \; \exists n_{0} \in \mathbb{N} :
    \quad \abs{\sum_{n= n_{0}}^{\infty} a_{n}} < \varepsilon $
\end{prop}
\begin{proof}
\item {}
    Έστω ότι η σειρά $ \sum_{n=1}^{\infty} a_{n} $ συγκλίνει. Τότε υπάρχει 
    $ S \in \mathbb{R} $ τέτοιος ώστε $ \lim_{n \to \infty} 
    {(S_{n})}_{n \in \mathbb{N}} = S \in \mathbb{R} $. Δηλαδή 
    \[
        \forall \varepsilon >0, \; \exists n_{0} \in \mathbb{N} \; : \; 
        \forall N \geq n_{0} \quad \abs{S_{N}-S} < \varepsilon 
    \] 
    Από την πρόταση~\ref{prop:diff}~\ref{prop:diff2} έχουμε ότι αφού 
    $ \sum_{n=1}^{\infty} a_{n} $ συγκλίνουσα ισχύει
    \begin{gather*}
        \sum_{n=1}^{\infty} a_{n} - \sum_{n= n_{0}+1}^{\infty} a_{n} = 
        \sum_{n= 1}^{n_{0}} a_{n} \Rightarrow \\ 
        \sum_{n= n_{0}+1}^{\infty} a_{n} = \sum_{n=1}^{\infty} a_{n} - 
        \sum_{n=1}^{n_{0}} a_{n} = S - S_{n} \Rightarrow \\
        \abs{\sum_{n= n_{0}}^{\infty} a_{n}} = \abs{S - S_{n}} < \varepsilon
    \end{gather*} 
\end{proof}

\begin{prop}
    Η αρμονική σειρά $ \sum_{n=1}^{\infty} \frac{1}{n} $ δεν συγκλίνει.
\end{prop}
\begin{proof}(Με άτοπο)
\item {}
    Έστω ότι η σειρά συγκλίνει στο $ S \in \mathbb{R} $. Τότε 
    $ \sum_{n=1}^{\infty} \frac{1}{n} 
    = S \in \mathbb{R} \Leftrightarrow \lim_{n \to \infty} {(S_{n})}_{n \in \mathbb{N}} 
    = S $, δηλαδή 
    \[
        \forall \varepsilon >0, \; \exists n_{0} \in \mathbb{N} \; : \; \forall n 
        \geq n_{0} \quad \abs{S_{n}-S} 
        < \varepsilon 
    \]
    Οπότε για $ \varepsilon = \frac{1}{4}, \; \exists n_{0} \in \mathbb{N} \; : \; 
    \forall n \geq n_{0} \quad \abs{S_{n}-S} < \frac{1}{4} $, και 
    $ \abs{S_{2 n_{0}}-S} < \frac{1}{4}  $ επειδή και κάθε  υπακολουθία της 
    $ {(S_{n})}_{n \in \mathbb{N}} $ θα συγκλίνει στο ίδιο όριο.
    Επομένως
    \[
        \abs{S_{2 n_{0} }- S_{n_{0}}} = \abs{S_{2 n_{0} } -S + S - S_{n_{0}}} \leq 
        \abs{S_{2 n_{0}}-S} + \abs{S_{n_{0}}-S} < \frac{1}{4} + \frac{1}{4} = 
        \frac{1}{2}
    \] 
    Από την άλλη μεριά
    \begin{align*}
        \abs{S_{2 n_{0}} - S_{n_{0}}} 
        &= \abs{\left(1+ \frac{1}{2} + \cdots + 
                \frac{1}{2 n_{0}}\right) - \left(1 +
        \frac{1}{2} + \cdots + \frac{1}{n_{0}}\right)} = \frac{1}{n_{0}+1} + \cdots + 
        \frac{1}{2 n_{0}} \geq \underbrace{\frac{1}{2 n_{0}} + \cdots + 
        \frac{1}{2 n_{0}}}_{n_{0} \; \text{φορές}} = \\ 
        &= n_{0}\cdot \frac{1}{2 n_{0}} = \frac{1}{2} \quad \text{άτοπο, άρα 
        η σειρά δεν συγκλίνει.}
    \end{align*}
\end{proof}

\begin{thm}
    Έστω $ \sum_{n=1}^{\infty} a_{n} $ σειρά θετικών όρων, δηλαδή $ a_{n} \geq 0, 
    \; \forall n \in \mathbb{N} $. Τότε η $ \sum_{n=1}^{\infty} a_{n} $ συγκλίνει 
    αν και μόνον αν είναι άνω φραγμένη.
\end{thm}
\begin{proof}
\item {}
    \begin{description}
        \item [($ \Rightarrow $)]
            Έστω ότι η σειρά $ \sum_{n=1}^{\infty} a_{n} $ συγκλίνει. Τότε η ακολουθία 
            $ {(S_{n})}_{n \in \mathbb{N}} $ συγκλίνει, επομένως από γνωστή πρόταση θα 
            είναι και φραγμένη, επομένως και άνω φραγμένη. Άρα και η σειρά 
            $ \sum_{n=1}^{\infty} a_{n} $ είναι άνω φραγμένη.
        \item [($ \Leftarrow $)]
            Έστω ότι η σειρά $ \sum_{n=1}^{\infty} a_{n} $ είναι άνω φραγμένη, δηλαδή 
            η ακολουθία $ {(S_{n})}_{n \in \mathbb{N}} $ είναι άνω φραγμένη, δηλαδή 
            \[
                S_{n} \leq M, \; \forall n \in \mathbb{N}
            \] 
            Θα δείξουμε ότι η ακολουθία $ {(S_{n})}_{n \in \mathbb{N}} $ είναι και 
            αύξουσα, όπότε από γνωστή πρόταση, θα συγκλίνει. Πράγματι
            έστω $ n_{0} \in \mathbb{N} $. Τότε
            \[
                a_{n_{0}+1} \geq 0 \Rightarrow S_{n_{0}+1} - S_{n_{0}} \geq 0 
                \Rightarrow S_{n_{0}+1} \geq S_{n_{0}}
            \] 
    \end{description}
\end{proof}

\begin{cor}
    Μια σειρά θετικών όρων ή συγκλίνει, ή αποκλίνει στο $ + \infty $.  
\end{cor}

\begin{rem}
    Μια σειρά στην οποία δεν γνωρίζουμε τα πρόσημα των όρων της, μπορει:
    \begin{enumerate}[i)]
        \item να συγκλίνει
        \item να αποκλίνει στο $ + \infty $ ή στο $ - \infty $
        \item τίποτε από τα παραπάνω.
    \end{enumerate}
\end{rem}



\section{Κριτήρια Σύγκλισης}

\subsection{Κριτήριο Σύγκρισης}
Έστω ότι $ 0 \leq a_{n} \leq b_{n}, \; \forall n \in \mathbb{N} $, τότε:
\begin{enumerate}[i)]
    \item $ \sum_{n=1}^{\infty} b_{n} $ συγκλίνει $ \Rightarrow \sum_{n=1}^{\infty} 
        a_{n}$ συγκλίνει.
    \item $ \sum_{n=1}^{\infty} a_{n} $ αποκλίνει $ \Rightarrow \sum_{n=1}^{\infty} 
        b_{n} $ αποκλίνει.
\end{enumerate}
\begin{proof}
\item {}
    \begin{enumerate}[i)]
        \item Έστω ότι η σειρά $ \sum_{n=1}^{\infty} b_{n} $ συγκλίνει. Τότε είναι 
            άνω φραγμένη, δηλαδή 
            \[
                \exists M>0 : \sum_{n=1}^{N} b_{N} \leq M, \; \forall N \in \mathbb{N}
                \Rightarrow \sum_{n=1}^{N} a_{N} \leq M, \; \forall N \in \mathbb{N}
            \]
            δηλαδή η $ \sum_{n=1}^{\infty} a_{n} $ είναι άνω φραγμένη. Είναι όμως και 
            σειρά θετικών όρων, οπότε από το προηγούμενο θεώρημα, συγκλίνει.

        \item Έστω $ M >0 $. Αφού η $ \sum_{n=1}^{\infty} a_{n} $ αποκλίνει στο $ 
            + \infty, \; \exists N_{0} \in \mathbb{N} \; : \; \sum_{n=1}^{N} a_{n} 
            \geq M, \; \forall N \in \mathbb{N} $. 
            Όμως τότε $ \sum_{n=1}^{N} b_{n} \geq M, \; \forall n \in \mathbb{N} $, 
            δηλαδή και η σειρά $ \sum_{n=1}^{\infty} b_{n} $ αποκλίνει στο $ 
            + \infty$.
    \end{enumerate}
\end{proof}

\begin{prop}
    Η σειρά $ \sum_{n=1}^{\infty} \frac{1}{n^{2}} $ συγκλίνει.
\end{prop}
\begin{proof}
\item {}
    Θέτουμε $ a_{n} = \frac{1}{(n+1)^{2}}, \; \forall n \in \mathbb{N} $ και 
    $ b_{n} = \frac{1}{n(n+1)}, \; \forall n \in \mathbb{N} $. Ισχύει, προφανώς ότι
    $ 0 \leq a_{n} \leq b_{n}, \; \forall n \in \mathbb{N} $. Επιπλέον έχουμε αποδείξει
    ότι $ \sum_{n=1}^{\infty} \frac{1}{n(n+1)} = 1 $, συγκλίνει, οπότε από κριτήριο 
    σύγκρισης και η σειρά 
    $ \sum_{n=1}^{\infty} a_{n} $ συγκλίνει. Όμως ισχύει ότι
    \[
        \sum_{n=1}^{\infty} a_{n} = \sum_{n=1}^{\infty} \frac{1}{(n+1)^{2}} = 
        \sum_{n=2}^{\infty} \frac{1}{n^{2}} 
    \] 
    δηλαδή η σειρά $ \sum_{n=2}^{\infty} \frac{1}{n^{2}} $ συγκλίνει και από γνωστή 
    πρόταση και η σειρά $ \sum_{n=1}^{\infty} \frac{1}{n^{2}} $ συγκλίνει.
\end{proof}

\begin{examples}
\item {}
    \begin{enumerate}
        \item Η σειρά $ \sum_{n=1}^{\infty} \frac{1}{n(n+1)} $ συγκλίνει. Πράγματι
            \[
                a_{n}= \frac{1}{n(n+1)} = \frac{1}{n^{2}+n} \leq \frac{1}{n^{2}} = 
                b_{n}, \; \forall n \in \mathbb{N}
            \] 
            και η $ \sum_{n=1}^{\infty} b_{n} = \sum_{n=1}^{\infty} \frac{1}{n^{2}} $ 
            συγκλίνει, οπότε από το κριτήριο 
            σύγκρισης και η $ \sum_{n=1}^{\infty} a_{n} = \sum_{n=1}^{\infty} 
            \frac{1}{n(n+1)} $ συγκλίνει. 

        \item Η σειρά $ \sum_{n=1}^{\infty} \frac{n^{2}+n+1}{n^{3}} $ αποκλίνει. Πράγματι
            \[
                a_{n} = \frac{n^{2}+n+1}{n^{3}} \geq \frac{n^{2}}{n^{3}} = \frac{1}{n} = 
                b_{n}, \; \forall n \in \mathbb{N} 
            \] 
            και η $ \sum_{n=1}^{\infty} b_{n} = \frac{1}{n} $ αποκλίνει, οπότε από 
            το κριτήριο σύγκρισης και η $ \sum_{n=1}^{\infty} a_{n} = 
            \sum_{n=1}^{\infty} \frac{n^{2}+n+1}{n^{3}} $ αποκλίνει.
    \end{enumerate}
\end{examples}

\subsection{Κριτήριο Ορίου}
\begin{thm}
    Έστω $ {(a_{n})}_{n \in \mathbb{N}} $ και $ {(b_{n})}_{n \in \mathbb{N}} $ δύο 
    ακολουθίες θετικών αριθμών, τέτοιες ώστε $ \lim_{n \to \infty} \frac{a_{n}}{b_{n}}
    \in \mathbb{R} $, τότε: 
    \begin{enumerate}[i)]
        \item $ \sum_{n=1}^{\infty} b_{n} $ συγκλίνει $ \Rightarrow 
            \sum_{n=1}^{\infty} a_{n} $ συγκλίνει.
        \item $ \sum_{n=1}^{\infty} a_{n} $ αποκλίνει (στο $+ \infty$) $ \Rightarrow 
            \sum_{n=1}^{\infty} b_{n} $ αποκλίνει (στο $+ \infty$).
    \end{enumerate}
\end{thm}
\begin{proof}
\item {}
    Έστω η ακολουθία $ {\left(\frac{a_{n}}{b_{n}}\right)}_{n \in \mathbb{N}} $ 
    συγκλίνει.  Τότε θα είναι και φραγμένη, επομένως
    \[
        \exists M>0 \; : \; \frac{a_{n}}{b_{n}} \leq M, \; \forall n \in \mathbb{N} 
    \] 
    Οπότε $ 0 \leq a_{n} \leq b_{n}, \; \forall n \in \mathbb{N} $.
    \begin{enumerate}[i)]
        \item Η σειρά $ \sum_{n=1}^{\infty} b_{n} $ συγκλίνει, άρα και η σειρά 
            $ \sum_{n=1}^{\infty} M\cdot b_{n} $ συγκλίνει, οπότε από το κριτήριο 
            σύγκρισης συγκλίνει και η $ \sum_{n=1}^{\infty} a_{n} $.
        \item Η σειρά $ \sum_{n=1}^{\infty} a_{n} $ αποκλίνει στο $ + \infty $, 
            οπότε από το κριτήριο σύγκρισης η σειρά 
            $ \sum_{n=1}^{\infty} M \cdot b_{n} $ αποκλίνει
            στο $ + \infty $. Τελικά και η σειρά $ \sum_{n=1}^{\infty} b_{n} $ 
            αποκλίνει στο $ + \infty $.
    \end{enumerate}
\end{proof}

\begin{cor}
    Έστω $ {(a_{n})}_{n \in \mathbb{N}} $ και $ {(b_{n})}_{n \in \mathbb{N}} $ δύο 
    ακολουθίες θετικών αριθμών, τέτοιες ώστε $ \lim_{n \to \infty} \frac{a_{n}}{b_{n}} 
    \in \mathbb{R} \setminus \{ 0 \}$. Τότε η σειρά $ \sum_{n=1}^{\infty} b_{n} $ 
    συγκλίνει αν και μόνον αν η σειρά $ \sum_{n=1}^{\infty} a_{n} $ συγκλίνει.
\end{cor}
\begin{proof}
    Επειδή $ \lim_{n \to \infty} \frac{a_{n}}{b_{n}} \neq 0 \Rightarrow \exists 
    \lim_{n \to \infty} \frac{b_{n}}{a_{n}}$, οπότε από την προηγούμενη πρόταση 
    έχουμε το ζητούμενο.
\end{proof}

\begin{examples}
\item {}
    \begin{enumerate}
        \item Η σειρά $ \sum_{n=1}^{\infty} \frac{n^{5}+8n}{n^{6}+6} $ αποκλίνει.
            \begin{proof}
            \item {}
                \begin{minipage}{0.3\textwidth}
                    \begin{myitemize}
                    \item $ a_{n} = \frac{n^{5}+8n}{n^{6}+6}, \; \forall n \in \mathbb{N} $ 
                        \hfill \tikzmark{a}
                    \item $ b_{n} = \frac{1}{n}, \; \forall n \in \mathbb{N} $
                        \hfill \tikzmark{b}
                    \end{myitemize}    
                \end{minipage}

                \mybrace{a}{b}[$ \frac{a_{n}}{b_{n}} = 
                \frac{\frac{n^{5}+8n}{n^{6}+6}}{\frac{1}{n}} = 
                \frac{n^{6}+8n^{2}}{n^{6}+6} \to 1 $]
                και επειδή η $ \sum_{n=1}^{\infty} b_{n} = \sum_{n=1}^{\infty} \frac{1}{n} $
                αποκλίνει, τότε και η $ \sum_{n=1}^{\infty} a_{n} = \sum_{n=1}^{\infty} 
                \frac{n^{5}+8n}{n^{6}+6}$ αποκλίνει.
            \end{proof}

        \item Η σειρά $ \sum_{n=1}^{\infty} \frac{1}{\sqrt[3]{n^{4}+1}} $ συγκλίνει.
            \begin{proof}
            \item {} 
                \begin{minipage}{0.3\textwidth}
                    \begin{myitemize}
                    \item $ a_{n} = \frac{1}{\sqrt[3]{n^{4}+1}}, \; 
                        \forall n \in \mathbb{N} $ \hfill \tikzmark{a}
                    \item $ b_{n} = \frac{1}{n^{\frac{4}{3}}}, \; 
                        \forall n \in \mathbb{N} $ \hfill \tikzmark{b}
                    \end{myitemize}    
                \end{minipage}

                \mybrace{a}{b}[$ \frac{a_{n}}{b_{n}} =
                \frac{\frac{1}{\sqrt[3]{n^{4}+1}}}{\frac{1}{\sqrt[3]{n^{4}}}} = 
                \sqrt[3]{\frac{n^{4}}{n^{4}+1}} \to 1$]
                και επειδή η $ \sum_{n=1}^{\infty} b_{n} = \sum_{n=1}^{\infty} 
                \frac{1}{n^{\frac{4}{3}}} $ συγκλίνει, τότε και η 
                $ \sum_{n=1}^{\infty} a_{n} = \sum_{n=1}^{\infty} 
                \frac{1}{\sqrt[3]{n^{4}+1}}$ συγκλίνει.
            \end{proof}
    \end{enumerate}
\end{examples}

\begin{prop}
    $ \sum_{n=1}^{\infty} a_{n} $ συγκλίνει $ \Rightarrow \lim_{n \to \infty} a_{n} 
    = 0 $.
\end{prop}
\begin{proof}
\item {}
    Θέτουμε $ S_{N} = \sum_{n=1}^{N} a_{n}, \; \forall N \in \mathbb{N} $. Γνωρίζουμε 
    ότι η $ {(S_{n})}_{n \in \mathbb{N}} $ συγκλίνει, άρα έστω $ \lim_{n \to \infty} S_{N} = S$.
    Τότε $ a_{N} = S_{N}-S_{N-1} \xrightarrow{N \to \infty} S-S = 0 $.
\end{proof}

\begin{cor}[Αντιθετοαντίστροφο]
    $ \lim_{n \to \infty} a_{n} \neq 0 $, ή $ \not \exists \lim_{n \to \infty} a_{n} 
    \Rightarrow \sum_{n=1}^{\infty} a_{n} $ αποκλίνει.
\end{cor}

\begin{examples}
    \begin{enumerate}
        \item Η σειρά $ \sum_{n=1}^{\infty} n^{2} $ δεν συγκλίνει, γιατί 
            $ \lim_{n \to \infty} n^{2} = \infty \neq 0$
        \item Η σειρά $ \sum_{n=1}^{\infty} \frac{n+1}{n} $ δεν συγκλίνει, γιατί 
            $ \lim_{n \to \infty} \frac{n+1}{n} = 1 \neq 0 $
        \item Η σειρά $ \sum_{n=1}^{\infty} (1 + \frac{1}{n} )^{n} $ δεν συγκλίνει, 
            γιατί $ \lim_{n \to \infty} (1+ \frac{1}{n} )^{n} = e \neq 0 $
        \item Η σειρά $ \sum_{n=1}^{\infty} \sin{\left(\frac{n \pi}{2}\right)} $ 
            δεν συγκλίνει, γιατί 

            \begin{minipage}{0.55\textwidth}
                \begin{myitemize}
                \item $ a_{2n} = \sin{(n \pi)} = 0, \; \forall n \in \mathbb{N} 
                    \Rightarrow a_{2n} \xrightarrow{n \to \infty} 0$ \hfill \tikzmark{a}
                \item $ a_{4n+1} = \sin{\left(2n \pi + \frac{\pi}{2}\right)}, \; 
                    \forall n \in \mathbb{N} \Rightarrow a_{4n+1} 
                    \xrightarrow{n \to  \infty} 1 $ \hfill \tikzmark{b}
                \end{myitemize}
            \end{minipage}

            \mybrace{a}{b}[$ {(a_{n})}_{n \in \mathbb{N}} $ δεν συγκλίνει]
            επομένως η $ \sum_{n=1}^{\infty} \sin{\left(\frac{n \pi}{2}\right) } $ 
            αποκλίνει.
    \end{enumerate}
\end{examples}

\section{Απόλυτη Σύγκλιση}

\begin{prop}
    $ \sum_{n=1}^{\infty} \abs{a_{n}} $ συγκλίνει 
    $ \Rightarrow \sum_{n=1}^{\infty} a_{n} $ συγκλίνει 
\end{prop}
\begin{proof}
    $ \abs{a_{n}} $ συγκλίνει $ \Rightarrow \sum_{n=1}^{\infty} 2 \abs{a_{n}} $ 
    συγκλίνει. Ισχύουν
    \[
        0 \leq a_{n} + \abs{a_{n}} \leq 2 \abs{a_{n}}, \; \forall n \in \mathbb{N} 
    \] 
    οπότε από το κριτήριο σύγκρισης, έπεται ότι $ \sum_{n=1}^{\infty} (a_{n}+ 
    \abs{a_{n}}) $ συγκλίνει. Τότε
    \[
        S_{N} = \sum_{n=1}^{N} a_{n} = \sum_{n=1}^{N} (a_{n}+ \abs{a_{n}}) - 
        \sum_{n=1}^{\infty} \abs{a_{n}}   
    \] 
    Από άλγεβρα των ορίων η $ {(S_{n})}_{n \in \mathbb{N}} $ συγκλίνει και άρα 
    έχουμε το ζητούμενο.
\end{proof}

\begin{examples}
\item {}
    \begin{enumerate}
        \item Η σειρά $ \sum_{n=1}^{\infty} (-1)^{n} \frac{1}{n^{2}} $ συγκλίνει, 
            γιατί 
            \[ 
                \sum_{n=1}^{\infty} \abs{a_{n}} = \sum_{n=1}^{\infty} \abs{(-1)^{n}
                \frac{1}{n^{2}}} = \sum_{n=1}^{\infty} \abs{(-1)^{n}} 
                \cdot \abs{\frac{1}{n^{2}} } =  
                \sum_{n=1}^{\infty} \abs{-1}^{n} \cdot 
                \frac{1}{n^{2}} = \sum_{n=1}^{\infty} \frac{1}{n^{2}} 
            \]
            η οποία συγκλίνει.
    \end{enumerate}
\end{examples}

\section{Κριτήριο Συμπύκνωσης του Cauchy}

\begin{prop}
    Έστω $ {(a_{n})}_{n \in \mathbb{N}} $ φθίνουσα ακολουθία θετικών όρων με 
    $ \lim_{n \to \infty} a_{n} =0 $. Τότε
    \[
        \sum_{n=1}^{\infty} a_{n} \; \text{συγκλίνει} \; 
        \Leftrightarrow \sum_{n=1}^{\infty} 2^{n} a_{n} \; \text{συγκλίνει}
    \] 
\end{prop}
\begin{proof}
    Χωρίς Απόδειξη
\end{proof}

\begin{cor}
    Η σειρά $ \sum_{n=1}^{\infty} \frac{1}{n^{\rho}} $ συγκλίνει $ \Leftrightarrow 
    \rho > 1$.
\end{cor}
\begin{proof}
\item {}
    Θέτουμε $ a_{n}= \frac{1}{n^{\rho}}, \; \forall n \in \mathbb{N} $. Η 
    ακολουϑία είναι φθίνουσα και θετική (απο γνωστό λήμμα). Οπότε σύμφωνα 
    με την προηγούμενη πρόταση έχουμε
    \[
        \sum_{n=1}^{\infty} 2^{n} a_{2^{n}} = \sum_{n=1}^{\infty} 2 ^{n} 
        \frac{1}{(2^{n})^{\rho}} = \sum_{n=1}^{\infty} \left(\frac{1}{2^{\rho -1}} 
        \right)^{n}
     \] 
     \begin{myitemize}
     \item Αν $ \rho >1 \Rightarrow \rho -1 >0 \Rightarrow 2^{\rho -1} > 2^{0} 
         \Rightarrow 2^{\rho -1} > 1 \Rightarrow \frac{1}{2^{\rho -1}} < 1 $, οπότε
         η σειρά $ \sum_{n=1}^{\infty} \left(\frac{1}{2^{\rho -1}} \right)^{n} $ 
         συγκλίνει ως γεωμετρική με λόγο $ \frac{1}{2^{\rho -1}} < 1 $
     \item Αν $ \rho \leq 1 \Rightarrow \rho -1 \leq 0 \Rightarrow 2^{\rho -1} \leq 
         2^{0} \Rightarrow 2^{\rho -1} \leq 1 \Rightarrow \frac{1}{2^{\rho -1}} \geq 1 $
         Από αυτό το γεγονός έχουμε ότι η ακολουθία 
         $ \lim_{n \to \infty} \left(\frac{1}{2^{\rho -1}}\right)^{n} $ 
         δεν συγκλίνει στο 0, οπότε η αρχική σειρά δεν συγκλίνει. Και επειδή είναι 
         σειρά θετικών όρων, έχουμε ότι αποκλίνει στο $ + \infty $.
     \end{myitemize}
\end{proof}


\end{document}
