\documentclass[main.tex]{subfiles}



\begin{document}

\section{Ορισμός}

\begin{dfn}
    Έστω $ f \colon A \subset \mathbb{R} \to \mathbb{R} $ συνάρτηση και $ x_{0} \in A $. 
    Η $f$ ονομάζεται συνεχής στο $ x_{0} \in \mathbb{R} $ αν $ \forall \varepsilon >0, 
    \; \exists \delta >0 \; : \forall x \in A \quad \abs{x- x_{0}} < \delta 
    \Rightarrow \abs{f(x) - f(x_{0})} < \varepsilon $
\end{dfn}

\begin{prop}
    Οι σταθερές συναρτήσεις $ f \colon \mathbb{R} \to \mathbb{R} $ με $ f(x)=c, \; c \in 
    \mathbb{R}$ είναι συνεχείς.
\end{prop}
\begin{proof}
\item {}
    Έστω $ \varepsilon >0 $ και $ x_{0} \in \mathbb{R} $. 
    Επιλέγουμε τυχαίο $ \delta >0 $ και έχουμε ότι $ \forall x \in \mathbb{R} $ 
    \[
        \abs{x- x_{0}} < \delta \Rightarrow \abs{f(x)- f(x_{0})} = \abs{c-c} = 0 < 
        \varepsilon 
     \]
\end{proof}

\begin{prop}
    Η ταυτοτική συνάρτηση $ f \colon \mathbb{R} \to \mathbb{R} $ με $ f(x)=x, \; x \in 
    \mathbb{R}$ είναι συνεχής στο $ \mathbb{R} $.
\end{prop}
\begin{proof}
\item {}
    Έστω $ \varepsilon >0 $ και $ x_{0} \in \mathbb{R} $. 
    Επιλέγουμε $ \delta = \varepsilon $ και έχουμε ότι $ \forall x \in \mathbb{R} $ 
    \[
        \abs{x- x_{0}} < \delta \Rightarrow \abs{f(x)-f(x_{0})} = \abs{x- x_{0}} < 
        \delta = \varepsilon
    \]
\end{proof}

\begin{prop}
    Η συνάρτηση $ f \colon \mathbb{R} \to \mathbb{R} $ με $ f(x)= \sin{x}, \; x \in 
    \mathbb{R}$ είναι συνεχής.
\end{prop}
\begin{proof}
\item {}
    Έστω $ \varepsilon >0 $ και $ x_{0} \in \mathbb{R} $. 
    \[
        \abs{\sin{x} - \sin{x_{0}}} = 2 \abs{\sin{\left(\frac{x - x_{0}}{2}\right)} 
        \cdot \cos{\left(\frac{x+ x_{0}}{2}\right)}} \leq 2 
        \abs{\sin{\left(\frac{x- x_{0}}{2}\right)}} \cdot 
        \abs{\cos{\left(\frac{x+ x_{0}}{2}\right)} } \leq 2 \frac{x- x_{0}}{2} = 
    \]
    Επιλέγουμε $ \delta = \varepsilon $ και έχουμε ότι $ \forall x \in \mathbb{R} $ 
\end{proof}
\end{document}
