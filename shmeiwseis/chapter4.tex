\documentclass[main.tex]{subfiles}



\begin{document}

\section{Ορισμός}

\begin{dfn}
    Έστω $ f \colon A \subset \mathbb{R} \to \mathbb{R} $ συνάρτηση και $ x_{0} \in A $. 
    Η $f$ ονομάζεται συνεχής στο $ x_{0} \in \mathbb{R} $ αν $ \forall \varepsilon >0, 
    \; \exists \delta >0 \; : \forall x \in A \quad \abs{x- x_{0}} < \delta 
    \Rightarrow \abs{f(x) - f(x_{0})} < \varepsilon $
\end{dfn}

\begin{prop}
    Οι σταθερές συναρτήσεις $ f \colon \mathbb{R} \to \mathbb{R} $ με $ f(x)=c, \; c \in 
    \mathbb{R}$ είναι συνεχείς.
\end{prop}
\begin{proof}
\item {}
    Έστω $ \varepsilon >0 $ και $ x_{0} \in \mathbb{R} $. 
    Επιλέγουμε τυχαίο $ \delta >0 $ και έχουμε ότι $ \forall x \in \mathbb{R} $ 
    \[
        \abs{x- x_{0}} < \delta \Rightarrow \abs{f(x)- f(x_{0})} = \abs{c-c} = 0 < 
        \varepsilon 
    \]
\end{proof}

\begin{prop}
    Η ταυτοτική συνάρτηση $ f \colon \mathbb{R} \to \mathbb{R} $ με $ f(x)=x, \; x \in 
    \mathbb{R}$ είναι συνεχής στο $ \mathbb{R} $.
\end{prop}
\begin{proof}
\item {}
    Έστω $ \varepsilon >0 $ και $ x_{0} \in \mathbb{R} $. 
    Επιλέγουμε $ \delta = \varepsilon $ και έχουμε ότι $ \forall x \in \mathbb{R} $ 
    \[
        \abs{x- x_{0}} < \delta \Rightarrow \abs{f(x)-f(x_{0})} = \abs{x- x_{0}} < 
        \delta = \varepsilon
    \]
\end{proof}

\begin{prop}
    Η συνάρτηση $ f \colon \mathbb{R} \to \mathbb{R} $ με $ f(x)= \sin{x}, \; x \in 
    \mathbb{R}$ είναι συνεχής.
\end{prop}
\begin{proof}
\item {}
    Έστω $ \varepsilon >0 $ και $ x_{0} \in \mathbb{R} $. 
    \[
        \abs{\sin{x} - \sin{x_{0}}} = 2 \abs{\sin{\left(\frac{x - x_{0}}{2}\right)} 
        \cdot \cos{\left(\frac{x+ x_{0}}{2}\right)}} = 2 
        \abs{\sin{\left(\frac{x- x_{0}}{2}\right)}} \cdot 
        \abs{\cos{\left(\frac{x+ x_{0}}{2}\right)}} \leq 2 \frac{\abs{x- x_{0}}}{2}
    \]
    Επιλέγουμε $ \delta = \varepsilon $ και έχουμε ότι $ \forall \varepsilon >0, \; 
    \exists \delta = \varepsilon > 0 \; : \; \forall x \in \mathbb{R} \quad 
    \abs{x- x_{0}} < \delta $
    \begin{align*}
        \abs{\sin{x} - \sin{x_{0}}} = \cdots \leq \abs{x - x_{0}} < \delta = 
        \varepsilon  
    \end{align*} 
\end{proof}

\begin{thm}[Αρχή Μεταφοράς]
    Μια συνάρτηση $ f \colon A \to \mathbb{R} $ είναι συνεχής στο $ x_{0} \in A $, 
    αν και μόνον αν, για κάθε ακολουθία $ (x_{n})_{n \in \mathbb{N}} $ με

    \vspace{\baselineskip}

    \begin{minipage}{0.25\textwidth}
        \begin{myitemize}
        \item $ x_{n} \in A, \forall n \in \mathbb{N} $ \hfill \tikzmark{a}
        \item $ \lim_{n \to \infty} x_{n} = x_{0} $ \hfill \tikzmark{b}
        \end{myitemize}
    \end{minipage}
    \mybrace{a}{b}[$ \lim_{n \to \infty} f(x_{n}) = f(x_{0}) $]
\end{thm}

\begin{proof}
\item {}
    \begin{description}
        \item [($ \Rightarrow $)]
            Έστω $ f $ συνεχής στο $ x_{0} $ και έστω ακολουθία 
            $ (x_{n})_{n \in \mathbb{N}} $ με $ x_{n} \in A $ και $ x_{n} 
            \xrightarrow{n \to \infty} x_{0} $.
            Θα δείξουμε ότι $ f(x_{n}) \xrightarrow{n \to \infty} x_{0} $.

            Έστω $ \varepsilon >0 $.

            Ζητάμε $ n_{0} \in \mathbb{N} \; : \; 
            \forall n \geq n_{0} \quad \abs{f(x_{n}) - f(x_{0})}<
            \varepsilon $.

            Από την συνέχεια της $f$ στο $ x_{0} $ έχουμε:
            \[
                \exists \delta >0 \; : \; \quad \forall x \in A \quad 
                \abs{x- x_{0}} < \delta \Rightarrow 
                \inlineequation[eq:contres]{\abs{f(x) - f(x_{0}) < \varepsilon}}
            \] 
            Επειδή $ x_{n} \xrightarrow{n \to \infty} x_{0} $ τότε $ \exists n_{0} 
            \in \mathbb{N} \; : \; \forall n \geq n_{0} \quad
            \abs{x_{n} - x_{0}} < \delta $. 

            Οπότε $ \forall n \geq n_{0}$ έχω ότι $ x_{n} \in A $ και 
            $ \abs{x_{n}- x_{0}} < \delta 
            \overset{\eqref{eq:contres}}{\Rightarrow } 
            \abs{f(x_{n}) - f(x_{0})} < \varepsilon $ 

        \item [($ \Leftarrow $)]
            Έστω ότι γιά κάθε ακολουθία $ (x_{n})_{n \in \mathbb{N}} $ 
            $ x_{n} \in A $ και $ x_{n} \xrightarrow{n \to \infty} x_{0} $ 
            έχουμε ότι $ f(x_{n}) \xrightarrow{n \to \infty} f(x_{0}) $.

            Θα δείξουμε ότι η $f$ είναι συνεχής στο $ x_{0} $. (Με άτοπο)

            Έστω ότι $f$ όχι συνεχής στο $ x_{0} $. Οπότε 
            \[
                \exists \varepsilon >0 \; : \; \exists x \in A \quad 
                \abs{x_{\delta} - x_{0}} < \delta \; \text{και} \; 
                \abs{f(x_{\delta})- f(x_{0})} \geq \varepsilon 
            \] 
            Άρα για 
            \begin{myitemize}
            \item $ \delta =1 \; \exists x_{1} \in A \; : \; \abs{x_{1}- x_{0}} < 1 
                $ και $ \abs{f(x_{1}) - f(x_{0})} \geq \varepsilon $
            \item $ \delta =2 \; \exists x_{2} \in A \; : \; \abs{x_{2}- x_{0}} < 
                \frac{1}{2}$ και $ \abs{f(x_{2}) - f(x_{0})} \geq \varepsilon $ 

                \hspace{0.2\textwidth} \vdots 
            \end{myitemize}
            Γενικά για κάθε $ n \in \mathbb{N} $ επιλέγουμε $ x_{n} \in A $ 
            τέτοιο ώστε $ \abs{x_{n}- x_{0}} < \frac{1}{n} $ και 
            $ \abs{f(x_{n}) - f(x_{0})} \geq \varepsilon  $

            Οπότε έχουμε
            \begin{myitemize}
            \item $ x_{n} \in A, \forall n \in \mathbb{N} $ 
            \item $ - \frac{1}{n} < x_{n} - x_{0} < \frac{1}{n}, \; 
                \forall n \in \mathbb{N} 
                \Leftrightarrow x_{0}- \frac{1}{n} < x_{n} < x_{0}+ \frac{1}{n} $ 
            \end{myitemize}
            και $ \lim_{n \to \infty} (x_{0}- \frac{1}{n}) = 
            \lim_{n \to \infty} (x_{0}+ \frac{1}{n}) = x_{0} $ άρα από Κριτήριο 
            Παρεμβολής και $ \lim_{n \to \infty} x_{n} = x_{0} $ και από υπόθεση 
            έχουμε ότι $ \lim_{n \to \infty} f(x_{n}) = f(x_{0}) $.

            Άτοπο γιατι $ \abs{f(x_{n})- f(x_{0})} \geq \delta, 
            \; \forall n \in \mathbb{N}$
    \end{description}
\end{proof}

\begin{prop}
    Έστω $ f,g \colon A \to \mathbb{R} $ συνεχείς στο $ x_{0} \in A $ και έστω 
    $ \lambda \in \mathbb{R} $. Τότε:
    \begin{enumerate}[i)]
        \item $ f+g $ συνεχής στο $ x_{0} $.
        \item $ f\cdot g $ συνεχής στο $ x_{0} $.
        \item $ \lambda f $ συνεχής στο $ x_{0} $.
        \item $ \frac{f}{g} $ συνεχής στο $ x_{0} $, αν $g \neq 0$ ($g$ δεν έχει
            ρίζες) 
    \end{enumerate}
\end{prop}

\begin{proof}
\item {}
    \begin{enumerate}[wide,labelwidth=!,labelindent=0pt]
        \item Έστω $ (x_{n})_{n \in \mathbb{N}} $ στο $A$ με 
            $ x_{n} \xrightarrow{n \to \infty} x_{0} $. Θα δείξουμε ότι 
            $ (f+g)(x) \xrightarrow{n \to \infty} (f+g)(x_{0}) $. 

            \begin{myitemize}
            \item Επειδή $f$ είναι συνεχής στο $ x_{0} $, έχουμε: 
                $ f(x_{n}) \xrightarrow{n \to \infty} f(x_{0}) $
            \item Επειδή $g$ είναι συνεχής στο $ x_{0} $, έχουμε: 
                $ g(x_{n}) \xrightarrow{n \to \infty} g(x_{0}) $
            \end{myitemize}

            Σύμφωνα με τον ορισμό του αρθοίσματος συναρτήσεων 
            \begin{gather*}
                (f+g)(x_{n}) = f(x_{n}) + g(x_{n}), \; \forall n \in \mathbb{N} \\
                (f+g)(x_{0}) = f(x_{0}) + g(x_{0}) 
            \end{gather*} 

            Επίσης $ (f(x_{n})+g(x_{n}))_{n \in \mathbb{N}} $ συγκλίνει ως άθροισμα 
            συγκλινουσών ακολουθιών και μάλιστα 
            \[
                \lim_{n \to \infty} (f(x_{n})+g(x_{n})) = f(x_{0}) + g(x_{0}) 
            \] 
            Άρα $ \lim_{n \to \infty} (f+g)(x_{n}) = (f+g)(x_{0}) $.
    \end{enumerate}
\end{proof}

\begin{prop}
    Η συνάρτηση $ f(x)=x^{2}, \; x \in \mathbb{R} $ είναι συνεχής.
\end{prop}

\begin{proof}
    Από προηγούμενη πρόταση, για $ f(x)=g(x) = x $, συνεχείς, έχουμε ότι     
    $ f\cdot g = x\cdot x = x^{2} $ θα είναι συνεχής.
\end{proof}

\begin{prop}
    Όλες οι πολυωνυμικές συναρτήσεις είναι συνεχείς.
\end{prop}

\begin{proof}
    Έστω $ f \colon \mathbb{R} \to \mathbb{R} $ με τύπο $ f(x) = a_{n} x^{n} + 
    a_{n-1} x^{n-1} + \cdots + a_{1}x + a_{0}$, όπου οι συντελεστές είναι πραγματικοί 
    αριθμοί. 
    \begin{myitemize}
    \item $ f_{0}(x) = a_{0} $, σταθερή αλλά συνεχής.
    \item $ f_{1}(x) = a_{1}x $, αριθμός επί $x$, άρα συνεχής.
    \item $ f_{2}(x) = a_{2}x^{2} $, αριθμός επί $ x^{2} $, άρα συνεχής.
    \item $ f_{3}(x) = a_{3}x^{3}$, αριθμός επί $ x^{3} $, άρα συνεχής.
    \end{myitemize}

    \begin{isx}
        Όλες οι συναρτήσεις της μορφής $ x^{n}, \; \forall n \in \mathbb{N} $ 
        είναι συνεχείς, άρα και το άθροισμά τους, σύμφωνα με την προηγούμενη πρόταση.
    \end{isx}

    \begin{proof}
        Με Μαθηματική Επαγωγή 
    \end{proof}
\end{proof}

\begin{thm}
    Έστω $ f \colon [a,b] \to \mathbb{R} $ συνεχής. Τότε $ \exists m,M \in \mathbb{R} $ 
    τέτοια ώστε $ m \leq f(x) \leq M, \; \forall x \in [a,b] $. Δηλαδή το $ f([a,b]) $ 
    είναι φραγμένο ή ισοδύναμα η $f$ είναι φραγμένη.
\end{thm}

\begin{proof}(Με άτοπο)
\item {}
    Έστω ότι το σύνολο $ f([a,b]) $ δεν είναι άνω φραγμένο.    

    Για κάθε $ n \in \mathbb{N} $ επιλέγω $ x_{n} \in [a,b] $ τέτοιο ώστε 
    $ f(x_{n}) > n $

    Δηλαδή $ a \leq x_{n} \leq b, \; \forall n \in \mathbb{N} $, οπότε η ακολουθία 
    $ (x_{n})_{n \in \mathbb{N}} $ είναι φραγμένη. Άρα από το θεώρημα 
    Bolzano-Weierstrass υπάρχει υπακολουθία της, έστω 
    $ (x_{k_{n}})_{n \in \mathbb{N}} $ συγκλίνουσα, και έστω 
    $ \lim_{n \to \infty} x_{k_{n}} = l \in \mathbb{R} $. 

    Επειδή $ a \leq x_{n} \leq b, \; \forall n \in \mathbb{N} $, έπεται, από γνωστή 
    πρόταση, ότι $ a \leq l \leq b \Rightarrow l \in [a,b]$.

    Επειδή $f$ συνεχής στο $ [a,b] $ θα είναι και συνεχής στο $ l \in [a,b] $, οπότε:
    \[
        f(x_{k_{n}})_{n \in \mathbb{N}} \xrightarrow{n \to \infty} f(l) \Rightarrow 
        (f(x_{k_{n}}))_{n \in \mathbb{N}} \quad \text{φραγμένη}
    \] 
    Άτοπο, γιατί $ f(x_{k_{n}}) > k_{n} > n $.

    Ομοίως για κάτω φραγμένη.
\end{proof}

\begin{examples}
\item {}
    \begin{enumerate}[wide,labelwidth=!,labelindent=0pt]
        \item Έστω η συνάρτηση $ f(x) = [x], \; x \in \mathbb{R} $. 
            \begin{enumerate}[i)]
                \item Η $f$ είναι συνεχής στο $ \frac{1}{2} $.
                \item Η $f$ δεν είναι συνεχής στο $1$.
            \end{enumerate}

            \begin{proof}
            \item {}
                \begin{enumerate}[i)]
                    \item Έχουμε ότι 
                        \[
                            [x] \leq x \leq [x]+1 
                        \] 
                        όπου $ [x], [x] + 1 \in \mathbb{Z} $. 

                        Έστω $ (x_{n})_{n \in \mathbb{N}}$ ακολουθία 
                        πραγματικών αριθμών με 
                        $ \lim_{n \to \infty} x_{n} = \frac{1}{2} $. Θα δείξουμε ότι 
                        \[ \lim_{n \to \infty} f(x_{n}) = 
                        f\left(\frac{1}{2}\right) \] 
                        Πράγματι:
                        \[
                            \lim_{n \to \infty} x_{n} = \frac{1}{2} \Rightarrow 
                            \exists n_{0} \in \mathbb{N} \; : \; \forall n 
                            \geq n_{0} \quad \abs{x_{n} - \frac{1}{2}} < \frac{1}{10} 
                        \]
                        Δηλαδή 
                        \begin{gather*}
                            -\frac{1}{10} < x_{n} - \frac{1}{2} < \frac{1}{10}, 
                            \; \forall n \geq n_{0} \Leftrightarrow \\
                            - \frac{1}{10} + \frac{1}{2} < x_{n} < 
                            \frac{1}{10} + \frac{1}{2}, \; \forall n \geq n_{0} 
                            \Rightarrow  \\ 0 < x <1, \; \forall n \geq n_{0} 
                            \Rightarrow \\ [x_{n}] = 0, \; \forall n \geq n_{0}
                        \end{gather*}
                        Άρα 
                        \[
                            \lim_{n \to \infty} f(x_{n}) = 
                            \lim_{n \to \infty} [x_{n}] = 
                            \lim_{n \to \infty} 0 = 0 = \left[\frac{1}{2} \right] = 
                            f\left(\frac{1}{2}\right), \; \forall n \geq n_{0} 
                        \] 

                    \item 
                        Επιλέγω $ (\tilde{x}_{n})_{ n \in \mathbb{N}} $ 
                        ακολουθία με τύπο 
                        $ \tilde{x}_{n} = 1 - \frac{1}{n}, \; \forall n \in 
                        \mathbb{N} $. Τότε
                        $ \lim_{n \to \infty} \tilde{x}_{n} = 1 $, όμως 
                        \[
                            \lim_{n \to \infty} f(\tilde{x}_{n}) = 
                            \lim_{n \to \infty} 
                            [\tilde{x}_{n}] = \lim_{n \to \infty} 0 = 0 \quad 
                            \text{ενώ} \quad  f(1) = [1] = 1 
                        \] 
                \end{enumerate}
            \end{proof}

        \item Η συνάρτηση $ f \colon \mathbb{R} \to \mathbb{R} $ με τύπο 
            $ f(x) = \chi _{\mathbb{Q}} $, δεν είναι συνεχής σε κανένα σημείο.

            \begin{proof} 
            \item {}
                \begin{description}
                    \item [Α᾽ Τρόπος: (Με αρχή Μεταφοράς)]
                        \[
                            f(x) = \chi _{\mathbb{Q}} = \begin{cases} 1, & x \in 
                            \mathbb{Q} \\ 0, & x \not \in \mathbb{Q}\end{cases} 
                        \]

                        Έστω $ x_{0} \in \mathbb{R} $. Τότε:
                        \begin{myitemize}
                        \item Από πυκνότητα ρητών, υπάρχει ακολουθία ρητών 
                            $ (q_{n})_{ n \in \mathbb{N}} $ με 
                            \[
                                \lim_{n \to \infty} q_{n} = x_{0} \Rightarrow 
                                \lim_{n \to \infty} f(q_{n}) = 
                                \lim_{n \to \infty} 1 = 1  
                            \]

                        \item Από πυκνότητα άρρητων, υπάρχει ακολουθία άρρητων 
                            $ (a_{n})_{n \in \mathbb{N}} $ με
                            \[
                                \lim_{n \to \infty} a_{n} = x_{0} \Rightarrow  
                                \lim_{n \to \infty} f(a_{n}) = 
                                \lim_{n \to \infty} 0 = 0 
                            \]
                        \end{myitemize}
                        Άρα $f$ ασυνεχής στο $ x_{0} $, όμως $ x_{0} $ τυχαίο, οπότε 
                        $f$ ασυνεχής παντού.

                    \item [Β᾽ Τρόπος: (Με άρνηση ορισμού συνέχειας)]
                    \item {}
                        Για $ \varepsilon = \frac{1}{2} > 0 $ θα δείξουμε ότι 
                        \[ 
                            \forall \delta >0, \; \exists x \in \mathbb{R} \; : \; 
                            \abs{x - x_{0}} < \delta  \quad \text{αλλά} \quad
                            \abs{f(x) - f(x_{0})} \geq \frac{1}{2}.
                        \] Πράγματι:
                        Έστω $ \delta > 0 $, τότε:
                        \begin{myitemize}
                        \item Αν $ x_{0} $ ρητός, τότε $ \exists$  άρρητος $a$ (από 
                            πυκνότητα άρρητων) με
                            $ \abs{a - x_{0}} < \delta $ και $ \abs{f(a) - f(x_{0})} = 
                            \abs{0 - 1} = 1 > \frac{1}{2}$
                        \item Αν $ x_{0} $ άρρητος, τότε $ \exists $ ρητός $q$ (από 
                            πυκνότητα ρητών) με 
                            $ \abs{q- x_{0}} < \delta $ και $ \abs{f(q) - f(x_{0})} = 
                            \abs{1 - 0} = 1 > \frac{1}{2}$
                        \end{myitemize}
                        \end{description}
                    \end{proof}
            \end{enumerate}

        \end{examples}

        \begin{prop}
            Έστω $ f \colon A \to \mathbb{R} $ συνεχής σ᾽ ένα σημείο $ x_{0} \in A $ 
            και $ g \colon f(A) \to \mathbb{R} $ συνεχής στο $ f(x_{0}) \in f(A) $. 
            Τότε η συνάρτηση $ g \circ f $ είναι συνεχής στο $ x_{0} \in A $.
        \end{prop}

        \begin{proof}
            Εύκολη με Αρχή Μεταφοράς
        \end{proof}

        \begin{prop}
            Η συνάρτηση $ \cos{} \colon \mathbb{R} \to \mathbb{R} $ είναι συνεχής.
        \end{prop}

        \begin{proof}
        \item {}
            \begin{myitemize}
            \item Ομοίως με την $ \sin{x} $
            \item Ή $ \cos{x} = \pm \sqrt{1- \sin^{2}{x}} $ είναι σύνθεση και 
                πράξεις συνεχών συναρτήσεων, με κατάλληλη προσαρμογή στο 
                κάθε τεταρτημόριο.
            \end{myitemize}
        \end{proof}

        \begin{prop}
            Η συνάρτηση $ f(x) = a^{x}, \; x \in \mathbb{R}, \; a>0 $ είναι συνεχής.
        \end{prop}

        \begin{proof}
        \item {}
            Έστω $ x_{0} \in \mathbb{R} $. Θα δείξουμε ότι η $f$ είναι συνεχής 
            στο $ x_{0} $, με την αρχής της μεταφοράς.

            Θεωρούμε την ακολουθία $ (x_{n})_{n \in \mathbb{N}} $ πραγματικών αριθμών 
            με $ \lim_{n \to \infty} x_{n} = x_{0} $ και θα δείξουμε ότι  
            $\lim_{n \to \infty} a^{x_{n}} = a^{x_{0}}$.

            Επιλέγουμε ρητούς $ q_{n} \in \left(x_{n} - \frac{1}{n}, x_{n}\right), 
            \; \forall n \in \mathbb{N} $ και $ \tilde{q}_{n} \in \left(x_{n}, x_{n}+ 
                \frac{1}{n} \right), \; \forall n \in \mathbb{N} $.

            \begin{description}
                \item [1η Περίπτωση: (a>1)]
                    \begin{gather*}
                        x_{n} - \frac{1}{n} < q_{n} < x_{n}, \; \forall n \in 
                        \mathbb{N} \Rightarrow \lim_{n \to \infty} q_{n} = x_{0} \\
                        x_{n} < \tilde{q}_{n} < x_{n}+ \frac{1}{n}, \; \forall n \in 
                        \mathbb{N} \Rightarrow \lim_{n \to \infty} \tilde{q}_{n} = x_{0}
                    \end{gather*}

                    Οπότε 
                    \[
                        \lim_{n \to \infty} a^{q_{n}} = 
                        \lim_{n \to \infty} a^{\tilde{q}_{n}} = a^{x_{0}}
                    \] 
                    και επίσης
                    \[
                        q_{n} < x_{n} < \tilde{q}_{n}, \; \forall n \in \mathbb{N} 
                        \Rightarrow a^{q_{n}} < a^{x_{n}} < a^{\tilde{q}_{n}}, \; 
                        \forall n \in \mathbb{N}
                    \] 
                    Οπότε από Κριτήριο παρεμβολής και 
                    $ \lim_{n \to \infty} a^{x_{n}} = a^{x_{0}} $ 
                \item [2η Περίπτωση: (0<a<1)]
                    Ομοίως με την 1η Περίπτωση.
                \item [3η Περίπτωση: (a=1)]
                    Αν $ a=1 \Rightarrow $ τότε η $f$ είναι σταθερή, και άρα συνεχής.
            \end{description}
        \end{proof}

        \begin{prop}
            Η συνάρτηση $ f(x) = x^{a}, \; x>0 $ είναι συνεχής $ \forall a 
            \in \mathbb{R} $.
        \end{prop}

        \begin{proof}
            Έστω $ x_{0}>0 $. Θα δείξουμε ότι η $f$ είναι συνεχής στο $ x_{0} $.
            \begin{description}
                \item [1η Περίπτωση: $(a \in \mathbb{N})$]
                    Αν $ a \in \mathbb{N} $ τότε η $ f $ είναι μονώνυμο και άρα είναι συνεχής στο 
                    $ \mathbb{R} $.
                \item [2η Περίπτωση: $(a \in \mathbb{Q}, a>0)$]
                    Αν $ a \in \mathbb{Q}, \; a >0  \Rightarrow a = \frac{\kappa}{\lambda}, \; 
                    \kappa, \lambda \in \mathbb{N} $. Τότε αν $ (x_{n})_{n \in \mathbb{N}} $ 
                    ακολουθία πραγματικών αριθμών με $ \lim_{n \to \infty} x_{n} = x_{0} $, τότε
                    \[
                        \lim_{n \to \infty} {x_{n}}^{\frac{1}{\lambda}} = 
                        {x_{0}}^{\frac{1}{\lambda}} \Rightarrow \lim_{n \to \infty} 
                        {x_{n}}^{\frac{\kappa}{\lambda} } = {x_{0}}^{\frac{\kappa}{\lambda}} 
                        \Rightarrow \lim_{n \to \infty} f(x_{n}) = f(x_{0})
                    \] 
                    όπου πρέπει $ x_{0} \geq 0 $.
                \item [3η Περίπτωση: $(a \in \mathbb{Q}, \; a<0) $]
                    Αν $ a<0 \Rightarrow -a>0 \Rightarrow \lim_{n \to \infty} {x_{n}}^{-a} = 
                    {x_{0}}^{-a} \Rightarrow \lim_{n \to \infty} {x_{n}}^{a} = {x_{0}}^{a} $
                \item [4η Περίπτωση: $ a \in \mathbb{R} \setminus \mathbb{Q} $] 
                    Χωρίς Απόδειξη.
            \end{description}
        \end{proof}

        \begin{thm}
            Έστω $ f \colon [a,b] \to \mathbb{R} $ συνεχής. Τότε $ \exists x_{1}, x_{2} \in 
            [a,b] \; : \; f(x_{1}) \leq f(x) \leq f(x_{2}), \; \forall x \in [a,b]$.
        \end{thm}

        \begin{proof}
            Όπως είδαμε στο θεώρημα Μέγιστης, Ελάχιστης τιμής, το σύνολο $ f([a,b]) $ είναι 
            φραγμένο. Γιατί φραγμένη συνάρτηση, σημαίνει ακριβώς φραγμένο σύνολο τιμών. 

            Από το αξίωμα της Πληρότητας, έχω ότι υπάρχουν τα $ \sup f([a,b]) $ και 
            $ \inf f([a,b]) $. Έστω ότι $ s = \sup f([a,b]) $. Θα δείξουμε ότι το σύνολο 
            $ f([a,b]) $ έχει μέγιστο στοιχείο, δηλαδή ότι $ s \in f([a,b]) $. Ζητάμε 
            $ x_{0} \in [a,b] $ τέτοιο ώστε $ s = f(x_{0}) $.

            Σύμφωνα με γνωστή πρόταση, $ \exists \tilde{x}_{n} \in [a,b], \; 
            \forall n \in \mathbb{N} $ τέτοιο ώστε 
            \inlineequation[eq:asd1]{\lim_{n \to \infty} f(\tilde{x}_{n}) = s}. 
            Η ακολουθία $ (x_{n})_{n \in \mathbb{N}} $ είναι φραγμένη, αφού $ x_{n} \in [a,b], 
            \; \forall n \in \mathbb{N} $, οπότε από το θεώρημα Bolzano-Weierstrass θα έχει 
            συγκλίνουσα υπακολουθία, δηλαδή:
            \[
                \lim_{n \to \infty} {x_{k}}_{n} = l \in [a,b] \Rightarrow 
                \inlineequation[eq:asd2]{\lim_{n \to \infty} f({x_{k}}_{n}) = f(l)}
            \]

            Από τις σχέσεις~\eqref{eq:asd1} και~\eqref{eq:asd2}, έχουμε ότι $ f(l) =s $, 
            δηλαδή το ζητούμενο.

            Ομοίως για το ελάχιστο.
        \end{proof}

        \begin{thm}[Bolzano]
            Έστω $ f \colon [a,b] \to \mathbb{R} $ συνεχής, με $ f(a) \cdot f(b) <0 $. 
            Τότε $ \exists \xi \in (a,b) \; : \; f(\xi) = 0 $.
        \end{thm}

        \begin{proof}
            %TODO
        \end{proof}

        \begin{thm}[Ενδιάμεσης Τιμής]
            ΄Εστω $ f \colon [a,b] \to \mathbb{R} $, συνεχής. Αν επιλέξουμε έναν αριθμό 
            $ \rho $ τέτοιον ώστε $ f(a) < \rho < f(b) $ ή $ f(b) < \rho < f(a) $, τότε 
            $ \rho \in f([a,b]) $.
        \end{thm}

        \begin{proof}
        \item {}
            \begin{description}
                \item [1η Περίπτωση: $(f(a)< \rho < f(b))$]
                \item {}
                    Θεωρούμε συνάρτηση $ F \colon [a,b] \to \mathbb{R} $ τέτοια ώστε $ 
                    F(x) = f(x) - \rho$. Τότε 

                    \begin{minipage}{0.3\textwidth}
                        \begin{myitemize}
                        \item $F(a) = f(a) - \rho < 0 $ \hfill \tikzmark{a}
                        \item $F(b) = f(b) - \rho >0 $ \hfill  \tikzmark{b}
                        \end{myitemize}
                    \end{minipage}
                    \mybrace{a}{b}[$ \exists x_{0} \in (a,b) \; : \; F(x_{0}) = 0 $]

                    Οπότε 
                    $ 
                    f(x_{0}) = \rho \Rightarrow \rho \in f([a,b])
                    $

                \item [2η Περίπτωση: $(f(b)< \rho < f(a))$]
                \item {}
                    Ομοίως
            \end{description}
        \end{proof}

        \begin{dfn}
            Ένα υποσύνολο $I$ των πραγματικών αριθμών, ονομάζεται διάστημα, αν 
            $ [x,y] \subseteq I, \; \forall x,y \in I $ με $ x<y $.
        \end{dfn}

        \begin{example}
            Το σύνολο $ A = (0,2) \cup (3,4) $ δεν είναι διάστημα, γιατί $ 1 \in A $ και 
            $ \frac{7}{2} \in A $, όμως το $ \left[1, \frac{7}{2}\right] \not\subseteq A $.
        \end{example}

        \begin{thm}
            Συνεχής εικόνα διαστήματος είναι διάστημα. (ΘΜΤ με άλλα λόγια)
        \end{thm}

        \begin{prop}
            Η εικόνα κλειστού διαστήματος είναι διάστημα. 
        \end{prop}

        \begin{proof}
            Συνδυασμός του Θεωρήματος Μέσης Τιμής και του θεωρήματος, ότι μια συνεχής σε 
            κλειστό διάστημα συνάρτηση, παίρνει μέγιστη και ελάχιστη τιμή σε αυτό το διάστημα.
        \end{proof}

        \begin{thm}[Σταθερού Σημείου]
            Έστω $ f \colon [0,1] \to [0,1] $ συνεχής. Τότε $ \exists x_{0} \in [0,1] \; 
            : \; f(x_{0}) = x_{0} $.
        \end{thm}

        \begin{proof}
        \item {}
            Αρκεί να δείξουμε ότι καμπύλη $ y= f(x) $ τέμνει γεωμετρικά τη διαγώνιο $ y=x $.
            Δηλαδή, αρκεί να δείξουμε ότι η συνεχής συνάρτηση $ H(x) = f(x)-x, \; x \in [0,1] $, 
            μηδενίζεται στο $ [0,1] $. Πράγματι:
            Αν $ f(0)=0 $ ή $ f(1)= 1 $, τότε έχουμε το ζητούμενο. Οπότε, έστω ότι 
            $ f(0) > 0 $ και $ f(1)<1 $, αφού η $ f $ παίρνει τιμές στο $ [0,1] $. Τότε:

            \begin{minipage}{0.3\textwidth}
                \begin{myitemize}
                \item $ H(0) = f(0) > 0 $ \hfill \tikzmark{a}
                \item $ H(1) = f(1)-1 < 0 $ \hfill \tikzmark{b}
                \end{myitemize}
            \end{minipage}
            \mybrace{a}{b}[$ \exists x_{0} \in (0,1) \; : \; H(x_{0}) = 0$]

            και επομένως $ f(x_{0}) = x_{0} $ και το θεώρημα έχει αποδειχθεί.
        \end{proof}

        \begin{thm}
            Έστω $ f \colon I \to \mathbb{R} $ συνεχής και 1-1. Τότε η $f$ είναι 
            γνησίως αύξουσα ή γνησίως φθίνουσα.
        \end{thm}

        \begin{thm}
            Έστω $ f \colon I \to \mathbb{R} $ συνεχής και 1-1. Τότε η $f$ αντιστρέφεται.
        \end{thm}

        \begin{prop}
            Η συνάρτηση $ e^{x} \colon \mathbb{R} \to (0,+\infty) $ είναι 1-1 και επί.
        \end{prop}

        \begin{proof}
        \item {}
            Η $ e^{x} $ είναι γνησίως αύξουσα, επομένως είναι και 1-1.

            Έστω $ y>0 $. Θα δείξουμε ότι υπάρχει $ x \in \mathbb{R} $ με $ e^{x}=y $. Πράγματι
            \begin{gather*}
                \lim_{n \to \infty} e^{n} = + \infty \Rightarrow \exists n_{1} \in 
                \mathbb{N} \; : e^{n_{1}} > y \\
                \lim_{n \to \infty} \left(\frac{1}{e}\right)^{n} = 0 \Rightarrow \exists n_{2} 
                \in \mathbb{N} \; : e^{- n_{2}} < y 
            \end{gather*} 

            Επομένως 
            \[
                e^{- n_{2}} < y < e^{n_{1}}
            \] 
            και από θεώρημα Ενδιάμεσης Τιμής, έχουμε ότι $ \exists x \in \mathbb{R} \; : \; 
            e^{x} = y$.


        \end{proof}

        \begin{dfn}
            Ορίζουμε $ \log{} \colon (0,+ \infty) \to \mathbb{R} $ με τύπο $ \log{y} = x $, 
            για το μοναδικό $ x \; : \; e^{x} = y $.
        \end{dfn}

        \section{Όριο Συνάρτησης}

        \begin{dfn}
            Έστω $ A \subseteq \mathbb{R} $. Το $ x_{0} $ ονομάζεται σημείο συσσώρευσης του 
            συνόλου $A$, αν υπάρχει ακολουθία $ (x_{n})_{n \in \mathbb{N}} $ τέτοια ώστε 
            \begin{myitemize}
            \item $ x_{n} \in A \setminus \{ x_{0} \} $
            \item $ \lim_{n \to \infty} x_{n} = x_{0} $
            \end{myitemize}
        \end{dfn}

        \begin{examples}
        \end{examples}

        \begin{dfn}
            Έστω $ f \colon A \to \mathbb{R} $ και $ x_{0} $ σημείο συσσώρευσης του $A$. 
            \begin{myitemize}
            \item Το όριο $ \lim_{x \to x_{0}} f(x) $ υπάρχει και είναι ο αριθμός 
                $ l \in \mathbb{R} $ αν 
                \[
                    \forall \varepsilon >0, \; \exists \delta >0 \; : \; \forall x \in A \quad 
                    \abs{x-x0} < \delta \Rightarrow \abs{f(x)-l} < \varepsilon
                \] 

            \item Το όριο $ \lim_{x \to x_{0}} f(x) $ υπάρχει και είναι $ + \infty $, αν
                \[
                    \forall M>0, \; \exists \delta >0 \; : \; \forall x \in A 
                    \quad \abs{x - x_{0}} < \delta \Rightarrow f(x) > M
                \] 

            \item Το όριο $ \lim_{x \to x_{0}} f(x) $ υπάρχει και είναι $ - \infty $, αν
                \[
                    \forall M>0, \; \exists \delta >0 \; : \; \forall x \in A 
                    \quad \abs{x - x_{0}} < \delta \Rightarrow f(x) < -M
                \] 
            \end{myitemize}
        \end{dfn}

        \begin{dfn}
            Έστω $ f \colon A \to \mathbb{R} $ με $A$ όχι άνω φραγμένο.
            \begin{myitemize}
            \item $ \lim_{x \to +\infty} f(x) = l \in \mathbb{R} \Leftrightarrow \forall 
                \varepsilon > 0, \; \exists M>0 \; : \; \forall x > M \quad 
                \abs{f(x)-l} < \varepsilon $ 
            \item $ \lim_{x \to +\infty} f(x) = + \infty \in \mathbb{R} \Leftrightarrow \forall 
                M > 0, \; \exists \overline{M} >0 \; : \; \forall x > \overline{M} \quad 
                f(x) > M $ 
            \end{myitemize}
        \end{dfn}

        \begin{prop}
            Να αποδείξετε ότι $ \lim_{x \to 0} \frac{\sin{x}}{x} = 1 $.
        \end{prop}

        \begin{examples}
        \end{examples}

        \begin{prop}
            Το όριο $ \lim_{x \to \infty} \sin{x} $ δεν υπάρχει.
        \end{prop}

        \section{Παράγωγος Συνάρτησης}

        \begin{dfn}
            Έστω $ f \colon (a,b) \to \mathbb{R} $ συνάρτηση και $ x_{0} \in (a,b) $. Αν 
            υπάρχει το $ \lim_{x \to x_{0}} \frac{f(x) - f(x_{0})}{x - x_{0}}$, τότε λέμε ότι 
            η $f$ είναι παραγωγίσιμη στο $ x_{0} $ και $ f'(x_{0}) = \lim_{x \to x_{0}} 
            \frac{f(x)-f(x_{0})}{x- x_{0}} $.
        \end{dfn}

        \begin{prop}
            Έστω $ f \colon \mathbb{R} \to \mathbb{R} $ με τύπο $ f(x)=c $, σταθερή. Τότε 
            $ f'(x) = 0, \; \forall x \in \mathbb{R} $. 
        \end{prop}

        \begin{proof}

        \end{proof}

        \begin{prop}
            Έστω $ f \colon \mathbb{R} \to \mathbb{R} $ με τύπο $ f(x)=x $. Τότε 
            $ f'(x) = 1, \; \forall x \in \mathbb{R} $. 
        \end{prop}

        \begin{proof}

        \end{proof}

        \begin{prop}
            Έστω $ f \colon \mathbb{R} \to \mathbb{R} $ με τύπο $ f(x)=x^{2} $. Τότε 
            $ f'(x) = 2x, \; \forall x \in \mathbb{R} $. 
        \end{prop}

        \begin{prop}
            Έστω $ f \colon \mathbb{R} \to \mathbb{R} $ με τύπο $ f(x) = \abs{x}, \; 
            x \in \mathbb{R}$. Τότε 
            \[
                f'(x) = \begin{cases} \phantom{-}1, & x>0 \\ -1, & x<0 \end{cases} 
            \] 
            και δεν υπάρχει η παράγωγος στο $0$.
            \end{prop}

            \begin{proof}

            \end{proof}

            \begin{examples}
            \end{examples}

            \begin{thm}
                Έστω $ f,g \colon (a,b) \to \mathbb{R} $ και $ x_{0} \in (a,b) $. Αν οι $ f,g $ 
                είναι παραγωγίσιμες στο $ x_{0} $, τότε:
                \begin{myitemize}
                \item $ (f+g)'(x_{0}) = f'(x_{0}) + g'(x_{0})$
                \item $ (f\cdot g)'(x_{0}) = f'(x_{0}) \cdot g(x_{0}) + g'(x_{0})\cdot f(x_{0}) $
                \item $ \left(\frac{f}{g}\right)'(x_{0}) = \frac{f'(x_{0})
                    \cdot g(x_{0}) - g'(x_{0})\cdot f(x_{0})}{g^{2}(x_{0})} $,
                    με την προυπόθεση ότι η $ \frac{f}{g} $ ορίζεται.
                \end{myitemize}
            \end{thm}

            \begin{proof}

            \end{proof}

            \begin{prop}
                Έστω $ f \colon (a,b) \to \mathbb{R} $ και $ x_{0} \in (a,b) $. Αν η $f$ είναι 
                παραγωγίσιμη στο $ x_{0} $, τότε η $f$ είναι συνεχής στο $ x_{0} $.
            \end{prop}

            \begin{proof}

            \end{proof}

            \begin{prop}
                Έστω $ f(x) = a_{n}x^{n} + \cdots + a_{1}x + a_{0} $ πολυώνυμο. Τότε 
                $ f'(x) = n a_{n}x^{n-1} + \cdots a_{1} $.
            \end{prop}

            \begin{proof}
                Εύκολη (Γ᾽ Λυκείου)
            \end{proof}

            \begin{prop}
                Έστω $ f \colon \mathbb{R} \to (-1,1)$, με τύπο, $ f(x) = \sin{x} $. Τότε 
                $ f'(x) = \cos{x} $.
            \end{prop}

            \begin{proof}

            \end{proof}

            \begin{prop}
                Η εκθετική συνάρτηση $ f \colon \mathbb{R} \to (0,+ \infty) $, με τύπο 
                $ f(x) = e^{x} $ είναι παραγωγίσιμη και ισχύει $ f'(x) = e^{x} $.
            \end{prop}

            \begin{proof}

            \end{proof}

            \section{Ακρότατα Συνάρτησης}

            \begin{dfn}
                Έστω $I$ διάστημα των πραγματικών αριθμών και $ f \colon I \to \mathbb{R} $. 
                Ένα εσωτερικό σημείο $ x_{0} $ του $I$ λέγεται κρίσιμο σημείο για την $f$, αν 
                $ f'(x_{0} ) = 0 $.
            \end{dfn}

            \begin{rem}
                Έστω $ f \colon [a,b] \to \mathbb{R} $ συνεχής συνάρτηση. Γνωρίζουμε ότι η $f$ 
                παίρνει μέγιστη και ελάχιστη τιμή στο $ [a,b] $. Αν $ x_{0} \in [a,b] $ και 
                $ f(x_{0}) = \max f $ ή $ f(x_{0}) = \min f $ τότε συμβαίνει αναγκαστικά 
                κάτι από τα παρακάτω:
                \begin{myitemize}
                \item $ x_{0} = a $ ή $ x_{0} =b $ (άκρα του διαστήματος)
                \item $ x_{0} \in (a,b) $ και $ f'(x_{0} ) = 0 $ (κρίσιμο σημείο)
                \item $ x_{0} \in (a,b) $ και η $f$ δεν είναι παραγωγίσιμη στο $ x_{0} $.
                \end{myitemize}
            \end{rem}

            \begin{examples}
            \end{examples}

            \begin{thm}[Rolle]
                Έστω $ f \colon [a,b] \to \mathbb{R} $. Υποθέτουμε ότι η $ f $ είναι συνεχής 
                στο $ [a,b] $ και παραγωγίσιμη στο $ (a,b) $. Έστω ότι $ f(a)=f(b) $. Τότε
                $ \exists x_{0} \in (a,b) \; : \; f'(x_{0}) = 0 $.
            \end{thm}

            \begin{proof}

            \end{proof}

            \begin{thm}[Μέσης Τιμής]
                Έστω $ f \colon [a,b] \to \mathbb{R} $ συνεχής στο $ [a,b] $ και παραγωγίσιμη 
                στο $ (a,b) $. Τότε $ \exists x_{0} \in (a,b) \; : \; f'(x_{0}) = 
                \frac{f(b)-f(a)}{b-a} $
            \end{thm}

            \begin{proof}
                \begin{rem}
                    Η υπόθεση ότι η $f$ είναι συνεχής στο κλειστό διάστημα $ [a,b] $ είναι 
                    απαραίτητη για να μπορώ να ισχυριστώ ότι $ f(a), f(b) $ επίσης μέγιστα της $f$.
                \end{rem}   
            \end{proof}

            \begin{thm}
                Έστω $ f \colon (a,b) \to \mathbb{R} $ παραγωγίσιμη συνάρτηση. Τότε:
                \begin{myitemize}
                \item $ f'(x_{0}) \geq 0, \; x \in (a,b) \Rightarrow f $ αύξουσα στο $ (a,b) $.
                \item $ f'(x_{0}) > 0, \; x \in (a,b) \Rightarrow f $ γνησίως αύξουσα στο $ (a,b) $.
                \item $ f'(x_{0}) \leq 0, \; x \in (a,b) \Rightarrow f $ φθίνουσα στο $ (a,b) $.
                \item $ f'(x_{0}) < 0, \; x \in (a,b) \Rightarrow f$ γνησίως φθίνουσα στο $ (a,b) $.
                \item $ f'(x_{0}) = 0, \; x \in (a,b) \Rightarrow f $ σταθερή στο $ (a,b) $.
                \end{myitemize}
            \end{thm}




            \end{document}
