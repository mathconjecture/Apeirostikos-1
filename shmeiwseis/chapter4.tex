\documentclass[main.tex]{subfiles}



\begin{document}

\section{Ορισμός}

\begin{dfn}
    Έστω $ f \colon A \subset \mathbb{R} \to \mathbb{R} $ συνάρτηση και $ x_{0} \in A $. 
    Η $f$ ονομάζεται συνεχής στο $ x_{0} \in \mathbb{R} $ αν $ \forall \varepsilon >0, 
    \; \exists \delta >0 \; : \forall x \in A \quad \abs{x- x_{0}} < \delta 
    \Rightarrow \abs{f(x) - f(x_{0})} < \varepsilon $
\end{dfn}

\begin{prop}
    Οι σταθερές συναρτήσεις $ f \colon \mathbb{R} \to \mathbb{R} $ με $ f(x)=c, \; c \in 
    \mathbb{R}$ είναι συνεχείς.
\end{prop}
\begin{proof}
\item {}
    Έστω $ \varepsilon >0 $ και $ x_{0} \in \mathbb{R} $. 
    Επιλέγουμε τυχαίο $ \delta >0 $ και έχουμε ότι $ \forall x \in \mathbb{R} $ 
    \[
        \abs{x- x_{0}} < \delta \Rightarrow \abs{f(x)- f(x_{0})} = \abs{c-c} = 0 < 
        \varepsilon 
    \]
\end{proof}

\begin{prop}
    Η ταυτοτική συνάρτηση $ f \colon \mathbb{R} \to \mathbb{R} $ με $ f(x)=x, \; x \in 
    \mathbb{R}$ είναι συνεχής στο $ \mathbb{R} $.
\end{prop}
\begin{proof}
\item {}
    Έστω $ \varepsilon >0 $ και $ x_{0} \in \mathbb{R} $. 
    Επιλέγουμε $ \delta = \varepsilon $ και έχουμε ότι $ \forall x \in \mathbb{R} $ 
    \[
        \abs{x- x_{0}} < \delta \Rightarrow \abs{f(x)-f(x_{0})} = \abs{x- x_{0}} < 
        \delta = \varepsilon
    \]
\end{proof}

\begin{prop}
    Η συνάρτηση $ f \colon \mathbb{R} \to \mathbb{R} $ με $ f(x)= \sin{x}, \; x \in 
    \mathbb{R}$ είναι συνεχής.
\end{prop}
\begin{proof}
\item {}
    Έστω $ \varepsilon >0 $ και $ x_{0} \in \mathbb{R} $. 
    \[
        \abs{\sin{x} - \sin{x_{0}}} = 2 \abs{\sin{\left(\frac{x - x_{0}}{2}\right)} 
        \cdot \cos{\left(\frac{x+ x_{0}}{2}\right)}} = 2 
        \abs{\sin{\left(\frac{x- x_{0}}{2}\right)}} \cdot 
        \abs{\cos{\left(\frac{x+ x_{0}}{2}\right)}} \leq 2 \frac{\abs{x- x_{0}}}{2}
    \]
    Επιλέγουμε $ \delta = \varepsilon $ και έχουμε ότι $ \forall \varepsilon >0, \; 
    \exists \delta = \varepsilon > 0 \; : \; \forall x \in \mathbb{R} \quad 
    \abs{x- x_{0}} < \delta $
    \begin{align*}
        \abs{\sin{x} - \sin{x_{0}}} = \cdots \leq \abs{x - x_{0}} < \delta = 
        \varepsilon  
    \end{align*} 
\end{proof}

\begin{thm}[Αρχή Μεταφοράς]
    Μια συνάρτηση $ f \colon A \to \mathbb{R} $ είναι συνεχής στο $ x_{0} \in A $, 
    αν και μόνον αν, για κάθε ακολουθία $ (x_{n})_{n \in \mathbb{N}} $ με

    \vspace{\baselineskip}

    \begin{minipage}{0.25\textwidth}
        \begin{myitemize}
        \item $ x_{n} \in A, \forall n \in \mathbb{N} $ \hfill \tikzmark{a}
        \item $ \lim_{n \to \infty} x_{n} = x_{0} $ \hfill \tikzmark{b}
        \end{myitemize}
    \end{minipage}
    \mybrace{a}{b}[$ \lim_{n \to \infty} f(x_{n}) = f(x_{0}) $]
\end{thm}

\begin{proof}
\item {}
    \begin{description}
        \item [($ \Rightarrow $)]
            Έστω $ f $ συνεχής στο $ x_{0} $ και έστω ακολουθία 
            $ (x_{n})_{n \in \mathbb{N}} $ με $ x_{n} \in A $ και $ x_{n} 
            \xrightarrow{n \to \infty} x_{0} $.
            Θα δείξουμε ότι $ f(x_{n}) \xrightarrow{n \to \infty} x_{0} $.

            Έστω $ \varepsilon >0 $.

            Ζητάμε $ n_{0} \in \mathbb{N} \; : \; 
            \forall n \geq n_{0} \quad \abs{f(x_{n}) - f(x_{0})}<
            \varepsilon $.

            Από την συνέχεια της $f$ στο $ x_{0} $ έχουμε:
            \[
                \exists \delta >0 \; : \; \quad \forall x \in A \quad 
                \abs{x- x_{0}} < \delta \Rightarrow 
                \inlineequation[eq:contres]{\abs{f(x) - f(x_{0}) < \varepsilon}}
            \] 
            Επειδή $ x_{n} \xrightarrow{n \to \infty} x_{0} $ τότε $ \exists n_{0} 
            \in \mathbb{N} \; : \; \forall n \geq n_{0} \quad
            \abs{x_{n} - x_{0}} < \delta $. 

            Οπότε $ \forall n \geq n_{0}$ έχω ότι $ x_{n} \in A $ και 
            $ \abs{x_{n}- x_{0}} < \delta 
            \overset{\eqref{eq:contres}}{\Rightarrow } 
            \abs{f(x_{n}) - f(x_{0})} < \varepsilon $ 

        \item [($ \Leftarrow $)]
            Έστω ότι γιά κάθε ακολουθία $ (x_{n})_{n \in \mathbb{N}} $ 
            $ x_{n} \in A $ και $ x_{n} \xrightarrow{n \to \infty} x_{0} $ 
            έχουμε ότι $ f(x_{n}) \xrightarrow{n \to \infty} f(x_{0}) $.

            Θα δείξουμε ότι η $f$ είναι συνεχής στο $ x_{0} $. (Με άτοπο)

            Έστω ότι $f$ όχι συνεχής στο $ x_{0} $. Οπότε 
            \[
                \exists \varepsilon >0 \; : \; \exists x \in A \quad 
                \abs{x_{\delta} - x_{0}} < \delta \; \text{και} \; 
                \abs{f(x_{\delta})- f(x_{0})} \geq \varepsilon 
            \] 
            Άρα για 
            \begin{myitemize}
            \item $ \delta =1 \; \exists x_{1} \in A \; : \; \abs{x_{1}- x_{0}} < 1 
                $ και $ \abs{f(x_{1}) - f(x_{0})} \geq \varepsilon $
            \item $ \delta =2 \; \exists x_{2} \in A \; : \; \abs{x_{2}- x_{0}} < 
                \frac{1}{2}$ και $ \abs{f(x_{2}) - f(x_{0})} \geq \varepsilon $ 

                \hspace{0.2\textwidth} \vdots 
            \end{myitemize}
            Γενικά για κάθε $ n \in \mathbb{N} $ επιλέγουμε $ x_{n} \in A $ 
            τέτοιο ώστε $ \abs{x_{n}- x_{0}} < \frac{1}{n} $ και 
            $ \abs{f(x_{n}) - f(x_{0})} \geq \varepsilon  $

            Οπότε έχουμε
            \begin{myitemize}
            \item $ x_{n} \in A, \forall n \in \mathbb{N} $ 
            \item $ - \frac{1}{n} < x_{n} - x_{0} < \frac{1}{n}, \; 
                \forall n \in \mathbb{N} 
                \Leftrightarrow x_{0}- \frac{1}{n} < x_{n} < x_{0}+ \frac{1}{n} $ 
            \end{myitemize}
            και $ \lim_{n \to \infty} (x_{0}- \frac{1}{n}) = 
            \lim_{n \to \infty} (x_{0}+ \frac{1}{n}) = x_{0} $ άρα από Κριτήριο 
            Παρεμβολής και $ \lim_{n \to \infty} x_{n} = x_{0} $ και από υπόθεση 
            έχουμε ότι $ \lim_{n \to \infty} f(x_{n}) = f(x_{0}) $.

            Άτοπο γιατι $ \abs{f(x_{n})- f(x_{0})} \geq \delta, 
            \; \forall n \in \mathbb{N}$
    \end{description}
\end{proof}

\begin{prop}
    Έστω $ f,g \colon A \to \mathbb{R} $ συνεχείς στο $ x_{0} \in A $ και έστω 
    $ \lambda \in \mathbb{R} $. Τότε:
    \begin{enumerate}[i)]
        \item $ f+g $ συνεχής στο $ x_{0} $.
        \item $ f\cdot g $ συνεχής στο $ x_{0} $.
        \item $ \lambda f $ συνεχής στο $ x_{0} $.
        \item $ \frac{f}{g} $ συνεχής στο $ x_{0} $, αν $g \neq 0$ ($g$ δεν έχει
            ρίζες) 
    \end{enumerate}
\end{prop}

\begin{proof}

\end{proof}

\begin{prop}
    Η συνάρτηση $ f(x)=x^{2}, \; x \in \mathbb{R} $ είναι συνεχής.
\end{prop}

\begin{proof}
    Από προηγούμενη πρόταση, για $ f(x)=g(x) = x $, συνεχείς, έχουμε ότι     
    $ f\cdot g = x\cdot x = x^{2} $ θα είναι συνεχής.
\end{proof}

\begin{prop}
    Όλες οι πολυωνυμικές συναρτήσεις είναι συνεχείς.
\end{prop}

\begin{proof}

\end{proof}

\begin{prop}
    Όλες οι συναρτήσεις της μορφής $ f(x)= x^{n}, \; \forall n \in \mathbb{N} $
    είναι συνεχείς.
\end{prop}

\begin{proof}
    Με Μαθηματική Επαγωγή 
\end{proof}

\begin{thm}
    Έστω $ f \colon [a,b] \to \mathbb{R} $ συνεχής. Τότε $ \exists m,M \in \mathbb{R} $ 
    τέτοια ώστε $ m \leq f(x) \leq M, \; \forall x \in [a,b] $. Δηλαδή το $ f([a,b]) $ 
    είναι φραγμένο ή ισοδύναμα η $f$ είναι φραγμένη.
\end{thm}

\begin{proof}
    
\end{proof}

\begin{prop}
    Έστω $ f \colon A \to \mathbb{R} $ συνεχής σ᾽ ένα σημείο $ x_{0} \in A $ και 
    $ g \colon f(A) \to \mathbb{R} $ συνεχής στο $ f(x_{0}) \in f(A) $. Τότε η 
    συνάρτηση $ g \circ f $ είναι συνεχής στο $ x_{0} \in A $.
\end{prop}

\begin{proof}
    Εύκολη με Αρχή Μεταφοράς
\end{proof}

\begin{prop}
    Η συνάρτηση $ \cos{} \colon \mathbb{R} \to \mathbb{R} $ είναι συνεχής.
\end{prop}

\begin{proof}
    
\end{proof}

\begin{prop}
    Η συνάρτηση $ f(x) = a^{x}, \; x \in \mathbb{R}, \; a>0 $ είναι συνεχής.
\end{prop}

\begin{proof}
    
\end{proof}

\begin{prop}
    Η συνάρτηση $ f(x) = x^{a}, \; x>0 $ είναι συνεχής $ \forall a \in \mathbb{R} $.
\end{prop}

\begin{proof}
    
\end{proof}

\begin{thm}
    Έστω $ f \colon [a,b] \to \mathbb{R} $ συνεχής. Τότε $ \exists x_{1}, x_{2} \in 
    [a,b] \; : \; f(x_{1}) \leq f(x) \leq f(x_{2}), \; \forall x \in [a,b]$.
\end{thm}

\begin{proof}
    
\end{proof}

\begin{thm}[Bolzano]
    Έστω $ f \colon [a,b] \to \mathbb{R} $ συνεχής, με $ f(a) \cdot f(b) <0 $. 
    Τότε $ \exists \xi \in (a,b) \; : \; f(\xi) = 0 $.
\end{thm}

\begin{proof}
    Έστω $ f \colon [a,b] \to \mathbb{R} $ συνεχής. Αν επιλέξουμε έναν αριθμό $ \rho $ 
    τέτοιο ώστε $ f(a) < q \rho < f(b) $ ή $ f(b) < \rho < f(a) $, τότε $ \rho \in 
    f([a,b]) $.
\end{proof}

\begin{proof}
    
\end{proof}

\begin{dfn}
    Ένα υποσύνολο $I$ των πραγματικών αριθμών, ονομάζεται διάστημα, αν 
    $ [x,y] \subseteq I, \; \forall x,y \in I $ με $ x<y $.
\end{dfn}

\begin{example}
    Το σύνολο $ A = (0,2) \cup (3,4) $ δεν είναι διάστημα, γιατί $ 1 \in A $ και 
    $ \frac{7}{2} \in A $, όμως το $ \left[1, \frac{7}{2}\right] \not\subseteq A $.
\end{example}

\begin{thm}
    Συνεχής εικόνα διαστήματος είναι διάστημα. (ΘΜΤ με άλλα λόγια)
\end{thm}

\begin{prop}
   Η εικόνα κλειστού διαστήματος είναι διάστημα. 
\end{prop}

\begin{proof}
    
\end{proof}

\begin{thm}[Σταθερού Σημείου]
    Έστω $ f \colon [0,1] \to [0,1] $ συνεχής. Τότε $ \exists x_{0} \in [0,1] \; 
    : \; f(x_{0}) = x_{0} $.
\end{thm}

\begin{proof}
    
\end{proof}

\begin{thm}
    Έστω $ f \colon I \to \mathbb{R} $ συνεχής και $ 1 - 1 $. Τότε η $f$ είναι 
    γνησίως αύξουσα ή γνησίως φθίνουσα.
\end{thm}

\begin{thm}
    Έστω $ f \colon I \to \mathbb{R} $ συνεχής και $ 1 - 1 $. Τότε η $f$ αντιστρέφεται.
\end{thm}

\begin{proof}
    
\end{proof}

\begin{dfn}
    Ορίζουμε $ \log{} \colon (0,+ \infty) \to \mathbb{R} $ με τύπο $ \log{y} = x $, 
    για το μοναδικό $ x \; : \; e^{x} = y $.
\end{dfn}

\section{Όριο Συνάρτησης}

\begin{dfn}
    Έστω $ A \subseteq \mathbb{R} $. Το $ x_{0} $ ονομάζεται σημείο συσσώρευσης του 
    συνόλου $A$, αν υπάρχει ακολουθία $ (x_{n})_{n \in \mathbb{N}} $ τέτοια ώστε 
    \begin{myitemize}
    \item $ x_{n} \in A \setminus \{ x_{0} \} $
    \item $ \lim_{n \to \infty} x_{n} = x_{0} $
    \end{myitemize}
\end{dfn}

\begin{examples}
\end{examples}

\begin{dfn}
    Έστω $ f \colon A \to \mathbb{R} $ και $ x_{0} $ σημείο συσσώρευσης του $A$. 
    \begin{myitemize}
    \item Το όριο $ \lim_{x \to x_{0}} f(x) $ υπάρχει και είναι ο αριθμός 
        $ l \in \mathbb{R} $ αν 
        \[
            \forall \varepsilon >0, \; \exists \delta >0 \; : \; \forall x \in A \quad 
            \abs{x-x0} < \delta \Rightarrow \abs{f(x)-l} < \varepsilon
        \] 

    \item Το όριο $ \lim_{x \to x_{0}} f(x) $ υπάρχει και είναι $ + \infty $, αν
        \[
            \forall M>0, \; \exists \delta >0 \; : \; \forall x \in A 
            \quad \abs{x - x_{0}} < \delta \Rightarrow f(x) > M
        \] 

    \item Το όριο $ \lim_{x \to x_{0}} f(x) $ υπάρχει και είναι $ - \infty $, αν
        \[
            \forall M>0, \; \exists \delta >0 \; : \; \forall x \in A 
            \quad \abs{x - x_{0}} < \delta \Rightarrow f(x) < -M
        \] 
    \end{myitemize}
\end{dfn}

\begin{dfn}
    Έστω $ f \colon A \to \mathbb{R} $ με $A$ όχι άνω φραγμένο.
    \begin{myitemize}
    \item $ \lim_{x \to +\infty} f(x) = l \in \mathbb{R} \Leftrightarrow \forall 
        \varepsilon > 0, \; \exists M>0 \; : \; \forall x > M \quad 
        \abs{f(x)-l} < \varepsilon $ 
    \item $ \lim_{x \to +\infty} f(x) = + \infty \in \mathbb{R} \Leftrightarrow \forall 
        M > 0, \; \exists \overline{M} >0 \; : \; \forall x > \overline{M} \quad 
        f(x) > M $ 
    \end{myitemize}
\end{dfn}

\begin{prop}
    Να αποδείξετε ότι $ \lim_{x \to 0} \frac{\sin{x}}{x} = 1 $.
\end{prop}

\begin{examples}
\end{examples}

\begin{prop}
    Το όριο $ \lim_{x \to \infty} \sin{x} $ δεν υπάρχει.
\end{prop}

\section{Παράγωγος Συνάρτησης}

\begin{dfn}
    Έστω $ f \colon (a,b) \to \mathbb{R} $ συνάρτηση και $ x_{0} \in (a,b) $. Αν 
    υπάρχει το $ \lim_{x \to x_{0}} \frac{f(x) - f(x_{0})}{x - x_{0}}$, τότε λέμε ότι 
    η $f$ είναι παραγωγίσιμη στο $ x_{0} $ και $ f'(x_{0}) = \lim_{x \to x_{0}} 
    \frac{f(x)-f(x_{0})}{x- x_{0}} $.
\end{dfn}

\begin{prop}
    Έστω $ f \colon \mathbb{R} \to \mathbb{R} $ με τύπο $ f(x)=c $, σταθερή. Τότε 
    $ f'(x) = 0, \; \forall x \in \mathbb{R} $. 
\end{prop}

\begin{proof}
    
\end{proof}

\begin{prop}
    Έστω $ f \colon \mathbb{R} \to \mathbb{R} $ με τύπο $ f(x)=x $. Τότε 
    $ f'(x) = 1, \; \forall x \in \mathbb{R} $. 
\end{prop}

\begin{proof}
    
\end{proof}

\begin{prop}
    Έστω $ f \colon \mathbb{R} \to \mathbb{R} $ με τύπο $ f(x)=x^{2} $. Τότε 
    $ f'(x) = 2x, \; \forall x \in \mathbb{R} $. 
\end{prop}

\begin{prop}
    Έστω $ f \colon \mathbb{R} \to \mathbb{R} $ με τύπο $ f(x) = \abs{x}, \; 
    x \in \mathbb{R}$. Τότε 
    \[
        f'(x) = \begin{cases} \phantom{-}1, & x>0 \\ -1, & x<0 \end{cases} 
    \] 
    και δεν υπάρχει η παράγωγος στο $0$.
\end{prop}

\begin{proof}
    
\end{proof}

\begin{examples}
\end{examples}

\begin{thm}
    Έστω $ f,g \colon (a,b) \to \mathbb{R} $ και $ x_{0} \in (a,b) $. Αν οι $ f,g $ 
    είναι παραγωγίσιμες στο $ x_{0} $, τότε:
    \begin{myitemize}
    \item $ (f+g)'(x_{0}) = f'(x_{0}) + g'(x_{0})$
    \item $ (f\cdot g)'(x_{0}) = f'(x_{0}) \cdot g(x_{0}) + g'(x_{0})\cdot f(x_{0}) $
    \item $ \left(\frac{f}{g}\right)'(x_{0}) = \frac{f'(x_{0})
            \cdot g(x_{0}) - g'(x_{0})\cdot f(x_{0})}{g^{2}(x_{0})} $,
            με την προυπόθεση ότι η $ \frac{f}{g} $ ορίζεται.
    \end{myitemize}
\end{thm}

\begin{proof}
    
\end{proof}

\begin{prop}
    Έστω $ f \colon (a,b) \to \mathbb{R} $ και $ x_{0} \in (a,b) $. Αν η $f$ είναι 
    παραγωγίσιμη στο $ x_{0} $, τότε η $f$ είναι συνεχής στο $ x_{0} $.
\end{prop}

\begin{proof}
    
\end{proof}

\begin{prop}
    Έστω $ f(x) = a_{n}x^{n} + \cdots + a_{1}x + a_{0} $ πολυώνυμο. Τότε 
    $ f'(x) = n a_{n}x^{n-1} + \cdots a_{1} $.
\end{prop}

\begin{proof}
    Εύκολη (Γ᾽ Λυκείου)
\end{proof}

\begin{prop}
    Έστω $ f \colon \mathbb{R} \to (-1,1)$, με τύπο, $ f(x) = \sin{x} $. Τότε 
    $ f'(x) = \cos{x} $.
\end{prop}

\begin{proof}
    
\end{proof}

\begin{prop}
    Η εκθετική συνάρτηση $ f \colon \mathbb{R} \to (0,+ \infty) $, με τύπο 
    $ f(x) = e^{x} $ είναι παραγωγίσιμη και ισχύει $ f'(x) = e^{x} $.
\end{prop}

\begin{proof}
    
\end{proof}

\section{Ακρότατα Συνάρτησης}

\begin{dfn}
    Έστω $I$ διάστημα των πραγματικών αριθμών και $ f \colon I \to \mathbb{R} $. 
    Ένα εσωτερικό σημείο $ x_{0} $ του $I$ λέγεται κρίσιμο σημείο για την $f$, αν 
    $ f'(x_{0} ) = 0 $.
\end{dfn}

\begin{rem}
    Έστω $ f \colon [a,b] \to \mathbb{R} $ συνεχής συνάρτηση. Γνωρίζουμε ότι η $f$ 
    παίρνει μέγιστη και ελάχιστη τιμή στο $ [a,b] $. Αν $ x_{0} \in [a,b] $ και 
    $ f(x_{0}) = \max f $ ή $ f(x_{0}) = \min f $ τότε συμβαίνει αναγκαστικά 
    κάτι από τα παρακάτω:
    \begin{myitemize}
    \item $ x_{0} = a $ ή $ x_{0} =b $ (άκρα του διαστήματος)
    \item $ x_{0} \in (a,b) $ και $ f'(x_{0} ) = 0 $ (κρίσιμο σημείο)
    \item $ x_{0} \in (a,b) $ και η $f$ δεν είναι παραγωγίσιμη στο $ x_{0} $.
    \end{myitemize}
\end{rem}

\begin{examples}
\end{examples}

\begin{thm}[Rolle]
    Έστω $ f \colon [a,b] \to \mathbb{R} $. Υποθέτουμε ότι η $ f $ είναι συνεχής 
    στο $ [a,b] $ και παραγωγίσιμη στο $ (a,b) $. Έστω ότι $ f(a)=f(b) $. Τότε
    $ \exists x_{0} \in (a,b) \; : \; f'(x_{0}) = 0 $.
\end{thm}

\begin{proof}

\end{proof}

\begin{thm}[Μέσης Τιμής]
    Έστω $ f \colon [a,b] \to \mathbb{R} $ συνεχής στο $ [a,b] $ και παραγωγίσιμη 
    στο $ (a,b) $. Τότε $ \exists x_{0} \in (a,b) \; : \; f'(x_{0}) = 
    \frac{f(b)-f(a)}{b-a} $
\end{thm}

\begin{proof}
 \begin{rem}
    Η υπόθεση ότι η $f$ είναι συνεχής στο κλειστό διάστημα $ [a,b] $ είναι 
    απαραίτητη για να μπορώ να ισχυριστώ ότι $ f(a), f(b) $ επίσης μέγιστα της $f$.
\end{rem}   
\end{proof}

\begin{thm}
    Έστω $ f \colon (a,b) \to \mathbb{R} $ παραγωγίσιμη συνάρτηση. Τότε:
    \begin{myitemize}
    \item $ f'(x_{0}) \geq 0, \; x \in (a,b) \Rightarrow f $ αύξουσα στο $ (a,b) $.
    \item $ f'(x_{0}) > 0, \; x \in (a,b) \Rightarrow f $ γνησίως αύξουσα στο $ (a,b) $.
    \item $ f'(x_{0}) \leq 0, \; x \in (a,b) \Rightarrow f $ φθίνουσα στο $ (a,b) $.
    \item $ f'(x_{0}) < 0, \; x \in (a,b) \Rightarrow f$ γνησίως φθίνουσα στο $ (a,b) $.
    \item $ f'(x_{0}) = 0, \; x \in (a,b) \Rightarrow f $ σταθερή στο $ (a,b) $.
    \end{myitemize}
\end{thm}


\end{document}
