\documentclass[main.tex]{subfiles}

\begin{document}



\section{Μαθηματική Επαγωγή}

\mythm{Έστω $ S \subseteq \mathbb{N} $ τέτοιο ώστε:
    \begin{minipage}[t]{0.3\textwidth}
        \begin{enumerate}[i)]
            \item  $ 1 \in S $ \hfill \tikzmark{a}
            \item  $ n \in S \Rightarrow n + 1 \in S $ \hfill \tikzmark{b}
        \end{enumerate} 
    \end{minipage}
\mybrace{a}{b}[$ S = \mathbb{N} $]}


\begin{example}
    Να αποδείξετε ότι $ 4^{n} \geq n^{2}, \; \forall n \in \mathbb{N} $.
\end{example}


\begin{proof}
\item {}
    \begin{myitemize}
        \item Για $ n=1 $, έχω: $ 4^{1} \geq 1^{2} $, ισχύει.
        \item Έστω ότι η ανισότητα ισχύει για $n$, δηλ. 
            $\inlineequation[eq:epagex1]{4^{n} \geq n^{2}}$
        \item Θα δείξουμε ότι ισχύει και για $ n+1 $. Πράγματι:
            \begin{align*}
                4^{n+1} = 4^{n}\cdot 4 \overset{\eqref{eq:epagex1}}{\geq}
                n^{2}\cdot 4 
                = 4n^{2} = n^{2} + 2n^{2} + n^{2} \geq n^{2} + 2n + 1 = (n+1)^{2}.
            \end{align*}
    \end{myitemize}
\end{proof}

\begin{example}
    Να αποδείξετε ότι $ 2^{n} \geq n^{3}, \; \forall n \geq 10 $.
\end{example}

\begin{proof}
\item {}
    \begin{myitemize}
        \item Για $ n=10 $, έχω: $ 2^{10} = 1024 \geq 1000 = 10^{3}  $, 
            ισχύει.
        \item Έστω ότι η ανισότητα ισχύει για $n$, δηλ. 
            $\inlineequation[eq:epagex2]{2^{n} \geq n^{3}}$.
        \item Θα δέιξουμε ότι ισχύει και για $ n+1 $. Πράγματι:
            \begin{align*}
                2^{n+1} = 2^{n} \cdot 2 \overset{\eqref{eq:epagex2}}{\geq} n^{3} 
                \cdot 2 = 2n^{3} = n^{3} + n^{3} &\geq n^{3} + 10n^{2} \\
                                                 &\geq n^{3} + 6n^{2} + 3n^{2} 
                                                 + n^{2} \\ 
                                                 &\geq n^{3} + 3n^{2} + 3n + 1 \\ 
                                                 &= (n+1)^{3}
            \end{align*} 
    \end{myitemize}
\end{proof}

\section{Ανισότητα Bernoulli}
\[
    \boxed{(1+a)^{n} \geq 1 + na, \quad \forall a \geq 1, \; \forall n \in
    \mathbb{N}}
\] 


\begin{proof}
\item {}
    \begin{myitemize}
        \item Για $ n=1 $, έχω: $ (1+a)^{1} = 1+a \geq 1+a = 1 + 1 \cdot a $
        \item Έστω ότι η ανισότητα ισχύει, για $ n $, δηλ. $\inlineequation[eq:bern]
            {(1+a)^{n} \geq 1 + na}$
        \item Θα δείξουμε ότι ισχύει και για $ n+1 $. Πράγματι:
            \begin{align*}
                (1+a)^{n+1} = (1+a)^{n}(1+a) \overset{\eqref{eq:bern}}
                {\underset{a \geq 1}{\geq}}
                (1+na)(1+a) &= 1 + a + na + na^{2} \\
                            &= 1 + (n+1)a + na^{2} \\
                            &\geq 1 + (n+1)a
            \end{align*}

    \end{myitemize}
\end{proof}

\section{Αθροίσματα}

\mydfn{Έστω $ a_{1}, a_{2}, \ldots, a_{n} \in \mathbb{R} $. Ορίζουμε 
    \[ \sum_{n=1}^{N} a_{n} = a_{1} + a_{2} + \cdots + a_{n} \]}

\begin{example}
    Να δείξετε ότι $ \sum_{n=3}^{6} n^{2} = \sum_{n=4}^{7} (n-1)^{2}   $
\end{example}

\begin{proof}
    \begin{align*}
        \sum_{n=4}^{7} (n-1)^{2} &= (4-1)^{2}+(5-1)^{2}+(6-1)^{2}+(7-1)^{2} 
        = 3^{2}+4^{2}+5^{2}+6^{2} = \sum_{n=1}^{3} n^{2} 
    \end{align*}
\end{proof}

\begin{example}
    Να δείξετε ότι $ \sum_{n=3}^{6} n^{2} = \sum_{n=1}^{4} (n+2)^{2}   $
\end{example}

\begin{proof}
    \begin{align*}
        \sum_{n=1}^{4} (n+2)^{2} = (1+2)^{2}+(2+2)^{2}+(3+2)^{2}+(4+2)^{2}=
        3^{2}+4^{2}+5^{2}+6^{2} = \sum_{n=3}^{6} n^{2} 
    \end{align*}
\end{proof}

\begin{example}
    Να δείξετε ότι $ \sum_{n=1}^{5} (2n+1) - \sum_{n=3}^{5} (2n+1)  =  
    \sum_{n=1}^{2} (2n+1) $
\end{example}

\begin{proof}
    \begin{align*}
        \sum_{n=1}^{5} (2n+1)- \sum_{n=3}^{5} (2n+1) = 
        (3+5+7++11) - (7+9+11) = 3 + 5 = \sum_{n=1}^{2} (2n+1) 
    \end{align*}
\end{proof}

\begin{example}
    Να δείξετε ότι $ \sum_{n=1}^{5} \frac{1}{n(n+1)} = 1 - \frac{1}{6}  $
\end{example}

\begin{proof}
    \begin{align*}
        \sum_{n=1}^{5} \frac{1}{n(n+1)} 
        &= \frac{1}{1\cdot 2} + \frac{1}{2 \cdot 3} + \frac{1}{3 \cdot 4} 
        + \frac{1}{4 \cdot 5} + \frac{1}{5 \cdot 6} \\
        &= \left(\frac{1}{1} - \frac{1}{2}\right) + \left(\frac{1}{2} 
        - \frac{1}{3} \right) + \left(\frac{1}{3} - \frac{1}{4}\right) 
        + \left(\frac{1}{4} - \frac{1}{5}\right) +
        \left(\frac{1}{5} - \frac{1}{6}\right)  = 1 - \frac{1}{6}
    \end{align*}
\end{proof}



\begin{example}
    Να δείξετε ότι $ \sum_{n=1}^{100} \frac{1}{n(n+1)} \leq \sum_{n=1}^{100}
    \frac{1}{n^{2}} $.
\end{example}

\begin{proof}
    \[
        \frac{1}{n(n+1)} = \frac{1}{n^{2}+n} \leq \frac{1}{n^{2}} ,\; 
        \forall n \in \mathbb{N} 
    \] 
    Οπότε,
    \[
        \sum_{n=1}^{100} \frac{1}{n(n+1)} \leq \sum_{n=1}^{100} 
        \frac{1}{n^{2}}
    \]
\end{proof}

\section{Απόλυτη Τιμή}

\mydfn{Για κάθε $ a \in \mathbb{R} $, θέτουμε
\[ \abs{a} = \begin{cases} a, & a \geq 0 \\ -a, & a < 0 \end{cases}  \]}

\begin{rem}
    Ισχύει, 
    \begin{myitemize}
        \item  $ \abs{a} \geq 0, \;  \forall a \in \mathbb{R} $
        \item $ a \leq \abs{a}, \; \forall a \in \mathbb{R} $
    \end{myitemize}
\end{rem}

\myprop{Έστω $ \theta > 0 $. Τότε 
    \[
        \abs{a} \leq \theta \Leftrightarrow - \theta \leq a \leq \theta  
\]}

\begin{proof}
\item {}
    \begin{description}
        \item [$(  \Rightarrow ) $] 
            Έστω ότι $ \abs{a} \leq \theta, \; a \in \mathbb{R} $ και 
            $ \theta >0 $. 
            \begin{myitemize}
                \item Έστω $ a \geq 0 $. Τότε $ \abs{a} = a $, οπότε:
                    $
                        0 \leq \abs{a} \leq \theta \Rightarrow 
                        0 \leq a \leq \theta \Rightarrow - \theta 
                        \leq a \leq \theta
                    $ 
                \item Έστω $ a < 0 $. Τότε $ \abs{a} = -a $, οπότε:
                    $
                        0 \leq \abs{a} \leq \theta \Rightarrow 
                        0 < -a \leq \theta \Rightarrow 
                        - \theta \leq a < 0 \Rightarrow 
                        - \theta \leq a \leq \theta 
                    $ 
            \end{myitemize}
        \item [$(\Leftarrow)$] Έστω $ - \theta \leq a \leq \theta $ 
            για $ a \in \mathbb{R} $ και $ \theta >0 $.
            \begin{myitemize}
                \item Έστω $ a \geq 0 $. Τότε $ \abs{a} = a $. Οπότε 
                    $ \abs{a} = a \leq \theta $.
                \item Έστω $ a <0 $. Τότε $ \abs{a} = -a $. Οπότε
                    $ \abs{a} = -a \leq \theta $.
            \end{myitemize}
    \end{description}
\end{proof}

\myprop{Για κάθε $ a, b \in \mathbb{R} $, ισχύει:
    \begin{enumerate}[i)]
        \item $ \abs{a+b} \leq \abs{a} + \abs{b}   $
        \item $ \abs{a} - \abs{b} \leq \abs{a+b}  $
\end{enumerate}}

\begin{proof}
\item {}
    \begin{enumerate}[i)]
        \item \[ a \leq \abs{a} \Rightarrow - \abs{a} \leq a \leq \abs{a}
            \] 
            \[
                b \leq \abs{b} \Rightarrow - \abs{b} \leq b \leq \abs{b} 
            \] 
            Με πρόσθεση, έχουμε: 
            \[
                - (\abs{a} + \abs{b} ) \leq a + b \leq ( \abs{a} + \abs{b}) 
            \] 
            Οπότε από την προηγούμενη πρόταση, ισχύει:
            $ \abs{a+b} \leq \abs{a} + \abs{b} $
        \item Θέτω $ x = a+b $ και $ y = -a $. Για τους $ x,y $ από το $ i) $ 
            ερώτημα, έχουμε:
            \[
                \abs{x+y} \leq \abs{x} + \abs{y} \Rightarrow \abs{b} \leq 
                \abs{a+b} + \abs{a} \Rightarrow \abs{a+b} \geq \abs{b} - \abs{a}
            \] 
    \end{enumerate} 
\end{proof} 

\begin{rem}
    Χρησιμοποιώντας το πρώτο υποερώτημα της προηγούμενης πρότασης, θέτωντας
    όπου $ b = -b $, έχουμε:
    \[
        \abs{a-b} \leq \abs{a} + \abs{b} 
    \] 
    Χρησιμοποιώντας το δεύτερο υποερώτημα της προηγούμενης πρότασης, θέτωντας
    όπου $ b = -b $, έχουμε:
    \[
        \abs{a} - \abs{b} \leq \abs{a-b} 
    \]
    Με εναλλαγή των ρόλων των $ a $ και $ b $, το δεύτερο υποερώτημα, επίσης 
    μας δίνει:
    \[
        \abs{b} - \abs{a} \leq \abs{b+a} 
    \] 
    Οπότε, τελικά έχουμε:
    \[
        \boxed{\abs{\abs{b} - \abs{a}} \leq \abs{a \pm b} \leq \abs{a} + \abs{b}  }
    \]
\end{rem}

\section{Μέγιστο και Ελάχιστο}

\mydfn{Έστω $ A \subseteq \mathbb{R} $. Λέμε ότι το $A$ έχει μέγιστο 
    (αντίστοιχα ελάχιστο) στοιχείο, αν υπάρχει $ x_{0} \in A $ τέτοιο 
    ώστε $ x \leq x_{0}, \; \forall x \in A $ (αντίστοιχα $ x_{0} \leq 
x, \; \forall x \in A$).}

\begin{rem}
\item {}
    \begin{myitemize}
        \item Το μέγιστο του συνόλου $A$, συμβολίζεται: $ \max A $
        \item Το ελάχιστο του συνόλου $A$, συμβολίζεται $ \min A $
    \end{myitemize}
\end{rem}

\begin{example}
    Έστω $ A = [0,3] = \{ x \in \mathbb{R} \; : \; 0 \leq x \leq 3 \} $. 
    Το $ x_{0}= 3 $ είναι μέγιστο του $A$, γιατί $ 3 \in A $ και 
    $ a \leq 3, \; \forall a \in A $. Ομοίως $ x_{0}= 0 $ είναι ελάχιστο 
    του $A$.
\end{example}

\begin{example}
    Να δείξετε ότι το σύνολο $ A = (0,3) $ δεν έχει μέγιστο στοιχείο.
\end{example}

\begin{proof}
\item {}
    $ A = (0,3) = \{ x \in \mathbb{R} \; : \; 0 < x < 3 \} $. 
    Έστω ότι $ x_{0} = \max A $. Έχουμε  $ x_{0} \in A \Rightarrow  x_{0} 
    < 3$. Άρα  $ (x_{0}, 3) \neq \emptyset \Rightarrow (x_{0},3) \cap A \neq 
    \emptyset \Rightarrow \exists a \in (x_{0},3) \cap A $. Δηλαδή $ 
    \exists a \in A$ τέτοιο ώστε $ a > x_{0} $. Άτοπο, γιατί $ x_{0}= \max A $.
\end{proof}


\myprop{Αν $ A \subseteq \mathbb{R} $ έχει μέγιστο στοιχείο, τότε αυτό είναι 
μοναδικό.}


\begin{proof}
    Έστω ότι το $A$ έχει δυο μέγιστα στοιχεία,  $ x_{0}, {x_{0}}' $,  με 
    $ x_{0} \neq {x_{0}}'$. 

    Τότε 
    \begin{myitemize}
        \item $ a \leq x_{0}, \; \forall a \in A \xRightarrow{{x_{0}}' 
            \in A } {x_{0}}'  \leq x_{0} $ \tikzmark{a}
        \item $ a \leq {x_{0}}', \; \forall a \in A \xRightarrow{x_{0} 
            \in A } x_{0} \leq {x_{0}}' $ \tikzmark{b}
            \mybrace{a}{b}[$ x_{0} = {x_{0}}' $, άτοπο.] 
    \end{myitemize}
\end{proof}

\myprop{Αν $ A \subseteq \mathbb{R} $ έχει ελάχιστο στοιχείο, τότε αυτό είναι 
μοναδικό.}

\begin{proof}
    Ομοίως 
\end{proof}

\mydfn{Έστω $ A \subseteq \mathbb{R}, \; A \neq \emptyset $. Λέμε ότι το 
    $A$ είναι άνω φραγμένο (αντίστοιχα κάτω φραγμένο), αν $ 
    \exists x_{0} \in \mathbb{R}$ τέτοιο ώστε $ a \leq x_{0}, \; 
\forall a \in A$ (αντίστοιχα $ x_{0} \leq a, \; \forall a \in A $).}

\begin{example}
\item {}
    \begin{enumerate}[i)]
        \item Το  σύνολο $ A = (0,3) $  έχει ως άνω φράγματα τους αριθμούς 
            $ 3, 4, 144, \ldots$ και κάτω φράγματα τους 
            $ 0, -1, -2, -128, \ldots $ 

        \item Το σύνολο $ B = (0,+\infty) $ δεν είναι άνω φραγμένο.
    \end{enumerate}
\end{example}

\begin{rem}
    Το  $ 3 $ είναι το ελάχιστο από τα άνω φράγματα του $A$, ενώ το $ 0
    $ είναι το μέγιστο από τα κάτω φράγματα του $A$. 
\end{rem}


\mydfn{Έστω $ A \subseteq \mathbb{R}, \; A \neq \emptyset $. Λέμε ότι το $A$ 
είναι φραγμένο, αν είναι άνω και κάτω φραγμένο.}

\myprop{Ένα σύνολο $ A \subseteq \mathbb{R}, \; A \neq \emptyset $ είναι 
    φραγμένο αν και μόνο αν $ \exists M>0 \; : \; \abs{a} \leq M, \; 
\forall a \in A$.}

\begin{proof}
\item {}
    \begin{description}
        \item [$ (\Rightarrow) $] Έστω $A$ φραγμένο. Τότε 
            $ \exists m',M' \in \mathbb{R} \; : \; m' \leq a \leq M', \; 
            \forall a \in A $.

            Επιλέγω $ M = \max \{ \abs{m'}, \abs{M'} \} $. Τότε $ M >0 $ και 
            \begin{gather*}
                -M \leq - \abs{m'} \leq m' \leq a \leq M' \leq \abs{M'} 
                \leq M, \quad \forall a \in A \\
                -M \leq a \leq M, \quad \forall a \in A \\
                \abs{a} \leq M, \quad \forall a \in A
            \end{gather*}

        \item [(Β΄ τρόπος)]
            Επιλέγω $ M > \max \{ \abs{m'}, \abs{M'} \} $. Τότε $ M >0 $ και 
            \begin{gather*}
                \abs{m'} < M \Leftrightarrow - M < m' < M \\
                \abs{M'} < M \Leftrightarrow -M < M' < M 
            \end{gather*}
            οπότε: 
            \begin{gather*}
                -M < m' \leq a \leq M' < M, \quad a \in A \\
                \abs{a} \leq M, \quad a \in A
            \end{gather*}


        \item [$ (\Leftarrow) $]
            Έστω ότι $ \exists M>0 \; : \; \abs{a} \leq M, \; \forall a \in 
            A \Leftrightarrow -M \leq a \leq M, \; \forall a \in A$. 

            Τότε προφανώς έχουμε ότι $-M $ και $ M $, 
            είναι αντίστοιχα άνω και κάτω φράγματα του $A$, και άρα $A$ 
            φραγμένο.
    \end{description} 
\end{proof}

\myprop{Έστω $ A \subseteq \mathbb{R}, A \neq \emptyset $ και έστω 
    $ -A = \{ x \in \mathbb{R} \; : \; x = -a, \; a \in A \} = \{ 
    -a \; : \; a \in A\} $. Αν $M$ είναι άνω φράγμα 
του $A$, τότε το $ -M $ είναι κάτω φράγμα του συνόλου $ -A $.}

\begin{proof}
    Έστω $M$ α.φ. του $A$. Τότε 
    \begin{align*}
        a &\leq M, \; \forall a \in A \Rightarrow  \\
        -a &\geq -M, \; \forall a \in A \xRightarrow{x= -a, a \in A} \\
        x &\geq -M, \; \forall x \in -A
    \end{align*}
    οπότε $ -M $  κ.φ. του $ -A $.
\end{proof}

\section{Supremum και Infimum}

\mydfn{Έστω $ A \subseteq \mathbb{R}, A \neq \emptyset $ και $A$ άνω 
    φραγμένο. Αν υπάρχει άνω φράγμα $s$ του $A$ τέτοιο ώστε 
    \[
        s \leq M, \quad \text{για κάθε M άνω φράγμα του A}, 
    \] 
τότε το $s$ ονομάζεται supremum του $A$ και συμβολίζεται $ s=\sup A $.}

\mydfn{Έστω $ A \subseteq \mathbb{R}, A \neq \emptyset $ και $A$ κάτω 
    φραγμένο. Αν υπάρχει κάτω φράγμα $s$ του $A$ τέτοιο ώστε 
    \[
        s \geq m, \quad \text{για κάθε m κάτω φράγμα του A}, 
    \] 
τότε το $s$ ονομάζεται infimum του $A$ και συμβολίζεται $ s=\inf A $.}

\begin{rem}
\item {}
    \begin{myitemize}
        \item Αν το $A$ δεν είναι άνω φραγμένο, τότε γράφουμε ότι $ \sup A = 
            + \infty $.
        \item Αν το $A$ δεν είναι κάτω φραγμένο, τότε γράφουμε ότι $ \inf A = 
            - \infty $.
    \end{myitemize}
\end{rem}

\begin{example}
    Έστω $ A = (0,3) $. Θα δείξουμε ότι $ \sup A = 3 $. Πράγματι, 
    το $ 3 $ προφανώς είναι α.φ. του $A$. Θα δείξουμε ότι είναι το
    ελάχιστο από τα άνω φράγματα του $A$. Έστω ότι δεν είναι, δηλαδή 
    $ \exists M $ α.φ. του $A$, ώστε $ M < 3 \Rightarrow (M,3) \neq 
    \emptyset \Rightarrow (M,3) \cap A \neq \emptyset \Rightarrow \exists 
    a \in (M,3) \cap A $. Τότε, έχουμε ότι $ a \in A $ και $ a > M $, 
    άτοπο, γιατί $M$ είναι ά.φ. του $A$. Ομοίως αποδεικνύουμε ότι $ 
    \inf A = 0$.
\end{example}

\section{Αξίωμα Πληρότητας}

Κάθε μη κενό και άνω φραγμένο υποσύνολο των πραγματικών αριθμών έχει 
supremum, δηλαδή έχει ελάχιστο άνω φράγμα.

\myprop{\label{prop:epsilon_prot}
    Έστω ότι για τον $a \in \mathbb{R}$ ισχύει ότι 
    \[
        0 \leq a < \varepsilon, \quad \forall \varepsilon >0 \Rightarrow a =0 
\]}



\begin{proof}
\item {}
    Έστω ότι $ 0 \leq a < \varepsilon, \quad \varepsilon >0 $ και 
    $ a \neq 0 \xRightarrow{a \geq 0} a > 0$. Τότε για $ \varepsilon = 
    a > 0$, έχουμε ότι $ 0 \leq a < a $, άτοπο.
\end{proof}

\myprop{
    Έστω $ (a_{n})_{n \in \mathbb{N}} $ ακολουθία, και έστω $ a \in \mathbb{R} $, ώστε:

    \vspace{\baselineskip}

    \begin{minipage}{0.3\textwidth}
        \begin{enumerate}[i)]
            \item $ 0 \leq a < a_{n}, \; \forall n \in \mathbb{N} $ \hfill \tikzmark{a}
            \item $ \lim_{n \to +\infty} a_{n} = 0 $ \hfill \tikzmark{b}
        \end{enumerate}
\end{minipage}

\mybrace{a}{b}[$ a = 0 $]}


\myprop{
    \begin{enumerate}[i)]
        \item Το $ \mathbb{N} $ δεν είναι άνω φραγμένο.
        \item $ \forall y > 0, \; \exists n \in \mathbb{N} \; ; \; 
            n > y$
        \item $ \forall \varepsilon >0, \; \exists n \in \mathbb{N} \; 
            : \; \frac{1}{n} < \varepsilon$
\end{enumerate}}

\begin{proof}
\item {}
    \begin{enumerate}[i)]
        \item Έστω ότι το $ \mathbb{N} $ είναι άνω φραγμένο. Λόγω ότι 
            είναι και μη κενό ($ 1 \in \mathbb{N} $), από το αξίωμα 
            Πληρότητας, έχουμε ότι υπάρχει το $ \sup \mathbb{N} $. 
            Τότε από τη χαρακτηριστική ιδιότητα του supremum, έχουμε 
            ότι για $ \varepsilon = 1 >0, \; \exists n \in \mathbb{N} 
            \; : \; \sup \mathbb{N}-1 < n \Leftrightarrow n+1 > \sup
            \mathbb{N} $, άτοπο.

        \item Έστω ότι δεν ισχύει η πρόταση. Τότε  $ \exists y >0, \; 
            \forall n \in \mathbb{N} \; : \; n \leq y$. Δηλαδή
            το  $y$  είναι ά.φ. του $\mathbb{N}$, άτοπο.

        \item Έστω ότι δεν ισχύει η πρόταση. Τότε $ \exists 
            \varepsilon >0 , \; \forall n \in \mathbb{N} \; : \; 
            \frac{1}{n} \geq \varepsilon  \Leftrightarrow \ n \leq 
            \frac{1}{\varepsilon} $, δηλαδή $ \frac{1}{\varepsilon} $ 
            α.φ. του $ \mathbb{N} $, άτοπο. 
    \end{enumerate}
\end{proof}

\myprop{Έστω $ A \subseteq \mathbb{R}, \; A \neq \emptyset $ και το $A$ έχει μέγιστο 
στοιχείο, έστω $ x_{0} = \max A $. Τότε $ \sup A = \max A = x_{0}$.}

\begin{proof}
\item {}
    $ x_{0} = \max A \Rightarrow a \leq x_{0}, \; \forall a \in A $, άρα $ x_{0} $ α.φ. 
    του $A$. Άρα το $A$ είναι άνω φραγμένο και επειδή $ A \neq \emptyset $ υπάρχει 
    το $ \sup A $. 

    Ισχύει $ x_{0} \leq \sup A $, γιατι $ \sup A $ α.φ. του $A$.
    Όμως $ x_{0} $ επίσης α.φ. του $A$, άρα $ \sup A \leq x_{0} $.

    Άρα $ x_{0}= \sup A $.
\end{proof}

\myprop{Έστω $ A \subseteq \mathbb{R}, \; A \neq \emptyset $ και το $A$ έχει ελάχιστο
στοιχείο, έστω $ x_{0} = \min A $. Τότε $ \inf A = \min A = x_{0}$.}

\begin{proof}
    Ομοίως 
\end{proof}

\section{Χαρακτηριστική Ιδιότητα του Supremum}

\myprop{Έστω $ A \subseteq \mathbb{R} $, $ A \neq \emptyset $ και άνω φραγμένο. 
Έστω $ s $ α.φ. του $A$. Τότε 
\[
     s= \sup A \Leftrightarrow \forall \varepsilon >0, \; \exists a \in A \; 
     : \; s - \varepsilon < a
 \]}

 \begin{proof}
 \item {}
     \begin{description}
         \item[$ (\Rightarrow) $] 
             Έστω $ s = \sup A $. Έστω $ \varepsilon >0 $. Έχουμε $ 
             s - \varepsilon < s $ άρα το $ s- \varepsilon $ δεν είναι 
             α.φ. του $A$, άρα $ \exists a \in A $ τέτοιο ώστε $ 
             a > s- \varepsilon$. 

         \item [$ (\Leftarrow) $] 
             Έστω ότι $ \forall \varepsilon >0, \; \exists a \in A \; : 
             \; s- \varepsilon < a$. Θ.δ.ο. $ s = \sup A $. 


             \begin{minipage}{0.23\textwidth}
                 \begin{myitemize}
                     \item $ A \neq \emptyset $ \hfill \tikzmark{a}
                     \item $ A $ άνω φραγμένο \hfill  \tikzmark{b}
                 \end{myitemize}
             \end{minipage}

             \mybrace{a}{b}[$ \exists $ το $ \sup A $]

             Έστω ότι $ s \neq \sup A $, και λόγω ότι $ s $ α.φ. του $A$ 
             εχουμε ότι $sup A < s $. 

             Επιλέγουμε $ \varepsilon = s - \sup A > 0 $

             Τότε από την υπόθεση έχουμε ότι 
             υπάρχει $ a \in A \; : \; s - \varepsilon < a \Rightarrow s 
             - s + \sup A < a \Rightarrow \sup A < a $, άτοπο, γιατί 
             $ \sup A $ α.φ. του $A$.  
     \end{description} 
 \end{proof}

 \myprop{Έστω $ A \subseteq \mathbb{R} $, $ A \neq \emptyset $ και άνω φραγμένο. 
Έστω $ s $ α.φ. του $A$. Τότε 
\[
    s = \sup A \Leftrightarrow \forall n \in \mathbb{N} \; \exists a_{n} \in A
    \; : \; s - \frac{1}{n} < a_{n} 
\]}
 
 \mythm{Έστω $ A \subseteq \mathbb{R} $, $ A \neq \emptyset $ και άνω φραγμένο. 
Έστω $ s $ α.φ. του $A$. Τότε 
\[
    s = \sup A \Leftrightarrow \text{υπάρχει ακολουθία στοιχείων του $A$ με όριο το $s$} 
 \]}

 \begin{proof}
 \item {}
     \begin{description}
         \item [$ (\Rightarrow) $]
             Έστω $ s = \sup A $. Τότε από τη χαρακτηριστική ιδιότητα του supremum έχουμε ότι:
            \begin{align*}
                \exists a_{1} \in A \; &: \; s - 1 < a_{1} \\
                \exists a_{2} \in A \; &: \; s - \frac{1}{2}  < a_{2} \\
                                       &\vdots \\
                \exists a_{n} \in A \; &: \; s - \frac{1}{n}  < a_{n} 
            \end{align*} 

            Επομένως υπάρχει ακολουθία, η $ (a_{n})_{n \in \mathbb{N}} $, στοιχείων του $A$, 
            με την ιδιότητα $ s - \frac{1}{n} < a_{n} \leq s, \; \forall n \in \mathbb{N}$. 

            Παρατηρούμε ότι $ \lim_{n \to +\infty} s- \frac{1}{n} = \lim_{n \to +\infty} s = s $ 

            Οπότε από το Κριτήριο Παρεμβολής, έχουμε ότι $ \lim_{n \to +\infty} a_{n}= s $, 
            το ζητούμενο.

        \item [$ (\Rightarrow) $]
            Έστω $ s $ α.φ. του $A$ και έστω ότι υπάρχει ακολουθία $ (a_{n})_{n \in \mathbb{N}} $ 
            στοιχείων του $A$ ώστε $ \lim_{n \to +\infty} a_{n}= s$. Θ.δ.ο. $ s = \sup A $. 
            Πράγματι, 

            Έστω ότι $ s \neq \sup A $. Τότε $ \sup A < s \Rightarrow s - \sup A > 0 > 0 $. 

            Επιλέγω $ \varepsilon = s - \sup A >0 $. Οπότε από τον ορισμό του οριου, έχουμε ότι 
            $ \exists n_{0} \in \mathbb{N} \; : \; \forall n \geq n_{0} \quad \abs{a_{n} - a} < 
            s - \sup A \Leftrightarrow \underbrace{-(s - \sup A) < a_{n} - s} < s - \sup A 
            \Leftrightarrow a_{n} > \sup A$, άτοπο, γιατι $ \sup A $ α.φ. του $A$.
     \end{description}
 \end{proof}


 \myprop{\label{prop:leqsup}
    Έστω $ A, B $ μη-κενά, άνω φραγμένα υποσύνολα του $ \mathbb{R} $. Αν $ A \subseteq 
B$ τότε $ \sup A \leq \sup B $.}


\begin{proof}
\item {}
    Αρκεί να δείξω ότι $ \sup B $ είναι α.φ. του $A$. Πράγματι:

    Έστω $ x \in A \overset{A \subseteq B}{\Rightarrow } x \in B \Rightarrow 
    x \leq \sup B, \forall x \in A$. Άρα το $ \sup B $ είναι α.φ. του $A$.
\end{proof}

\myprop{Έστω $ A \subseteq \mathbb{R}, A \neq \emptyset $ και έστω $ \lambda A = 
    \{ x \in \mathbb{R} \; : \; x = \lambda a, \; a \in A \} = \{ \lambda a \; : \; 
    a \in A\} $. Τότε
    \begin{enumerate}[i)]
        \item $ A $ άνω φραγμένο και $ \lambda >0 \Rightarrow \exists $ το $ \sup A $
            και $ \sup (\lambda A) = \lambda \sup A $
        \item $ A $ κάτω φραγμένο και $ \lambda <0 \Rightarrow \exists $ το $ \sup A $
            και $ \sup (\lambda A) = \lambda \inf A $
\end{enumerate}}

\begin{proof}
\item {}
    \begin{enumerate}[i)]
        \item $A$ άνω φραγμένο $ \Rightarrow \exists M \in \mathbb{R} $ τέτοιο ώστε
            \begin{align*}
                a \leq M, \; \forall a \in A
                &\Rightarrow \lambda a \leq \lambda M, \; \forall a \in A \\
                & \Rightarrow x \leq \lambda M, \; \forall x \in \lambda A 
            \end{align*}
            δηλαδή, το  $ \lambda M $ είναι α.φ. του $ \lambda A $.

            Άρα το $ \lambda A $ είναι άνω φραγμένο και μη-κενό, γιατί $ A \neq 
            \emptyset $, άρα υπάρχει το $ \sup (\lambda A) $.

            Έστω $ s= \sup A $, το οποίο επίσης υπάρχει, γιατί $ A \neq \emptyset $ 
            και $A$ άνω φραγμένο. 

            \begin{myitemize}
                \item Αποδεικνύουμε ότι $ \lambda s $ είναι α.φ. του $ \lambda A $. 
                    Πράγματι
                    \begin{align*}
                        a \leq s, \; \forall a \in A 
                        & \Rightarrow \lambda a \leq \lambda s, \; \forall a \in A \\
                        & \Rightarrow x \leq \lambda s, \; \forall x \in \lambda A
                    \end{align*}
                    άρα το $ \lambda s $ είναι α.φ. του $ \lambda A $.

                \item Αποδεικνύουμε ότι $ \lambda s $ είναι το ελάχιστο άνω φράγμα 
                    του $ \lambda A$.  Πράγματι 

                    Έστω $M$ άνω φράγμα του $ \lambda A $ με $ M < \lambda s 
                    \overset{\lambda >0} {\Rightarrow} \frac{M}{\lambda} < s  $, 
                    άτοπο, γιατί $ \frac{ M}{\lambda} $ α.φ. του $A$ και $ s= \sup A $.

                    Πράγματι, αφού $ M $ ά.φ. του $ \lambda A $, τότε
                    \begin{align*}
                        x \leq M, \; \forall x \in \lambda A 
            &\Rightarrow \lambda a \leq M, \; \forall a \in A \\
            &\Rightarrow a \leq \frac{M}{\lambda}, \; \forall a \in A
                    \end{align*} 
                    άρα $ \frac{M}{\lambda} $ είναι α.φ. του Α.
            \end{myitemize}

        \item Ομοίως
    \end{enumerate}
\end{proof}

\myprop{Έστω $ A, B $, μη-κενά και άνω φραγμένα υποσύνολα του $ \mathbb{R} $. Τότε υπάρχει 
το $ \sup (A \cup B) $ και $ \sup (A \cup B) = \max \{ \sup A, \sup B \} $.}

\begin{proof}
    Έστω $ x \in A \cup B \Rightarrow \begin{cases} x \in A \overset{A \; \text{α.φ.}}{
        \Rightarrow} \exists M_{A} \in \mathbb{R} \; : \; x \leq M_{A} \\
        x \in B \overset{B \; \text{α.φ.}}{ \Rightarrow} \exists M_{B} \in \mathbb{R} 
        \; : \; x \leq M_{B}  \\
    \end{cases} $  
    Άρα $ A \cup B $ είναι άνω φραγμένο.

    Επίσης $ A \cup B \neq emtpy $, γιατί $ A, B \neq \emptyset $. 

    Άρα υπάρχει το $ \sup (A \cup B) $.

    $ A,B $ μη-κενά και άνω φραγμένα, άρα υπάρχουν τα $ \sup A, \sup B $ 
    και έστω χ.β.γ.  ότι $ \sup A \leq \sup B $. Τότε:
    \begin{gather*}
        a \leq \sup A \leq \sup B, \; \forall a \in A \\
        b \leq \sup B, \; \forall b \in B 
    \end{gather*}
    άρα $ x \leq \sup B, \; \forall x \in A \cup B  $, δηλαδή το $ \sup B $ 
    είναι α.φ. του $ A \cup B $, άρα $ \inlineequation[eq:1steq]{\sup (A \cup B) 
    \leq \sup B} $.

    Όμως $ B \subseteq A \cup B \Rightarrow \inlineequation[eq:2ndeq]{\sup B \leq 
    \sup (A \cup B)} $, από την πρόταση~\ref{prop:leqsup}. 
    Οπότε από τις σχέσεις~\eqref{eq:1steq} και \eqref{eq:2ndeq}, έχουμε ότι $
    \sup (A \cup B) = \sup B = \max \{ \sup A, \sup B \} $.
\end{proof}

\myprop{Έστω $ A, B $, μη-κενά και κάτω φραγμένα υποσύνολα του $ \mathbb{R} $. 
    Τότε υπάρχει το $ \inf (A \cup B) $ και $ \inf (A \cup B) = \min \{ \inf A, 
\inf B \} $.}

\begin{proof}
    Ομοίως
\end{proof}

\begin{lem}\label{lem:ineqs}
    Έστω $ x,y >0 $ και $ n \in \mathbb{N} $. Τότε
    \begin{enumerate}[i)]
        \item $ x \leq y \Leftrightarrow x^{n} \leq y^{n} $
        \item $ x < y \Leftrightarrow x^{n} < y^{n} $ \label{lem:ineqreal2}
        \item $ x = y \Leftrightarrow x^{n} = y^{n} $ \label{lem:ineqreal3}
    \end{enumerate}
\end{lem}

\mythm{Υπάρχει μοναδικός θετικός αριθμός $ x \in \mathbb{R} $ τέτοιος ώστε $ x^{2}=2 $.}

\begin{proof}
\item {}
    Έστω $ A = \{ a \in \mathbb{R} \; : \; a > 0, a^{2} < 2 \}  $. Έχουμε $ 1 \in A 
    \Rightarrow A \neq \emptyset $ 
    και αν $ a \in A \Rightarrow a^{2} < 2 \Rightarrow a^{2} < 4
    \overset{\ref{lem:ineqreal2}}{\Rightarrow} a <2,\; \forall a \in A $, 
    άρα το 2 είναι άνω φράγμα του $A$. Άρα υπάρχει το supremum του Α, έστω 
    $ x = \sup A $. 

    Θα δείξουμε ότι $ x > 0 $ και $ x^{2} = 2 $

    Πράγματι, $ 1 \in A \Rightarrow 1 \leq x $, γιατί $x$ α.φ. του $A$, οπότε $ x >0 $.

    Θα δείξουμε ότι $ x^{2} = 2 $ αποκλείοντας τις περιπτώσεις $ x^{2} <2 $ και 
    $ x^{2} > 2 $.
    \begin{myitemize}
        \item Έστω $ x^{2} > 2 $. Επιλέγω $ \varepsilon = \frac{1}{2} \min \{ x, 
            \frac{x^{2}-2}{2x}\} $.

            Καταρχάς $ \varepsilon > 0 $, γιατί $ x>0 $ και $ x^{2} -2 >0 $. 

            Επίσης από ορισμό του $ \varepsilon $ έχουμε ότι  $\varepsilon < x $ και 
            $ \varepsilon < \frac{x^{2}-2}{2x}$. 

            Άρα $ \varepsilon < \frac{x^{2}-2}{2x} \Rightarrow 
            2x \varepsilon < x^{2} - 2 \Leftrightarrow 2 < x^{2} -2x \varepsilon 
            \overset{\varepsilon ^{2}>0}{\leq}
            x^{2} -2x \varepsilon + \varepsilon ^{2} \Rightarrow 2 
            < (x- \varepsilon )^{2}   $

            Έστω $ a \in A \Rightarrow a^{2} <2 \Rightarrow a^{2}<2< 
            (x- \varepsilon )^{2} \Rightarrow a^{2}< (x- \varepsilon )^{2} 
            \Rightarrow a < x- \varepsilon, \; \forall a \in A$, 
            άρα $ x - \varepsilon $ α.φ. του $A$, άτοπο, γιατί $ x= \sup A $.

        \item Έστω $ x^{2}<2 $. Ομοίως.

            Οπότε $ x^{2}=2 $. 

            Τέλος, για να αποδείξουμε τη μοναδικότητα, έχουμε:

            Έστω ότι υπάρχουν $ 2 $ αριθμοί, έστω $ x,y $ τέτοιοι ώστε $ x^{2} =2 $ 
            και $ y^{2}=2 $. Τότε $ x^{2}=y^{2} \overset{\ref{lem:ineqreal3}}
            {\Rightarrow} x =y $.
    \end{myitemize}
\end{proof}

\mythm{Για κάθε $ x_{0} >0 $ και για κάθε $ n \in \mathbb{N} $ υπάρχει μοναδικός, θετικός 
πραγμτικός αριθμός $ p $, τέτοιος ώστε $ p^{n}= x_{0} $.}

\begin{cor}\label{cor:ineqs}
    Έστω $ x,y >0 $ και $ n \in \mathbb{N} $. Τότε
    \begin{enumerate}[i)]
        \item $ x \leq y \Leftrightarrow x^{\frac{1}{n}} \leq y^{\frac{1}{n}} $
        \item $ x <y \Leftrightarrow x^{\frac{1}{n}} < y^{\frac{1}{n}} $
        \item $ x =y \Leftrightarrow x^{\frac{1}{n}} = y^{\frac{1}{n}} $
    \end{enumerate}
\end{cor}

\begin{proof}
\item {}
    Αν $ x,y>0 $, τότε από το θεώρημα ύπαρξης $n$-οστής ρίζας υπάρχουν $ \tilde{x}  
    = x^{\frac{1}{n}} $ και $ \tilde{y} =y^{\frac{1}{n}} $, οπότε από το 
    λήμμα~\ref{lem:ineqs} έχουμε $ \tilde{x} \leq \tilde{y}  \Leftrightarrow 
    \tilde{x} ^{n} \leq \tilde{y} ^{n} $,
    οπότε $ x^{\frac{1}{n}} \leq y^{\frac{1}{n}} \Leftrightarrow x \leq y $.
\end{proof}

\begin{lem}\label{lem:ineqq}
    Έστω $ x,y \geq 0 $ με $ x<y $ και $ q \in \mathbb{Q}, \; q >0 $. 
    Τότε: $ x^{q} <y^{q} $
\end{lem}

\begin{proof}
\item {}
    Έστω $ p \in \mathbb{Q} $. Τότε υπάρχουν $ n,m \in \mathbb{N} $, τέτοιοι ώστε 
    $ q = \frac{n}{m} $. Τότε 

    \[ x<y \overset{\text{Πορ.}~\ref{cor:ineqs}}{\Rightarrow} x^{\frac{1}{m}} 
        < y^{\frac{1}{m}} \overset{\text{Λημ.}~\ref{lem:ineqs}}{\Rightarrow} 
    x^{\frac{n}{m}} < y^{\frac{n}{m}} \Rightarrow x^{q} < y^{q}  \] 
\end{proof}

\begin{lem}
    Έστω $ a > 0 $ και $ q_{1}, q_{2} \in \mathbb{Q} $ με $ q_{1} < q_{2} $. 
    \begin{enumerate}[i)]
        \item $ a>1 \Rightarrow a^{q_{1}} < a^{q_{2}} $
        \item $ a<1 \Rightarrow a^{q_{1}} > a^{q_{2}} $
    \end{enumerate}
\end{lem}

\begin{proof}
\item {}
    \begin{enumerate}[i)]
        \item \label{q1q2} $ a>1 \overset{\text{Λημ.~\ref{lem:ineqq}}}{\Rightarrow} 
            a^{q_{2}- q_{1}} > 1^{q_{2} - q_{1}} \overset{a^{q_{1}}>0}{\Rightarrow} 
            a^{q_{2}} > a^{q_{1} } $

        \item $ a<1 \Rightarrow \frac{1}{a} >1 \overset{\ref{q1q2}}{\Rightarrow} 
            (\frac{1}{a} )^{q_{1}} < (\frac{1}{a} )^{q_{2} } \Rightarrow a^{q_{1}} 
            > a^{q_{2}} $
    \end{enumerate}
\end{proof}


\section{Ακέραιο Μέρος}

\myprop{Έστω $ x \in \mathbb{R} $. Τότε υπάρχει μοναδικός ακέραιος αριθμός 
$ m \in \mathbb{Z} $ τέτοιος ώστε $\inlineequation[eq:akermer]{m \leq x < m +1}$.}

\begin{proof}
    Έστω $ x \in \mathbb{R} $. Επειδή $ \mathbb{N} $ όχι άνω φραγμένο, τότε από την 
    Αρχιμήδεια Ιδιότητα, $ \exists n \in \mathbb{N} $ με $ n > x $. Έστω $ S = \{ 
    n \in \mathbb{N} \; : \; n > x\} \Rightarrow S \neq \emptyset $ (υπάρχει το $n$). 
    Άρα το $S$ έχει ελάχιστο στοιχείο, έστω $ n_{0} $. Άρα $ n_{0}-1 \leq x \leq 
    n_{0} \Rightarrow m \leq n \leq m+1$, δηλαδή $ m= n_{0}-1 = [x] $.

    Αποδεικνύουμε την μοναδικότητα:

    Έστω $ m' $ με $ m' \leq x \leq m'+1 $ και έστω $ m' \leq m $. Τότε 
    $ m' +1 \leq m \leq x $, άτοπο.
\end{proof}

\mydfn{Έστω $ x \in \mathbb{R} $. Ο μοναδικός ακέραιος που ικανοποιεί την 
σχέση~\eqref{eq:akermer} ονομάζεται ακέραιο μέρος του $x$ και συμβολίζεται $ [x] $.}

\begin{example}
\item {}
    \begin{enumerate}[i)]
        \item $ [3]=3 $
        \item $ [3,14] = 3  $
        \item $ [-3,14] =-4 $
    \end{enumerate}
\end{example}

\section{Ρητοί και Άρρητοι}

\begin{lem}
\item {}
    \begin{enumerate}[i)]
        \item $n$ άρτιος $ \Leftrightarrow n^{2} $ άρτιος.
        \item $ n $ περιττός $ \Leftrightarrow n^{2} $ περιττός.
    \end{enumerate}
\end{lem}

\begin{proof}
\item {}
    \begin{enumerate}[i)]
        \item 
            \begin{description}
                \item [($ \Rightarrow $ )] 
                    Έστω $ n $ άρτιος $ \Rightarrow n =2k, \; k \in \mathbb{Z} 
                    \Rightarrow n^{2} = (2k)^{2} = 4k^{2} = 2\cdot (2k)^{2} $ άρτιος. 
                \item [($ \Leftarrow $)] Έστω $ n^{2} $ άρτιος και $n$ περιττός. Τότε 
                    $ n \cdot n = n^{2} $ περιττός. Άτοπο.
            \end{description}

        \item Ομοίως
    \end{enumerate}
\end{proof}

\begin{rem}
    Στις αποδείξεις τους παραπάνω λήμματος, χρησιμοποιήσαμε ότι 
    \begin{enumerate}[i)]
        \item άρτιος $ \cdot $ άρτιος = άρτιος
        \item περιττός $ \cdot $ περιτός = περιττός
        \item άρτιος $ \cdot $ περιττός = άρτιος
    \end{enumerate}
\end{rem}

\mythm{Ο $ \sqrt{2} $ είναι άρρητος.}

\begin{proof}
    Έστω $ \sqrt{2} $ όχι άρρητος. Άρα $ \sqrt{2} $ ρητός, δηλαδή $ \exists m,n 
    \in \mathbb{Z} $, με $ (m,n)=1 $, δηλαδή $ m,n $ πρώτοι μεταξύ τους
    τ.ω. $ \sqrt{2} = \frac{m}{n} \Rightarrow 2 = \frac{m^{2}}{n^{2}} \Rightarrow 
    m^{2} = 2n^{2} \Rightarrow m^{2}$ είναι άρτιος $ \Rightarrow m $ άρτιος 
    $ \Rightarrow m = 2k, \; k \in \mathbb{Z}$. 

    Άρα $ (2k)^{2} = 2n^{2} \Rightarrow 4k^{2}=2n^{2} \Rightarrow n^{2} = 2k^{2} 
    \Rightarrow n^{2} $ άρτιος $ \Rightarrow n $ άρτιος. Άτοπο, γιατί $ (m,n)=1 $.
\end{proof}

\begin{example}
    Υπάρχει αριθμός $ a \in \mathbb{R} $ τέτοιος ώστε $ a^{2} $ άρρητος και $ a^{4} $ 
    ρητός; 
\end{example}

\begin{proof}
    Ναί, ο $ a= \sqrt[4]{2} $. Πράγματι, $ a^{2} = \sqrt{2} $ άρρητος, και $ 
    a^{4} = 2$ ρητός.
\end{proof}

\begin{example}
    Υπάρχουν αριθμοί, $ a,b $ άρρητοι, ώστε $ a+b, a\cdot b $ να είναι ρητοί;
\end{example}

\begin{proof}
    Ναι οι $ a= \sqrt{2} $ και $ b= - \sqrt{2} $, οι οποίοι είναι και οι δύο άρρητοι 
    και $ a+b= \sqrt{2} - \sqrt{2} = 0 $ ρητός, και $ a\cdot b = \sqrt{2} \cdot (- 
    \sqrt{2}) = -2 $, επίσης ρητός.
\end{proof}

\begin{example}
    Ο $ \sqrt{3} $ είναι άρρητος.
\end{example}

\begin{proof}
    Έστω $ \sqrt{3} $ όχι άρρητος. Άρα $ \sqrt{3} $ ρητός, δηλαδή $ \exists m,n 
    \in \mathbb{Z} $, με $ (m,n)=1 $, δηλαδή $ m,n $ πρώτοι μεταξύ τους
    τ.ω. $ \sqrt{3} = \frac{m}{n} \Rightarrow 3 = \frac{m^{2}}{n^{2}} \Rightarrow 
    m^{2} = 3n^{2} \Rightarrow 3 \mid m^{2} \Rightarrow 3 \mid m  \Rightarrow m 
    \Rightarrow m = 3k, \; k \in \mathbb{Z}$. 

    Άρα $ (3k)^{2} = 3n^{2} \Rightarrow 9k^{2}=3n^{2} \Rightarrow n^{2} = 3k^{2} 
    \Rightarrow 3 \mid n^{2} \Rightarrow  3 \mid n$,  ατοπο, γιατί $ (m,n)=1 $.
\end{proof}

\begin{lem}
    $ 3 \mid m^{2} \Rightarrow 3 \mid m $
\end{lem}
 
\begin{proof}
    Έστω ότι $ 3 \nmid m \Rightarrow m = 3n +1 $ ή $ m = 3n+2 $. 
    Αν $ m=3n+1 \Rightarrow m^{2} = (3n+1)^{2} = \ldots 3(3n^{2}+2n)+1 \Rightarrow 
    3 \nmid m^{2}$, άτοπο και αν $ m =3n+2 \Rightarrow m^{2}=(3n+2)^{2} = \ldots = 
    3(3n^{2}+4n+1)+1 \Rightarrow 3 \nmid m^{2}$ άτοπο.
\end{proof}

\begin{rem}
    Ομοίως αποδεικνύονται και ότι οι $ \sqrt{5}$ και  $ \sqrt{6} $ είναι άρρητοι, 
    και γενικότερα ισχύει η επόμενη πρόταση.
\end{rem}

\myprop{$ \sqrt{n} $ άρρητος $ \Leftrightarrow n $ όχι τετράγωνο κάποιου φυσικού αριθμού.}

\begin{example}
    Να δείξετε ότι ο αριθμός $ \sqrt[3]{2} $ είναι άρρητος.
\end{example}

\begin{proof}
    Έστω $ \sqrt[3]{2} $ όχι άρρητος. Άρα $ \sqrt[3]{2} $ ρητός, δηλαδή $ \exists m,n 
    \in \mathbb{Z} $, με $ (m,n)=1 $, δηλαδή $ m,n $ πρώτοι μεταξύ τους
    τ.ω. $ \sqrt[3]{2} = \frac{m}{n} \Rightarrow 2 = \frac{m^{3}}{n^{3}} \Rightarrow 
    m^{3} = 2n^{3} \Rightarrow m^{3} $ άρτιος $ \Rightarrow m $ άρτιος. 

    Άρα $ (2k)^{3} = 2n^{3} \Rightarrow 2k^{3}=2n^{3} \Rightarrow n^{3} = 4k^{3} 
    \Rightarrow n $ άρτιος,  ατοπο, γιατί $ (m,n)=1 $.
\end{proof}

\begin{example}
    Να δείξετε ότι $ \sqrt{2} + \sqrt{3} $ είναι άρρητος.
\end{example}

\begin{proof}
    Έστω ότι $ \sqrt{2} + \sqrt{3} $ είναι ρητός. Τότε και ο $ (\sqrt{2} + \sqrt{3} )
    ^{2} = 2 + 2 \sqrt{6} + 3 = 2 \sqrt{6} + 5 $ είναι ρητός, δηλαδή ο $ \sqrt{6} $ 
    είναι ρητός, άτοπο.
\end{proof}

\myprop{
    $
    \left.
        \begin{tabular}{l}
            $q$ ρητός \\
            $r$ άρρητος
        \end{tabular}
\right\}  \Rightarrow (q+r) $ άρρητος}

\begin{proof}
\item {}
    Έστω $(q+r)$ άρρητος. Τότε ο $ r = (q+r)-r $ είναι ρητός ως άθροισμα δύο 
    ρητών. Άτοπο.
\end{proof}

\myprop{
    $
    \left.
        \begin{tabular}{l}
            $r$ άρρητος \\
            $s$ άρρητος
        \end{tabular}
\right\}  \Rightarrow (r+s) $ ρητός ή άρρητος.}

\begin{proof}
    Για παράδειγμα ο αριθμός $ \sqrt{2} + \sqrt{3} $ είναι άρρητος, αλλά ο αριθμός
    $ \underbrace{r}_{\text{άρρητος}}+ \underbrace{(q-r)}_{\text{άρρητος}} =q $ 
    είναι ρητός, αν ο $q$ είναι ρητός.
\end{proof}

\myprop{
    $
    \left.
        \begin{tabular}{l}
            $q$ ρητός \\
            $r$ άρρητος
        \end{tabular}
    \right\}  \Rightarrow  $ \begin{tabular}{l}
        $ (q \cdot r) $ ρητός $ \Leftrightarrow q =0 $ \\
        $(q \cdot r)$   άρρητος $ \Leftrightarrow q \neq 0 $
\end{tabular}}

\begin{proof}
    Πράγματι, αν $ q=0 $ (ρητός) και $ r $ άρρητος, τότε $ q \cdot r =0 $ είναι ρητός, 
    αλλά αν $ q \neq 0 $ τότε $ (q \cdot r) $ είναι άρρητος, γιατί αλλιώς ο 
    $ r = \underbrace{(q \cdot r)}_{\text{ρητός}} \cdot \underbrace{q^{-1}}_{\text{
    ρητός}} $ είναι ρητός ως γινόμενο ρητών. Άτοπο.
\end{proof}


\section{Πυκνότητα Ρητών και Άρρητων}

\myprop{Σε κάθε ανοιχτό διάστημα πραγματικών αριθμών, υπάρχει ρητός.}

\begin{proof}
\item {}
    Θα δείξουμε ότι $ (a,b) $ διάστημα στο $ \mathbb{R} \Rightarrow \exists q 
    \in \mathbb{Q} $, τ.ω. $ a < q < b $. Πράγματι:

    Αν $ a<b \Rightarrow b-a >0 \overset{\text{Αρχ.Ιδιοτ.}}{\Rightarrow} \exists 
    n_{0} \in \mathbb{N} \; : \; \frac{1}{n_{0}} < b-a \Rightarrow b n_{0} - a n_{0} 
    >1 \Rightarrow n_{0}b > 1+ n_{0}a$.

    Τότε 
    \begin{gather*}
        [n_{0}a] \leq n_{0}a < [n_{0}a]+1 \leq n_{0}a +1 < n_{0}b \Leftrightarrow \\
        n_{0}a < [n_{0}a]+1 < n_{0}b \Leftrightarrow \\
        a < \underbrace{\frac{[n_{0}a]+1}{n_{0}}}_{\text{ρητός}}< b
    \end{gather*}
\end{proof}

\myprop{Σε κάθε ανοιχτό διάστημα πραγματικών αριθμών, υπάρχει άρρητος.}

\begin{proof}
    Θα δείξουμε ότι $ (a,b) $ διάστημα στο $ \mathbb{R} \Rightarrow \exists r \in 
    \mathbb{R} \setminus \mathbb{Q} $ τ.ω. $a < r < b$. Πράγματι:

    Έστω $ \sqrt{2} $ (τυχαίος) άρρητος. Τότε

    $ a < b \Leftrightarrow a - \sqrt{2} < b- \sqrt{2} \overset{\text{Πυκν. Ρητών}}{\Rightarrow} \exists q \in \mathbb{Q} \; : \;  a - \sqrt{2} < q < b - 
    \sqrt{2} \Leftrightarrow  a < \underbrace{q + \sqrt{2}}_{\text{άρρητος}} < b $ 
\end{proof}


\end{document}
