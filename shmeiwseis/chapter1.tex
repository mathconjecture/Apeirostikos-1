\documentclass[main.tex]{subfiles}

\begin{document}



\section{Μαθηματική Επαγωγή}

\begin{thm}
    Έστω $ S \subseteq \mathbb{N} $ τέτοιο ώστε:
\begin{minipage}[t]{0.25\textwidth}
    \begin{enumerate}[i)]
        \item  $ 1 \in S $ \hfill \tikzmark{a}
        \item  $ n \in S \Rightarrow n + 1 \in S $ \hfill \tikzmark{b}
    \end{enumerate} 
\end{minipage}
\mybrace{a}{b}[$ S = \mathbb{N} $]

\end{thm}

\begin{example}
    Να αποδείξετε ότι $ 4^{n} \geq n^{2}, \; \forall n \in \mathbb{N} $.
    \begin{proof}
    \item {}
        \begin{itemize}
            \item Για $ n=1 $, έχω: $ 4^{1} \geq 1^{2} $, ισχύει.
            \item Έστω ότι η ανισότητα ισχύει για $n$, δηλ. 
                $\inlineequation[eq:epagex1]{4^{n} \geq n^{2}}$
            \item Θα δείξουμε ότι ισχύει και για $ n+1 $. Πράγματι:
                \begin{align*}
                4^{n+1} = 4^{n}\cdot 4 \overset{\eqref{eq:epagex1}}{\geq} n^{2}\cdot 4 
                = 4n^{2} = n^{2} + 2n^{2} + n^{2} \geq n^{2} + 2n + 1 = (n+1)^{2}.
                \end{align*}
        \end{itemize}
    \end{proof}
\end{example}

\begin{example}
    Να αποδείξετε ότι $ 2^{n} \geq n^{3}, \; \forall n \geq 10 $.
    \begin{proof}
    \item {}
        \begin{itemize}
            \item Για $ n=10 $, έχω: $ 2^{10} = 1024 \geq 1000 = 10^{3}  $, 
                ισχύει.
            \item Έστω ότι η ανισότητα ισχύει για $n$, δηλ. 
                $\inlineequation[eq:epagex2]{2^{n} \geq n^{3}}$.
            \item Θα δέιξουμε ότι ισχύει και για $ n+1 $. Πράγματι:
               \begin{align*}
                   2^{n+1} = 2^{n} \cdot 2 \overset{\eqref{eq:epagex2}}{\geq} n^{3} 
                   \cdot 2 = 2n^{3} = n^{3} + n^{3} &\geq n^{3} + 10n^{2} \\
                                 &\geq n^{3} + 6n^{2} + 3n^{2} + n^{2} \\ 
                                 &\geq n^{3} + 3n^{2} + 3n + 1 \\ 
                                 &= (n+1)^{3}
               \end{align*} 
        \end{itemize}
    \end{proof}
\end{example}

\section{Ανισότητα Bernoulli}
\[
    \boxed{(1+a)^{n} \geq 1 + na, \quad \forall a \geq 1, \; \forall n \in
    \mathbb{N}}
 \] 
 \begin{proof}
     \begin{itemize}
         \item Για $ n=1 $, έχω: $ (1+a)^{1} = 1+a \geq 1+a = 1 + 1 \cdot a $
         \item Έστω ότι η ανισότητα ισχύει, για $ n $, δηλ. $\inlineequation[eq:bern]
             {(1+a)^{n} \geq 1 + na}$
         \item Θα δείξουμε ότι ισχύει και για $ n+1 $. Πράγματι:
             \begin{align*}
                 (1+a)^{n+1} = (1+a)^{n}(1+a) \overset{\eqref{eq:bern}}
                 {\smash{\underset{a \geq 1}{\geq}}}
                 (1+na)(1+a) &= 1 + a + na + na^{2} \\
                             &= 1 + (n+1)a + na^{2} \\
                             &\geq 1 + (n+1)a
             \end{align*}

     \end{itemize}
 \end{proof}

\section{Αθροίσματα}

\begin{dfn}
    Έστω $ a_{1}, a_{2}, \ldots, a_{n} \in \mathbb{R} $. Ορίζουμε 
    \[ \sum_{n=1}^{N} a_{n} = a_{1} + a_{2} + \cdots + a_{n} \]
\end{dfn}

\begin{example}
    Να δείξετε ότι $ \sum_{n=3}^{6} n^{2} = \sum_{n=4}^{7} (n-1)^{2}   $
\end{example}

\begin{proof}
    \begin{align*}
        \sum_{n=4}^{7} (n-1)^{2} &= (4-1)^{2}+(5-1)^{2}+(6-1)^{2}+(7-1)^{2} = 3^{2}+4^{2}+5^{2}+6^{2} = \sum_{n=1}^{3} n^{2} 
    \end{align*}
\end{proof}

\begin{example}
    Να δείξετε ότι $ \sum_{n=3}^{6} n^{2} = \sum_{n=1}^{4} (n+2)^{2}   $
\end{example}
 
\begin{proof}
    \begin{align*}
        \sum_{n=1}^{4} (n+2)^{2} = (1+2)^{2}+(2+2)^{2}+(3+2)^{2}+(4+2)^{2}=
        3^{2}+4^{2}+5^{2}+6^{2} = \sum_{n=3}^{6} n^{2} 
 \end{align*}
\end{proof}

\begin{example}
    Να δείξετε ότι $ \sum_{n=1}^{5} (2n+1) - \sum_{n=3}^{5} (2n+1)  =  
    \sum_{n=1}^{2} (2n+1) $
\end{example}

\begin{proof}
    \begin{align*}
        \sum_{n=1}^{5} (2n+1)- \sum_{n=3}^{5} (2n+1) = 
        (3+5+7++11) - (7+9+11) = 3 + 5 = \sum_{n=1}^{2} (2n+1) 
    \end{align*}
\end{proof}

\begin{example}
    Να δείξετε ότι $ \sum_{n=1}^{5} \frac{1}{n(n+1)} = 1 - \frac{1}{6}  $
\end{example}

\begin{proof}
    \begin{align*}
        \sum_{n=1}^{5} \frac{1}{n(n+1)} &= \frac{1}{1\cdot 2} + \frac{1}{
        2 \cdot 3} + \frac{1}{3 \cdot 4} + \frac{1}{4 \cdot 5} + \frac{1}{
    5 \cdot 6} \\
    &= \left(\frac{1}{1} - \frac{1}{2}\right) + \left(\frac{1}{2} - \frac{1}{3}
    \right) + \left(\frac{1}{3} - \frac{1}{4}\right) + \left(\frac{1}{4} - \frac{1}{5}\right) +
    \left(\frac{1}{5} - \frac{1}{6}\right)  = 1 - \frac{1}{6}
    \end{align*}
\end{proof}



\begin{example}
    Να δείξετε ότι $ \sum_{n=1}^{100} \frac{1}{n(n+1)} \leq \sum_{n=1}^{100}
    \frac{1}{n^{2}} $.
\end{example}

\begin{proof}
    \[
        \frac{1}{n(n+1)} = \frac{1}{n^{2}+n} \leq \frac{1}{n^{2}} ,\; 
        \forall n \in \mathbb{N} 
     \] 
     Οπότε,
     \[
         \sum_{n=1}^{100} \frac{1}{n(n+1)} \leq \sum_{n=1}^{100} 
     \frac{1}{n^{2}}
      \]
\end{proof}

\section{Απόλυτη Τιμή}

\begin{dfn}
    Για κάθε $ a \in \mathbb{R} $, θέτουμε
   \[
       \abs{a} = \begin{cases} a, & a \geq 0 \\
       -a, & a < 0\end{cases}  
    \] 
\end{dfn}

\begin{rem}
    Ισχύει, 
    \begin{itemize}
\item  $ \abs{a} \geq 0, \;  \forall a \in \mathbb{R} $
\item $ a \leq \abs{a}, \; \forall a \in \mathbb{R} $
    \end{itemize}
\end{rem}

\begin{prop}
    Έστω $ \theta > 0 $. Τότε 
    \[
        \abs{a} \leq \theta \Leftrightarrow - \theta \leq a \leq \theta  
     \]

     \begin{proof}
     \item {}
         \begin{description}
             \item [$(  \Rightarrow ) $] 
                 Έστω ότι $ \abs{a} \leq \theta, \; a \in \mathbb{R} $ και 
                 $ \theta >0 $. 
                 \begin{itemize}
                     \item Έστω $ a \geq 0 $. Τότε $ \abs{a} = a $, οπότε:
                         \[
                             0 \leq \abs{a} \leq \theta \Rightarrow 
                             0 \leq a \leq \theta \Rightarrow - \theta 
                              \leq a \leq \theta
                          \] 
                      \item Έστω $ a < 0 $. Τότε $ \abs{a} = -a $, οπότε:
                          \[
                              0 \leq \abs{a} \leq \theta \Rightarrow 
                              0 < -a \leq \theta \Rightarrow 
                               - \theta \leq a < 0 \Rightarrow 
                               - \theta \leq a \leq \theta 
                           \] 
                 \end{itemize}
             \item [$(\Leftarrow)$] Έστω $ - \theta \leq a \leq \theta $ 
                 για $ a \in \mathbb{R} $ και $ \theta >0 $.
                 \begin{itemize}
                     \item Έστω $ a \geq 0 $. Τότε $ \abs{a} = a $. Οπότε 
                         $ \abs{a} = a \leq \theta $.
                     \item Έστω $ a <0 $. Τότε $ \abs{a} = -a $. Οπότε
                         $ \abs{a} = -a \leq \theta $.
                 \end{itemize}
         \end{description}
     \end{proof}
\end{prop}

\begin{prop}
    Για κάθε $ a, b \in \mathbb{R} $, ισχύει:
    \begin{enumerate}[i)]
        \item $ \abs{a+b} \leq \abs{a} + \abs{b}   $
        \item $ \abs{a} - \abs{b} \leq \abs{a+b}  $
    \end{enumerate}
\end{prop}

    \begin{proof}
    \item {}
        \begin{enumerate}[i)]
            \item \[ a \leq \abs{a} \Rightarrow - \abs{a} \leq a \leq \abs{a}
        \] 
        \[
            b \leq \abs{b} \Rightarrow - \abs{b} \leq b \leq \abs{b} 
         \] 
         Με πρόσθεση, έχουμε: 
         \[
             - (\abs{a} + \abs{b} ) \leq a + b \leq ( \abs{a} + \abs{b}) 
          \] 
Οπότε από την προηγούμενη πρόταση, ισχύει:
$ \abs{a+b} \leq \abs{a} + \abs{b} $
\item Θέτω $ x = a+b $ και $ y = -a $. Για τους $ x,y $ από το $ i) $ 
    ερώτημα, έχουμε:
    \[
        \abs{x+y} \leq \abs{x} + \abs{y} \Rightarrow \abs{b} \leq 
        \abs{a+b} + \abs{a} \Rightarrow \abs{a+b} \geq \abs{b} - \abs{a}
     \] 
\end{enumerate} 
\end{proof} 

\begin{rem}
    Χρησιμοποιώντας το πρώτο υποερώτημα της προηγούμενης πρότασης, θέτωντας
    όπου $ b = -b $, έχουμε:
    \[
        \abs{a-b} \leq \abs{a} + \abs{b} 
     \] 
    Χρησιμοποιώντας το δεύτερο υποερώτημα της προηγούμενης πρότασης, θέτωντας
    όπου $ b = -b $, έχουμε:
\[
    \abs{a} - \abs{b} \leq \abs{a-b} 
 \]
 Με εναλλαγή των ρόλων των $ a $ και $ b $, το δεύτερο υποερώτημα, επίσης 
 μας δίνει:
 \[
     \abs{b} - \abs{a} \leq \abs{b+a} 
  \] 
Οπότε, τελικά έχουμε:
\[
    \boxed{\abs{\abs{b} - \abs{a}} \leq \abs{a \pm b} \leq \abs{a} + \abs{b}  }
 \]
 
\end{rem}

\section{Μέγιστο και Ελάχιστο}

\begin{dfn}
    Έστω $ A \subseteq \mathbb{R} $. Λέμε ότι το $A$ έχει μέγιστο 
    (αντίστοιχα ελάχιστο) στοιχείο, αν υπάρχει $ x_{0} \in A $ τέτοιο 
    ώστε $ x \leq x_{0}, \; \forall x \in A $ (αντίστοιχα $ x_{0} \leq 
    x, \; \forall x \in A$).
\end{dfn}

\begin{rem}
\item {}
    \begin{itemize}
        \item Το μέγιστο του συνόλου $A$, συμβολίζεται: $ \max A $
        \item Το ελάχιστο του συνόλου $A$, συμβολίζεται $ \min A $
    \end{itemize}
\end{rem}

\begin{example}
    Έστω $ A = [0,3] = \{ x \in \mathbb{R} \; : \; 0 \leq x \leq 3 \} $. 
    Το $ x_{0}= 3 $ είναι μέγιστο του $A$, γιατί $ 3 \in A $ και 
    $ a \leq 3, \; \forall a \in A $. Ομοίως $ x_{0}= 0 $ είναι ελάχιστο 
    του $A$.
\end{example}

\begin{example}
    Να δείξετε ότι το σύνολο $ A = (0,3) $ δεν έχει μέγιστο στοιχείο.
\end{example}

\begin{proof}
\item {}
    $ A = (0,3) = \{ x \in \mathbb{R} \; : \; 0 < x < 3 \} $. 
Έστω ότι $ x_{0} = \max A $. Έχουμε  $ x_{0} \in A \Rightarrow  x_{0} 
< 3$. Άρα  $ (x_{0}, 3) \neq \emptyset \Rightarrow (x_{0},3) \cap A \neq 
\emptyset \Rightarrow \exists a \in (x_{0},3) \cap A $. Δηλαδή $ 
\exists a \in A$ τέτοιο ώστε $ a > x_{0} $. Άτοπο, γιατί $ x_{0}= \max A $.
\end{proof}

\begin{prop}
    Αν $ A \subseteq \mathbb{R} $ έχει μέγιστο στοιχείο, τότε αυτό είναι 
    μοναδικό.
\end{prop}

\begin{proof}
    Έστω ότι το $A$ έχει δυο μέγιστα στοιχεία,  $ x_{0}, {x_{0}}' $,  με 
    $ x_{0} \neq {x_{0}}'$. 

    Τότε 

    \begin{itemize}
        \item $ a \leq x_{0}, \; \forall a \in A \xRightarrow{{x_{0}}' 
            \in A } {x_{0}}'  \leq x_{0} $ \tikzmark{a}
        \item $ a \leq {x_{0}}', \; \forall a \in A \xRightarrow{x_{0} 
            \in A } x_{0} \leq {x_{0}}' $ \tikzmark{b}
            \mybrace{a}{b}[$ x_{0} = {x_{0}}' $, άτοπο.] 
    \end{itemize}
\end{proof}

\begin{prop}
    Αν $ A \subseteq \mathbb{R} $ έχει ελάχιστο στοιχείο, τότε αυτό είναι 
    μοναδικό.
\end{prop}

\begin{proof}
   Ομοίως 
\end{proof}

 \begin{dfn}
     Έστω $ A \subseteq \mathbb{R}, \; A \neq \emptyset $. Λέμε ότι το 
     $A$ είναι άνω φραγμένο (αντίστοιχα κάτω φραγμένο), αν $ 
     \exists x_{0} \in \mathbb{R}$ τέτοιο ώστε $ a \leq x_{0}, \; 
     \forall a \in A$ (αντίστοιχα $ x_{0} \leq a, \; \forall a \in A $).
 \end{dfn}

 \begin{example}
 \item {}
     \begin{enumerate}[i)]
         \item Το  σύνολο $ A = (0,3) $  έχει άνω φράγματα τους $ 3, 4, 144, 
             \ldots$ και κάτω φράγματα τους $ 0, -1, -2, -128, \ldots $ 

         \item Το σύνολο $ B = (0,+\infty) $ δεν είναι άνω φραγμένο.
     \end{enumerate}
 \end{example}

 \begin{rem}
     Το  $ 3 $ είναι το ελάχιστο από τα άνω φράγματα του $A$, ενώ το $ 0
     $ είναι το μέγιστο από τα κάτω φράγματα του $A$. 
 \end{rem}

 \begin{prop}
     Έστω $ A \subseteq \mathbb{R}, A \neq \emtpy $ και έστω 
     $ -A = \{ x \in \mathbb{R} \; : \; x = -a, \; a \in A \} = \{ 
     -a \; : \; a \in A\} $. Αν $M$ είναι άνω φράγμα 
     του $A$, τότε το $ -M $ είναι κάτω φράγμα του συνόλου $ -A $.
 \end{prop}

 \begin{proof}
   Έστω $M$ α.φ. του $A$. Τότε 
   \begin{align*}
       a &\leq M, \; \forall a \in A \Rightarrow  \\
       -a &\geq -M, \; \forall a \in A \xRightarrow{x= -a, a \in A} \\
       x &\geq -M, \; \forall x \in -A
   \end{align*}
 οπότε $ -M $  κ.φ. του $ -A $.
 \end{proof}

 \begin{dfn}
     Έστω $ A \subseteq \mathbb{R}, \; A \neq \emptyset $. Λέμε ότι το $A$ 
     είναι φραγμένο, αν είναι άνω και κάτω φραγμένο. 
 \end{dfn}

 \begin{prop}
     Ένα σύνολο $ A \subseteq \mathbb{R}, \; A \neq \emptyset $ είναι 
     φραγμένο αν και μόνο αν $ \exists M>0 \; : \; \abs{a} \leq M, \; 
     \forall a \in A$.
 \end{prop}

 \begin{proof}
 \item {}
    \begin{description}
        \item [$ (\Rightarrow) $] Έστω $A$ φραγμένο. Τότε 
            $ \exists m',M' \in \mathbb{R} \; : \; m' \leq a \leq M', \; 
            \forall a \in A $.
       
            Επιλέγω $ M = \max \{ \abs{m'}, \abs{M'} \} $. Τότε $ M >0 $ και 
            \begin{gather*}
                -M \leq - \abs{m'} \leq m' \leq a \leq M' \leq \abs{M'} 
                \leq M, \quad \forall a \in A \\
                -M \leq a \leq M, \quad \forall a \in A \\
                \abs{a} \leq M, \quad \forall a \in A
            \end{gather*}

        \item [(Β΄ τρόπος)]
            Επιλέγω $ M > \max \{ \abs{m'}, \abs{M'} \} $. Τότε $ M >0 $ και 
\begin{gather*}
    \abs{m'} < M \Leftrightarrow - M < m' < M \\
    \abs{M'} < M \Leftrightarrow -M < M' < M 
\end{gather*}
 οπότε: 
 \begin{gather*}
     -M < m' \leq a \leq M' < M, \quad a \in A \\
     \abs{a} \leq M, \quad a \in A
 \end{gather*}
        

        \item [$ (\Leftarrow) $]
            Έστω ότι $ \exists M>0 \; : \; \abs{a} \leq M, \; \forall a \in 
            A \Leftrightarrow -M \leq a \leq M, \; \forall a \in A$. 
            
            Τότε προφανώς έχουμε ότι $-M $ και $ M $, 
            είναι αντίστοιχα άνω και κάτω φράγματα του $A$, και άρα $A$ 
            φραγμένο.
    \end{description} 
 \end{proof}

 \section{Supremum και Infimum}

 \begin{dfn}
     Έστω $ A \subseteq \mathbb{R}, A \neq \emptyset $ και $A$ άνω 
     φραγμένο. Αν υπάρχει άνω φράγμα $s$ του $A$ τέτοιο ώστε 
\[
    s \leq M, \quad \text{για κάθε M άνω φράγμα του A}, 
 \] 
 τότε το $s$ ονομάζεται supremum του $A$ και συμβολίζεται $ s=sup A $.
 \end{dfn}

 \begin{rem}
     Αν το $A$ δεν είναι άνω φραγμένο, τότε γράφουμε ότι $ \sup A = + \infty $.
 \end{rem}

 \begin{example}
     Έστω $ A = (0,3) $. Θα δείξουμε ότι $ \sup A = 3 $. Πράγματι, 
     το $ 3 $ προφανώς είναι α.φ. του $A$. Θα δείξουμε ότι είναι το
     ελάχιστο από τα άνω φράγματα του $A$. Έστω ότι δεν είναι, δηλαδή 
     $ \exists M $ α.φ. του $A$, ώστε $ M < 3 \Rightarrow (M,3) \neq 
     \emptyset \Rightarrow (M,3) \cap A \neq \emptyset \Rightarrow \exists 
     a \in (M,3) \cap A $. Τότε, έχουμε ότι $ a \in A $ και $ a > M $, 
     άτοπο, γιατί $M$ είναι ά.φ. του $A$. Ομοίως αποδεικνύουμε ότι $ 
     \inf A = 0$.
 \end{example}

 \section{Αξίωμα Πληρότητας}

 Κάθε μη κενό και άνω φραγμένο υποσύνολο των πραγματικών αριθμών έχει 
 supremum, δηλαδή έχει ελάχιστο άνω φράγμα.

 \begin{prop}
     Έστω ότι για τον $a \in \mathbb{R}$ ισχύει ότι 
     \[
        0 \leq a < \varepsilon, \quad \forall \varepsilon >0 \Rightarrow a =0 
      \] 
 \end{prop}

      \begin{proof}
      \item {}
         Έστω ότι $ 0 \leq a < \varepsilon, \quad \varepsilon >0 $ και 
         $ a \neq 0 \xRightarrow{a \geq 0} a > 0$. Τότε για $ \varepsilon = 
         a > 0$, έχουμε ότι $ 0 \leq a < a $, άτοπο.
      \end{proof}

      \begin{prop}[Αρχιμήδεια Ιδιότητα]
      \item {}
          \begin{enumerate}[i)]
              \item Το $ \mathbb{N} $ δεν είναι άνω φραγμένο.
              \item $ \forall y > 0, \; \exists n \in \mathbb{N} \; ; \; 
                  n > y$
              \item $ \forall \varepsilon >0, \; \exists n \in \mathbb{N} \; 
                  : \; \frac{1}{n} < \varepsilon$
          \end{enumerate}
      \end{prop}

      \begin{proof}
      \item {}
          \begin{enumerate}[i)]
              \item Έστω ότι το $ \mathbb{N} $ είναι άνω φραγμένο. Λόγω ότι 
                  είναι και μη κενό ($ 1 \in \mathbb{N} $), από το αξίωμα 
                  Πληρότητας, έχουμε ότι υπάρχει το $ \sup \mathbb{N} $. 
                 Τότε από τη χαρακτηριστική ιδιότητα του supremum, έχουμε 
                 ότι για $ \varepsilon = 1 >0, \; \exists n \in \mathbb{N} 
                 \; : \; \sup \mathbb{N}-1 < n \Leftrightarrow n+1 > \sup
                 \mathbb{N} $, άτοπο.

             \item Έστω ότι δεν ισχύει η πρόταση. Τότε  $ \exists y >0, \; 
                 \forall n \in \mathbb{N} \; : \; n \leq y$. Δηλαδή
                 το  $y$  είναι ά.φ. του $\mathbb{N}$, άτοπο.

             \item Έστω ότι δεν ισχύει η πρόταση. Τότε $ \exists 
                 \varepsilon >0 , \; \forall n \in \mathbb{N} \; : \; 
                 \frac{1}{n} \geq \varepsilon  \Leftrightarrow \ n \leq 
                 \frac{1}{\varepsilon} $, δηλαδή $ \frac{1}{\varepsilon} $ 
                 α.φ. του $ \mathbb{N} $, άτοπο. 
          \end{enumerate}
      \end{proof}

\end{document}
